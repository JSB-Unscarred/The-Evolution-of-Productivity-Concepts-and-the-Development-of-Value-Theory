% !TeX root = ../2019080346_Mason.tex

\chapter{引言}

\section{研究背景及意义}

新质生产力作为新时代中国特色社会主义经济思想的重要创新成果,其理论内涵与实践价值具有深远意义。该概念由习近平总书记在2023年新时代推动东北全面振兴座谈会上首次提出\cite{YinBoGuLaoLaoBaWoDongBeiDeChongYaoShiMingFenLiPuXieDongBeiQuanMianZhenXingXinPianZhang2023}。而后,习近平总书记又在不同场合多次提到要发展新质生产力,2024年\cite[17]{LiQiangZhengFuGongZuoBaoGao2024Nian3Yue5RiZaiDiShiSiJieQuanGuoRenMinDaiBiaoDaHuiDiErCiHuiYiShang2024}、2025年\cite[19]{LiQiangZhengFuGongZuoBaoGao2025Nian3Yue5RiZaiDiShiSiJieQuanGuoRenMinDaiBiaoDaHuiDiSanCiHuiYiShang2025}政府工作报告和中共二十届三中全会\cite[8]{ZhongGuoGongChanDangDiErShiJieZhongYangWeiYuanHuiDiSanCiQuanTiHuiYiGongBao2024}也强调要发展新质生产力。在中央的重视下,新质生产力很快成为当前经济发展和经济学界的主旋律。

根据笔者的文献调研,当前的相关文献主要通过与旧质生产力的对比来阐释“新质”,从马克思的著作中找到与生产力概念相关的原文来阐释“生产力”。在“新质”方面,从理论内涵上看,新质生产力是一种由更多生产要素的高质量组合,以关键性颠覆性技术突破为特点的一种生产力\cite[141-142]{GaoFanXinZhiShengChanLiDeTiChuLuoJiDuoWeiNeiHanJiShiDaiYiYi2023}\cite[1-2]{ZhouWenLunXinZhiShengChanLiNeiHanTeZhengYuChongYaoZhaoLiDian2023};从发展过程来看,新质生产力是一个相对于旧质生产力的概念,是对旧质生产力的更新换代。从蒸汽机到电动机再到计算机,每一种技术都代表了各个时期的新质生产力\cite[28]{CaiJiMingXinZhiShengChanLiDeFaZhanDuiJieZhiChuangZaoHeJingJiZengChangDeGongXian2024};从现实角度来看,当今世界正经历新一轮科技革命\cite[06]{XiJinPingJiaKuaiJianSheKeJiQiangGuoShiXianGaoShuiPingKeJiZiLiZiQiang2022},战略新兴产业和未来产业是新质生产力的主要载体\cite[9]{XiJinPingJingJiSiXiangYanJiuZhongXinXinZhiShengChanLiDeNeiHanTeZhengHeFaZhanChongDian2024}。在“生产力”方面,根据马克思“生产力当然始终是有用的、具体的劳动的生产力,它事实上只决定有目的的生产活动在一定时间内的效率”\cite[59]{ZhongGongZhongYangMaKeSiEnGeSiLieNingSiDaLinZhuZuoBianYiJuMaKeSiEnGeSiWenJiDi5Juan2009}的定义,目前的文献一致肯定了新质生产力作为一种生产力在使用价值创造中所起的作用及其对经济发展的重要贡献\cite[16]{CaiJiMingXinZhiShengChanLiCanYuJieZhiChuangZaoDeLiLunTanTaoHeShiJianYingYong2024}\cite{YangYuZhenXinZhiShengChanLiLiLunDuiMaKeSiShengChanLiShengChanGuanXiLiLunDeShouZhengHeChuangXin2025}\cite{XieFuShengMaKeSiDeShengChanLiLiLunYuFaZhanXinZhiShengChanLi2024}\cite{HuYingZaiLunXinZhiShengChanLiDeNeiHanTeZhengYuXingChengLuJingYiMaKeSiShengChanLiLiLunWeiShiJiao2024},但只有少数文献基于广义价值论分析了新质生产力参与价值创造的原理\cite{CaiJiMingXinZhiShengChanLiDeFaZhanDuiJieZhiChuangZaoHeJingJiZengChangDeGongXian2024}\cite{CaiJiMingXinZhiShengChanLiCanYuJieZhiChuangZaoDeLiLunTanTaoHeShiJianYingYong2024}。在笔者看来,这一问题的原因是绝大部分现有的文献都遵循了马克斯在《资本论》中的叙述:“不管生产力发生了什么变化,同一劳动在同样的时间内提供的价值量总是相同的。”\cite[60]{ZhongGongZhongYangMaKeSiEnGeSiLieNingSiDaLinZhuZuoBianYiJuMaKeSiEnGeSiWenJiDi5Juan2009}

然而,谷书堂教授提出的“价值总量之谜”\cite[6-7]{GuShuTangQiuJieJieZhiZongLiangZhiMiLiangTiaoSiLuDeBiJiao2002}却又暗示着新质生产力的不断涌现推动了社会价值总量的上升。具体而言,国内生产总值(GDP)是一定时期内社会价值量的总和\cite[107]{ChenDeDiGuoFangJingJiDaCiDian2001},这意味着在投入劳动总量基本不变的情况下,以不变价格衡量的GDP(实际GDP)也会保持基本不变,而这显然与我国改革开放以来的实践事实相矛盾。目前来看,唯有将马克思“劳动生产力与价值量成正比”原理的适用范围从单个生产者推广到部门和全社会的广义价值论,能比较合理、严谨地解释这一谜题\cite{CaiJiMingJiShuJinBuJingJiZengChangYuJieZhiZongLiangZhiMiJiYuGuangYiJieZhiLunDeJieShi2019},因此,笔者希望用广义价值论对历史上的”生产力”作一次系统的梳理,以期为新质生产力研究中的“生产力”概念作一个新的阐释。

在此基础上,笔者又发现现有的文献并没有对广义价值论中的各种生产力概念与各种价值理论中的生产力概念进行比较分析,而价值理论又是经济学体系的理论基础\cite[118]{CaiJiMingCongGuDianZhengZhiJingJiXueDaoZhongGuoTeSeSheHuiZhuYiZhengZhiJingJiXueJiYuZhongGuoShiJiaoDeZhengZhiJingJiXueYanBianShangCe2023}。鉴于建立中国特色社会主义政治经济学的目标要求我们积极汲取西方经济学的合理成分\cite[81]{ChengEnFuChongJianZhongGuoJingJiXueChaoYueMaKeSiYuXiFangJingJiXue2000},笔者打算着眼于生产力参与价值创造的原理,从广义价值论出发,系统梳理各种生产力概念,并将其与不同价值理论中的生产力概念进行对比分析。这一方面有助于对生产力理论,特别是新质生产力理论的进一步研究,以更好地指导发展新质生产力的实践;另一方面也有助于广义价值论的进一步发展,为中国特色社会主义政治经济学大厦的构建添砖加瓦。


\section{研究思路}

第一章 \quad 引言

这部分主要介绍本文的研究背景和研究意义。

第二章 \quad 广义价值论的基本原理

这部分主要介绍广义价值论的基本原理并系统整理各种生产力的概念及其与价值创造的联系。

第三章 \quad 古典政治经济学中的生产力概念与价值理论

这部分主要对比分析亚当斯密(重点)、李嘉图(重点)、马尔萨斯、约翰穆勒的价值理论及生产力概念。

第四章 \quad 新古典经济学中的生产力概念与价值理论

这部分主要对比分析奥地利学派的边际效用价值论、马歇尔的均衡价格论以及斯拉法价值论。

第五章 \quad 总结和展望

总结评价前文的各种生产力概念和价值理论,为新质生产力的研究提供思考。

