% !TeX root = ../2019080346_Mason.tex

\chapter{古典政治经济学中的生产力概念与价值理论}

在完成了对广义价值论的介绍后,笔者将对古典政治经济学中的生产力概念和价值理论进行梳理。

\section{古典政治经济学的界定}

在开始梳理之前,我们首先对古典政治经济学的范围做出定义。

经济学界对古典政治经济学有很多种界定,马克思是最早使用“古典”一词来泛指一段时期的政治经济学理论的\cite[7]{YueHan*MeiNaDe*KaiEnSiJiuYeLiXiHeHuoBiTongLunChongYiBen2021}。他认为“古典政治经济学在英国从威廉$\cdot$配第开始,到李嘉图结束,在法国从阿吉尔贝尔开始,到西斯蒙第结束”\cite[56]{QiaEr*MaKeSiZhengZhiJingJiXuePiPanYingWenBan2022}。除了马克思以外,还有许多经济学家也对古典政治经济学作了口径不一的划分\cite[5-8]{CaiJiMingCongGuDianZhengZhiJingJiXueDaoZhongGuoTeSeSheHuiZhuYiZhengZhiJingJiXueJiYuZhongGuoShiJiaoDeZhengZhiJingJiXueYanBianShangCe2023}。由于当前学界主流的观点\cite[56]{QiaEr*MaKeSiZhengZhiJingJiXuePiPanYingWenBan2022}\cite[45]{ChenDaiSunCongGuDianJingJiXuePaiDaoMaKeSiRuoGanZhuYaoXueShuoFaZhanLueLun2014}\cite[12]{CaiJiMingCongGuDianZhengZhiJingJiXueDaoZhongGuoTeSeSheHuiZhuYiZhengZhiJingJiXueJiYuZhongGuoShiJiaoDeZhengZhiJingJiXueYanBianShangCe2023}认为古典政治经济学的基本倾向是劳动价值论,同时马克思本人也赞同劳动价值论,
故笔者在本文中将马克思也一并列入古典政治经济学的范畴。在参考了大量的文献后,笔者进一步从古典政治经济学家中选择亚当$\cdot$斯密(Adam  Smith)、大卫$\cdot$李嘉图(David Ricardo)、马克思(Karl Heinrich Marx)、马尔萨斯(Thomas Robert Malthus)、萨伊(Jean-Baptiste Say)及约翰$\cdot$斯图亚特$\cdot$穆勒(John Stuart Mill)及作为古典政治经济学的代表,依次展开分析。

\section{亚当$\cdot$斯密}
 
亚当$\cdot$斯密在1776年发表的《国富论》中系统地阐释了古典政治经济学的基本思想\cite[120]{CaiJiMingCongGuDianZhengZhiJingJiXueDaoZhongGuoTeSeSheHuiZhuYiZhengZhiJingJiXueJiYuZhongGuoShiJiaoDeZhengZhiJingJiXueYanBianShangCe2023}\cite[90]{YanZhiJieXiFangJingJiXueShuoShiJiaoChengDiErBan2013}。接下来,笔者将首先介绍亚当$\cdot$斯密笔下的价值理论,然后梳理其生产力概念。

\subsection{亚当$\cdot$斯密的价值理论}

斯密是第一个明确地区分使用价值和交换价值的经济学家\cite[122]{CaiJiMingCongGuDianZhengZhiJingJiXueDaoZhongGuoTeSeSheHuiZhuYiZhengZhiJingJiXueJiYuZhongGuoShiJiaoDeZhengZhiJingJiXueYanBianShangCe2023}。他认为使用价值指的是某个商品对使用者的效用,而交换价值指的是某个商品对其他商品的购买力\cite[24]{YaDang*SiMiGuoFuLun2015}。应当说,这种区分是自然而严谨的,直到现在经济学界仍然在沿用这一区分。

接下来,斯密提出了自己在《国富论》中“为要探讨支配商品交换价值原则”\cite[24]{YaDang*SiMiGuoFuLun2015}而必须回答的三个问题,这三个问题中的概念构成了斯密价值理论的体系,笔者对斯密价值理论的介绍也将从这些概念入手。为了让斯密的表述更加清晰,笔者把这三个问题重新表述如下:

1.“什么是交换价值的尺度?”\cite[24]{YaDang*SiMiGuoFuLun2015}

2.“真实价格是由什么构成的?”\cite[24]{YaDang*SiMiGuoFuLun2015}

3.“市场价格围绕自然价格波动的原因是什么?”\cite[24]{YaDang*SiMiGuoFuLun2015}

下面,我们按照这三个问题的顺序依次进行分析。

\subsubsection{交换价值的尺度}

首先,斯密认为“劳动是衡量一切商品交换价值的真实尺度”\Cite[25]{YaDang*SiMiGuoFuLun2015}。具体而言,随着社会分工的深入发展,个体所需的物品中仅小部分可通过自身劳动供给,剩下的绝大部分都依赖他人的劳动。所以,个体的财富状况取决于其可支配的社会劳动量\cite[25]{YaDang*SiMiGuoFuLun2015}。这也就是说,一个人拥有的商品的价值,等于这件商品能够支配他人劳动的数量\cite[25]{YaDang*SiMiGuoFuLun2015}。斯密又进一步解释了选用劳动作为尺度的原因:“只有用劳动作标准,才能在一切时代和一切地方比较各种商品的价值”\cite[31]{YaDang*SiMiGuoFuLun2015}。这里,笔者想强调斯密的意思是说交换价值的尺度不仅是劳动,而且是商品可以支配的、他人(社会)的劳动。

但斯密又说:“一切商品的价值,通常不是按劳动估定的。”\cite[26]{YaDang*SiMiGuoFuLun2015}这是因为劳动存在异质性:“它们的不同困难程度和精巧程度,也必须加以考虑”\cite[26]{YaDang*SiMiGuoFuLun2015}。进而,斯密认为通过市场的议价行为可以“消除”劳动的异质性,让不同质的劳动可以通过商品交换的方式实现相互交换,于是,劳动交换这一抽象的概念取得了商品交换这一具体的形式\cite[26]{YaDang*SiMiGuoFuLun2015}。随着货币的出现,物物交换演进为商品与货币的交换,商品取得了货币价格的形式。这种以货币价格为尺度的价格被斯密称作是“名义价格”,而商品的真实价格还是以购买的劳动为尺度的\cite[28]{YaDang*SiMiGuoFuLun2015}。

根据以上内容,笔者认为,斯密笔下的“交换价值”与广义价值论中的交换价值相同,即不同商品之间的交换比例。而当货币出现并被广泛使用之后,交换价值又转换为名义价格。这里的名义价格,应当就是我们可以观察到的市场价格\cite[293]{YueSeFu*XiongBiTeJingJiFenXiShiDi1Juan2017}。

\subsection{真实价格的构成}

但是,斯密没有给出“价值”的明确定义,只是说“同一真实价格的价值,往往相等”\cite[28]{YaDang*SiMiGuoFuLun2015}。有学者认为,斯密并没有从交换价值中抽象出价值的概念\cite[71]{ChenDaiSunCongGuDianJingJiXuePaiDaoMaKeSiRuoGanZhuYaoXueShuoFaZhanLueLun2014},笔者也认同这一观点。在此基础上,笔者进一步认为斯密的意思是“交换价值”在货币出现后转换为“名义价格”,而“真实价格”是“名义价格”围绕波动的中心\cite[52]{YaDang*SiMiGuoFuLun2015}。因此,斯密笔下的“真实价格”才符合广义价值论中对价值的定义。

一般认为,斯密在第一篇第六章中提出了两种价值论\cite[97]{YanZhiJieXiFangJingJiXueShuoShiJiaoChengDiErBan2013}\cite[126]{CaiJiMingCongGuDianZhengZhiJingJiXueDaoZhongGuoTeSeSheHuiZhuYiZhengZhiJingJiXueJiYuZhongGuoShiJiaoDeZhengZhiJingJiXueYanBianShangCe2023}。首先,斯密认为“在资本累积和土地私有尚未发生以前的初期野蛮社会,获取各种物品所需要的劳动量之间的比例,似乎是各种物品相互交换的唯一标准。”\cite[41]{YaDang*SiMiGuoFuLun2015}于是我们可以说,单要素的劳动价值论是亚当斯密提出的第一种价值论。

而当资本积累到一定程度时,则“劳动者对原材料增加的价值”就可以被分解为两个部分,一部分是劳动者的工资,另一部分是雇主的利润\cite[42]{YaDang*SiMiGuoFuLun2015}。当然,衡量这三个组成部分价值的尺度,仍然是其能够支配的、他人(社会)的劳动\cite[43-44]{YaDang*SiMiGuoFuLun2015}。由于每一期的劳动可以被下一年再次投入生产,所以全社会每年的劳动产物能够买的劳动量将远超当年实际耗费的劳动量\cite[48]{YaDang*SiMiGuoFuLun2015}。于是我们又可以说,多要素的劳动价值论是亚当斯密提出的第二种价值论,也就是说耗费的劳动量不能再单独决定能够买的劳动量了,由于其它要素的参与,单位消耗的劳动量可以购买更多的劳动量\cite[138]{CaiJiMingCongGuDianDaoXianDaiZhengZhiJingJiXueGaiNianDeYanBianJianPingXinZhengZhiJingJiXueDeFaZhan2012}。事实上,这种不等关系对应着广义价值论中两个生产者的单位耗费劳动因综合生产力的相对大小不同而能购买不等量劳动的结论\cite[294]{CaiJiMingCongGuDianZhengZhiJingJiXueDaoZhongGuoTeSeSheHuiZhuYiZhengZhiJingJiXueJiYuZhongGuoShiJiaoDeZhengZhiJingJiXueYanBianShangCe2023}。

\subsection{对以上内容的两方面争议}

在此基础上,有许多经济学家对斯密的观点提出了批评,这些批评主要集中在两个方面:价值决定和价值尺度。

从价值决定的角度看,部分经济思想史学者指出,亚当$\cdot$斯密的价值理论存在逻辑矛盾:斯密一方面提出了劳动价值论的观点,另一方面又有生产费用论的成分\cite[136]{CaiJiMingCongGuDianZhengZhiJingJiXueDaoZhongGuoTeSeSheHuiZhuYiZhengZhiJingJiXueJiYuZhongGuoShiJiaoDeZhengZhiJingJiXueYanBianShangCe2023}\cite[294]{YueSeFu*XiongBiTeJingJiFenXiShiDi1Juan2017}\cite[47]{ZhongGongZhongYangMaKeSiEnGeSiLieNingSiDaLinZhuZuoBianYiJuMaKeSiEnGeSiQuanJiDi26JuanDi1Ce1972}。但如前文所述,笔者更支持这样一种观点:斯密的价值理论是基于不同历史阶段的多要素价值论\cite[136]{CaiJiMingCongGuDianZhengZhiJingJiXueDaoZhongGuoTeSeSheHuiZhuYiZhengZhiJingJiXueJiYuZhongGuoShiJiaoDeZhengZhiJingJiXueYanBianShangCe2023},即当生产要素只有劳动时,商品的价值全部由劳动单一决定,此时斯密的价值理论是单要素劳动价值论;而当生产要素包括了土地和资本时,土地、资本、劳动都会影响商品的价值,此时斯密的价值理论是多要素劳动价值论\cite{peachAdamSmithsLabor2020}\cite[21]{MaKe*BuLaoGeJingJiLiLunDeHuiGu2018}\cite[70-71]{meekStudiesLaborTheory1973}。

从价值尺度的角度看,李嘉图批评斯密同时提出了耗费劳动尺度说和购买劳动尺度说\cite[7]{DaWei*LiJiaTuZhengZhiJingJiXueJiFuShuiYuanLi2021}。但如前文所述,斯密始终是把能购买的劳动作为尺度的\cite[142]{CaiJiMingCongGuDianZhengZhiJingJiXueDaoZhongGuoTeSeSheHuiZhuYiZhengZhiJingJiXueJiYuZhongGuoShiJiaoDeZhengZhiJingJiXueYanBianShangCe2023}\cite[63]{meekStudiesLaborTheory1973}。斯密是把价值尺度和价值决定分开考察的\cite[73]{ChenDaiSunCongGuDianJingJiXuePaiDaoMaKeSiRuoGanZhuYaoXueShuoFaZhanLueLun2014},他首先提出了衡量“交换价值”的外在尺度是可以购买的劳动\cite{rodriguezherreraAdamSmithsConcept2016},接着再进一步提出在土地尚未私有、资本尚未积累时,决定“真实价格”的是商品生产耗费的劳动。这时,“真实价格”的尺度和耗费的劳动是统一的。而在土地私有、资本累积的发达社会,包括直接耗费的劳动在内的多种要素共同决定了“真实价格”。这时,“真实价格”的尺度和耗费的劳动便不再统一,但参与商品生产的总劳动量(包括直接和间接)仍然等于商品能购买的总劳动量。\cite[67-69]{CaiJiMingLunHaoFeiDeLaoDongYuGouMaiDeLaoDongZaiJieZhiLiLunZhongDeZuoYong2022}

关于价值尺度的问题笔者将在本章末尾的总结部分进行深入的分析。

接下来,我们来分析斯密的第三个问题。

\subsection{市场价格围绕自然价格波动的原因}

首先,我们有必要对“自然价格”的概念进行阐释。按照斯密的观点,商品生产所花费的地租、工资和利润构成了商品的“自然价格”,而“自然价格”又“恰恰相当于其价值”\cite[49]{YaDang*SiMiGuoFuLun2015}。这里,斯密略去了对价值向生产价格的转化问题的分析\footnote{即由于竞争使剩余价值在各生产部门资本家之间按资本量平均分配,剩余价值转化为平均利润,价值转化生产价格的过程\cite{XieFuShengXiFangXueZheGuanYuMaKeSiJieZhiZhuanXingLiLunYanJiuShuPing2000}。}
,而直接得出了类似马克思笔下生产价格的“自然价格”概念。

斯密指出,商品的市场价格会受到供给和有效需求的支配而围绕着商品的自然价格上下波动,这里的有效需求指的是愿意支付商品“自然价格”者的需求\cite[50]{YaDang*SiMiGuoFuLun2015}。于是,长期的商品价格会在自然价格处达到均衡\cite[50]{YaDang*SiMiGuoFuLun2015}。另外,斯密还指出商品的市场价格会受到货币本身价值变动的影响\cite[28-31]{YaDang*SiMiGuoFuLun2015}。

至此,我们已经回答了斯密提出的三个问题,基本梳理了斯密的价值理论体系如下图所示。

\begin{figure}
    \centering
    \caption{亚当$\cdot$斯密的价值理论}
    \label{figures:AdamSmith_Value_Theory}
    \includegraphics[width=\textwidth]{figures/AdamSmith_Value_Theory.pdf}
\end{figure}

接下来,笔者将梳理斯密笔下的生产力概念。

\subsection{亚当$\cdot$斯密的生产力概念}

在《国富论》的开篇,斯密指出分工是提升劳动生产力的最重要因素\cite[3]{YaDang*SiMiGuoFuLun2015}。这是因为通过分工有以下三个优点:第一,劳动者能够熟能生巧;第二,分工可以避免被劳动者切换工作时的时间损失;第三,借助简化劳动的机械,一个劳动者可以完成多个劳动者的工作。\cite[6]{YaDang*SiMiGuoFuLun2015}这里,我们还可以看到斯密笔下的分工不仅是简单的通过劳动力排列组合的方面,而且包括了劳动技能提升和要素使用的方面。

如此来看,斯密笔下的“劳动生产力”对应的是广义价值论中的绝对生产力概念。纵观《国富论》全文,我们也找不到其它的生产力概念。

\subsection{生产力与价值的关系}

按照广义价值论的分析,绝对生产力与单位商品的价值是负相关的关系。虽然斯密认为资本和土地要素的投入会影响商品的价值量,但这些投入都可以被视为是物化劳动的投入,所以绝对生产力与单位商品的价值仍然是负相关的关系。另外,由于斯密认为耗费的劳动和购买的劳动可以不相等,所以我们也不难据此推出绝对生产力和单位劳动的价值量正相关的结论。


\section{大卫$\cdot$李嘉图}

大卫$\cdot$李嘉图是英国产业革命时代的经济学家,继承和发展了斯密价值理论中蕴含的单一劳动价值论思想,对价值决定于劳动时间的原理作了比较透彻的表述与发展\cite[iv]{DaWei*LiJiaTuZhengZhiJingJiXueJiFuShuiYuanLi2021}。在这一部分,笔者将探讨李嘉图理论体系中的生产力概念和价值理论。

\subsection{大卫$\cdot$李嘉图的价值理论}

在《政治经济学及赋税原理》第一章中,李嘉图对他的价值理论做了比斯密更为严谨和系统的介绍,但他同样没有区分开交换价值和价值。

李嘉图首先继承了斯密对使用价值和价值的区分,然后指出商品的交换价值有两个来源——一是来源于商品的稀有性,二是来源于制造商品所必需的劳动量\cite[5-6]{DaWei*LiJiaTuZhengZhiJingJiXueJiFuShuiYuanLi2021}:“对于那些劳动无法增加其数量的商品,例如一些稀有的雕像和图画,它们的价值只由其稀少性决定;对于另外那些数量可以通过劳动增加的商品,其价值“几乎完全取决于各商品上所费的相对劳动量”。\cite[6]{DaWei*LiJiaTuZhengZhiJingJiXueJiFuShuiYuanLi2021}但由于前种商品在市场上只占极小一部分,所以李嘉图将分析的重点放在了后者上。

其中,“相对劳动量”并非指实际耗费的劳动量,而是指在最不利生产条件下所必需的劳动量\cite[16]{LiRenJunJieZhiLiLun2004}。李嘉图指出:规定一切商品的交换价值的不是在优越的生产条件下的生产者所耗费的较少量的劳动,而是“不享有这种便利的人进行生产时所必须投入的较大量劳动”\cite[58]{DaWei*LiJiaTuZhengZhiJingJiXueJiFuShuiYuanLi2021}。也就是说,在李嘉图看来,商品价值量是由边际生产条件(即最劣等生产环境)下耗费的必要劳动量来调节的\cite[9]{ChenZhenYuLunSheHuiBiYaoLaoDongShiJianXueShuoCongGuDianXuePaiDaoMaKeSiDeFaZhan1990}\footnote{马克思则认为在工业部门,商品的价值是由部门平均生产条件下所必需的劳动来决定的\cite[16]{LiRenJunJieZhiLiLun2004}。}。

另外,李嘉图又指出商品的价值“不取决于付给这种劳动的报酬的多少”\cite[5]{DaWei*LiJiaTuZhengZhiJingJiXueJiFuShuiYuanLi2021},工资的高低只会影响价值的分配而不会影响价值的决定。具体来说,李嘉图认为工资的高低对商品生产者的利润来说是很重要的,工资高低和利润高低成反比。但因为不同行业的工资是一致的,工资的升降在不同行业间也是同步的,所以工资的高低不会影响商品之间的相对价值\cite[19]{DaWei*LiJiaTuZhengZhiJingJiXueJiFuShuiYuanLi2021}。

总的来说,李嘉图价值理论的基本观点是劳动价值论——价值是由耗费的劳动决定的\cite[142]{CaiJiMingCongGuDianZhengZhiJingJiXueDaoZhongGuoTeSeSheHuiZhuYiZhengZhiJingJiXueJiYuZhongGuoShiJiaoDeZhengZhiJingJiXueYanBianShangCe2023}。

和斯密一样,李嘉图也认为生产商品所耗费劳动的异质性可以通过市场消除。在李嘉图看来,商品的相对价值会在市场交换中形成且“估价的尺度一经形成就很少发生变动”,劳动的异质性实际上已经体现在商品的价格比例中了,所以市场“消除”了劳动的异质性\cite[13-14]{DaWei*LiJiaTuZhengZhiJingJiXueJiFuShuiYuanLi2021}。

随后,李嘉图又指出资本的使用不违背劳动价值论。李嘉图首先把资本看作是物化劳动,他指出:影响商品交换价值的“不仅是指投在商品的直接生产过程中的劳动,而且也包括投在实现该种劳动所需要的一切器具或机器上的劳动”\cite[17]{DaWei*LiJiaTuZhengZhiJingJiXueJiFuShuiYuanLi2021}。因此,无论是商品直接耗费劳动的减少还是通过资本间接耗费劳动的减少,都会使得商品的相对价值下降\cite[18]{DaWei*LiJiaTuZhengZhiJingJiXueJiFuShuiYuanLi2021}。

在此基础上,李嘉图又进一步研究了资本变化情况下的价值决定。首先,李嘉图认识到具有不同固定资本和流动资本比例的资本对价值决定会产生影响。具体而言,使用固定资本占比较高的资本生产的商品的价值会高于使用流动资本占比较高的资本生产的商品的价值。李嘉图举例说,假定有两个人各雇佣100人工作1年,其中一人是织造业者,选择制造机器;另外一人是农场主,选择栽种谷物。那么根据价值由耗费劳动决定的基本观点,在第一年年末时机器的价值等于谷物的价值。在第二年,这两人继续各雇佣100人工作1年,机器的所有者制造棉织品,另外一人仍然选择栽种谷物。那么在第二年末,棉织品+棉织机和谷物(两年累积)的价值仍应是相等的。但是,由于一般利润率\footnote{即等量资本带来等量回报形成的利润率}的存在,拥有机器的两人实际上是把第一年的利润加入到了各自的资本中,而栽种谷物的一人把第一年的利润消费掉了,所以棉织品+棉织机的价值应该要受到补偿,也就是把第一年所应得的利润一并计入第二年的资本额内,并依据这个资本额计算第二年的利润。于是,棉织品+棉织机的价值实际上会大于谷物的价值\cite[24-25]{DaWei*LiJiaTuZhengZhiJingJiXueJiFuShuiYuanLi2021}\cite[119]{YanZhiJieXiFangJingJiXueShuoShiJiaoChengDiErBan2013}。

从数值上看,假设每个劳动者一年的工资是50镑,一般利润率为10\%,那么我们可以得到表格如下\cite[19]{LiRenJunJieZhiLiLun2004}:

\begin{table}[!h]
    \caption{资本构成不同对价值决定的影响}
    \begin{tabularx}{\textwidth}{|>{\centering\arraybackslash}p{1.2cm}|>{\centering\arraybackslash}X|>{\centering\arraybackslash}X|>{\centering\arraybackslash}X|>{\centering\arraybackslash}X|>{\centering\arraybackslash}X|>{\centering\arraybackslash}X|}
    \toprule
        & \multicolumn{3}{c|}{织造业者}                                                                   & \multicolumn{3}{c|}{农场主}                                                        \\ \hline
        & 固定资本 & 流动资本        & 商品价值                         & 固定资本 & 流动资本        & 商品价值             \\ \hline
    第一年 & $0$    & $100 \times 50=5000$ & $5000 \times 1.1=5500$             & $0$    & $100 \times 50=5000$ & $5000 \times 1.1=5500$ \\ \hline
    第二年 & $5500$ & $100 \times 50=5000$ & $(5500+5000) \times 1.1 -5500=6050$ & $0$    & $100 \times 50=5000$ & $5000 \times 1.1=5500$ \\ \bottomrule
    \end{tabularx}
\end{table}

接着,李嘉图又察觉到工资波动会对使用不同固定资本和流动资本比例的资本生产的商品的价值产生不同的影响\cite[117-119]{YanZhiJieXiFangJingJiXueShuoShiJiaoChengDiErBan2013}。具体而言,工资涨落使得那些使用固定资本占比较高的资本所生产的商品价值跌落,而那些使用流动资本占比较高的的资本所生产的商品价值上升。

沿用前面的例子,假设每个劳动者第一年的工资是50镑,第一年的一般利润率为10\%;第二年的一般利润率因工资上涨而下降为9\%,那么我们可以得到表格如下\cite[21]{LiRenJunJieZhiLiLun2004}:

\begin{table}[!h]
    \caption{工资变动对价值决定的影响}
    \begin{tabularx}{\textwidth}{|>{\centering\arraybackslash}p{1.2cm}|>{\centering\arraybackslash}X|>{\centering\arraybackslash}X|>{\centering\arraybackslash}X|>{\centering\arraybackslash}X|>{\centering\arraybackslash}X|>{\centering\arraybackslash}X|}
    \toprule
        & \multicolumn{3}{c|}{织造业者}                                                                   & \multicolumn{3}{c|}{农场主}                                                        \\ \hline
        & 固定资本 & 流动资本        & 商品价值                         & 固定资本 & 流动资本        & 商品价值             \\ \hline
    第一年 & $0$    & $100 \times 50=5000$ & $5000 \times 1.1=5500$             & $0$    & $100 \times 50=5000$ & $5000 \times 1.1=5500$ \\ \hline
    第二年 & $5500$ & $100 \times 50.46=5046$ & $(5500+5046) \times 1.09 -5500=5995$ & $0$    & $100 \times 50.46=5000$ & $5046 \times 1.09=5500$ \\ \bottomrule
    \end{tabularx}
\end{table}

最后,李嘉图还意识到资本周转速度的差异会对商品的价值产生影响。具体而言,使用周转周期长的资本生产的商品的价值会高于使用周转周期短的资本生产的商品。假定有甲、乙两个资本家,甲资本家用1000镑雇佣20个工人生产一年得到一批半成品,第二年再用1000镑雇佣20人对半成品进行加工,第二年末完成生产;乙资本家则在1年中用2000镑雇佣40个工人生产商品,第一年年末就送上市场\cite[27-28]{DaWei*LiJiaTuZhengZhiJingJiXueJiFuShuiYuanLi2021}。再假定一般利润率为10\%,那我们可以得到表格如下\cite[20]{LiRenJunJieZhiLiLun2004}:

\begin{table}[!h]
    \caption{资本周转差异对价值决定的影响}
    \begin{tabularx}{\textwidth}{|>{\centering\arraybackslash}p{1.2cm}|>{\centering\arraybackslash}X|>{\centering\arraybackslash}X|>{\centering\arraybackslash}X|>{\centering\arraybackslash}X|>{\centering\arraybackslash}X|>{\centering\arraybackslash}X|}
    \toprule
        & \multicolumn{3}{c|}{甲}                                                                   & \multicolumn{3}{c|}{乙}                                                        \\ \hline
        & 固定资本 & 流动资本        & 商品价值                         & 固定资本 & 流动资本        & 商品价值             \\ \hline
    第一年 & $0$    & $20 \times 50=1000$ & $1000 \times 1.1=1100$             & $0$    & $40 \times 50=2000$ & $2000 \times 1.1=2200$ \\ \hline
    第二年 & $110000$ & $20 \times 50=1000$ & $(1100+1000) \times 1.1 =2310$ & $0$    & $0 $ & $0$ \\ \bottomrule
    \end{tabularx}
\end{table}

如果再考虑到资本周转时间的差异,那么工资变动会进一步影响商品的价值决定。对此,李嘉图总结道:商品价值对工资涨落的敏感性取决于固定资本在全部资本中的占比;那些固定资本占比高或是资本周转时间长的商品的相对价值会因为工资变动而跌落,反之则会上涨\cite[26]{DaWei*LiJiaTuZhengZhiJingJiXueJiFuShuiYuanLi2021}。

虽然以上的结论实际上已经违背了“价值是由耗费的劳动决定的”这一基本观点,但李嘉图认为“商品价值变动的这一原因的影响是比较小的。工资上涨到使利润跌落百分之一时,在前述假定情况下生产出来的商品的相对价值只会发生百分之一的变动”\cite[26]{DaWei*LiJiaTuZhengZhiJingJiXueJiFuShuiYuanLi2021}。由此,李嘉图的劳动价值论也被人戏称为“93\%的劳动价值论”\cite{georgej.stiglerRicardo93Labor1958}。

\subsection{大卫$\cdot$李嘉图的生产力概念}

李嘉图的著作中没有出现明确的生产力概念,我们只能从他的文字中总结出他对生产力的看法。例如,他在分析黄金的价值变动时说:如果“由于发现了更丰饶的新矿山,或是由于更有利地使用机器,用较少的劳动量就可以获得一定量的黄金,那么,我就有理由说,黄金相对于其他商品的价值发生变动的原因,是它的生产已经比较便利,或获得时所必需的劳动量已经减少”\cite[11]{DaWei*LiJiaTuZhengZhiJingJiXueJiFuShuiYuanLi2021}。李嘉图在别处分析生产力的改进时也是指商品生产所必需的劳动量的减少,所以笔者认为李嘉图的理论中也只有绝对生产力的概念。

另一方面,李嘉图也注意到商品的价值取决于“相对劳动量”而不是绝对劳动量\cite[5]{DaWei*LiJiaTuZhengZhiJingJiXueJiFuShuiYuanLi2021},“如果生产其他商品所需的劳动有所增减,那么我们已经说过,这种情形一定会立即造成其相对价值的变动。”\cite[21]{DaWei*LiJiaTuZhengZhiJingJiXueJiFuShuiYuanLi2021}由此看来,李嘉图其实具备了提出比较生产力的潜力,只不过李嘉图并没有把研究重点放在生产力上。

由于李嘉图也没有区分部门、个体的绝对生产力变动带来的影响,笔者在此先不探讨李嘉图学说中生产力与价值的关系,而是先介绍与李嘉图价值理论相似的马克思的价值理论,并在马克思的理论基础上对绝对生产力与价值的关系做出进一步分析。

最后,李嘉图在解释国际贸易时讨论了国家的比较优势,引入了机会成本,因此实际上提出了相对生产力的概念。李嘉图构建了一个英国葡萄牙两国贸易模型:英国生产单位毛呢需投入100人$\cdot$年的​劳动量,等值葡萄酒则需120人$\cdot$年;葡萄牙生产单位毛呢需要90人$\cdot$年的劳动量,等值葡萄酒仅需要80人$\cdot$年;如果两国可以进行贸易,那么葡萄牙即使能够以90人$\cdot$年的劳动量生产毛呢,但仍会选择从需要100人$\cdot$年的​劳动量生产毛呢的英国进口;因为对葡萄牙来说,把资本全部用于生产葡萄酒然后同英国交换得到的毛呢将多于把资本投入本国毛呢生产而得到的毛呢\cite[111-112]{DaWei*LiJiaTuZhengZhiJingJiXueJiFuShuiYuanLi2021}。在这个例子中,尽管葡萄牙在两种商品的生产商占据绝对优势,但由于葡萄牙在葡萄酒的生产商具有比较优势,英国在生产毛呢商具有比较优势,所以双方仍能从国际贸易中获得好处(节约劳动)。这实际上就是广义价值论中的比较优势原理在国际贸易中的运用。所以,笔者认为李嘉图事实上已经提出了相对生产力的概念,并事实上借助了相对生产力系数来判断交换双方的比较优势。

但是,李嘉图没能把比较优势原理运用到一国内的商品交换过程中,他认为支配国内商品交换的价值法则不适用于国际的商品交换,反之亦然\cite[110]{DaWei*LiJiaTuZhengZhiJingJiXueJiFuShuiYuanLi2021}。事实上,广义价值论正是吸收了李嘉图的思想,把这一原理拓展到了一般的商品交换过程之中\cite[157]{CaiJiMingCongGuDianZhengZhiJingJiXueDaoZhongGuoTeSeSheHuiZhuYiZhengZhiJingJiXueJiYuZhongGuoShiJiaoDeZhengZhiJingJiXueYanBianShangCe2023}。

\section{卡尔$\cdot$马克思}

一般认为,马克思继承和发展了李嘉图的劳动价值论\cite[347]{YueSeFu*XiongBiTeJingJiFenXiShiDi2Juan2017}\cite[84]{ChenDaiSunCongGuDianJingJiXuePaiDaoMaKeSiRuoGanZhuYaoXueShuoFaZhanLueLun2014}。

\subsection{马克思的价值理论}

从价值理论来看,马克思首先区分了交换价值和价值:价值是交换价值的基础,交换价值是价值的表现形态\cite[86-88]{ChenDaiSunCongGuDianJingJiXuePaiDaoMaKeSiRuoGanZhuYaoXueShuoFaZhanLueLun2014}。根本上来说,李嘉图之所以不能做出这种区分,是因为李嘉图把劳动价值论“仅仅是作为一种假设”,“用来说明相对价格(能观察到的市场价格)的实际长期正常状态”\cite[348]{YueSeFu*XiongBiTeJingJiFenXiShiDi2Juan2017},于是李嘉图遇到了“商品价值量取决于社会必要劳动时间与等量资本获取等量利润”的矛盾\cite[144]{CaiJiMingCongGuDianZhengZhiJingJiXueDaoZhongGuoTeSeSheHuiZhuYiZhengZhiJingJiXueJiYuZhongGuoShiJiaoDeZhengZhiJingJiXueYanBianShangCe2023}\cite[21-28]{DaWei*LiJiaTuZhengZhiJingJiXueJiFuShuiYuanLi2021};而马克思则是把劳动看成价值的实质——价值就是凝结的劳动本身,于是马克思遇到了价值向生产价格的转形问题\footnote{对“转形问题”的争论可见(谢富胜,2000)\cite{XieFuShengXiFangXueZheGuanYuMaKeSiJieZhiZhuanXingLiLunYanJiuShuPing2000}。}\cite[348-350]{YueSeFu*XiongBiTeJingJiFenXiShiDi2Juan2017}\cite[159]{CaiJiMingCongGuDianZhengZhiJingJiXueDaoZhongGuoTeSeSheHuiZhuYiZhengZhiJingJiXueJiYuZhongGuoShiJiaoDeZhengZhiJingJiXueYanBianShangCe2023}。

再次,马克思通过区分劳动与劳动力解释“资本与劳动交换”的问题。马克思认为工人让渡的是劳动力使用权而非劳动本身。当劳动力作为特殊商品进入生产领域,其使用过程(即活劳动)创造的价值超越自身价值(工资),形成被资本家无偿占有的价值剩余,由此消解了资本交换与价值规律的逻辑悖论。\cite[615,581-606]{ZhongGongZhongYangMaKeSiEnGeSiLieNingSiDaLinZhuZuoBianYiJuMaKeSiEnGeSiWenJiDi5Juan2009}\cite[157-158]{CaiJiMingCongGuDianZhengZhiJingJiXueDaoZhongGuoTeSeSheHuiZhuYiZhengZhiJingJiXueJiYuZhongGuoShiJiaoDeZhengZhiJingJiXueYanBianShangCe2023}\cite[348]{YueSeFu*XiongBiTeJingJiFenXiShiDi2Juan2017}。

总的来说,马克思完善了劳动价值论,其价值理论可以简单地表述为:商品的价值由生产商品的社会必要劳动时间决定\cite[51-52]{ZhongGongZhongYangMaKeSiEnGeSiLieNingSiDaLinZhuZuoBianYiJuMaKeSiEnGeSiWenJiDi5Juan2009}。

\subsection{马克思的生产力理论}

马克思虽然对生产力有不同的提法,但各种提法的本质是清晰且一贯的\cite{YangQiaoYuShengChanLiGaiNianCongSiMiDaoMaKeSiDeSiXiangPuXi2013}\cite{DingXiaoPingZhengQueLiJieMaKeSiZhuYiDeShengChanLiGaiNian2021}——“生产力当然始终是有用的、具体的劳动的生产力,它事实上只决定有目的的生产活动在一定时间内的效率”\cite[59]{ZhongGongZhongYangMaKeSiEnGeSiLieNingSiDaLinZhuZuoBianYiJuMaKeSiEnGeSiWenJiDi5Juan2009}——这和广义价值论中绝生产力的定义是相符的。

但笔者注意到,马克思在研究人类社会早期的分工交换时曾经间接地提出过相对生产力的概念。在《<资本论>第三册增补》中有这样一个例子:在商品经济发展的初期,进行商品交换的主要是劳动的农民。这些农民借助自己家庭的帮助,在自己的田地上进行农业、畜牧业和手工业的生产,并拿除必需品之外剩下的剩余产品同其它农民家庭进行交换。这些农民之所以进行交换,并非因为自己不会生产这些物品,而是因为得不到原料或者因为通过交换得到的物品要更好或更便宜\cite[1015-1016]{ZhongGongZhongYangMaKeSiEnGeSiLieNingSiDaLinZhuZuoBianYiJuMaKeSiEnGeSiWenJiDi7Juan2009}。这里,“农民进行交换不是因为自己不会生产,而是因为通过交换得到的物品要更好或更便宜”的观点反映了马克思是基于可变分工体系来研究早期人类社会的,而且马克思实际上间接地意识到了相对生产力在决定分工方向时的重要作用。因此,笔者认为马克思在研究人类社会早期的分工交换时提出了相对生产力的概念。

马克思对国际商品交换的研究虽然没有系统地完成,但在已有地论述中可以看到他继承了李嘉图将国内商品交换和国际商品交换区分开来的分析方法,认为支配国内商品交换的价值规律在支配国际商品交换时需要进行修正\cite[28-29]{LiBangXiDangDaiZiBenZhuYiYuBuPingDengJiaoHuanJiYuMaKeSiZhuYiDePiPanXingYanJiu2025}\cite[112]{ZhongGongZhongYangMaKeSiEnGeSiLieNingSiDaLinZhuZuoBianYiJuMaKeSiEnGeSiQuanJiDi26JuanDi3Ce1972}。马克思指出,假设生产效率较高的国家没有因为竞争压力而把商品价格降低到和价值相等的程度,那么这些国家的劳动在世界市场上会被算作强度较大的劳动;所以,同样的劳动在不同的国家会创造不等量的价值,进而拥有不同的价格\cite[645]{ZhongGongZhongYangMaKeSiEnGeSiLieNingSiDaLinZhuZuoBianYiJuMaKeSiEnGeSiWenJiDi7Juan2009}。从这个观点来看,我们应当可以说马克思在研究国际贸易时提出了比较生产力的概念。倘若我们把此处的“生产效率”理解为一国的综合生产力,并通过与其它国家的综合生产力相比较得到一国相对于另一国的比较生产力,那么不同国家间的劳动相互折算的过程实际上就是根据比较生产力决定不同国家商品价值的过程。这一逻辑同广义价值论是相符的。所以,笔者认为马克思在研究国际商品交换时实际上提出了综合生产力和比较生产力的概念。

在研究国内商品交换时,马克思指出“社会劳动生产力”在不同部门具有不同的发展水平,“社会劳动生产力”“和一定量劳动所推动的生产资料量成正比”\cite[183]{ZhongGongZhongYangMaKeSiEnGeSiLieNingSiDaLinZhuZuoBianYiJuMaKeSiEnGeSiWenJiDi7Juan2009}。也就是说,资本构成\footnote{不变资本同可变资本占比的比值\cite[183]{ZhongGongZhongYangMaKeSiEnGeSiLieNingSiDaLinZhuZuoBianYiJuMaKeSiEnGeSiWenJiDi7Juan2009}。}高的部门具有更高的“社会劳动生产力”。由于“商品不只是当做商品来交换,而是当作资本的产品来交换”\cite[196]{ZhongGongZhongYangMaKeSiEnGeSiLieNingSiDaLinZhuZuoBianYiJuMaKeSiEnGeSiWenJiDi7Juan2009},所以商品马克思又提出了“生产价格”\footnote{这引起了著名的“转形问题”\cite{XieFuShengXiFangXueZheGuanYuMaKeSiJieZhiZhuanXingLiLunYanJiuShuPing2000}。}的概念——把商品的成本价格加上按照一般利润率计算的资本利润而得到的价格\cite[177]{ZhongGongZhongYangMaKeSiEnGeSiLieNingSiDaLinZhuZuoBianYiJuMaKeSiEnGeSiWenJiDi7Juan2009}。马克思又进一步指出:“社会劳动生产力”高(也就是资本构成高)的部门商品的生产价格会高于其价值\cite[183-184]{ZhongGongZhongYangMaKeSiEnGeSiLieNingSiDaLinZhuZuoBianYiJuMaKeSiEnGeSiWenJiDi7Juan2009}。当然,持劳动价值论观点的马克思不把“生产价格”看作商品的价值。但是,倘若我们抛开劳动价值论,把“生产价格”看作商品的价值,那么“社会劳动生产力”高的部门的单位商品的价值就高于“社会劳动生产力”低的部门的单位商品的价值。也就是说,部门的“社会劳动生产力”同单位商品的价值正相关。这样看来,这里的“社会劳动生产力”就同广义价值论中的比较生产力具有相同的作用。因此,笔者认为在马克思研究国内商品交换的理论中也在一定程度上暗含了比较生产力的概念。

\subsection{生产力与价值的关系}

基于绝对生产力不同维度,马克思针对生产力与价值的关系给出了三个对立统一的命题\cite[273]{CaiJiMingCongGuDianZhengZhiJingJiXueDaoZhongGuoTeSeSheHuiZhuYiZhengZhiJingJiXueJiYuZhongGuoShiJiaoDeZhengZhiJingJiXueYanBianShangCe2023}。

第一,劳动生产力与价值量负相关。马克思指出,“劳动生产力越高,生产一种物品所需要的劳动时间就越少,凝结在该物品中的劳动量就越小,该物品的价值就越小”\cite[53]{ZhongGongZhongYangMaKeSiEnGeSiLieNingSiDaLinZhuZuoBianYiJuMaKeSiEnGeSiWenJiDi5Juan2009}。这里的“劳动生产力”应当是指广义价值论中的部门绝对生产力,而这里的“价值量”应当是指单位商品的价值量\cite[273]{CaiJiMingCongGuDianZhengZhiJingJiXueDaoZhongGuoTeSeSheHuiZhuYiZhengZhiJingJiXueJiYuZhongGuoShiJiaoDeZhengZhiJingJiXueYanBianShangCe2023}。该命题与广义价值论的判断是一致的。但马克思还说单位商品的价值量和劳动生产力成反比\cite[53-54]{ZhongGongZhongYangMaKeSiEnGeSiLieNingSiDaLinZhuZuoBianYiJuMaKeSiEnGeSiWenJiDi5Juan2009},这就与广义价值论的判断出现矛盾。在广义价值论中,部门绝对生产力的提高会带动部门综合生产力提高,进而驱动部门比较生产力系数发生变化,最终导致单位商品价值量的降幅总是比部门绝对生产力的提高幅度更小,所以部门绝对生产力与单位商品价值量只能构成负相关而不是反比的关系\cite[274, 282]{CaiJiMingCongGuDianZhengZhiJingJiXueDaoZhongGuoTeSeSheHuiZhuYiZhengZhiJingJiXueJiYuZhongGuoShiJiaoDeZhengZhiJingJiXueYanBianShangCe2023}。出现这种矛盾的根本原因,是马克思没有考虑部门综合生产能力对价值决定的影响。

第二,劳动生产力与价值量正相关。马克思指出:“生产力特别高的劳动起了自乘的劳动的作用,$\cdots$,在同样的时间内,它所创造的价值比同种社会平均劳动要多。”\cite[370]{ZhongGongZhongYangMaKeSiEnGeSiLieNingSiDaLinZhuZuoBianYiJuMaKeSiEnGeSiWenJiDi5Juan2009}此处的“劳动生产力”应当是指广义价值论中的个别绝对生产力,而此处的“价值量”应当是指单个生产者在单位劳动时间内所创造的价值量\cite[273]{CaiJiMingCongGuDianZhengZhiJingJiXueDaoZhongGuoTeSeSheHuiZhuYiZhengZhiJingJiXueJiYuZhongGuoShiJiaoDeZhengZhiJingJiXueYanBianShangCe2023}。该命题与广义价值论的判断也是一致的。

第三,劳动生产力与价值量不相关。马克思指出,“不管生产力发生了什么变化,同一劳动在同样的时间内提供的价值量总是相同的”\cite[60]{ZhongGongZhongYangMaKeSiEnGeSiLieNingSiDaLinZhuZuoBianYiJuMaKeSiEnGeSiWenJiDi5Juan2009}。这里的“劳动生产力”应当是指部门绝对生产力,而这里的“价值量”应当是指部门总劳动创造的价值量\cite[274]{CaiJiMingCongGuDianZhengZhiJingJiXueDaoZhongGuoTeSeSheHuiZhuYiZhengZhiJingJiXueJiYuZhongGuoShiJiaoDeZhengZhiJingJiXueYanBianShangCe2023}。该命题与广义价值论的判断出现了矛盾。出现这种矛盾的根本原因,是马克思否认了非劳动要素对价值决定的作用\cite[274]{CaiJiMingCongGuDianZhengZhiJingJiXueDaoZhongGuoTeSeSheHuiZhuYiZhengZhiJingJiXueJiYuZhongGuoShiJiaoDeZhengZhiJingJiXueYanBianShangCe2023}。

最后,在国际贸易领域,马克思用“生产效率”间接地提出了综合生产力和比较生产力的概念,并指出同样的劳动在“生产效率”高的国家能创造出更多的价值。因此我们可以说马克思认为综合生产力与单位劳动创造的价值量是正相关的,这与广义价值论的判断也是一致的。

总的来说,马克思在研究国内商品交换时并没有提出绝对生产力以外的概念,并且不承认非劳动要素对价值决定的作用,但他区分了个体和部门的绝对生产力变化对价值的不同影响,将单一劳动价值论发展到了极致。

\section{马尔萨斯和萨伊}

至此,我们完成了对李嘉图和马克思的介绍。不难看出,他们的经济思想主要以继承和发展斯密的单要素劳动价值论为主。接下来,笔者将介绍继承和发展了斯密多要素价值论的马尔萨斯、萨伊的价值理论并分析其生产力概念。笔者之所以把这两位经济学家放在一起,一方面是因为两人生活在同一个时代\cite[132,140]{YanZhiJieXiFangJingJiXueShuoShiJiaoChengDiErBan2013},另一方面是因为一般认为他们都按照各自的理解继承了斯密的多要素价值论\cite[169]{CaiJiMingCongGuDianZhengZhiJingJiXueDaoZhongGuoTeSeSheHuiZhuYiZhengZhiJingJiXueJiYuZhongGuoShiJiaoDeZhengZhiJingJiXueYanBianShangCe2023}。

\subsection{马尔萨斯的价值理论}

马尔萨斯在区分使用价值和交换价值的基础上对交换价值作了进一步细分,他认为“价值”一词可以被分解为三个方面:第一是使用价值,即物品的效用;第二是名义交换价值,即以货币表示的价值(如果还没有出现货币,则商品的交换价值通过其它任何一种商品表现出来\cite[32]{BiLuo*SiLaFaDaWeiLiJiaTuQuanJiDi2JuanMaErSaSiZhengZhiJingJiXueYuanLiPingZhu2013});第三是实际交换价值,即必需品、享用品和劳动的价值。\cite[42]{BiLuo*SiLaFaDaWeiLiJiaTuQuanJiDi2JuanMaErSaSiZhengZhiJingJiXueYuanLiPingZhu2013}马尔萨斯重点研究的是“实际交换价值”。

然而,值得指出的是,马尔萨斯并没有把价值从价格中抽象出来,他认为“任何时间与地点的商品的自然价值\footnote{笔者认为这里就是指“实际交换价值”}”是“商品处于自然或一般状态下由原始生产成本或供应条件决定的估价”\cite[132]{MaErSaSiZhengZhiJingJiXueDingYi2023}。换句话说,马尔萨斯的“价值”或者“实际交换价值”在广义价值论中对应的是交换价值的概念而非价值概念,马尔萨斯并没有揭开价值的面纱。

马尔萨斯虽然认为“实际交换价值”就是“估价”,但他意识到用贵金属或是其它商品来衡量这一“估价”是很困难的\cite[98]{BiLuo*SiLaFaDaWeiLiJiaTuQuanJiDi2JuanMaErSaSiZhengZhiJingJiXueYuanLiPingZhu2013},
所以他认为应当用商品能支配的劳动而非生产耗费的劳动作为“实际交换价值”的尺度\footnote{事实上马尔萨斯认为用单一的商品能支配的劳动作为价值尺度都是不够精准的,马尔萨斯认为在某些情况下,由谷物和劳动两种尺度组合成的新的尺度优于任何一个单独的尺度\cite[98-105]{BiLuo*SiLaFaDaWeiLiJiaTuQuanJiDi2JuanMaErSaSiZhengZhiJingJiXueYuanLiPingZhu2013}。}\cite[60-82, 92-97]{BiLuo*SiLaFaDaWeiLiJiaTuQuanJiDi2JuanMaErSaSiZhengZhiJingJiXueYuanLiPingZhu2013}\cite[133]{MaErSaSiZhengZhiJingJiXueDingYi2023}。这是因为:第一,“用价值中的绝大部分来进行交换的是,生产性或非生产性的劳动”\cite[92]{BiLuo*SiLaFaDaWeiLiJiaTuQuanJiDi2JuanMaErSaSiZhengZhiJingJiXueYuanLiPingZhu2013};第二,“只有与劳动交换的商品的价值,能够表达商品对社会需要和爱好的配合程度,能够表达同消费者的愿望与人数对照下,商品供给的丰裕程度。”\cite[92]{BiLuo*SiLaFaDaWeiLiJiaTuQuanJiDi2JuanMaErSaSiZhengZhiJingJiXueYuanLiPingZhu2013};第三,“资本的积累,以及其增加财富和人口的效能...取决于其换取劳动的力量”\cite[93]{BiLuo*SiLaFaDaWeiLiJiaTuQuanJiDi2JuanMaErSaSiZhengZhiJingJiXueYuanLiPingZhu2013}。最后,用商品生产耗费的劳动作为价值尺度则存在很多例外\cite[60-82]{BiLuo*SiLaFaDaWeiLiJiaTuQuanJiDi2JuanMaErSaSiZhengZhiJingJiXueYuanLiPingZhu2013},既然例外如此之多,例外反倒成了法则\cite[171]{CaiJiMingCongGuDianZhengZhiJingJiXueDaoZhongGuoTeSeSheHuiZhuYiZhengZhiJingJiXueJiYuZhongGuoShiJiaoDeZhengZhiJingJiXueYanBianShangCe2023}。

接着,马尔萨斯又提出一切商品的实际交换价值取决于市场的供给和需求的相对关系\cite[43]{BiLuo*SiLaFaDaWeiLiJiaTuQuanJiDi2JuanMaErSaSiZhengZhiJingJiXueYuanLiPingZhu2013}。这里,需求指的是“购买的力量和愿望的结合”\cite[43]{BiLuo*SiLaFaDaWeiLiJiaTuQuanJiDi2JuanMaErSaSiZhengZhiJingJiXueYuanLiPingZhu2013},供给指的是“商品的生产和卖出商品的意向的结合”\cite[43]{BiLuo*SiLaFaDaWeiLiJiaTuQuanJiDi2JuanMaErSaSiZhengZhiJingJiXueYuanLiPingZhu2013}。在此基础上,马尔萨斯又进一步指出尽管商品的生产成本是商品供给的必要条件,对商品价格有很大的影响\cite[50,55]{BiLuo*SiLaFaDaWeiLiJiaTuQuanJiDi2JuanMaErSaSiZhengZhiJingJiXueYuanLiPingZhu2013},但生产成本本身还是由供求法则决定的,所以商品的实际交换价值根本上是由供求法则来决定的\cite[59]{BiLuo*SiLaFaDaWeiLiJiaTuQuanJiDi2JuanMaErSaSiZhengZhiJingJiXueYuanLiPingZhu2013}。

总的来说,马尔萨斯把商品所能支配的劳动作为衡量“实际交换价值”的尺度,并认为是供需关系决定了商品的“实际交换价值”。

\subsection{马尔萨斯的生产力理论}

马尔萨斯与其所处时代的其他经济学家一样,并没有提出绝对生产力以外的生产力概念。马尔萨斯实际上是在分析财富的增长时间接地分析了生产力。

马尔萨斯指出,财富和价值有着根本性的区别,财富的多少“部分取决于产品的数量,部分取决于产品对社会的需要和力量的适应”\cite[292]{BiLuo*SiLaFaDaWeiLiJiaTuQuanJiDi2JuanMaErSaSiZhengZhiJingJiXueYuanLiPingZhu2013}。光看这段话,似乎可以认为马尔萨斯的“财富”与使用价值是相等的概念,但是笔者认为,马尔萨斯的“财富”指的应当是全社会商品的价值总量。因为马尔萨斯在后文中指出人们为了取得商品所作的牺牲是财富存在的唯一原因\cite[292]{BiLuo*SiLaFaDaWeiLiJiaTuQuanJiDi2JuanMaErSaSiZhengZhiJingJiXueYuanLiPingZhu2013},以及“...,对财富的不断增长,...,就非有对商品需求的不断增长的配合不可”\cite[355]{BiLuo*SiLaFaDaWeiLiJiaTuQuanJiDi2JuanMaErSaSiZhengZhiJingJiXueYuanLiPingZhu2013}。马尔萨斯之所以认为财富和价值有着根本性的区别,还是因为他没能从交换价值中抽象出价值的概念,而交换价值是某一个比例,只有大小而无数量,所以马尔萨斯必须引入“财富”来衡量价值的数量或大小。

紧接着,马尔萨斯分析了“财富增长的直接原因”,并指出“资本的积累、土地的肥力和节省劳动的发明”是提高供给的三大原因\cite[355]{BiLuo*SiLaFaDaWeiLiJiaTuQuanJiDi2JuanMaErSaSiZhengZhiJingJiXueYuanLiPingZhu2013}。但这还不够,马尔萨斯还强调“生产力与分配手段结合的必要性”\cite[356]{BiLuo*SiLaFaDaWeiLiJiaTuQuanJiDi2JuanMaErSaSiZhengZhiJingJiXueYuanLiPingZhu2013}:从正面看,生产力的提升本身不能保证财富的增长,为了让生产力充分发挥作用,还需要“这样一种情况的产品分配,和产品对消费者的需要的这样一种情况的适应,从而使全部产品的交换价值\footnote{这里可以看到因马尔萨斯没能区分价值和交换价值而产生的谬误}不断提高”\cite[356]{BiLuo*SiLaFaDaWeiLiJiaTuQuanJiDi2JuanMaErSaSiZhengZhiJingJiXueYuanLiPingZhu2013};从反面看,“假使一个国家所有的公路和运河都被破坏,其产品的分配手段根本受到阻碍,其产品的整个价值将显著下降”\cite[357]{BiLuo*SiLaFaDaWeiLiJiaTuQuanJiDi2JuanMaErSaSiZhengZhiJingJiXueYuanLiPingZhu2013}。

\subsection{马尔萨斯的生产力与价值的关系}

根据以上的分析,我们不难看出马尔萨斯笔下的生产力是绝对生产力,且马尔萨斯认为当绝对生产力能适应消费者的需求时,绝对生产力的提升与全社会商品的总价值量正相关;当绝对生产力不能适应消费者的需求时,绝对生产力的提升与全社会商品的总价值量不一定相关。

另一方面,马尔萨斯也间接地指出了绝对生产力与单位商品价值量之间的关系。他说“用改进的机器,在同样成本下取得同样质量的更多的商品时,财富与价值之间的区别是明显的;然而,即使就这里的情况说,这一增益量的拥有者,也只是从消费看来,而不是从交换看来,比前富裕”\cite[291]{BiLuo*SiLaFaDaWeiLiJiaTuQuanJiDi2JuanMaErSaSiZhengZhiJingJiXueYuanLiPingZhu2013}。不难看出,马尔萨斯认为绝对生产力与单个生产者单位劳动创造的价值量是不相关的。这与广义价值论的判断是不同的,其原因是马尔萨斯采取了供求价值论,商品的数量增加会影响商品的供求关系。最后,在供求价值论下,绝对生产力与单位商品的价值量仍应是负相关的,这一结论与广义价值论的判断是一致的。

\subsection{萨伊的价值理论}

萨伊把使用价值和交换价值放到一起来分析价值,形成了一种综合了生产要素论、生产费用论和效用论的价值论\footnote{原文认为还有“供求价值论”,但笔者认为萨伊并没有引入“供求价值论”。}\cite[138]{YanZhiJieXiFangJingJiXueShuoShiJiaoChengDiErBan2013}。

萨伊认为,“物品满足人类需要的内在力量叫做效用”,效用构成了商品价值的基础\cite[59]{SaYiZhengZhiJingJiXueGaiLunCaiFuDeShengChanFenPeiHeXiaoFei2020}。萨伊进一步指出,价格是测量价值的尺度,价值又是测量效用的尺度\cite[60]{SaYiZhengZhiJingJiXueGaiLunCaiFuDeShengChanFenPeiHeXiaoFei2020}。

在此基础上,萨伊又认为价值或者效用是劳动、资本和自然力共同创造的\cite[78]{SaYiZhengZhiJingJiXueGaiLunCaiFuDeShengChanFenPeiHeXiaoFei2020},劳动、资本和自然力是价值的三大来源。接着,萨伊又指出了价值的决定方式。首先,萨伊指出商品的价值决定于“在生产方面所作的努力”\cite[351]{SaYiZhengZhiJingJiXueGaiLunCaiFuDeShengChanFenPeiHeXiaoFei2020},这种努力该如何衡量?萨伊又进一步指出:生产力的价值来源于创造社会所需要的效用的能力,与其在生产过程中的重要性成正比,表现为生产费用\cite[352]{SaYiZhengZhiJingJiXueGaiLunCaiFuDeShengChanFenPeiHeXiaoFei2020}。也就是说,生产费用衡量了生产商品所付出的努力,进而决定了商品的价值。这里,萨伊强调了“生产劳动的市值,是基于许多产品相比较的价值”\cite[352]{SaYiZhengZhiJingJiXueGaiLunCaiFuDeShengChanFenPeiHeXiaoFei2020},也就是说,生产费用决定于某一生产要素潜在的创造最大效用的能力。萨伊还引用了一个例子来阐释这一思想:生产一件生产费用为四法郎而售价为三法郎的物品的生产力的价值不是三法郎而是四法郎。因为既然该生产力的生产费用为四法郎,那么其本来能够创造四法郎的价值。但是在这种情况下,该生产力仅仅创造了三法郎的价值\cite[352]{SaYiZhengZhiJingJiXueGaiLunCaiFuDeShengChanFenPeiHeXiaoFei2020}。

前文提到,萨伊认为价格是衡量价值的尺度。那价格是如何决定的呢?萨伊指出价格受到供求法则的支配——货物的价格随和需求成正比,和供给成反比\cite[256]{SaYiZhengZhiJingJiXueGaiLunCaiFuDeShengChanFenPeiHeXiaoFei2020}。然而,萨伊并没有指出由生产费用决定的价值是如何转化为受供求关系支配的价格的,这种空缺使得萨伊的价值理论看上去比较混乱。例如,有的学者认为萨伊的价值概念有三重内涵:第一是要素成本意义上的价值,指的是获取商品的必要代价,体现为劳动、资本、土地的生产要素支出;第二是市场交易价值,即市价,指的是供求关系调节形成的即时交换比率;第三是真正的价值,指的是来自生产费用的物品的效用\cite[138]{YanZhiJieXiFangJingJiXueShuoShiJiaoChengDiErBan2013}\cite[175]{CaiJiMingCongGuDianZhengZhiJingJiXueDaoZhongGuoTeSeSheHuiZhuYiZhengZhiJingJiXueJiYuZhongGuoShiJiaoDeZhengZhiJingJiXueYanBianShangCe2023}。在笔者看来,萨伊的价值理论没有那么复杂。如前所述,萨伊认为价值的基础是效用,价值的来源是劳动、资本和土地的协作生产,价值又决定于商品生产过程中所使用的这三种要素的生产费用,而生产费用又取决于某生产力能创造的最大效用。所以,萨伊的价值理论是效用论和生产费用论的综合。不过在解释价格和价值之间的关系时,萨伊遇到了和马克思“转形问题”类似的价值向价格转化的问题。

\subsection{萨伊的生产力理论}

不难推断,持效用价值论观点的萨伊没有提出绝对生产力以外的生产力概念,他认为:“所谓生产,不是创造物质,而是创造效用。”\cite[60]{SaYiZhengZhiJingJiXueGaiLunCaiFuDeShengChanFenPeiHeXiaoFei2020}萨伊进一步用“劳力”来衡量生产的效率,也就是衡量绝对生产力的大小,他指出:“我所说的劳力,是指任何一种劳动工作时所进行的继续不断的动作,或在从事任何一种劳动工作的某一部分时所进行的继续不断的动作。”\cite[90]{SaYiZhengZhiJingJiXueGaiLunCaiFuDeShengChanFenPeiHeXiaoFei2020}这里的“劳力”又可以按照价值来源的三要素被分为:人的劳力、自然的劳力和机器的劳力\cite[90]{SaYiZhengZhiJingJiXueGaiLunCaiFuDeShengChanFenPeiHeXiaoFei2020}。而绝对生产力的提升,“在于减少生产同一数量产品所需的劳力,或与此类似,在于扩大同一数量人力所能获得的产品量”\cite[91]{SaYiZhengZhiJingJiXueGaiLunCaiFuDeShengChanFenPeiHeXiaoFei2020}

\subsection{生产力与价值的关系}

广义价值论关于绝对生产力与单位商品价值量负相关的判断和萨伊的判断是一致的——“由于人类所掌握的生产手段实际上变得更有力量,所以创造出来的产品在数量上总是增加,而在价值上总是成比例地减少。”\cite[369]{SaYiZhengZhiJingJiXueGaiLunCaiFuDeShengChanFenPeiHeXiaoFei2020}这里的“价值”应当是指单位商品的价值量。

当然,由于萨伊没有统一价值的来源——效用和价值的决定——生产费用,以上的结论显然是不严谨的。

\section{约翰$\cdot$穆勒}

 穆勒被尊称为古典经济学的集大成者,对各种观点的调和折中是其经济思想公认的特征\cite[165]{YanZhiJieXiFangJingJiXueShuoShiJiaoChengDiErBan2013}\cite[176-178]{CaiJiMingCongGuDianZhengZhiJingJiXueDaoZhongGuoTeSeSheHuiZhuYiZhengZhiJingJiXueJiYuZhongGuoShiJiaoDeZhengZhiJingJiXueYanBianShangCe2023}。
 
 \subsection{约翰$\cdot$穆勒的价值理论}

 穆勒继承了斯密对交换价值和使用价值的区分,并将其研究重点放在了交换价值上。和斯密一样,穆勒认为商品的价值“是指它的一般购买力,即拥有这一物品对于一般可购商品所具有的支配力”,而价格则是用货币表示的价值\cite[493]{YueHan*MuLeZhengZhiJingJiXueYuanLiJiQiZaiSheHuiZheXueShangDeRuoGanYingYongShangJuan1991}。

 穆勒强调,“价值是一个相对的术语”\cite[495]{YueHan*MuLeZhengZhiJingJiXueYuanLiJiQiZaiSheHuiZheXueShangDeRuoGanYingYongShangJuan1991}。由于穆勒笔下的价值就是交换价值,其尺度是可以用其交换到的其它商品或货币,所以不可能出现所有商品的价值一起提高的情况。另外,所以穆勒在价值尺度问题上认为价值尺度是一种充当参照物的物品:将任意两种物品分别与其进行价值对比,即可揭示它们之间的价值关系。\cite[102]{YueHan*MuLeZhengZhiJingJiXueYuanLiJiQiZaiSheHuiZheXueShangDeRuoGanYingYongXiaJuan1991}从这个定义出发,由于价值的相对性,所以“任何商品在一定的时间和地点都可以用作价值尺度;因为如果我们知道两种物品各自与任何第三种物品相交换的比率,就总是可以推知这两种物品相互交换的比率”\cite[102]{YueHan*MuLeZhengZhiJingJiXueYuanLiJiQiZaiSheHuiZheXueShangDeRuoGanYingYongXiaJuan1991}。但是穆勒自己也意识到,这种定义下的价值尺度的并不是斯密、马克思或马尔萨斯所讨论的价值尺度——“政治经济学家所寻求的,$\cdots$是同一物品在不同的时间和地点的价值尺度”\cite[103]{YueHan*MuLeZhengZhiJingJiXueYuanLiJiQiZaiSheHuiZheXueShangDeRuoGanYingYongXiaJuan1991};政治经济学家们是希望能找到这样一种媒介,​一种商品的价值,仅需通过与这一通用媒介进行对比即可确定,无需再与其他任何特定商品逐一比较\cite[103]{YueHan*MuLeZhengZhiJingJiXueYuanLiJiQiZaiSheHuiZheXueShangDeRuoGanYingYongXiaJuan1991}。显然,“政治经济学家所寻求的”那种价值尺度是和穆勒所认为的价值的相对性在根本上矛盾的,所以穆勒认为价值尺度是不存在的\cite[104]{YueHan*MuLeZhengZhiJingJiXueYuanLiJiQiZaiSheHuiZheXueShangDeRuoGanYingYongXiaJuan1991}。但穆勒没有完全否定以往的政治经济学家,他又指出:“把这个概念称为生产费用的尺度,或许更为恰当”\footnote{这实际上也反映了穆勒的价值理论中生产费用论的倾向。}\cite[104]{YueHan*MuLeZhengZhiJingJiXueYuanLiJiQiZaiSheHuiZheXueShangDeRuoGanYingYongXiaJuan1991}。但由于“生产费用不变的商品是没有的”,所以生产费用尺度实际上“同交换价值尺度一样是不存在的”\cite[105]{YueHan*MuLeZhengZhiJingJiXueYuanLiJiQiZaiSheHuiZheXueShangDeRuoGanYingYongXiaJuan1991}。最后,尽管在价值尺度方面取得进展,穆勒却意识到了许多古典政治经济学犯了混淆价值尺度和价值决定的错误,他指出:“我们不应该把价值尺度的概念同价值规定者或决定价值的原理相混淆。”\cite[107]{YueHan*MuLeZhengZhiJingJiXueYuanLiJiQiZaiSheHuiZheXueShangDeRuoGanYingYongXiaJuan1991}笔者认为,在区分价值尺度和价值决定一点上,穆勒是相当值得肯定的。

 穆勒还认为,商品要有价值,一方面必须具备一定的效用,另一方面必须是稀缺的\footnote{也就是在获得上存在困难。}\cite[499]{YueHan*MuLeZhengZhiJingJiXueYuanLiJiQiZaiSheHuiZheXueShangDeRuoGanYingYongShangJuan1991}。而且商品的价值是由竞争决定的,一方面“买主力求贱买,卖主则力求贵卖”,另一方面“质量相同的物品,在同一市场上不能有两种价格\footnote{根据上下文,这里也可以理解为“价值”}”\cite[497]{YueHan*MuLeZhengZhiJingJiXueYuanLiJiQiZaiSheHuiZheXueShangDeRuoGanYingYongShangJuan1991}。穆勒又将价值(交换价值)区分为暂时价值(市场价值)和永久价值(自然价值),前者是指受到供求法则支配的经常变动的价值,后者是指前者波动的中心\cite[2]{YueHan*MuLeZhengZhiJingJiXueYuanLiJiQiZaiSheHuiZheXueShangDeRuoGanYingYongXiaJuan1991}。为了研究自然价值,穆勒进一步把商品分成三类:第一类是供给绝对有限,其数量不能任意增加的商品,例如古代雕塑、需要特殊土壤和气候才能生产的葡萄酒等\cite[502]{YueHan*MuLeZhengZhiJingJiXueYuanLiJiQiZaiSheHuiZheXueShangDeRuoGanYingYongShangJuan1991};第二类是用劳动和资本可以任意增加的商品,如大多数的工业产品\cite[502,532]{YueHan*MuLeZhengZhiJingJiXueYuanLiJiQiZaiSheHuiZheXueShangDeRuoGanYingYongShangJuan1991};第三类是“用劳动和费用可以无限量地增加,但不能以固定数量的劳动和费用无限量地增加”,例如农产品\cite[502]{YueHan*MuLeZhengZhiJingJiXueYuanLiJiQiZaiSheHuiZheXueShangDeRuoGanYingYongShangJuan1991}。“这类商品不是有一种生产费用,而是有几种生产费用”\cite[532]{YueHan*MuLeZhengZhiJingJiXueYuanLiJiQiZaiSheHuiZheXueShangDeRuoGanYingYongShangJuan1991}。这三类商品有着不同的价值决定法则。

 第一类商品的价值由供求法则决定——商品的价值因需求增长或供给短缺而上升,因需求减弱或供给过剩而下降。这一机制将持续作用,直到市场上的买方力量与卖方力量在均衡价值处重新取得匹配。”\cite[506-507]{YueHan*MuLeZhengZhiJingJiXueYuanLiJiQiZaiSheHuiZheXueShangDeRuoGanYingYongShangJuan1991}

 第二类商品的价值由生产费用决定。穆勒认为,这类商品都有一个“最低价值”,这个“最低价值”是由生产费用加上通常的利润构成的,\cite[510-511]{YueHan*MuLeZhengZhiJingJiXueYuanLiJiQiZaiSheHuiZheXueShangDeRuoGanYingYongShangJuan1991}。而这里的生产费用指的是劳动数量\cite[517]{YueHan*MuLeZhengZhiJingJiXueYuanLiJiQiZaiSheHuiZheXueShangDeRuoGanYingYongShangJuan1991}。值得注意的是,这里的“劳动数量”包括生产过程中所投入的资本,因为资本也是由先前的劳动创造出来的\cite[517]{YueHan*MuLeZhengZhiJingJiXueYuanLiJiQiZaiSheHuiZheXueShangDeRuoGanYingYongShangJuan1991}。另外,这里的“劳动数量”也不是用劳动工资来衡量的,因为劳动工资的升降,会同时影响所有商品,所以不会改变商品之间的相对价值\cite[519-520]{YueHan*MuLeZhengZhiJingJiXueYuanLiJiQiZaiSheHuiZheXueShangDeRuoGanYingYongShangJuan1991}。但是,如果不同行业之间的工资有差异,那么这种差异就会“改变不同商品的相对生产费用”,并影响商品的价值\cite[521]{YueHan*MuLeZhengZhiJingJiXueYuanLiJiQiZaiSheHuiZheXueShangDeRuoGanYingYongShangJuan1991}。穆勒还指出了价值由生产费用决定和由供求法则决定这两种决定方式之间的关系,他认为对于第二类商品,其价值“并非取决于需求和供给;相反地,需求和供给取决于价值”\cite[515]{YueHan*MuLeZhengZhiJingJiXueYuanLiJiQiZaiSheHuiZheXueShangDeRuoGanYingYongShangJuan1991}。

第三类商品的价值由“以最大费用生产并运至市场的那部分供应量的生产费用”\cite[535]{YueHan*MuLeZhengZhiJingJiXueYuanLiJiQiZaiSheHuiZheXueShangDeRuoGanYingYongShangJuan1991}决定。这里的“以最大费用生产”应当是指最劣等土地的生产费用。由于这里的最劣等土地是指那些仅能产生普通利润而不能产生地租的土地\footnote{这说明穆勒否认绝对地租的存在\cite[176]{YanZhiJieXiFangJingJiXueShuoShiJiaoChengDiErBan2013}},所以穆勒进一步认为地租不是这类商品生产费用的组成部分,也就不影响这类商品的价值\cite[596]{YueHan*MuLeZhengZhiJingJiXueYuanLiJiQiZaiSheHuiZheXueShangDeRuoGanYingYongShangJuan1991}。最后,我们可以推断第三类商品价值生产费用决定和供求法则决定这两种决定方式之间的关系和第二类商品是一致的。

注意,以上的分析都是建立在不同行业的资本、劳动投入占比相同而且占用资本时间相同的前提之下的,如果这个前提不成立,那么按照等量资本带来等量利润的原则,商品将不仅仅按照生产所需的劳动数量比例进行交换,而且一般利润的每一次升降都会影响价值\cite[526-527]{YueHan*MuLeZhengZhiJingJiXueYuanLiJiQiZaiSheHuiZheXueShangDeRuoGanYingYongShangJuan1991}。可见,穆勒也没能较好地解释“转形问题”,而只是强行增加了价值决定的法则。

总的来说,穆勒价值理论的最大特点就是折衷,他通过对各种商品进行分类,将劳动价值论、供需价值论、生产费用价值论糅合在了一起。另外,正如熊彼特所言,穆勒自己的贡献是“充分发展了供给与需求分析”\cite[359]{YueSeFu*XiongBiTeJingJiFenXiShiDi2Juan2017}。他成功地将商品价值由生产费用决定和由供需法则决定这两种决定方式统一在了一起。

 \subsection{约翰$\cdot$穆勒的生产力概念}

穆勒认为,劳动生产的是效用\cite[60]{YueHan*MuLeZhengZhiJingJiXueYuanLiJiQiZaiSheHuiZheXueShangDeRuoGanYingYongShangJuan1991}。效用有三类:第一类是“固定和体现在外界物体中的效用”\cite[62]{YueHan*MuLeZhengZhiJingJiXueYuanLiJiQiZaiSheHuiZheXueShangDeRuoGanYingYongShangJuan1991},即平常认为的物的效用;第二类是“固定和体现在人身上的效用”,也就是“对自己和别人有用的品质”\cite[62]{YueHan*MuLeZhengZhiJingJiXueYuanLiJiQiZaiSheHuiZheXueShangDeRuoGanYingYongShangJuan1991},笔者理解为是人力资本;第三类则是“存在于所提供地服务中”的效用\cite[62]{YueHan*MuLeZhengZhiJingJiXueYuanLiJiQiZaiSheHuiZheXueShangDeRuoGanYingYongShangJuan1991}。据此,我们不难推测穆勒和前文提到的其他经济学家一样,认识到了绝对生产力的存在。

接着,穆勒总结了五条绝对生产力提高的原因:第一是有利的自然条件;第二是较大的劳动干劲;第三是较高的技能和知识;第四是较高的整个社会的知识水平和相互信任程度;第五是较高的安全感。\cite[123-135]{YueHan*MuLeZhengZhiJingJiXueYuanLiJiQiZaiSheHuiZheXueShangDeRuoGanYingYongShangJuan1991}

但如前文所述,穆勒还强调了价值概念的相对性,穆勒讨论并否认了“所有商品价值一起提高”的可能性,这就使他拥有了探究比较生产力的潜力。假使所有商品的绝对生产力都提高,那么某一商品的价值变化就要考虑其它商品的绝对生产力变化之幅度。遗憾的是,穆勒并没有在此深究。

\subsection{生产力与价值的关系}

根据穆勒对价值决定的理解,我们不难推断:如果资本、劳动投入占比不变且占用资本时间不变,那么绝对生产力提高会使单位商品的生产费用下降,进而使单位商品的价值下降。

\section{小结}

\subsection{劳动作为衡量交换价值的尺度}

如前所述,亚当$\cdot$斯密在国富论中首次提出了“交换价值\footnote{这里应当是价“价值”而非“交换价值;这里笔者先按照几位古典政治经济学家的原文使用“交换价值”一词;交换价值和价值的关系在后文中阐述。}的真实尺度”问题,并指出劳动是这一问题的答案。而后李嘉图、马克思、马尔萨斯都认为劳动是衡量交换价值的尺度。值得注意的是,马克思认为古典政治经济学的根本倾向是劳动价值论,但这一论断忽视了古典政治经济学不同理论之间的差异性\footnote{由于学界对古典政治经济学的划分不尽相同,对古典政治经济学家思想的理解也不尽相同,所以有学者认为古典政治经济学所得出的结论是“多要素供求价值论”\cite[179]{CaiJiMingCongGuDianZhengZhiJingJiXueDaoZhongGuoTeSeSheHuiZhuYiZhengZhiJingJiXueJiYuZhongGuoShiJiaoDeZhengZhiJingJiXueYanBianShangCe2023}。}。根据笔者对古典政治经济学的认识,笔者认为更有说服力的结论是:古典政治经济学的基本倾向是把劳动作为衡量交换价值的尺度;这里的价值尺度,指的是衡量商品价值量的社会标准,强调价值的社会属性。

一方面,古典政治经济学家意识到,在社会分工建立起来后,一个生产者或许因为不能生产某一需要的商品,或许因为通过交换得到的商品更好更便宜,所以会愿意用自己的劳动生产一些不愿自己消费的商品来交换他物。然而,商品都具有作为使用价值的异质性,为了与其它商品进行交换,商品的拥有者会根据在这一商品上花费的劳动\footnote{包括活劳动和物化劳动}与他人进行交换。这是因为这个生产者为了生产者这一商品所投入的就是自己的劳动,也只能知道自己为了这一商品投入了多少劳动。而那些与他进行商品交换的生产者,也是以同样的考量与他进行交换。因此,劳动成为了商品交换时所依据的尺度。换言之,劳动成为了衡量交换价值的尺度。\cite[1016-1017]{ZhongGongZhongYangMaKeSiEnGeSiLieNingSiDaLinZhuZuoBianYiJuMaKeSiEnGeSiWenJiDi7Juan2009}\cite[25]{YaDang*SiMiGuoFuLun2015}\cite[133]{MaErSaSiZhengZhiJingJiXueDingYi2023}

另一方面,交换价值会随着时间和地点的不同上下波动,具有偶然性和相对性。因此,古典政治经济学家所能找到的不变的、普遍的交换价值的尺度,只有劳动。\cite[49-51]{ZhongGongZhongYangMaKeSiEnGeSiLieNingSiDaLinZhuZuoBianYiJuMaKeSiEnGeSiWenJiDi5Juan2009}\cite[28-30]{YaDang*SiMiGuoFuLun2015}

\subsection{从交换价值到价值}

然而,交换价值是一种使用价值同另一种使用价值的比例,其本身没有单位,也没有数量大小上的意义。所以前文中古典政治经济学家使用“交换价值的尺度”这一说法是不合理的——一个没有大小的量怎么能被“衡量”呢?事实上,古典政治经济学家所衡量的其实不是交换价值,而是价值。只是在政治经济学发展的早期阶段,经济学家们还没有能力从交换价值中抽象出价值的概念。直到马克思首次区分了交换价值和价值,价值的面纱才被揭开\cite[86-87]{ChenDaiSunCongGuDianJingJiXuePaiDaoMaKeSiRuoGanZhuYaoXueShuoFaZhanLueLun2014}。

笔者认为,政治经济学逐步区分交换价值和价值的过程可以从两个角度来理解。

首先,从价值实体的角度来看,假设某种商品可以和多种商品进行交换,那么这一商品就具有了许多种交换价值,这些交换价值之间必然是可以相互替代的。所以,正如马克思在揭示交换价值本质时所指出的:第一,同一商品表现出的多种交换价值必然指向某种内在的同一性实体;第二,交换价值本身仅是这个同一实体在流通领域的“表现形式”\cite[49]{ZhongGongZhongYangMaKeSiEnGeSiLieNingSiDaLinZhuZuoBianYiJuMaKeSiEnGeSiWenJiDi5Juan2009}。也就是说,这些不同的交换价值都可以被化为“一种等量的共同的东西”\cite[49]{ZhongGongZhongYangMaKeSiEnGeSiLieNingSiDaLinZhuZuoBianYiJuMaKeSiEnGeSiWenJiDi5Juan2009}。而且,这种“共同的东西”既不是使用价值——使用价值具有异质性,也不是交换价值——交换价值仅仅是一个比例,没有大小,也无法被“衡量”。进而,马克思把这种共同的东西称为价值\cite[50]{ZhongGongZhongYangMaKeSiEnGeSiLieNingSiDaLinZhuZuoBianYiJuMaKeSiEnGeSiWenJiDi5Juan2009}。值得注意的是,尽管马克思已经能从交换价值中抽象出价值来,但是笔者却不认同马克思推演的逻辑。马克思在意识到不同的交换价值是在衡量一个既不是使用价值也不是交换价值的“共同的东西”之后,就使用排除法得到商品只剩下“劳动产品的属性”\cite[50-51]{ZhongGongZhongYangMaKeSiEnGeSiLieNingSiDaLinZhuZuoBianYiJuMaKeSiEnGeSiWenJiDi5Juan2009},再把商品体耗费的具体劳动的成分撇开,那么商品体剩下的就是“无差别的人类劳动的单纯凝结”。最后,马克思认为这种凝结的抽象劳动,就是商品的价值实体\cite[51]{ZhongGongZhongYangMaKeSiEnGeSiLieNingSiDaLinZhuZuoBianYiJuMaKeSiEnGeSiWenJiDi5Juan2009}。马克思论证过程的的问题主要在于:第一,马克思使用了排除法得到商品体只剩下劳动产品的属性,但根据古典政治经济学中多要素价值论的观点,商品体在排除使用价值后还有可能剩下多种生产要素产品的属性。劳动产品的属性只是排除了使用价值后剩下的属性之一,而不是剩下的唯一属性;第二,若主张使用价值的异质性使其不能决定价值,则同理可证劳动的异质性亦不具备价值决定功能;若承认具体劳动可抽象为无差别人类劳动,则此抽象逻辑同样可以将使用价值抽象为无差别的效用\cite[84]{CaiJiMingLunJieZhiJueDingYuJieZhiFenPeiDeTongYi2003}。也就是说,将商品的价值和“无差别的人类劳动的单纯凝结”等同起来并不是一个不证自明的过程,马克思的论证逻辑是存在问题的。但是,笔者认为马克思将价值从交换价值中抽象出来的逻辑过程是正确的,价值确实是一种既不同于使用价值也不同于交换价值的,客观存在的实体。

正因如此,笔者进一步认为前文中不同经济学家对“交换价值的尺度”的争论实际上是对“价值尺度”的争论。交换价值本身只是两种使用价值的比例,没有“大小”的概念,因此也无法被“衡量”。而价值作为一个实体,理论上是可以被衡量的。不同经济学家所争论的,正是价值应该用什么尺度来衡量,或者说价值的单位究竟是什么。

其次,人们对价值认识的演进和商品交换的发展是两个相反的过程。在商品交换的早期阶段,商品的交换价值表现为两个使用价值的比例关系\cite[49]{ZhongGongZhongYangMaKeSiEnGeSiLieNingSiDaLinZhuZuoBianYiJuMaKeSiEnGeSiWenJiDi5Juan2009},也就是简单的交换价值形式\footnote{在资本论中,马克思用的是“简单价值形式”一词,但由于马克思没有绝对严谨地区分价值和交换价值\cite[37]{ZhongGongZhongYangMaKeSiEnGeSiLieNingSiDaLinZhuZuoBianYiJuMaKeSiEnGeSiWenJiDi8Juan2009},所以这里笔者采取了蔡继明教授的更严谨的提法\cite[145]{CaiJiMingJieZhiZhengLunHuiGuYuZhanWang2008}。};随着交换范围的扩大和交换价值种类的增加,简单的交换价值形式演化为扩大的交换价值形式;当一般等价物出现后,所有的商品都借助一般等价物进行交换,交换价值形式就发展为以一般等价物衡量的一般形式;当货币产生后,交换价值形式便最终发展为价格形式。在价格形式出现后,人们发现尽管市场价格受供求波动影响,但从长期来看其始终会围绕着一个相对稳定的轴心运动——这个轴心,或者说调节价格运动的规律,被亚当$\cdot$斯密称为“自然价格”、马尔萨斯谓之“自然价值”,最终在政治经济学发展中凝练为“价值”概念。至此,价值作为价格运动规律的本质规定性得以确立。\cite[145]{CaiJiMingJieZhiZhengLunHuiGuYuZhanWang2008}\footnote{事实上马克思也把价值的这种内涵以价值作用的方式表述了出来\cite[199]{ZhongGongZhongYangMaKeSiEnGeSiLieNingSiDaLinZhuZuoBianYiJuMaKeSiEnGeSiWenJiDi7Juan2009}。}随着人们进一步认识到价格只是交换价值的一种形式,价值的内涵也进一步一般化为调节交换价值的规律。

正如马克思所言:“对人类生活形式的思索,从而对这些形式的科学分析,总是采取同实际发展相反的道路。这种思索是从事后开始的,就是说,是从发展过程的完成的结果开始的。$\cdots$因此,只有商品价格的分析才导致价值量的决定,只有商品共同的货币表现才导致商品的价值性质的确定。”\cite[93]{ZhongGongZhongYangMaKeSiEnGeSiLieNingSiDaLinZhuZuoBianYiJuMaKeSiEnGeSiWenJiDi5Juan2009}在现实层面,价值研究确实肇始于对价格现象的经验观察;在理论层面,支配交换价值的规律作为价值的内在本质,也符合人类认知从现象到本质的渐进过程,体现了历史发展的内在连贯性。

上述对价值实体认知的演进,自然引向价值决定机制的核心争论——究竟何种劳动形态构成价值尺度?这需要从价值源泉与价值尺度之间的关系展开分析。

\subsection{价值尺度和价值源泉的对立统一}

前文的论述表明,把劳动作为价值的尺度是古典政治经济学的基本倾向\footnote{值得一提的是,作为新古典价值论代表人物的马歇尔也有用支配的劳动作为价值尺度的倾向\cite{perskyMarshallsNeoClassicalLaborValues1999}}。但正如笔者在亚当$\cdot$斯密相关章节中探讨的学术争议所示,古典政治经济学家们在该用商品生产所耗费的劳动还是用商品所能支配的劳动来作为价值尺度的问题上产生了严重的分歧。总的来说,劳动价值论派的李嘉图、马克思认为应当用耗费的劳动作为价值尺度,而多要素价值论派的斯密、马尔萨斯则认为应当用支配的劳动作为价值尺度。

\subsubsection{商品所能支配的劳动作为商品价值的尺度}

在笔者看来,用商品所能支配的劳动作为价值尺度无疑是更合适的。

首先,正如马克思所强调的——价值本身具有社会属性\cite[61]{ZhongGongZhongYangMaKeSiEnGeSiLieNingSiDaLinZhuZuoBianYiJuMaKeSiEnGeSiWenJiDi5Juan2009}。

恩格斯曾在《<资本论>第三册增补》中举了一个例子来论证应当用商品生产所耗费的劳动作为商品价值的尺度\cite[1015-1018]{ZhongGongZhongYangMaKeSiEnGeSiLieNingSiDaLinZhuZuoBianYiJuMaKeSiEnGeSiWenJiDi7Juan2009}。恩格斯说,在商品经济发展的初期,进行商品交换的主要是劳动的农民。这些农民借助自己家庭的帮助,在自己的田地上进行农业、畜牧业和手工业的生产,并拿除必需品之外剩下的剩余产品同其它农民家庭进行交换。这些农民之所以进行交换,并非因为自己不会生产这些物品\footnote{这一对可变分工体系的认识与广义价值论的假设有异曲同工之妙,可惜恩格斯并没有把可变分工体系推广到一般的商品经济。},而是因为得不到原料或者因为通过交换得到的物品要更好或更便宜。这些农民在生产用于交换的产品时——无论是为了补偿工具还是为了加工原料——所耗费的只有自己的劳动。于是,恩格斯得出用耗费的劳动作为价值尺度结论:“在这里,花在这些产品上的劳动时间不仅对于互相交换的产品量的数量规定来说是唯一合适的尺度;在这里,也根本不可能有别的尺度。”\cite[1016]{ZhongGongZhongYangMaKeSiEnGeSiLieNingSiDaLinZhuZuoBianYiJuMaKeSiEnGeSiWenJiDi7Juan2009}但如果我们进一步深入思考,会发现恩格斯的这一结论是不能成立的。

正如恩格斯所言,农民之所以选择交换,不是因为自己不能生产,而是因为受到资源约束或者因为通过交换能节约自己的劳动。当农民在用劳动作为尺度衡量是否要进行交换的时候,他衡量的是“如果不进行交换,为了得到同样的产品我要多耗费多少劳动?”以及“为了得到同样的产品,如果进行交换我能省下多少劳动?”于是,这些“多耗费的劳动”或者“省下的劳动”就是这个农民眼中,通过交换得到的商品的价值。这里笔者想提请读者注意,农民是通过自己的劳动衡量了别人生产的商品的价值。换句话说,假设有农民A、B,分别花费劳动$L_A$和$L_B$生产商品$C_A$和$C_B$,现在农民A想要通过交换得到商品$C_B$,于是农民A会衡量“多耗费的劳动”或者“省下的劳动”。我们记农民A若生产商品$C_B$比农民B生产商品$C_B$多耗费的劳动为$\Delta L_A$。此时,$\Delta L_A$所衡量的,正是农民B生产的商品$C_B$的价值。同样,农民B也是以同样的逻辑用$\Delta L_B$衡量了商品$C_A$的价值\footnote{当然,这里所指的各种劳动量都可以按照马克思所述的“抽象劳动”概念来理解。}。至此,笔者的分析应当说是忠于恩格斯的分析逻辑的。接下来,根据马克思劳动价值论的等价交换原则,要让上述交换能够长久地持续,农民A必须拿出在农民B看来有足够价值的商品与农民B进行交换;所以要让交换持续,$\Delta L_A = \Delta L_B$必须成立。再根据马克思劳动价值论所认为的“等量劳动带来等量价值”的价值决定原理,农民A必须付出与$\Delta L_B$相等的劳动$L_A$才能创造出$\Delta L_B$的价值,所以我们得到$\Delta L_B = L_A$。也就是说,商品$C_B$实际上能够支配$L_A$量的劳动;进而我们可以说,商品$C_B$的价值是由能够支配$L_A$量的劳动来衡量的;这等价于说商品所能支配的劳动是商品价值的尺度。同样地,我们可以得到$\Delta L_A = L_B$,所以最终有$L_B =\Delta L_A = \Delta L_B = L_A$。以上的分析表明,按照恩格斯分析的逻辑,在简单交换中商品所能支配的劳动是商品价值的尺度;并且,用商品能够支配的劳动作为价值的尺度与马克思的劳动价值论并没有发生任何结论上的冲突。

在成熟的商品经济中,商品能够支配的劳动仍然是价值的尺度。马克思在《资本论》第一卷的第三章中指出,货币的作用之一是价值尺度\cite[114-124]{ZhongGongZhongYangMaKeSiEnGeSiLieNingSiDaLinZhuZuoBianYiJuMaKeSiEnGeSiWenJiDi5Juan2009}。而一种贵金属,例如金,为什么可以成为货币?马克思认为这是因为金本身是劳动产品,因而具有潜在可变的价值\cite[118]{ZhongGongZhongYangMaKeSiEnGeSiLieNingSiDaLinZhuZuoBianYiJuMaKeSiEnGeSiWenJiDi5Juan2009}。那么,如果我们用金来衡量某一件商品的价值,实际上是把由商品生产者耗费的劳动与金的生产者所耗费的抽象劳动进行了比较与折算。于是这件商品的价值不仅是被金衡量出来了,而且是本质上被金的生产者所耗费的抽象劳动衡量出来了。可见,在成熟的商品经济中,把商品能够支配的劳动作为商品价值的尺度也是符合马克思分析的逻辑和结论的。但笔者也注意到,马克思曾在《哲学贫困中》中批评“把用商品中所包含的劳动量来衡量的商品价值和用“劳动价值”来衡量的商品价值混为一谈”的行为\cite[97]{ZhongGongZhongYangMaKeSiEnGeSiLieNingSiDaLinZhuZuoBianYiJuMaKeSiEnGeSiQuanJiDi4Juan1958}。对此,笔者认为马克思所批评的主要是“用‘劳动价值’作为价值尺度”的观点,因为在马克思看来,商品所能支配的劳动就是指“劳动者的报酬”,而如果用劳动者的报酬作为价值尺度,那就相当于把生产费用作为价值尺度,会出现生产费用取决于价值而价值又用生产费用衡量的循环论证问题\cite[98]{ZhongGongZhongYangMaKeSiEnGeSiLieNingSiDaLinZhuZuoBianYiJuMaKeSiEnGeSiQuanJiDi4Juan1958}。\cite[5]{ZhangLeiShengMaKeSiLaoDongJieZhiLunYanJiuDeLiShiZhengTiXing2015}所以,笔者想要强调的是,笔者所指的支配劳动绝不是指劳动者的报酬,而是指一般的、平均的社会劳动。

事实上,如果从广义价值论的角度来看,耗费的劳动是具体的特殊的个别劳动,而支配的劳动是将不同的、具体的个别劳动折算得到的一般的、平均的社会劳动。

最后,笔者想要强调的是,将不同的、具体的个别劳动折算得到的一般的、平均的社会劳动的方式在不同的价值理论中是不一样的。例如马克思认为劳动本身具有二重性——随着机器的使用和分工的细化,劳动之间的差别在不断缩小\cite[96]{ZhongGongZhongYangMaKeSiEnGeSiLieNingSiDaLinZhuZuoBianYiJuMaKeSiEnGeSiQuanJiDi4Juan1958};“在使用机器的企业中,这个工人的劳动和那个工人的劳动几乎没有什么差别;工人彼此间的区别,只是他们在劳动中所化的时间不等。”\cite[97]{ZhongGongZhongYangMaKeSiEnGeSiLieNingSiDaLinZhuZuoBianYiJuMaKeSiEnGeSiQuanJiDi4Juan1958}也就是说,商品耗费的劳动本身就具有抽象劳动的属性;而斯密等经济学家则认是市场上的讨价还价消除了劳动的异质性。目前,经济学界对此还没有一个统一的认识,所以笔者认为在这一点上仍然存在研究的空间\footnote{笔者认为不同折算方式的本质,是对“等价交换”概念的不同理解,可参考《论耗费的劳动与购买的劳动在价值理论中的作用》\cite[69]{CaiJiMingLunHaoFeiDeLaoDongYuGouMaiDeLaoDongZaiJieZhiLiLunZhongDeZuoYong2022}。}。不过,无论是哪种转换方式,都强调了价值的社会属性。因此从社会属性的角度出发,不难看出用支配的劳动比耗费的劳动作为价值尺度更能表现价值的社会属性。

\subsubsection{参与价值决定的所有要素作为价值源泉}

前文提到,古典政治经济学的基本倾向是把劳动作为价值尺度。然而,所有的古典政治经济学家都承认耗费的劳动作为一种价值源泉,在价值决定中发挥了重要作用。事实上,从古典学派内部诸多学者到新古典经济学派,直至广义价值论体系,均承认非劳动要素同样参与价值决定过程,因而具备价值源泉属性。

笔者将从以下几方面来论证应当把参与价值决定的所有要素作为价值源泉。

首先,承认参与价值决定的所有要素作为价值源泉并不是指非劳动要素可以直接替代劳动决定价值,而是指与劳动共同决定价值,只不过二者的作用或贡献都是用购买劳动来折算和表现的。按照广义价值论的观点,非劳动要素是通过影响特定生产者劳动的绝对生产力,进而影响特定生产者相对于交换对手而言的比较生产力,最终参与价值决定的。

其次,从经济现实来看,中共十九届四中全会指出,要“健全劳动、资本、土地、知识、技术、管理、数据等生产要素由市场评价贡献、按贡献决定报酬的机制”\cite[39]{ZhongGuoGongChanDangDiShiJiuJieZhongYangWeiYuanHuiDiSiCiQuanTiHuiYiWenJianHuiBian2019}。笔者认为,这里的“贡献”指的就是各生产要素对价值创造的贡献——倘若我们承认上述六种非劳动生产要素对提升绝对生产力有帮助,根据前文和广义价值论的基本原理,那实际上就是承认了非劳动生产要素对价值创造的贡献\footnote{关于这方面的讨论,参见(蔡继明,2023)\cite[26-56]{CaiJiMingCongGuDianZhengZhiJingJiXueDaoZhongGuoTeSeSheHuiZhuYiZhengZhiJingJiXueJiYuZhongGuoShiJiaoDeZhengZhiJingJiXueYanBianXiaCe2023}}。笔者认为,十九届四中全会把按生产要素贡献分配的分配制度纳入社会主义基本经济制度的范畴\cite[5]{XieFuZhanWanShanJiBenJingJiZhiDuTuiJinGuoJiaZhiLiTiXiXianDaiHuaXueXiGuanCheZhongGongShiJiuJieSiZhongQuanHuiJingShenBiTan2020},一方面是因为经济现实的变化让各种生产要素作为价值源泉的身份越来越凸显。例如,十九届四中全会首次将“数据”增列为生产要素,体现了现代经济增长的新特征\cite[6]{CaiJiMingLunShuJuYaoSuAnGongXianCanYuFenPeiDeJieZhiJiChuJiYuGuangYiJieZhiLunDeShiJiao2023}\cite[5]{XieFuZhanWanShanJiBenJingJiZhiDuTuiJinGuoJiaZhiLiTiXiXianDaiHuaXueXiGuanCheZhongGongShiJiuJieSiZhongQuanHuiJingShenBiTan2020}。另一方面是因为将按生产要素贡献分配确定社会主义基本经济制度,体现了收入分配制度尊重知识、尊重人才、尊重创新的导向,可以更好地解放和发展社会生产力、推动经济高质量发展\cite[4-5]{XieFuZhanWanShanJiBenJingJiZhiDuTuiJinGuoJiaZhiLiTiXiXianDaiHuaXueXiGuanCheZhongGongShiJiuJieSiZhongQuanHuiJingShenBiTan2020}。

\subsubsection{价值源泉向价值因素的转化}

基于广义价值论的视角,价值形成过程应区分为两个阶段:在生产阶段,各种生产要素通过劳动过程将具体化、私人化的劳动转化为特定商品的使用价值。此时,各种价值源泉仅转化为潜在的价值因素,但尚未形成可量化的价值实体。因此笔者把这一过程称为价值源泉向价值因素转化的过程。在交换阶段,通过社会化的市场竞争和供需双方的博弈,具体劳动才真正实现向抽象劳动的转化,价值实体才在质上最终形成,在量上决定下来。因此笔者把这一过程称为价值决定的过程。

笔者区分价值源泉向价值因素的转化过程和和价值的决定过程有以下三个原因。 首先,包括广义价值论在内的许多价值理论都认为价值决定是在商品交换时完成的。也就是说,在商品被生产出来后到交换完成之前的这段过程中,商品的价值的质还没有形成,商品的价值的量也没有决定下来。一方面,商品价值的决定是一个社会的过程,而只有商品交换是在社会中完成的;在完成交换之前,商品耗费的劳动只是具体的、私人的劳动,只有在商品交换完成后,这些耗费的劳动才能转化为抽象的、社会的劳动,社会意义上的价值才能出现。因此,为了区分商品的生产和交换对价值的不同影响,笔者把价值形成划归到商品的生产过程,把价值决定划归到商品的交换过程。其次,这种区分和规定具有普适性。如果我们考虑商品的使用价值,那么其形成正是在商品的生产阶段,其决定是在商品的消费阶段,其尺度是消费者的效用;显然,在商品被生产出来之后和被消费之前,其使用价值只是潜在的、没有决定也没有表现出来的,只有当消费者真正消费商品时,商品的使用价值才被效用衡量而决定下来。上述对使用价值形成和决定的描述也适用于对效用价值论的分析。最后,这种区分是对马克思主义政治经济学的一种继承。尽管马克思不承认非劳动要素也参与价值形成,但他在早期\footnote{如前文所述,马克思在后期认为耗费的劳动本身包含了抽象劳动,因此商品的价值在生产过程就已经形成了。}的著作中认为:“商品的价值由生产成本即劳动决定,是通过竞争的作用实现的”\cite[3]{ZhangLeiShengMaKeSiLaoDongJieZhiLunYanJiuDeLiShiZhengTiXing2015}。 这就是说,马克思也意识到商品的价值在生产完成后,交换完成前处于一种潜在的状态,只有当商品在充满竞争的市场上完成了交换,商品的价值才真正决定下来。换句话说,马克思也意识到了价值源泉向价值因素的转化和价值决定之间的区别,只是他没有将其诉诸文字。

\subsubsection{价值决定——价值源泉和价值尺度的对立统一}

总的来说,耗费的劳动作为价值的源泉实现价值形成,支配的劳动作为价值尺度实现价值衡量。而价值的源泉和价值的尺度在商品交换的过程中实现了对立统一,商品的价值被决定下来。

首先,从微观的层面考察,价值源泉与价值尺度的对立统一并不体现为量上的完全等同,而在于两者可通过特定机制实现折算转换。广义价值论揭示了一个核心规律:部门间耗费劳动与支配劳动的比值等于其比较生产力的相对水平。具体表现为,当某部门比较生产力高于另一部门时,该部门能以较少耗费劳动置换对方较多劳动量,反之则需付出更多劳动;当两部门比较生产力之比等于1时,此时耗费劳动量恰与支配劳动量相等,这种特殊状态正对应着马克思劳动价值论中关于价值决定的基本论断。需要特别说明的是,广义价值论之所以强调比较生产力对价值决定的作用,在于其理论框架不仅考察劳动要素,还纳入非劳动生产要素对绝对生产力的实际影响,这类要素同样会作用于比较生产力的形成过程。

其次,从宏观总量的角度来说,由于耗费的劳动总量等于支配的劳动总量等于全社会价值总量\cite[71-72]{CaiJiMingLunHaoFeiDeLaoDongYuGouMaiDeLaoDongZaiJieZhiLiLunZhongDeZuoYong2022},所以作为价值源泉的耗费劳动和作为价值尺度的支配劳动是对立统一的。对于商品经济来说,任何支配的劳动都必然是要耗费的劳动——价值实体只能取决于耗费的劳动;任何耗费的劳动也都必然是要被支配的劳动——正如商品是用于交换的产品,自己耗费的劳动也是为了支配他人劳动而付出的劳动。在斯密的价值理论中,价值源泉和价值尺度是对立统一的。斯密认为,由每个国家全年总劳动耗费所得的所有商品一定可以被分解为工资、利润和地租三个部分。而这三个部分又可以用劳动作为价值尺度进行衡量从而被折算为所有商品能支配的劳动。因此,从一个国家特定时期的总量上看,支配的劳动等于耗费的劳动。\cite[41-48]{YaDang*SiMiGuoFuLun2015}在马克思的劳动价值论中,价值源泉和价值尺度也是对立统一的。马克思也曾指出过两种不同含义的社会必要劳动\cite[29]{CaiJiMingBiYaoLaoDongIHeBiYaoLaoDongIIGongTongJueDingJieZhi1995}\cite[19]{GuShuTangDuiJieZhiJueDingHeJieZhiGuiLuDeZaiTanTao1982}:第一种含义的社会必要劳动时间是指“在现有的社会正常的生产条件下制造某种使用价值所需要的劳动时间”\cite[52]{ZhongGongZhongYangMaKeSiEnGeSiLieNingSiDaLinZhuZuoBianYiJuMaKeSiEnGeSiWenJiDi5Juan2009},第二种含义的社会必要劳动时间则是指“既然社会要满足需要,并为此目的而生产某种物品,它就必须为这种物品进行支付”,“所以,社会购买这些物品的方法,就是把它所能利用的劳动时间的一部分用来生产这些物品,也就是说,用该社会所能支配的劳动时间的一定量来购买这些物品”\cite[208]{ZhongGongZhongYangMaKeSiEnGeSiLieNingSiDaLinZhuZuoBianYiJuMaKeSiEnGeSiWenJiDi7Juan2009};或者说是“当时社会平均生产条件下生产市场上这种商品的社会必需总量所必要的劳动时间”\cite[722]{ZhongGongZhongYangMaKeSiEnGeSiLieNingSiDaLinZhuZuoBianYiJuMaKeSiEnGeSiWenJiDi7Juan2009}。如果我们把第一种含义的社会必要劳动时间看作在商品生产中耗费的劳动量,把第二种含义的社会必要劳动时间看作为满足社会对某种商品的需求而被商品所支配的总劳动量。那么前者代表供给,是价值源泉;后者代表需求,是价值尺度。当两者总量相等时,也就是供求平衡而均衡价格的以确定时,全社会所有商品的两种含义的社会必要劳动时间总量也必然是相等的\cite[72]{CaiJiMingLunHaoFeiDeLaoDongYuGouMaiDeLaoDongZaiJieZhiLiLunZhongDeZuoYong2022}。