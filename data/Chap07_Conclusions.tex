% !TeX root = ../2019080346_Mason.tex

\chapter{结论}

\section{研究成果}

总的来说,笔者在本文中取得了以下三个成果。

第一,重新构建了广义价值论的分析框架,用更严谨的方式重新证明了一些结论,并将广义价值论中各种生产力创造价值的机制系统地总结为一张表格。

第二,系统梳理了古典政治经济学、新古典经济学中的价值论和斯拉法价值论,纠正了许多经济学者对亚当·斯密、马克思、马歇尔等人的误解;回顾了价值概念的提出过程,详细地分析了价值形成、价值决定和价值尺度这三个概念的区别和联系。同时,笔者还在广义价值论的基础上分析了各种价值理论中所出现的生产力概念,并总结了提出各种生产力概念的必备条件。

第三,从绝对生产力、相对生产力、比较生产力和社会总和生产力对应的四个维度解析了新质生产力的概念,阐明了新质生产力和价值创造之间的联系。

\section{未来研究展望}

笔者在写作过程中也发现以下三个可以深入研究的方向:

第一,针对本文对广义价值论的一些结论的证明还不完善的问题,可以在未来的研究中进一步将广义价值论公理化、体系化、严谨化;同时,还可以尝试用计量的方法验证广义价值论的部分结论,使其更加符合当下主流的经济学研究范式。

第二,针对本文的文献谱系完整性有待提升的现状,可以在价值理论溯源研究中进行双向拓展:纵向可以延伸至16世纪重商主义学派与18世纪法国重农学派,横向拓展应涵盖德国社会主义者赫斯、德国经济学家李斯特\cite[33]{YangQiaoYuShengChanLiGaiNianCongSiMiDaoMaKeSiDeSiXiangPuXi2013}等对生产力理论有重要贡献的经济学家。

第三,针对本文在新质生产力方面的剖析不够深入的问题,笔者认为未来研究的重点是建立对新质生产力的系统性分析框架,可以剖析现有的四种维度之间的相互关系,也可以尝试针对每个维度建立计量指标以推动实证研究,还可以针对每个维度给出发展新质生产力的具体建议。
