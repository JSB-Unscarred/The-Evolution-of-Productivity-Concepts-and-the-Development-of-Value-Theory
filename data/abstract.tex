% !TeX root = ../thuthesis-example.tex

% 中英文摘要和关键字

\begin{abstract}

新质生产力作为新时代中国特色社会主义经济思想的重要创新,其价值创造机制的理论阐释亟待突破传统劳动价值论的局限。本文以广义价值论为分析框架,系统梳理了经济思想史中的价值理论及生产力概念,揭示不同价值理论下生产力参与价值创造的机制。研究指出:广义价值论通过引入比较生产力、综合生产力等概念,将比较优势原理的应用范围从国际贸易扩展到一般的商品交换,既继承古典政治经济学劳动价值论的核心,又融合新古典边际分析及斯拉法体系的生产关联性,成功解释了生产力进步推动社会价值总量增长的理论悖论。研究还提出新质生产力的四维理论内涵——从绝对生产力的维度来看,新质生产力的本质是绝对生产力;从相对生产力的维度来看,新质生产力是相对生产力重构引发的比较优势转换;从比较生产力的的维度来看,新质生产力是价值创造导向的比较生产力跃迁;从社会总和生产力的维度来看,新质生产力能够驱动社会总价值量的跃迁。本文为构建中国特色政治经济学提供了价值理论创新路径,为发展新质生产力的政策实践奠定了微观机理基础。

  % 关键词用“英文逗号”分隔,输出时会自动处理为正确的分隔符
  \thusetup{
    keywords = {新质生产力, 价值理论, 生产力概念, 古典政治经济学, 新古典经济学},
  }
\end{abstract}

\begin{abstract*}

As a significant innovation in the economic thought of socialism with Chinese characteristics in the new era, the theoretical interpretation of new-quality productive forces urgently requires transcending the limitations of traditional labor value theory. This paper employs the generalized value theory as an analytical framework to systematically examine value theories and the conceptual evolution of productive forces in the history of economic thought, revealing mechanisms through which productive forces participate in value creation across different theoretical systems. The study demonstrates that generalized value theory, by introducing concepts such as comparative productivity and comprehensive productivity, extends the application scope of comparative advantage principles from international trade to general commodity exchange. It not only inherits the core tenets of classical political economy's labor value theory but also integrates neoclassical marginal analysis and production interdependencies from the Sraffian system, thereby resolving the theoretical paradox of how productivity advances drive the growth of total social value. Furthermore, the research proposes a four-dimensional theoretical framework for new-quality productive forces: 1) From the absolute productivity perspective, it fundamentally represents a breakthrough in technological revolution-driven labor efficiency; 2) As relative productivity, it reflects comparative advantage transformation through factor synergy optimization; 3) In terms of comparative productivity, it signifies value-creation-oriented leaps in production capabilities; 4) Regarding social aggregate productivity, it embodies systemic innovation ecosystems driving quantum leaps in total social value. This paper provides innovative pathways for value theory development in constructing Chinese-style political economy and establishes micro-mechanistic foundations for policy implementation in fostering new-quality productive forces.

% Use comma as separator when inputting

\thusetup{
  keywords* = {new-quality productive forces, value theory, concept of productive forces, classical political economy, neoclassical economics},
}
\end{abstract*}
