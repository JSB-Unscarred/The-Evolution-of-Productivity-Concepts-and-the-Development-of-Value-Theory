% !TeX root = ../2019080346_Mason.tex

\chapter{古典政治经济学中的生产力概念与价值理论}

在完成了对广义价值论的介绍后,笔者将对古典政治经济学中的生产力概念和价值理论进行梳理。

\section{古典政治经济学的界定}

在开始梳理之前,我们首先对古典政治经济学的范围做出定义。

经济学界对古典政治经济学有很多种界定,马克思是最早使用“古典”一词来泛指一段时期的政治经济学理论的\cite[7]{YueHan*MeiNaDe*KaiEnSiJiuYeLiXiHeHuoBiTongLunChongYiBen2021}。他认为“古典政治经济学在英国从威廉·配第开始,到李嘉图结束,在法国从阿吉尔贝尔开始,到西斯蒙第结束”\cite[41]{ZhongGongZhongYangMaKeSiEnGeSiLieNingSiDaLinZhuZuoBianYiJuMaKeSiEnGeSiQuanJiDi13Juan1962}。除了马克思以外,还有许多经济学家也对古典政治经济学作了口径不一的划分\cite[5-8]{CaiJiMingCongGuDianZhengZhiJingJiXueDaoZhongGuoTeSeSheHuiZhuYiZhengZhiJingJiXueJiYuZhongGuoShiJiaoDeZhengZhiJingJiXueYanBianShangCe2023}。正如马克思所说:“把交换价值归结于劳动时间或相同的社会劳动,是古典政治经济学一个半世纪以上的研究得出的批判性的最后成果。”\cite[41]{ZhongGongZhongYangMaKeSiEnGeSiLieNingSiDaLinZhuZuoBianYiJuMaKeSiEnGeSiQuanJiDi13Juan1962}笔者和学界主流的观点\cite[45]{ChenDaiSunCongGuDianJingJiXuePaiDaoMaKeSiRuoGanZhuYaoXueShuoFaZhanLueLun2014}\cite[12]{CaiJiMingCongGuDianZhengZhiJingJiXueDaoZhongGuoTeSeSheHuiZhuYiZhengZhiJingJiXueJiYuZhongGuoShiJiaoDeZhengZhiJingJiXueYanBianShangCe2023}都认为古典政治经济学的基本倾向是劳动价值论。鉴于马克思本人也赞同劳动价值论,
故笔者在本文中将马克思也一并列入古典政治经济学的范畴。在参考了大量的文献后,笔者进一步从古典政治经济学家中选择亚当·斯密、大卫·李嘉图、詹姆斯·穆勒、马尔萨斯、萨伊、约翰·斯图亚特·穆勒、马克思、斯拉法作为古典政治经济学的代表,依次展开分析。

\section{亚当·斯密}
 
亚当·斯密在1776年发表的《国民财富的性质和原因的研究》中最早建立了古典政治经济学的体系\cite[120]{CaiJiMingCongGuDianZhengZhiJingJiXueDaoZhongGuoTeSeSheHuiZhuYiZhengZhiJingJiXueJiYuZhongGuoShiJiaoDeZhengZhiJingJiXueYanBianShangCe2023}\cite[90]{YanZhiJieXiFangJingJiXueShuoShiJiaoChengDiErBan2013}。接下来,笔者将首先介绍亚当·斯密笔下的价值理论,然后梳理其生产力概念。

\subsection{亚当·斯密的价值理论}

斯密是第一个明确地区分使用价值和交换价值的经济学家\cite[122]{CaiJiMingCongGuDianZhengZhiJingJiXueDaoZhongGuoTeSeSheHuiZhuYiZhengZhiJingJiXueJiYuZhongGuoShiJiaoDeZhengZhiJingJiXueYanBianShangCe2023}。他说:“价值一词有两个不同的含义。它有时表示特定物品的效用,有时又表示由于占有某物而取得的对他种货物的购买力。前者可叫做使用价值,后者可叫做交换价值。”\cite[24]{YaDang*SiMiGuoFuLun2015}应当说,这种区分是自然而严谨的,直到现在经济学界仍然在沿用这一区分。

接下来,斯密提出了自己在《国富论》中"为要探讨支配商品交换价值原则"\cite[24]{YaDang*SiMiGuoFuLun2015}而必须回答的三个问题,这三个问题中的概念构成了斯密价值理论的体系,笔者对斯密价值理论的介绍也将从这些概念入手。这三个问题分别是:

1.“什么是交换价值的真实尺度,换言之,构成一切商品真实价格的,究竟是什么?”\cite[24]{YaDang*SiMiGuoFuLun2015}

2.“构成真实价格的各部分,究竟是什么?”\cite[24]{YaDang*SiMiGuoFuLun2015}

3.“什么情况使上述价格的某些部分或全部,有时高于其自然价格或普通价格,有时又低于其自然价格或普通价格?”\cite[24]{YaDang*SiMiGuoFuLun2015}

我们可以看到,斯密要研究的,是“交换价值”“真实价格”“自然价格(普通价格)”这几个概念,我们可以据此把这三个问题转化为更清晰的表达:

1.“什么是交换价值的尺度?”

2.“真实价格是由什么构成的?”

3.“市场价格围绕自然价格波动的原因是什么?”

下面,我们依次对这三个问题进行回答和分析。

\subsubsection{交换价值的尺度}

首先,斯密认为“劳动是衡量一切商品交换价值的真实尺度”\Cite[25]{YaDang*SiMiGuoFuLun2015}。具体而言,“自分工完全确立以来,各人所需要的物品,仅有极小部分仰给于自己劳动,最大部分却须仰给于他人劳动。所以,他是贫是富,要看他能够支配多少劳动”\cite[25]{YaDang*SiMiGuoFuLun2015}“一个人占有某货物,但不愿自己消费,而愿用以交换他物,对他来说,这货物的价值,等于使他能购买或能支配的劳动量。”\cite[25]{YaDang*SiMiGuoFuLun2015}斯密又解释了选用劳动作为尺度的原因,即“只有用劳动作标准,才能在一切时代和一切地方比较各种商品的价值”\cite[31]{YaDang*SiMiGuoFuLun2015}。

但斯密又说:“劳动虽是一切商品交换价值的真实尺度,但一切商品的价值,通常不是按劳动估定的。”\cite[26]{YaDang*SiMiGuoFuLun2015}这是因为不同的劳动存在异质性,“它们的不同困难程度和精巧程度,也必须加以考虑”\cite[26]{YaDang*SiMiGuoFuLun2015}。进而,斯密认为通过市场的议价行为可以消除劳动的异质性,让不同质的劳动可以通过商品交换的方式实现相互交换\cite[26]{YaDang*SiMiGuoFuLun2015}。于是,劳动交换这一抽象的概念取得了商品交换这一具体的形式。随着货币的出现,物物交换演进为商品与货币的交换,商品取得了货币价格的形式。斯密把商品的货币价格称为“名义价格”,而商品的真实价格还是劳动\cite[28]{YaDang*SiMiGuoFuLun2015}。也就是说,斯密首先提出了衡量“交换价值”的外在尺度是劳动,接着再进一步探讨了“交换价值”的决定是在交换过程中完成的。

根据以上内容,笔者认为,斯密笔下的“交换价值”与广义价值论中的交换价值相同,即不同商品之间的交换比例。而当货币出现并被广泛使用之后,交换价值又转换为名义价格。这里的名义价格,应当就是我们可以观察到的市场价格\cite[293]{YueSeFu*XiongBiTeJingJiFenXiShiDi1Juan2017}。

\subsection{真实价格的构成}

但是,斯密没有给出“价值”的明确定义,只是说“同一真实价格的价值,往往相等”\cite[28]{YaDang*SiMiGuoFuLun2015}。有学者认为,斯密并没有从交换价值中抽象出价值的概念\cite[71]{ChenDaiSunCongGuDianJingJiXuePaiDaoMaKeSiRuoGanZhuYaoXueShuoFaZhanLueLun2014},笔者也认同这一观点。在此基础上,笔者进一步认为斯密的意思是“交换价值”在货币出现后转换为“名义价格”,而“真实价格”是“名义价格”围绕波动的中心\cite[52]{YaDang*SiMiGuoFuLun2015}。因此,斯密笔下的“真实价格”才是广义价值论中所定义的价值概念。

一般认为,斯密在第一篇第六章中提出了两种价值论\cite[97]{YanZhiJieXiFangJingJiXueShuoShiJiaoChengDiErBan2013}\cite[126]{CaiJiMingCongGuDianZhengZhiJingJiXueDaoZhongGuoTeSeSheHuiZhuYiZhengZhiJingJiXueJiYuZhongGuoShiJiaoDeZhengZhiJingJiXueYanBianShangCe2023}。首先,斯密认为“在资本累积和土地私有尚未发生以前的初期野蛮社会,获取各种物品所需要的劳动量之间的比例,似乎是各种物品相互交换的唯一标准。”\cite[41]{YaDang*SiMiGuoFuLun2015}于是我们可以说,单要素的劳动价值论是亚当斯密提出的第一种价值论。

而当“资本一经在个别人手中积聚起来”,则“劳动者对原材料增加的价值”“就分为两个部分,其中一部分支付劳动者的工资,另一部分支付雇主的利润”\cite[42]{YaDang*SiMiGuoFuLun2015}。当然,“这三个组成部分各自的真实价值,由各自所能购买或所能支配的劳动量来衡量”\cite[43-44]{YaDang*SiMiGuoFuLun2015}。由于每一期的劳动可以被下一年再次投入生产,所以“社会全部劳动年产物所能购买或支配的劳动量,远远超过这年产物生产制造乃至运输所需要的劳动量”\cite[48]{YaDang*SiMiGuoFuLun2015}。于是我们又可以说,多要素的劳动价值论是亚当斯密提出的第二种价值论,也就是说商品生产中耗费的劳动量不能在单独决定商品能够买的劳动量了,由于其它要素的参与,商品能够购买的劳动量会大于商品生产中耗费的劳动量\cite[138]{CaiJiMingCongGuDianDaoXianDaiZhengZhiJingJiXueGaiNianDeYanBianJianPingXinZhengZhiJingJiXueDeFaZhan2012}。

这里,许多经济学家认为斯密的经济思想中既有劳动价值论的观点又有生产费用论的成分,批评斯密的价值理论是自相矛盾的\cite[136]{CaiJiMingCongGuDianZhengZhiJingJiXueDaoZhongGuoTeSeSheHuiZhuYiZhengZhiJingJiXueJiYuZhongGuoShiJiaoDeZhengZhiJingJiXueYanBianShangCe2023}\cite[294]{YueSeFu*XiongBiTeJingJiFenXiShiDi1Juan2017}。但是,笔者更支持这样一种观点:斯密的价值理论是一元多要素价值论\cite[136]{CaiJiMingCongGuDianZhengZhiJingJiXueDaoZhongGuoTeSeSheHuiZhuYiZhengZhiJingJiXueJiYuZhongGuoShiJiaoDeZhengZhiJingJiXueYanBianShangCe2023},即当只有劳动一种要素稀缺时价值由劳动单一决定,是单一的劳动价值论;而当资本和土地也成为稀缺要素时劳动的价值决定会受到其它要素的影响,是多要素价值论。

接下来,我们来分析斯密的第三个问题。

\subsection{市场价格围绕自然价格波动的原因}

首先,我们有必要对“自然价格”的概念进行阐释。按照斯密的观点,“生产、制造这商品乃至运送这商品到市场所使用的按自然率支付的地租、工资和利润“构成了商品的自然价格,自然价格又“恰恰相当于其价值”\cite[49]{YaDang*SiMiGuoFuLun2015}。这里,斯密略去了对价值向生产价格的转化问题的分析\footnote{即由于竞争使剩余价值在各生产部门资本家之间按资本量平均分配,剩余价值转化为平均利润,价值转化生产价格的过程\cite[99]{ZhongXiangTanMaKeSiLaoDongJieZhiLunZhengLunJiQiShi2018}。}
,而直接得出了类似马克思笔下生产价格的“自然价格”概念。

斯密指出,商品的市场价格会受到供给和有效需求的比例支配围绕着自然价格波动,其中,有效需求指的是愿意支付商品自然价格的人的需求\cite[50]{YaDang*SiMiGuoFuLun2015}。于是,长期的商品价格会在自然价格处达到均衡\cite[50]{YaDang*SiMiGuoFuLun2015}。

至此,我们已经回答了斯密提出的三个问题,基本梳理了斯密的价值理论体系如下图所示。

接下来,笔者将梳理斯密笔下的生产力概念。

\begin{figure}
    \centering
    \caption{亚当·斯密的价值理论}
    \label{figures:AdamSmith_Value_Theory}
    \includegraphics[width=\textwidth]{figures/AdamSmith_Value_Theory.pdf}
\end{figure}

\subsection{亚当·斯密的生产力概念}

在《国富论》的开篇,斯密指出:“劳动生产力最大的增进,以及运用劳动时所表现的更大的熟练、技巧和判断力,似乎都是分工的结果。”\cite[3]{YaDang*SiMiGuoFuLun2015}这是因为通过分工——“第一,劳动者的技巧因业专而日进;第二,由一种工作转到另一种工作,通常须损失不少时间,有了分工,就可以免除这种损失;第三,许多简化劳动和缩减劳动的机械的发明,使一个人能够做许多人的工作”\cite[6]{YaDang*SiMiGuoFuLun2015}。这里,我们还可以看到斯密笔下的分工不仅是简单的通过劳动力排列组合的方面,而且包括了劳动技能提升和要素使用的方面。

如此来看,斯密笔下的“劳动生产力”对应的是广义价值论中的绝对生产力概念,纵观《国富论》全文,也找不到其它的生产力概念。

\subsection{生产力与价值的关系}

按照广义价值论的分析,绝对生产力与单位商品的价值是负相关的关系。但在斯密的价值理论中,不同的使用价值是在市场议价的过程中转换为真实价格(广义价值论中的价值)的,“一点钟艰苦程度较高的劳动的生产物,往往可交换两点钟艰苦程度较低的劳动的生产物”\cite[41]{YaDang*SiMiGuoFuLun2015},而且随着资本和土地要素的加入,商品能够购买的劳动量会大于商品生产中耗费的劳动量。那么,绝对生产力的提升一定是和单位劳动的价值量正相关。但是,绝对生产力和单位商品的价值量却是不确定的,因为在斯密的理论体系中,总价值量的提升和总产量的提升幅度是不确定的。

总而言之,我们可以把亚当·斯密理论体系中的生产力与价值决定机制总结为下表:

\begin{table}
    \caption{亚当·斯密的生产力与价值决定机制}
    \label{table:AdamSmith}
     \begin{tabularx}{\textwidth}{|c<{\centering}|c<{\centering}|X<{\centering}|}
        \toprule
        生产力类型    &具体变量    &价值决定机制 \\ \midrule

        绝对生产力    &“劳动生产力”    &与单位劳动价值正相关 \\ \bottomrule
     \end{tabularx}
\end{table}

\section{}