% !TeX root = ../2019080346_Mason.tex

\chapter{引言}

\section{研究背景及意义}

新质生产力作为中国特色社会主义经济思想的重要创新成果,其具有丰富的理论内涵与重要的实践价值。该概念最早由习近平总书记在2023年新时代推动东北全面振兴座谈会上提出\cite{YinBoGuLaoLaoBaWoDongBeiDeChongYaoShiMingFenLiPuXieDongBeiQuanMianZhenXingXinPianZhang2023}。而后,习近平总书记又在不同场合多次提到要发展新质生产力,2024年\cite[17]{LiQiangZhengFuGongZuoBaoGao2024Nian3Yue5RiZaiDiShiSiJieQuanGuoRenMinDaiBiaoDaHuiDiErCiHuiYiShang2024}、2025年\cite[19]{LiQiangZhengFuGongZuoBaoGao2025Nian3Yue5RiZaiDiShiSiJieQuanGuoRenMinDaiBiaoDaHuiDiSanCiHuiYiShang2025}政府工作报告和中共二十届三中全会\cite[8]{ZhongGuoGongChanDangDiErShiJieZhongYangWeiYuanHuiDiSanCiQuanTiHuiYiGongBao2024}也强调要发展新质生产力。在中央的重视下,新质生产力很快成为当前经济发展和经济学研究的主旋律。

根据笔者的文献调研,当前的相关文献主要通过与旧质生产力的对比来阐释“新质”,从马克思的著作中找到与生产力概念相关的原文来阐释“生产力”。在“新质”方面,从理论内涵上看,新质生产力是一种由更多生产要素的高质量组合,以关键性颠覆性技术突破为特点的一种生产力\cite[141-142]{GaoFanXinZhiShengChanLiDeTiChuLuoJiDuoWeiNeiHanJiShiDaiYiYi2023}\cite[1-2]{ZhouWenLunXinZhiShengChanLiNeiHanTeZhengYuChongYaoZhaoLiDian2023};从发展过程来看,新质生产力是一个相对于旧质生产力的概念,是对旧质生产力的更新换代。从蒸汽机到电动机再到计算机,每一种技术都代表了各个时期的新质生产力\cite[28]{CaiJiMingXinZhiShengChanLiDeFaZhanDuiJieZhiChuangZaoHeJingJiZengChangDeGongXian2024};从现实角度来看,当今世界正经历新一轮科技革命\cite[06]{XiJinPingJiaKuaiJianSheKeJiQiangGuoShiXianGaoShuiPingKeJiZiLiZiQiang2022},战略新兴产业和未来产业是新质生产力的主要载体\cite[9]{XiJinPingJingJiSiXiangYanJiuZhongXinXinZhiShengChanLiDeNeiHanTeZhengHeFaZhanChongDian2024}。在“生产力”方面,基于马克思将生产力定义为具体劳动效率的经典论述\cite[59]{ZhongGongZhongYangMaKeSiEnGeSiLieNingSiDaLinZhuZuoBianYiJuMaKeSiEnGeSiWenJiDi5Juan2009},目前的文献一致肯定了新质生产力创造使用价值的能力及其对经济发展的推动作用\cite[16]{CaiJiMingXinZhiShengChanLiCanYuJieZhiChuangZaoDeLiLunTanTaoHeShiJianYingYong2024}\cite{YangYuZhenXinZhiShengChanLiLiLunDuiMaKeSiShengChanLiShengChanGuanXiLiLunDeShouZhengHeChuangXin2025}\cite{XieFuShengMaKeSiDeShengChanLiLiLunYuFaZhanXinZhiShengChanLi2024}\cite{HuYingZaiLunXinZhiShengChanLiDeNeiHanTeZhengYuXingChengLuJingYiMaKeSiShengChanLiLiLunWeiShiJiao2024},但只有少数文献基于广义价值论分析了新质生产力参与价值创造的原理\cite{CaiJiMingXinZhiShengChanLiDeFaZhanDuiJieZhiChuangZaoHeJingJiZengChangDeGongXian2024}\cite{CaiJiMingXinZhiShengChanLiCanYuJieZhiChuangZaoDeLiLunTanTaoHeShiJianYingYong2024}。在笔者看来,这一问题的原因是绝大部分现有的文献都遵循了马克斯等量劳动创造等量价值的观点\cite[60]{ZhongGongZhongYangMaKeSiEnGeSiLieNingSiDaLinZhuZuoBianYiJuMaKeSiEnGeSiWenJiDi5Juan2009}

然而,谷书堂教授提出的“价值总量之谜”\cite[6-7]{GuShuTangQiuJieJieZhiZongLiangZhiMiLiangTiaoSiLuDeBiJiao2002}却又暗示着新质生产力的不断涌现推动了社会价值总量的上升。具体而言,国内生产总值(GDP)是一定时期内社会价值量的总和\cite[107]{ChenDeDiGuoFangJingJiDaCiDian2001}\cite[659]{BaoLuo*SaMouErSenJingJiXueDiShiJiuBan2012},这意味着在投入劳动总量基本不变的情况下,以不变价格衡量的GDP(实际GDP)也会保持基本不变,而这显然与我国改革开放以来的实践事实相矛盾。目前来看,唯有将马克思“劳动生产力与价值量成正比”原理的适用范围从单个生产者推广到部门和全社会的广义价值论,能比较合理、严谨地解释这一谜题\cite{CaiJiMingJiShuJinBuJingJiZengChangYuJieZhiZongLiangZhiMiJiYuGuangYiJieZhiLunDeJieShi2019},因此,笔者希望用广义价值论对经济思想史上的”生产力”作一次系统的梳理,以期为新质生产力研究中的“生产力”概念作一个新的阐释。

在此基础上,笔者又发现现有的文献并没有对广义价值论中的各种生产力概念与各种价值理论中的生产力概念进行比较分析,而价值理论又是经济学体系的理论基础\cite[118]{CaiJiMingCongGuDianZhengZhiJingJiXueDaoZhongGuoTeSeSheHuiZhuYiZhengZhiJingJiXueJiYuZhongGuoShiJiaoDeZhengZhiJingJiXueYanBianShangCe2023}。鉴于建立中国特色社会主义政治经济学的目标要求我们积极汲取西方经济学的合理成分\cite[81]{ChengEnFuChongJianZhongGuoJingJiXueChaoYueMaKeSiYuXiFangJingJiXue2000},笔者打算着眼于生产力参与价值创造的原理,从广义价值论出发,系统梳理各种生产力概念,并将其与不同价值理论中的生产力概念进行对比分析。这一方面有助于对生产力理论,特别是新质生产力理论的进一步研究,以更好地指导发展新质生产力的实践;另一方面也有助于广义价值论的进一步发展,为中国特色社会主义政治经济学大厦的构建添砖加瓦。


\section{研究思路}

本文从新质生产力对价值决定的现实作用与传统劳动价值论的矛盾出发,首先介绍价值理论的最新成果——广义价值论的基本原理,而后通过广义价值论框架系统地对经济思想史进行梳理。本文首先回顾了古典政治经济学,在完成了对代表性经济学家的价值理论和生产力概念的描述后,笔者总结了古典政治经济学的基本倾向是把劳动作为衡量交换价值的尺度,并揭示了经济学对价值的认识发展的过程,最后揭示了作为价值源泉的各种生产要素与作为价值尺度的支配劳动之间的对立统一。继而,本文又指出新古典经济学偏重需求侧分析而忽视供给侧系统性关联,斯拉法体系虽隐含更多的生产力概念却未明确生产力与价值量的动态关系。在此批判性梳理基础上,本文根据广义价值论价值论中的绝对生产力、相对生产力、比较生产力和社会总和生产力提出了对新质生产力的四维理论内涵:在绝对生产力维度表现为技术革命驱动的效率跃升;在相对生产力维度重塑动态比较优势;在比较生产力维度重构价值创造机制;在社会总和生产力维度实现社会总价值量的跃升。

\section{文章结构}
第一章 \quad 引言

第二章 \quad 广义价值论的基本原理

该部分主要介绍广义价值论的基本原理并系统整理各种生产力的概念及其与价值创造的联系。

第三章 \quad 古典政治经济学中的生产力概念与价值理论

该部分主要对比分析亚当斯密(重点)、李嘉图(重点)、马尔萨斯、约翰穆勒的价值理论及生产力概念。

第四章 \quad 新古典经济学中的生产力概念与价值理论

该部分主要分析两种边际价值论和马歇尔的均衡价值论及各自理论中的生产力概念。

第五章 \quad 斯拉法价值论

该部分主要介绍斯拉法的价值理论,并浅析了其理论中的生产力概念。

第六章 \quad 价值理论演进与新质生产力理论初探

该部分对价值理论和生产力概念进行全面回顾,并试着对新质生产力的多维内涵进行探讨。

第七章 \quad 结论

该部分总结全文的研究成果,并针对本文的局限展望未来研究方向。

