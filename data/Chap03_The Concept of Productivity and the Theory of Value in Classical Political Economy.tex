% !TeX root = ../2019080346_Mason.tex

\chapter{古典政治经济学中的生产力概念与价值理论}

在完成了对广义价值论的介绍后,笔者将对古典政治经济学中的生产力概念和价值理论进行梳理。

\section{古典政治经济学的界定}

在开始梳理之前,我们首先对古典政治经济学的范围做出定义。

经济学界对古典政治经济学有很多种界定,马克思是最早使用“古典”一词来泛指一段时期的政治经济学理论的\cite[7]{YueHan*MeiNaDe*KaiEnSiJiuYeLiXiHeHuoBiTongLunChongYiBen2021}。他认为“古典政治经济学在英国从威廉·配第开始,到李嘉图结束,在法国从阿吉尔贝尔开始,到西斯蒙第结束”\cite[56]{QiaEr*MaKeSiZhengZhiJingJiXuePiPanYingWen2022}。除了马克思以外,还有许多经济学家也对古典政治经济学作了口径不一的划分\cite[5-8]{CaiJiMingCongGuDianZhengZhiJingJiXueDaoZhongGuoTeSeSheHuiZhuYiZhengZhiJingJiXueJiYuZhongGuoShiJiaoDeZhengZhiJingJiXueYanBianShangCe2023}。正如马克思所说:“把交换价值归结于劳动时间或相同的社会劳动,是古典政治经济学一个半世纪以上的研究得出的批判性的最后成果。”\cite[41]{ZhongGongZhongYangMaKeSiEnGeSiLieNingSiDaLinZhuZuoBianYiJuMaKeSiEnGeSiQuanJiDi13Juan1962}笔者和学界主流的观点\cite[45]{ChenDaiSunCongGuDianJingJiXuePaiDaoMaKeSiRuoGanZhuYaoXueShuoFaZhanLueLun2014}\cite[12]{CaiJiMingCongGuDianZhengZhiJingJiXueDaoZhongGuoTeSeSheHuiZhuYiZhengZhiJingJiXueJiYuZhongGuoShiJiaoDeZhengZhiJingJiXueYanBianShangCe2023}都认为古典政治经济学的基本倾向是劳动价值论。鉴于马克思本人也赞同劳动价值论,
故笔者在本文中将马克思也一并列入古典政治经济学的范畴。在参考了大量的文献后,笔者进一步从古典政治经济学家中选择亚当·斯密、大卫·李嘉图、詹姆斯·穆勒、马尔萨斯、萨伊、约翰·斯图亚特·穆勒、马克思、斯拉法作为古典政治经济学的代表,依次展开分析。

\section{亚当·斯密}
 
亚当·斯密在1776年发表的《国民财富的性质和原因的研究》中最早建立了古典政治经济学的体系\cite[120]{CaiJiMingCongGuDianZhengZhiJingJiXueDaoZhongGuoTeSeSheHuiZhuYiZhengZhiJingJiXueJiYuZhongGuoShiJiaoDeZhengZhiJingJiXueYanBianShangCe2023}\cite[90]{YanZhiJieXiFangJingJiXueShuoShiJiaoChengDiErBan2013}。接下来,笔者将首先介绍亚当·斯密笔下的价值理论,然后梳理其生产力概念。

\subsection{亚当·斯密的价值理论}

斯密是第一个明确地区分使用价值和交换价值的经济学家\cite[122]{CaiJiMingCongGuDianZhengZhiJingJiXueDaoZhongGuoTeSeSheHuiZhuYiZhengZhiJingJiXueJiYuZhongGuoShiJiaoDeZhengZhiJingJiXueYanBianShangCe2023}。他说:“价值一词有两个不同的含义。它有时表示特定物品的效用,有时又表示由于占有某物而取得的对他种货物的购买力。前者可叫做使用价值,后者可叫做交换价值。”\cite[24]{YaDang*SiMiGuoFuLun2015}应当说,这种区分是自然而严谨的,直到现在经济学界仍然在沿用这一区分。

接下来,斯密提出了自己在《国富论》中"为要探讨支配商品交换价值原则"\cite[24]{YaDang*SiMiGuoFuLun2015}而必须回答的三个问题,这三个问题中的概念构成了斯密价值理论的体系,笔者对斯密价值理论的介绍也将从这些概念入手。这三个问题分别是:

1.“什么是交换价值的真实尺度,换言之,构成一切商品真实价格的,究竟是什么?”\cite[24]{YaDang*SiMiGuoFuLun2015}

2.“构成真实价格的各部分,究竟是什么?”\cite[24]{YaDang*SiMiGuoFuLun2015}

3.“什么情况使上述价格的某些部分或全部,有时高于其自然价格或普通价格,有时又低于其自然价格或普通价格?”\cite[24]{YaDang*SiMiGuoFuLun2015}

我们可以看到,斯密要研究的,是“交换价值”“真实价格”“自然价格(普通价格)”这几个概念,我们可以据此把这三个问题转化为更清晰的表达:

1.“什么是交换价值的尺度?”

2.“真实价格是由什么构成的?”

3.“市场价格围绕自然价格波动的原因是什么?”

下面,我们依次对这三个问题进行回答和分析。

\subsubsection{交换价值的尺度}

首先,斯密认为“劳动是衡量一切商品交换价值的真实尺度”\Cite[25]{YaDang*SiMiGuoFuLun2015}。具体而言,“自分工完全确立以来,各人所需要的物品,仅有极小部分仰给于自己劳动,最大部分却须仰给于他人劳动。所以,他是贫是富,要看他能够支配多少劳动”\cite[25]{YaDang*SiMiGuoFuLun2015}“一个人占有某货物,但不愿自己消费,而愿用以交换他物,对他来说,这货物的价值,等于使他能购买或能支配的劳动量。”\cite[25]{YaDang*SiMiGuoFuLun2015}斯密又解释了选用劳动作为尺度的原因,即“只有用劳动作标准,才能在一切时代和一切地方比较各种商品的价值”\cite[31]{YaDang*SiMiGuoFuLun2015}。这里,笔者想强调斯密的意思是:交换价值的尺度不仅是劳动,而且是商品可以购买或支配的劳动。

但斯密又说:“劳动虽是一切商品交换价值的真实尺度,但一切商品的价值,通常不是按劳动估定的。”\cite[26]{YaDang*SiMiGuoFuLun2015}这是因为不同的劳动存在异质性,“它们的不同困难程度和精巧程度,也必须加以考虑”\cite[26]{YaDang*SiMiGuoFuLun2015}。进而,斯密认为通过市场的议价行为可以“消除”劳动的异质性,让不同质的劳动可以通过商品交换的方式实现相互交换,于是,劳动交换这一抽象的概念取得了商品交换这一具体的形式\cite[26]{YaDang*SiMiGuoFuLun2015}。随着货币的出现,物物交换演进为商品与货币的交换,商品取得了货币价格的形式。这种以货币价格为尺度的价格被斯密称作是“名义价格”,而商品的真实价格还是以购买的劳动为尺度的\cite[28]{YaDang*SiMiGuoFuLun2015}。

根据以上内容,笔者认为,斯密笔下的“交换价值”与广义价值论中的交换价值相同,即不同商品之间的交换比例。而当货币出现并被广泛使用之后,交换价值又转换为名义价格。这里的名义价格,应当就是我们可以观察到的市场价格\cite[293]{YueSeFu*XiongBiTeJingJiFenXiShiDi1Juan2017}。

\subsection{真实价格的构成}

但是,斯密没有给出“价值”的明确定义,只是说“同一真实价格的价值,往往相等”\cite[28]{YaDang*SiMiGuoFuLun2015}。有学者认为,斯密并没有从交换价值中抽象出价值的概念\cite[71]{ChenDaiSunCongGuDianJingJiXuePaiDaoMaKeSiRuoGanZhuYaoXueShuoFaZhanLueLun2014},笔者也认同这一观点。在此基础上,笔者进一步认为斯密的意思是“交换价值”在货币出现后转换为“名义价格”,而“真实价格”是“名义价格”围绕波动的中心\cite[52]{YaDang*SiMiGuoFuLun2015}。因此,斯密笔下的“真实价格”才符合广义价值论中对价值的定义。

一般认为,斯密在第一篇第六章中提出了两种价值论\cite[97]{YanZhiJieXiFangJingJiXueShuoShiJiaoChengDiErBan2013}\cite[126]{CaiJiMingCongGuDianZhengZhiJingJiXueDaoZhongGuoTeSeSheHuiZhuYiZhengZhiJingJiXueJiYuZhongGuoShiJiaoDeZhengZhiJingJiXueYanBianShangCe2023}。首先,斯密认为“在资本累积和土地私有尚未发生以前的初期野蛮社会,获取各种物品所需要的劳动量之间的比例,似乎是各种物品相互交换的唯一标准。”\cite[41]{YaDang*SiMiGuoFuLun2015}于是我们可以说,单要素的劳动价值论是亚当斯密提出的第一种价值论。

而当“资本一经在个别人手中积聚起来”,则“劳动者对原材料增加的价值”“就分为两个部分,其中一部分支付劳动者的工资,另一部分支付雇主的利润”\cite[42]{YaDang*SiMiGuoFuLun2015}。当然,“这三个组成部分各自的真实价值,由各自所能购买或所能支配的劳动量来衡量”\cite[43-44]{YaDang*SiMiGuoFuLun2015}。由于每一期的劳动可以被下一年再次投入生产,所以“社会全部劳动年产物所能购买或支配的劳动量,远远超过这年产物生产制造乃至运输所需要的劳动量”\cite[48]{YaDang*SiMiGuoFuLun2015}。于是我们又可以说,多要素的劳动价值论是亚当斯密提出的第二种价值论,也就是说耗费的劳动量不能再单独决定能够买的劳动量了,由于其它要素的参与,单位消耗的劳动量可以购买更多的劳动量\cite[138]{CaiJiMingCongGuDianDaoXianDaiZhengZhiJingJiXueGaiNianDeYanBianJianPingXinZhengZhiJingJiXueDeFaZhan2012}。这种不等关系对应着广义价值论中两个生产者的单位耗费劳动因综合生产力的相对大小不同而能购买不等量劳动的结论\cite[294]{CaiJiMingCongGuDianZhengZhiJingJiXueDaoZhongGuoTeSeSheHuiZhuYiZhengZhiJingJiXueJiYuZhongGuoShiJiaoDeZhengZhiJingJiXueYanBianShangCe2023}。

\subsection{对以上内容的两方面争议}

在此基础上,有许多经济学家对斯密的观点提出了批评,这些批评主要集中在两个方面:价值决定和价值尺度。

从价值决定的角度看,许多经济学家认为斯密的经济思想中既有劳动价值论的观点又有生产费用论的成分,批评斯密的价值理论是自相矛盾的\cite[136]{CaiJiMingCongGuDianZhengZhiJingJiXueDaoZhongGuoTeSeSheHuiZhuYiZhengZhiJingJiXueJiYuZhongGuoShiJiaoDeZhengZhiJingJiXueYanBianShangCe2023}\cite[294]{YueSeFu*XiongBiTeJingJiFenXiShiDi1Juan2017}。但如前文所述,笔者更支持这样一种观点:斯密的价值理论是一元多要素价值论\cite[136]{CaiJiMingCongGuDianZhengZhiJingJiXueDaoZhongGuoTeSeSheHuiZhuYiZhengZhiJingJiXueJiYuZhongGuoShiJiaoDeZhengZhiJingJiXueYanBianShangCe2023},即当只有劳动一种要素稀缺时价值由劳动单一决定,是单要素劳动价值论;而当资本和土地也成为稀缺要素时劳动的价值决定会受到其它要素的影响,是多要素劳动价值论。

从价值尺度的角度看,李嘉图批评斯密同时提出了耗费劳动尺度说和购买劳动尺度说\cite[7]{DaWei*LiJiaTuZhengZhiJingJiXueJiFuShuiYuanLi2021}。但如前文所述,斯密始终是把能购买的劳动作为尺度的\cite[142]{CaiJiMingCongGuDianZhengZhiJingJiXueDaoZhongGuoTeSeSheHuiZhuYiZhengZhiJingJiXueJiYuZhongGuoShiJiaoDeZhengZhiJingJiXueYanBianShangCe2023}。斯密是把价值尺度和价值决定分开考察的\cite[73]{ChenDaiSunCongGuDianJingJiXuePaiDaoMaKeSiRuoGanZhuYaoXueShuoFaZhanLueLun2014},他首先提出了衡量“交换价值”的外在尺度是可以购买的劳动,接着再进一步提出在土地尚未私有、资本尚未积累时,决定“真实价格”的是商品生产耗费的劳动。这时,“真实价格”的尺度和耗费的劳动是统一的。而在土地私有、资本累积的发达社会,包括直接耗费的劳动在内的多种要素共同决定了“真实价格”。这时,“真实价格”的尺度和耗费的劳动便不再统一,但参与商品生产的总劳动量(包括直接和间接)仍然等于商品能购买的总劳动量。

另外,

接下来,我们来分析斯密的第三个问题。

\subsection{市场价格围绕自然价格波动的原因}

首先,我们有必要对“自然价格”的概念进行阐释。按照斯密的观点,“生产、制造这商品乃至运送这商品到市场所使用的按自然率支付的地租、工资和利润“构成了商品的自然价格,自然价格又“恰恰相当于其价值”\cite[49]{YaDang*SiMiGuoFuLun2015}。这里,斯密略去了对价值向生产价格的转化问题的分析\footnote{即由于竞争使剩余价值在各生产部门资本家之间按资本量平均分配,剩余价值转化为平均利润,价值转化生产价格的过程\cite{XieFuShengXiFangXueZheGuanYuMaKeSiJieZhiZhuanXingLiLunYanJiuShuPing2000}。}
,而直接得出了类似马克思笔下生产价格的“自然价格”概念。

斯密指出,商品的市场价格会受到供给和有效需求的比例支配围绕着自然价格波动,其中,有效需求指的是愿意支付商品自然价格的人的需求\cite[50]{YaDang*SiMiGuoFuLun2015}。于是,长期的商品价格会在自然价格处达到均衡\cite[50]{YaDang*SiMiGuoFuLun2015}。另外,斯密还指出商品的市场价格会受到货币本身价值变动的影响\cite[28-31]{YaDang*SiMiGuoFuLun2015}。

至此,我们已经回答了斯密提出的三个问题,基本梳理了斯密的价值理论体系如下图所示。

接下来,笔者将梳理斯密笔下的生产力概念。

\begin{figure}
    \centering
    \caption{亚当·斯密的价值理论}
    \label{figures:AdamSmith_Value_Theory}
    \includegraphics[width=\textwidth]{figures/AdamSmith_Value_Theory.pdf}
\end{figure}

\subsection{亚当·斯密的生产力概念}

在《国富论》的开篇,斯密指出:“劳动生产力最大的增进,以及运用劳动时所表现的更大的熟练、技巧和判断力,似乎都是分工的结果。”\cite[3]{YaDang*SiMiGuoFuLun2015}这是因为通过分工——“第一,劳动者的技巧因业专而日进;第二,由一种工作转到另一种工作,通常须损失不少时间,有了分工,就可以免除这种损失;第三,许多简化劳动和缩减劳动的机械的发明,使一个人能够做许多人的工作”\cite[6]{YaDang*SiMiGuoFuLun2015}。这里,我们还可以看到斯密笔下的分工不仅是简单的通过劳动力排列组合的方面,而且包括了劳动技能提升和要素使用的方面。

如此来看,斯密笔下的“劳动生产力”对应的是广义价值论中的绝对生产力概念。纵观《国富论》全文,我们也找不到其它的生产力概念。

\subsection{生产力与价值的关系}

按照广义价值论的分析,绝对生产力与单位商品的价值是负相关的关系。虽然斯密认为资本和土地要素的投入会影响商品的价值量,但这些投入都可以被视为是物化劳动的投入,所以绝对生产力与单位商品的价值仍然是负相关的关系。同样,我们也不难推出绝对生产力和单位劳动的价值量正相关的结论。

总而言之,我们可以把亚当·斯密理论体系中的生产力与价值决定机制总结为下表:

\begin{table}
    \caption{亚当·斯密的生产力与价值决定机制}
    \label{table:AdamSmith}
     \begin{tabularx}{\textwidth}{|c<{\centering}|c<{\centering}|X<{\centering}|}
        \toprule
        生产力类型    &具体变量    &价值决定机制 \\ \midrule

        绝对生产力    &“劳动生产力”    &与单位劳动价值正相关 \\ \bottomrule
     \end{tabularx}
\end{table}

\section{大卫·李嘉图}

大卫·李嘉图是英国产业革命时代的经济学家,继承和发展了斯密价值理论中的单一劳动价值论的部分,对价值决定于劳动时间的原理作了比较透彻的表述与发展\cite[iv]{DaWei*LiJiaTuZhengZhiJingJiXueJiFuShuiYuanLi2021}。在这一部分,笔者将探讨李嘉图理论体系中的生产力概念和价值理论。

\subsection{大卫·李嘉图的价值理论}

李嘉图在《政治经济学及赋税原理》第一章中对他的价值理论做了比斯密更为严谨和系统的介绍,但他同样没有区分开交换价值和价值。

李嘉图首先继承了斯密对使用价值和价值的区分,然后指出交换价值有两个来源——一个是商品的稀有性,另一个是获取商品时所必需的劳动量\cite[5-6]{DaWei*LiJiaTuZhengZhiJingJiXueJiFuShuiYuanLi2021}。但由于那些单由稀有性决定价值的商品在市场上只占极小一部分,而那些“数量可以由人类劳动增加、生产可以不受限制地进行竞争地商品”占了绝大多数,所以李嘉图将分析的重点放在了后者上,并指出“决定这一商品交换另一商品时所应付出的数量的尺度,几乎完全取决于各商品上所费的相对劳动量”\cite[6]{DaWei*LiJiaTuZhengZhiJingJiXueJiFuShuiYuanLi2021}。

李嘉图同样探讨了劳动的异质性。李嘉图指出,他关注的只是商品相对价值的变动,而商品的相对价值会在市场交换中形成且“估价的尺度一经形成就很少发生变动”,所以“比较同一商品在不同时期的价值时,我们无需考虑这种商品所需劳动的相对熟练程度和强度”,即市场“消除”了劳动的异质性\cite[13-14]{DaWei*LiJiaTuZhengZhiJingJiXueJiFuShuiYuanLi2021}。

随后,李嘉图又指出影响商品交换价值的“不仅是指投在商品的直接生产过程中的劳动,而且也包括投在实现该种劳动所需要的一切器具或机器上的劳动”\cite[17]{DaWei*LiJiaTuZhengZhiJingJiXueJiFuShuiYuanLi2021},而且“劳动使用的节约必然会使商品的相对价值下降,无论这种节约是发生在制造这种商品本身所需的劳动方面,还是发生在构造协助生产这种商品的资本所需的劳动方面。”\cite[18]{DaWei*LiJiaTuZhengZhiJingJiXueJiFuShuiYuanLi2021}

据此,我们不难总结出李嘉图认为商品的价值是由耗费的劳动决定的。

\subsection{大卫·李嘉图的生产力概念}

李嘉图的著作中没有出现明确的生产力概念,我们只能从他的文字中总结出他对生产力的看法。例如,他在分析黄金的价值变动时说:如果“由于发现了更丰饶的新矿山,或是由于更有利地使用机器,用较少的劳动量就可以获得一定量的黄金,那么,我就有理由说,黄金相对于其他商品的价值发生变动的原因,是它的生产已经比较便利,或获得时所必需的劳动量已经减少”\cite[11]{DaWei*LiJiaTuZhengZhiJingJiXueJiFuShuiYuanLi2021}。李嘉图在别处分析生产力的改进时也是指商品生产时劳动量的减少,所以笔者认为李嘉图的理论中也只有绝对生产力的概念。

由于李嘉图也没有区分部门、个体的绝对生产力变动带来的影响,笔者在此先不探讨生产力与价值的关系,而是先介绍与李嘉图价值理论相似的马克思的价值理论,并在马克思的理论基础上对生产力与价值的关系做出进一步分析。

\section{卡尔·马克思}

一般认为,马克思继承和发展了李嘉图的劳动价值论\cite[347]{YueSeFu*XiongBiTeJingJiFenXiShiDi2Juan2017}\cite[84]{ChenDaiSunCongGuDianJingJiXuePaiDaoMaKeSiRuoGanZhuYaoXueShuoFaZhanLueLun2014}。

\subsection{马克思的价值理论}

从价值理论来看,马克思首先区分了交换价值和价值:价值是交换价值的基础,交换价值是价值的表现形态\cite[86-88]{ChenDaiSunCongGuDianJingJiXuePaiDaoMaKeSiRuoGanZhuYaoXueShuoFaZhanLueLun2014}。根本上来说,李嘉图之所以不能做出这种区分,是因为李嘉图把劳动价值论“仅仅是作为一种假设”,“用来说明相对价格(能观察到的市场价格)的实际长期正常状态”\cite[348]{YueSeFu*XiongBiTeJingJiFenXiShiDi2Juan2017},于是李嘉图遇到了“等量劳动创造等量价值和等量资本获得等量利润的矛盾”\cite[144]{CaiJiMingCongGuDianZhengZhiJingJiXueDaoZhongGuoTeSeSheHuiZhuYiZhengZhiJingJiXueJiYuZhongGuoShiJiaoDeZhengZhiJingJiXueYanBianShangCe2023}\cite[21-28]{DaWei*LiJiaTuZhengZhiJingJiXueJiFuShuiYuanLi2021};而马克思则是把劳动看成价值的实质——价值就是凝结的劳动本身,于是马克思遇到了价值向生产价格的转形问题\cite[348-350]{YueSeFu*XiongBiTeJingJiFenXiShiDi2Juan2017}\cite[159]{CaiJiMingCongGuDianZhengZhiJingJiXueDaoZhongGuoTeSeSheHuiZhuYiZhengZhiJingJiXueJiYuZhongGuoShiJiaoDeZhengZhiJingJiXueYanBianShangCe2023}。

再次,马克思区分了劳动与劳动力:工人出卖的是劳动力而不是劳动,劳动力的使用(劳动)创造的超过劳动力价值的价值(剩余价值)被资本家无偿占有,解决了“资本与劳动力交换违反劳动价值论”的矛盾\cite[615,581-606]{ZhongGongZhongYangMaKeSiEnGeSiLieNingSiDaLinZhuZuoBianYiJuMaKeSiEnGeSiWenJiDi5Juan2009}\cite[157-158]{CaiJiMingCongGuDianZhengZhiJingJiXueDaoZhongGuoTeSeSheHuiZhuYiZhengZhiJingJiXueJiYuZhongGuoShiJiaoDeZhengZhiJingJiXueYanBianShangCe2023}\cite[348]{YueSeFu*XiongBiTeJingJiFenXiShiDi2Juan2017}。

总的来说,马克思完善了劳动价值论,其价值理论可以简单地表述为:商品的价值由生产商品的社会必要劳动时间决定\cite[51-52]{ZhongGongZhongYangMaKeSiEnGeSiLieNingSiDaLinZhuZuoBianYiJuMaKeSiEnGeSiWenJiDi5Juan2009}。

\subsection{马克思的生产力理论}

马克思虽然对生产力有不同的提法,但各种提法的本质是清晰且一贯的\cite{YangQiaoYuShengChanLiGaiNianCongSiMiDaoMaKeSiDeSiXiangPuXi2013}\cite{DingXiaoPingZhengQueLiJieMaKeSiZhuYiDeShengChanLiGaiNian2021}——“生产力当然始终是有用的、具体的劳动的生产力,它事实上只决定有目的的生产活动在一定时间内的效率”\cite[59]{ZhongGongZhongYangMaKeSiEnGeSiLieNingSiDaLinZhuZuoBianYiJuMaKeSiEnGeSiWenJiDi5Juan2009}——这和广义价值论中绝生产力的定义是相符的。

\subsection{生产力与价值的关系}

在此基础上,马克思根据绝对生产力的描述对象不同,针对生产力与价值的关系给出了三个看似相互矛盾实则内在统一的命题\cite[273]{CaiJiMingCongGuDianZhengZhiJingJiXueDaoZhongGuoTeSeSheHuiZhuYiZhengZhiJingJiXueJiYuZhongGuoShiJiaoDeZhengZhiJingJiXueYanBianShangCe2023}。

第一,劳动生产力与加质量负相关。马克思指出,“劳动生产力越高,生产一种物品所需要的劳动时间就越少,凝结在该物品中的劳动量就越小,该物品的价值就越小”\cite[53]{ZhongGongZhongYangMaKeSiEnGeSiLieNingSiDaLinZhuZuoBianYiJuMaKeSiEnGeSiWenJiDi5Juan2009}。这里的“劳动生产力”应当是指广义价值论中的部门绝对生产力,而这里的“价值量”应当是指单位商品的价值量\cite[273]{CaiJiMingCongGuDianZhengZhiJingJiXueDaoZhongGuoTeSeSheHuiZhuYiZhengZhiJingJiXueJiYuZhongGuoShiJiaoDeZhengZhiJingJiXueYanBianShangCe2023}。该命题与广义价值论的判断是一致的。但马克思还说单位商品的价值量和劳动生产力成反比\cite[53-54]{ZhongGongZhongYangMaKeSiEnGeSiLieNingSiDaLinZhuZuoBianYiJuMaKeSiEnGeSiWenJiDi5Juan2009},这就与广义价值论的判断出现矛盾。在广义价值论中,部门绝对生产力的提高会影响部门综合生产力,进而影响部门比较生产力系数,使得单位商品价值量的降低幅度小于部门绝对生产力的提高幅度,于是部门绝对生产力与单位商品价值量仅仅是负相关而不成反比\cite[274, 282]{CaiJiMingCongGuDianZhengZhiJingJiXueDaoZhongGuoTeSeSheHuiZhuYiZhengZhiJingJiXueJiYuZhongGuoShiJiaoDeZhengZhiJingJiXueYanBianShangCe2023}。出现这种矛盾的根本原因,是马克思没有考虑部门综合生产能力对价值决定的影响。

第二,劳动生产力与价值量正相关。马克思指出:“生产力特别高的劳动起了自乘的劳动的作用,或者说,在同样的时间内,它所创造的价值比同种社会平均劳动要多。”\cite[370]{ZhongGongZhongYangMaKeSiEnGeSiLieNingSiDaLinZhuZuoBianYiJuMaKeSiEnGeSiWenJiDi5Juan2009}这里的“劳动生产力”应当是指广义价值论中的个别绝对生产力,而这里的“价值量”应当是指单个生产者在单位劳动时间内所创造的价值总量\cite[273]{CaiJiMingCongGuDianZhengZhiJingJiXueDaoZhongGuoTeSeSheHuiZhuYiZhengZhiJingJiXueJiYuZhongGuoShiJiaoDeZhengZhiJingJiXueYanBianShangCe2023}。该命题与广义价值论的判断也是一致的。

第三,劳动生产力与价值量不相关。马克思指出,“不管生产力发生了什么变化,同一劳动在同样的时间内提供的价值量总是相同的”\cite[60]{ZhongGongZhongYangMaKeSiEnGeSiLieNingSiDaLinZhuZuoBianYiJuMaKeSiEnGeSiWenJiDi5Juan2009}。这里的“劳动生产力”应当是指部门绝对生产力,而这里的“价值量”应当是指部门商品价值总量\cite[274]{CaiJiMingCongGuDianZhengZhiJingJiXueDaoZhongGuoTeSeSheHuiZhuYiZhengZhiJingJiXueJiYuZhongGuoShiJiaoDeZhengZhiJingJiXueYanBianShangCe2023}。该命题与广义价值论的判断出现了矛盾。出现这种矛盾的根本原因,是马克思否认了非劳动要素对价值决定的作用\cite[274]{CaiJiMingCongGuDianZhengZhiJingJiXueDaoZhongGuoTeSeSheHuiZhuYiZhengZhiJingJiXueJiYuZhongGuoShiJiaoDeZhengZhiJingJiXueYanBianShangCe2023}。

总的来说,马克思并没有提出绝对生产力以外的概念,并且不承认非劳动要素对价值决定的作用,但他区分了个体和部门的绝对生产力变化对价值的不同影响,将单一劳动价值论发展到了极致。

\section{马尔萨斯和萨伊}

至此,我们完成了对李嘉图和马克思的介绍。不难看出,他们的经济思想主要以继承和发展斯密的单要素劳动价值论为主。接下来,笔者将介绍继承和发展了斯密多要素价值论的马尔萨斯、萨伊的价值理论并分析其生产力概念。笔者之所以把这两位经济学家放在一起,一方面是因为两人生活在同一个时代\cite[132,140]{YanZhiJieXiFangJingJiXueShuoShiJiaoChengDiErBan2013},另一方面是因为一般认为他们都按照各自的理解继承了斯密的多要素价值论\cite[169]{CaiJiMingCongGuDianZhengZhiJingJiXueDaoZhongGuoTeSeSheHuiZhuYiZhengZhiJingJiXueJiYuZhongGuoShiJiaoDeZhengZhiJingJiXueYanBianShangCe2023}。

\subsection{马尔萨斯的价值理论}

马尔萨斯在区分使用价值和交换价值的基础上对交换价值作了进一步细分,他认为“价值”一词可以被分解为三个方面:第一是使用价值,即物品的效用;第二是名义交换价值,即以货币表示的价值(如果还没有出现货币,则商品的交换价值通过其它任何一种商品表现出来\cite[32]{BiLuo*SiLaFaDaWeiLiJiaTuQuanJiDi2JuanMaErSaSiZhengZhiJingJiXueYuanLiPingZhu2013});第三是实际交换价值,即必需品、享用品和劳动的价值\cite[42]{BiLuo*SiLaFaDaWeiLiJiaTuQuanJiDi2JuanMaErSaSiZhengZhiJingJiXueYuanLiPingZhu2013}。马尔萨斯重点研究的是“实际交换价值”。

然而,值得指出的是,马尔萨斯并没有把价值从价格中抽象出来,他认为“任何时间与地点的商品的自然价值\footnote{笔者认为这里指的就是“实际交换价值”}”是“商品处于自然或一般状态下由原始生产成本或供应条件决定的估价”\cite[132]{MaErSaSiZhengZhiJingJiXueDingYi2023}。换句话说,马尔萨斯的“价值”或者“实际交换价值”在广义价值论中对应的是交换价值的概念而非价值概念,马尔萨斯并没有直接揭开价值的面纱。


马尔萨斯虽然认为“实际交换价值”就是“估价”,但他意识到用贵金属或是其它商品来衡量这一“估价”是很困难的\cite[98]{BiLuo*SiLaFaDaWeiLiJiaTuQuanJiDi2JuanMaErSaSiZhengZhiJingJiXueYuanLiPingZhu2013},
所以他认为应当用商品能支配的劳动而非生产耗费的劳动作为“实际交换价值”的尺度\footnote{事实上马尔萨斯认为用单一的商品能支配的劳动作为价值尺度都是不够精准的,马尔萨斯认为在某些情况下,由谷物和劳动两种尺度组合成的新的尺度优于任何一个单独的尺度\cite[98-105]{BiLuo*SiLaFaDaWeiLiJiaTuQuanJiDi2JuanMaErSaSiZhengZhiJingJiXueYuanLiPingZhu2013}。}\cite[60-82, 92-97]{BiLuo*SiLaFaDaWeiLiJiaTuQuanJiDi2JuanMaErSaSiZhengZhiJingJiXueYuanLiPingZhu2013}\cite[133]{MaErSaSiZhengZhiJingJiXueDingYi2023}。这是因为:第一,“用价值中的绝大部分来进行交换的是,生产性或非生产性的劳动”\cite[92]{BiLuo*SiLaFaDaWeiLiJiaTuQuanJiDi2JuanMaErSaSiZhengZhiJingJiXueYuanLiPingZhu2013};第二,“只有与劳动交换的商品的价值,能够表达商品对社会需要和爱好的配合程度,能够表达同消费者的愿望与人数对照下,商品供给的丰裕程度。”\cite[92]{BiLuo*SiLaFaDaWeiLiJiaTuQuanJiDi2JuanMaErSaSiZhengZhiJingJiXueYuanLiPingZhu2013};第三,“资本的积累,以及其增加财富和人口的效能...取决于其换取劳动的力量”\cite[93]{BiLuo*SiLaFaDaWeiLiJiaTuQuanJiDi2JuanMaErSaSiZhengZhiJingJiXueYuanLiPingZhu2013}。最后,用商品生产耗费的劳动作为价值尺度则存在很多例外\cite[60-82]{BiLuo*SiLaFaDaWeiLiJiaTuQuanJiDi2JuanMaErSaSiZhengZhiJingJiXueYuanLiPingZhu2013},既然例外如此之多,例外反倒成了法则\cite[171]{CaiJiMingCongGuDianZhengZhiJingJiXueDaoZhongGuoTeSeSheHuiZhuYiZhengZhiJingJiXueJiYuZhongGuoShiJiaoDeZhengZhiJingJiXueYanBianShangCe2023}。

接着,马尔萨斯又提出一切商品的实际交换价值“取决于以这一商品易取那一商品的力量和愿望”\cite[43]{BiLuo*SiLaFaDaWeiLiJiaTuQuanJiDi2JuanMaErSaSiZhengZhiJingJiXueYuanLiPingZhu2013},也就是取决于市场的供给和需求的相对关系。这里,需求指的是“购买的力量和愿望的结合”\cite[43]{BiLuo*SiLaFaDaWeiLiJiaTuQuanJiDi2JuanMaErSaSiZhengZhiJingJiXueYuanLiPingZhu2013},供给指的是“商品的生产和卖出商品的意向的结合”\cite[43]{BiLuo*SiLaFaDaWeiLiJiaTuQuanJiDi2JuanMaErSaSiZhengZhiJingJiXueYuanLiPingZhu2013}。在此基础上,马尔萨斯又进一步指出尽管商品的生产成本是商品供给的必要条件,对商品价格有很大的影响\cite[50,55]{BiLuo*SiLaFaDaWeiLiJiaTuQuanJiDi2JuanMaErSaSiZhengZhiJingJiXueYuanLiPingZhu2013},但生产成本本身还是由供求法则决定的,所以商品的实际交换价值根本上是由供求法则来决定的\cite[59]{BiLuo*SiLaFaDaWeiLiJiaTuQuanJiDi2JuanMaErSaSiZhengZhiJingJiXueYuanLiPingZhu2013}。

总的来说,马尔萨斯把商品所能支配的劳动作为衡量“实际交换价值”的尺度,并认为是供需关系决定了商品的“实际交换价值”。

\subsection{马尔萨斯的生产力理论}

马尔萨斯与其所处时代的其他经济学家一样,并没有提出绝对生产力以外的生产力概念。马尔萨斯实际上是在分析财富的增长时间接地分析了生产力。

马尔萨斯指出,财富和价值有着根本性的区别,财富的多少“部分取决于产品的数量,部分取决于产品对社会的需要和力量的适应”\cite[292]{BiLuo*SiLaFaDaWeiLiJiaTuQuanJiDi2JuanMaErSaSiZhengZhiJingJiXueYuanLiPingZhu2013}。光看这段话,似乎可以认为马尔萨斯的“财富”与使用价值是相等的概念,但是笔者认为,马尔萨斯的“财富”指的应当是全社会商品的价值总量。因为马尔萨斯在后文中又指出“在现实情况中,商品的价值,也就是人们为了取得这些商品所愿意作出的牺牲,可以说是任何数量的财富存在的唯一原因”\cite[292]{BiLuo*SiLaFaDaWeiLiJiaTuQuanJiDi2JuanMaErSaSiZhengZhiJingJiXueYuanLiPingZhu2013},以及“...,对财富的不断增长,...,就非有对商品需求的不断增长的配合不可”\cite[355]{BiLuo*SiLaFaDaWeiLiJiaTuQuanJiDi2JuanMaErSaSiZhengZhiJingJiXueYuanLiPingZhu2013}。马尔萨斯之所以认为财富和价值有着根本性的区别,还是因为他没能从交换价值中抽象出价值的概念,而交换价值是某一个比例,只有大小而无数量,所以马尔萨斯必须引入“财富”来衡量价值的数量。

而后,马尔萨斯分析了“财富增长的直接原因”,并指出“资本的积累、土地的肥力和节省劳动的发明”是提高供给的三大原因\cite[text]{BiLuo*SiLaFaDaWeiLiJiaTuQuanJiDi2JuanMaErSaSiZhengZhiJingJiXueYuanLiPingZhu2013}。但这还不够,马尔萨斯还强调“生产力与分配手段结合的必要性”\cite[356]{BiLuo*SiLaFaDaWeiLiJiaTuQuanJiDi2JuanMaErSaSiZhengZhiJingJiXueYuanLiPingZhu2013}:从正面看,“...,单是生产力,不论巨大到什么程度是不足以保证财富在适当的程度上的增长的”,为了让生产力的提升充分发挥作用,还需要“这样一种情况的产品分配,和产品对消费者的需要的这样一种情况的适应,从而使全部产品的交换价值\footnote{这里可以看到因马尔萨斯没能区分价值和交换价值而产生的谬误}不断提高”\cite[356]{BiLuo*SiLaFaDaWeiLiJiaTuQuanJiDi2JuanMaErSaSiZhengZhiJingJiXueYuanLiPingZhu2013};从反面看,“假使一个国家所有的公路和运河都被破坏,其产品的分配手段根本受到阻碍,其产品的整个价值将显著下降”\cite[357]{BiLuo*SiLaFaDaWeiLiJiaTuQuanJiDi2JuanMaErSaSiZhengZhiJingJiXueYuanLiPingZhu2013}。

\subsection{马尔萨斯的生产力与价值的关系}

根据以上的分析,我们不难看出马尔萨斯笔下的生产力是绝对生产力,且马尔萨斯认为当绝对生产力能适应消费者的需求时,绝对生产力的提升与全社会商品的总价值量正相关;当绝对生产力不能适应消费者的需求时,绝对生产力的提升与全社会商品的总价值量不一定相关。

另一方面,马尔萨斯也间接地指出了绝对生产力与单位商品价值量之间的关系。他说“用改进的机器,在同样成本下取得同样质量的更多的商品时,财富与价值之间的区别是明显的;然而,即使就这里的情况说,这一增益量的拥有者,也只是从消费看来,而不是从交换看来,比前富裕”\cite[291]{BiLuo*SiLaFaDaWeiLiJiaTuQuanJiDi2JuanMaErSaSiZhengZhiJingJiXueYuanLiPingZhu2013}。不难看出,马尔萨斯认为绝对生产力与单个生产者单位劳动创造的价值量是不相关的。这与广义价值论的判断是不同的,其原因是马尔萨斯采取了供求价值论,商品的数量增加会影响商品的供求关系。最后,在供求价值论下,绝对生产力与单位商品的价值量仍应是负相关的,这一结论与广义价值论的判断是一致的。

\section{萨伊的价值理论}

萨伊把使用价值和交换价值放到一起来分析价值,形成了一种综合了生产要素论、供求价值论、生产费用论和效用论的价值论\cite[138]{YanZhiJieXiFangJingJiXueShuoShiJiaoChengDiErBan2013}。

萨伊认为,“物品满足人类需要的内在力量叫做效用”,“物品的效用就是物品价值的基础,而物品的价值就是财富所由构成的”\cite[59]{SaYiZhengZhiJingJiXueGaiLunCaiFuDeShengChanFenPeiHeXiaoFei2020}。萨伊进一步指出,价格是测量物品价值的尺度,物品的价值又是测量物品效用的尺度\cite[60]{SaYiZhengZhiJingJiXueGaiLunCaiFuDeShengChanFenPeiHeXiaoFei2020}。

在此基础上,萨伊指出了价值的来源:“事实已经证明,所生产出来的价值,都是归因于劳动、资本和自然力这三者的作用和协力,其中以能耕种的土地为最重要因素但不是唯一因素。”\cite[78]{SaYiZhengZhiJingJiXueGaiLunCaiFuDeShengChanFenPeiHeXiaoFei2020}也就是说,效用或价值是由劳动、资本和土地共同创造的。

接着,萨伊又指出了价值的决定方式。首先,萨伊指出商品的价值决定于“要获得它们必须付出的代价,而代价就是在生产方面所作的努力”\cite[351]{SaYiZhengZhiJingJiXueGaiLunCaiFuDeShengChanFenPeiHeXiaoFei2020},这种努力该如何衡量?萨伊又进一步指出:“使生产力有价值的,乃是创造那需要所从以产生效用的能力。这个价值的大小和这件物品在生产事业中所提供的合作的重要性成比例,而就各个产品说,这个价值构成所谓的生产费用”\cite[352]{SaYiZhengZhiJingJiXueGaiLunCaiFuDeShengChanFenPeiHeXiaoFei2020}。也就是说,生产费用衡量了生产商品所付出的努力,进而决定了商品的价值。这里,萨伊强调了“生产劳动的市值,是基于许多产品相比较的价值”\cite[352]{SaYiZhengZhiJingJiXueGaiLunCaiFuDeShengChanFenPeiHeXiaoFei2020},也就是说,生产费用决定于某一生产要素潜在的创造最大效用的能力。萨伊还引用了一个例子来阐释这一思想:生产一件生产费用为四法郎而售价为三法郎的物品的生产力的价值不是三法郎而是四法郎。因为既然该生产力的生产费用为四法郎,那么其本来能够创造四法郎的价值。但是在这种情况下,该生产力仅仅创造了三法郎的价值\cite[352]{SaYiZhengZhiJingJiXueGaiLunCaiFuDeShengChanFenPeiHeXiaoFei2020}。

前文提到,萨伊认为价格是衡量价值的尺度。那价格是如何决定的呢?萨伊指出价格受到供求法则的支配——“在一定的时间和地点,一种货物的价格,随着需求的增加与供给的减少而成比例地上升;反过来也是一样。换句话说,物价的上升和需求成正比例,但和供给成反比例。”\cite[256]{SaYiZhengZhiJingJiXueGaiLunCaiFuDeShengChanFenPeiHeXiaoFei2020}

然而,萨伊并没有指出由生产费用决定的价值是如何转化为受供求关系支配的价格的,这种空缺使得萨伊的价值理论看上去比较混乱。例如,有的学者认为萨伊笔下价值有三种含义:一般意义的价值是指获得商品就必须支付的代价,即劳动、资本和土地;市价是指供求所影响和决定的价值;价值则是指来自生产费用的物品的效用\cite[138]{YanZhiJieXiFangJingJiXueShuoShiJiaoChengDiErBan2013}\cite[175]{CaiJiMingCongGuDianZhengZhiJingJiXueDaoZhongGuoTeSeSheHuiZhuYiZhengZhiJingJiXueJiYuZhongGuoShiJiaoDeZhengZhiJingJiXueYanBianShangCe2023}。在笔者看来,萨伊的价值理论没有那么复杂。如前所述,萨伊认为价值的基础是效用,价值的来源是劳动、资本和土地的协作生产,价值又决定于商品生产过程中所使用的这三种要素的生产费用,而生产费用又决定于某生产力能创造的最大效用。只不过在解释价格和价值之间的关系时,萨伊遇到了和马克思“转型问题”类似的价值向价格转化的问题。

\section{约翰·穆勒}