% !TeX root = ../2019080346_Mason.tex

\chapter{广义价值论的基本原理}

\section{广义价值论简介}

广义价值论的思想是由蔡继明在1985年发表的论文《比较利益说和劳动价值论》\cite{CaiJiMingBiJiaoLiYiShuoYuLaoDongJieZhiLun1985}中首次提出的。这种价值理论借鉴了李嘉图的比较优势原理,将其适用范围从国际贸易领域拓展到一般的分工交换领域,使分工交换内生于价值决定,揭示了机会成本在价值决定过程中的作用,从而将传统的价值理论作为特列纳入了统一的价值决定框架之中,因此被称为广义价值论\cite[221]{LiRenJunJieZhiLiLun2004}。

接下来,笔者将对广义价值论的基本原理进行介绍。

\section{分工交换}

广义价值论是建立在对社会分工交换的认识之上的。

\subsection{比较利益是社会分工和交换产生的条件}

广义价值论认为,社会的分工和交换是同一事物的不同方面,两者之间不存在因果关系。它们的产生是因为比较利益(Comparative Benefit,CB)的存在。比较利益是生产者通过分工交换而得到的收益(价值或效用)与自给自足时的收益(价值或效用)之差,也可以通过交换得到的收益与其让渡产品的机会成本(Opportuinity Cost,OC)之差来表示。只有当双方的比较利益同时为正时,社会分工和交换才有可能\cite[32]{CaiJiMingLunFenGongYuJiaoHuanDeQiYuanHeJiaoHuanBiLiDeQueDingGuangYiJieZhiLunGangShang1999}。

事实上,广义价值论正是建立在这样一个基本公理之上:

\begin{axiom}
    \label{Bijiaoliyishishehuifengonghejiaohuanchanshengdetiaojian}
    比较利益是社会分工和交换产生的条件。
\end{axiom}

为了进一步揭示比较利益的来源,需要引入几种生产力概念。

\subsection{绝对生产力与相对生产力}

为了引入生产力的概念,首先需要阐释使用价值。按照马克思的观点,使用价值就是商品体本身,是构成财富的物质的内容,其来源于商品的有用性\cite[48-49]{ZhongGongZhongYangMaKeSiEnGeSiLieNingSiDaLinZhuZuoBianYiJuMaKeSiEnGeSiWenJiDi5Juan2009}。在此基础上,我们可以给出绝对生产力(Absolute Productivity)的定义:

\begin{definition}
    绝对生产力是指单位劳动耗费所生产的使用价值量\cite[47]{CaiJiMingCongXiaYiJieZhiLunDaoGuangYiJieZhiLunXiuDingBan2022}
\end{definition}

令$q_{ij}$表示生产者$i$在耗费单位劳动所生产产品$j$的使用价值量,也就是生产者$i$的绝对生产力;$t_{ij}$表示生产者$i$在生产单位使用价值的产品$j$所必需要耗费的劳动\footnote{既包括活劳动也包括物化劳动,物化劳动本身不能创造价值而只能转移价值,但可以使活劳动取得自乘的效果\cite{ChengEnFuXinDeHuoLaoDongJieZhiYiYuanLunLaoDongJieZhiLiLunDeDangDaiTuoZhan2001}}量(用劳动时间来衡量\cite[51]{ZhongGongZhongYangMaKeSiEnGeSiLieNingSiDaLinZhuZuoBianYiJuMaKeSiEnGeSiWenJiDi5Juan2009}),则有:
\begin{equation}
    \label{jueduishengchanli}
    q_{ij}=\frac{1}{t_{ij}}
\end{equation}

马克思认为“社会必要劳动时间是在现有的社会正常的生产条件下,在社会平均的劳动熟练程度和劳动强度下制造某种使用价值所需要的劳动时间”\cite[52]{ZhongGongZhongYangMaKeSiEnGeSiLieNingSiDaLinZhuZuoBianYiJuMaKeSiEnGeSiWenJiDi5Juan2009}。类似地,我们也可以称$t_{ij}$为个体必要劳动时间。如果我们把生产者i替换为一个部门,则我们也可以称之为部门必要劳动时间。那么,个体必要劳动时间该如何转化为部门必要劳动时间呢?从数值上看,部门必要劳动时间是个体必要劳动时间以使用价值产量为权重的加权平均\cite[53]{LinGangGuanYuSheHuiBiYaoLaoDongShiJianYiJiLaoDongShengChanLuYuJieZhiLiangGuanXiWenTiDeTanTao2005};从机制上看,这种转换是某一特定部门内部在市场上的相互竞争带来的。

基于此,我们可以定义绝对优势(Absolute Advantage)的概念:

\begin{definition}
    绝对优势是指一个生产者在特定产品上拥有绝对优势是指该生产者在特定产品上拥有较高的绝对生产力。
\end{definition}

绝对生产力也是马克思所认为的“劳动生产力”。马克思认为“生产力当然始终是有用的、具体的劳动的生产力,它事实上只决定有目的的生产活动在一定时间内的效率”\cite[59]{ZhongGongZhongYangMaKeSiEnGeSiLieNingSiDaLinZhuZuoBianYiJuMaKeSiEnGeSiWenJiDi5Juan2009},这和绝对生产力的定义是相符的。因此,绝对生产力“是由多种情况决定的,其中包括:工人的平均熟练程度,科学的发展水平和它在工艺上应用的程度,生产过程的社会结合,生产资料的规模和效能,以及自然条件”\cite[53]{ZhongGongZhongYangMaKeSiEnGeSiLieNingSiDaLinZhuZuoBianYiJuMaKeSiEnGeSiWenJiDi5Juan2009}。

由于不同的使用价值是异质的,所以不同产品对应的绝对生产力有着不同的单位,是不能相互进行比较的。但是,借助于相对生产力(Relative Productivity,RP)的概念却可以进行间接的比较。具体来说:

\begin{definition}
    相对生产力是指同一生产者在不同产品上的绝对生产力之比\cite[48]{CaiJiMingCongXiaYiJieZhiLunDaoGuangYiJieZhiLunXiuDingBan2022}。
\end{definition}

假设市场上有产品1、2,生产者1、2,则生产者1的相对生产力$RP_1$和生产者2的相对生产力$RP_2$分别为
\begin{equation}
    \begin{cases}
        \mathit{RP}_1=\frac{q_{11}}{q_{12}}\\
        \mathit{RP}_2=\frac{q_{21}}{q_{22}}
    \end{cases}
\end{equation}

然而,相对生产力的概念并没有消除使用价值的异质性,因此也就不能直接比较两个生产者的相对生产力。然而,我们仍然希望能分析一个生产者的比较优势(Comparitive Advantage),也就是:

\begin{definition}
    比较优势是指一个生产者生产一种产品的绝对生产力与自身生产另一种产品的绝对生产力之间的相对大小\cite[49]{CaiJiMingCongXiaYiJieZhiLunDaoGuangYiJieZhiLunXiuDingBan2022}。
\end{definition}

为了进行这样的分析,我们需要引入相对生产力系数(Relative Productivity Coefficient)的概念如下:

\begin{definition}
    相对生产力系数是一个生产者在不同产品上的相对生产力与另一个生产者在不同产品上的相对生产力之比\cite[48]
    {CaiJiMingCongXiaYiJieZhiLunDaoGuangYiJieZhiLunXiuDingBan2022},即
    \begin{equation}
        \mathit{RP}_{1/2} = \frac{q_{11}/q_{12}}{q_{21}/q_{22}} = \frac{q_{11}q_{22}}{q_{21}q_{12}}
    \end{equation}
\end{definition}

相对生产力系数在分子分母上同时出现了两种使用价值的乘积,将反映使用价值异质性的量纲约分后,使用价值的异质性已经被消除,且相对生产力系数是无量纲的,其大小直接反映了生产者的比较优势,具体如表\ref{table:RP_{1/2}}\footnote{引号代表相对大小}所示。

\begin{table}
    \centering
    \caption{相对生产力系数的含义\cite[48-49]{CaiJiMingCongXiaYiJieZhiLunDaoGuangYiJieZhiLunXiuDingBan2022}}
    \label{table:RP_{1/2}}
    \begin{tabular}{|l|l|}
    \hline
        $\mathit{RP}_{1/2}<1$ & $q_{11} \quad "<" \quad q_{12}; \quad q_{21} \quad ">" \quad q_{22}$ \\ \hline
        $\mathit{RP}_{1/2}=1$ & $q_{11} \quad "=" \quad q_{12}; \quad q_{21} \quad "=" \quad q_{22}$ \\ \hline
        $\mathit{RP}_{1/2}>1$ & $q_{11} \quad ">" \quad q_{12}; \quad q_{21} \quad "<" \quad q_{22}$ \\ \hline
    \end{tabular}
\end{table}

\subsection{比较优势是分工交换的充要条件}

现有的文献使用了几何的方法来证明比较优势是分工交换的充要条件。接下来,笔者将试着用代数的方式给出更为严谨的证明。

我们用使用价值来衡量比较利益\cite[63]{CaiJiMingGuangYiJieZhiLun2001},令$T_1$、$T_2$分别为生产者1、2为了交换而投入的劳动量,则生产者1、2的比较利益分别为\footnote{原文中这里有笔误}\footnote{这里假设了线性生产可能性边界,如果假设为非线性生产可能性边界也不会改变结论,只是增加了分析难度\cite[285]{LiRenJunJieZhiLiLun2004}。}
\begin{equation}
    \mathit{CB}_1 = q_{22}T_2 - q_{12}T_1 ; \quad \mathit{CB}_2 = q_{11}T_1 - q_{21}T_2  
\end{equation}

则如果我们可以证明当$RP_{1/2} \neq 1$时,两个生产者能够通过分工交换获得正的比较利益,就可以推断出相对生产力差别决定的比较优势带来了比较利益。下面,我们证明当$RP_{1/2} > 1$时,生产者1、2可以通过分工交换(生产者1生产产品1而生产者2生产产品2)获得正的比较利益。这等价于证明:

\begin{proposition}
    已知正数 $ q_{11}, q_{12}, q_{21}, q_{22} $ 满足 $\frac{q_{11}q_{22}}{q_{12}q_{21}} > 1$,存在正数 $T_1,T_2 > 0$ 使得以下两个不等式同时成立:
    $$
        \begin{cases}
            q_{22}T_2 - q_{12}T_1 > 0, \\
            q_{11}T_1 - q_{21}T_2 > 0.
        \end{cases}
    $$    
\end{proposition}

\begin{proof}
    令 $ r = \frac{T_2}{T_1} $,则 $ r > 0 $,可将原不等式改写为:
    $$
    \begin{cases}
        q_{22}r > q_{12} \quad \Rightarrow \quad r > \frac{q_{12}}{q_{22}}, \\
        q_{11} > q_{21}r \quad \Rightarrow \quad r < \frac{q_{11}}{q_{21}}.
    \end{cases}
    $$

    由 $\frac{q_{11}q_{22}}{q_{12}q_{21}} > 1$ ,可得
    $$
        \frac{q_{11}}{q_{21}} > \frac{q_{12}}{q_{22}}.
    $$

    因此区间 $\left( \frac{q_{12}}{q_{22}}, \frac{q_{11}}{q_{21}} \right)$ 非空,存在 $ r $ 满足
    $$
        \frac{q_{12}}{q_{22}} < r < \frac{q_{11}}{q_{21}}.
    $$

    具体地, $r = \frac{1}{2}\left( \frac{q_{12}}{q_{22}} + \frac{q_{11}}{q_{21}} \right)$ 必是一个解。

    取$T_1 = 1$,则$T_2 =  r T_1$,则已找到正数$T_1,T_2 > 0$。
\end{proof}

这样,根据前文的公理\ref{Bijiaoliyishishehuifengonghejiaohuanchanshengdetiaojian},我们得到了比较优势是分工交换的充分条件。同样,我们也可以证明比较优势是分工交换的必要条件如下:

\begin{proposition}
    已知正数 $ q_{11}, q_{12}, q_{21}, q_{22} $,若存在正数 $ T_1, T_2 > 0 $ 使得以下两个不等式同时成立:
    $$
        \begin{cases}
            q_{22}T_2 - q_{12}T_1 > 0, \\
            q_{11}T_1 - q_{21}T_2 > 0,
        \end{cases}
    $$
    则必有 $\displaystyle \frac{q_{11}q_{22}}{q_{12}q_{21}} > 1$。
\end{proposition}

\begin{proof}
    令 $ r = \frac{T_2}{T_1} $,由 $ T_1, T_2 > 0 $ 知 $ r > 0 $。将原不等式改写为:
    $$
        \begin{cases}
            q_{22}r > q_{12} \quad \Rightarrow \quad r > \frac{q_{12}}{q_{22}}, \\
            q_{11} > q_{21}r \quad \Rightarrow \quad r < \frac{q_{11}}{q_{21}}.
        \end{cases}
    $$

    由于存在 $ r > 0 $ 满足 $\frac{q_{12}}{q_{22}} < r < \frac{q_{11}}{q_{21}}$,这表明区间 $\left( \frac{q_{12}}{q_{22}}, \frac{q_{11}}{q_{21}} \right)$ 非空,因此有:
    $$
        \frac{q_{11}}{q_{21}} > \frac{q_{12}}{q_{22}}.
    $$

    将不等式两边同乘以正数 $ q_{21}q_{22} $,得:
    $$
        q_{11}q_{22} > q_{12}q_{21}.
    $$

    因此,$\displaystyle \frac{q_{11}q_{22}}{q_{12}q_{21}} > 1$,命题得证。

\end{proof}

这样,我们就得到了比较优势和社会分工交换之间的关系如下:

\begin{theorem}
    比较优势是社会分工交换的充要条件。
\end{theorem}

在接下来的分析中,我们会引入商品交换价值的概念,由于我们已经得到:相对生产力系数与1的大小关系决定了分工交换的方向,所以相对生产力系数也就决定了商品交换价值的单位。

\subsection{均衡交换比例的确定}

至此,我们已经分析得到了社会分工和交换的根本原因是比较优势的存在。那么不同商品之间的交换比例又是如何确定的呢?按照马克思的定义,确定这种交换比例也就确定了商品的交换价值\cite[49]{ZhongGongZhongYangMaKeSiEnGeSiLieNingSiDaLinZhuZuoBianYiJuMaKeSiEnGeSiWenJiDi5Juan2009}。

为了分析交换比例,我们引入记号$r_{2/1}=\frac{x_2}{x_1}$代表商品1、2的市场交换比例,同时仍然假定生产者1用商品1交换由生产者2生产的商品2。根据公理\ref{Bijiaoliyishishehuifengonghejiaohuanchanshengdetiaojian},我们可以得到交换必须满足的条件:生产者1用$q_{11}$所换到的商品2必须大于其机会成本$q_{12}$,且要小于生产者2为换取商品1而付出的机会成本$q_{21}$;同时,生产者2用$q_{22}$换取的商品1必须大于其机会成本$q_{21}$,但必须小于生产者1为换取商品2所付出的机会成本$q_{12}$\cite[64]{CaiJiMingCongXiaYiJieZhiLunDaoGuangYiJieZhiLunXiuDingBan2022}。若用公式表达,则有\cite[64]{CaiJiMingCongXiaYiJieZhiLunDaoGuangYiJieZhiLunXiuDingBan2022}:
\begin{equation}
    \label{jiaohuantiaojian}
    \frac{q_{12}}{q_{11}} = \frac{t_{11}}{t_{12}} < r_{2/1} < \frac{t_{21}}{t_{22}} = \frac{q_{22}}{q_{21}}
\end{equation}

原则上,任何满足式\ref{jiaohuantiaojian}的交换比例都是可以达到的,但是交换比例的变动也意味着比较利益在生产者1、2之间的分配比例变动,故只有当交换比例落在一个合理值上或围绕某个合理值波动时,交换才能长久地进行下去。那么,什么样的值才是合理的呢?广义价值论的缔造者提出了下述公理(来自于对人的行为假设\cite[413]{LiRenJunGuangYiJieZhiLunDeLuoJiYuZhengLun2009}):
\begin{axiom}
    在供求一致的情况下,均衡交换比例是根据比较利益率均等原则决定的\cite[65]{CaiJiMingCongXiaYiJieZhiLunDaoGuangYiJieZhiLunXiuDingBan2022}。
\end{axiom}

其中,
\begin{definition}
    比较利益率是比较利益与机会成本的比率\cite[65]{CaiJiMingCongXiaYiJieZhiLunDaoGuangYiJieZhiLunXiuDingBan2022}。
\end{definition}

为了分析的简便,我们用时间成本来表示比较利益率,有:
\begin{equation}
    \label{bijiaoliyilv}
    \mathit{CB}^{\prime}_1 = \frac{x_2 t_{12} - x_1 t_{11}}{x_1 t_{11}}; \quad \mathit{CB}^{\prime}_2 = \frac{x_2 t_{21} - x_1 t_{22}}{x_1 t_{22}}
\end{equation}

若我们用$r^{\mathit{min}}_{2/1}$和$r^{\mathit{max}}_{2/1}$分别表示最低和最高交换比例$\mathit{R}_{2/1}$表示均衡交换比例;$\mathit{CB}^{\prime}_i$表示生产者$i$的比较利益率,$\mathit{CB}^{\prime}_{i=j}$表示平均比较利益率,则我们可以得到\cite[66]{CaiJiMingCongXiaYiJieZhiLunDaoGuangYiJieZhiLunXiuDingBan2022}:

\begin{equation}
    \begin{cases}
        0 = \frac{q_{11}}{q_{12}}r^{\mathit{min}}_{2/1} - 1 < \mathit{CB}^{\prime}_1 < \frac{q_{11}}{q_{12}}r^{\mathit{max}}_{2/1} - 1 = \mathit{RP}_{1/2} - 1 \\
        0 = \frac{q_{22}}{q_{12}r^{\mathit{max}}_{2/1}} - 1 < \mathit{CB}^{\prime}_2 < \frac{q_{22}}{q_{12}r^{\mathit{min}}_{2/1}} - 1 = \mathit{RP}_{1/2} - 1
    \end{cases}
\end{equation}

在均衡状态下:
\begin{equation}
    \mathit{CB}^{\prime}_1 = \mathit{CB}^{\prime}_2 = \mathit{CB}^{\prime}_{i=j}
\end{equation}

代入式\ref{bijiaoliyilv},我们可以得到:
\begin{equation}
    \label{junhengjiaohuanbili}
    \mathit{R}_{2/1} = \sqrt{\frac{t_{11}t_{21}}{t_{12}t_{22}}} = \sqrt{\frac{q_{12}q_{22}}{q_{11}q_{21}}}
\end{equation}

\section{社会平均生产力}

观察上式,我们发现$\sqrt{t_{11}t_{21}}$和$\sqrt{t_{12}t_{22}}$分别是两个生产者(部门)生产商品1和商品2的个体(部门)必要劳动时间的几何平均。因此,我们可以接着定义:
\begin{definition}
    社会必要劳动时间是生产某一商品所有的部门必要劳动时间的几何平均。
\end{definition}

然而,这里的社会必要劳动时间和马克思笔下的“社会必要劳动时间”不是一个概念。前文提到,马克思笔下的“社会必要劳动时间”在广义价值论的体系中只能算是部门必要劳动时间,这种差别源自对分工体系的不同认识。

\subsection{不同的分工体系与“社会必要劳动时间”}
分工体系是一个与劳动异质性密切相关的概念,劳动异质性不仅表现为生产力大小方面的相对差异,也表现为生产力种类方面的差异,而根据参与分工的生产者专业化分工方向能否改变,我们可以将分工体系分为三类\cite[111]{CaiJiMingCongXiaYiJieZhiLunDaoGuangYiJieZhiLunXiuDingBan2022}。

第一类是不变分工体系,也就是生产者分工方向唯一且固定不可改变的分工体系\cite[55-56]{CaiJiMingGuangYiJieZhiLun2001}。在这类分工体系中,某一生产者只能生产特定的商品,而在其它商品上的生产力为0。例如,一个经济部门甲依山而居,可以种植苹果;另一个经济部门乙是海岛居民,可以出海捕鱼。那么甲在海鱼上的生产力为0,乙在水果上的生产力为0,这就是一个典型的不变分工体系。

第二类是可变分工体系,也就是是生产者分工方向可以改变的分工体系\cite[55-56]{CaiJiMingGuangYiJieZhiLun2001},在这类分工体系中,分工的方向取决于相对生产力系数$\mathit{RP}_{1/2}$的取值,生产者会生产自己占有比较优势的商品,但也具备生产其它商品的潜在的能力\footnote{或者同时还生产其它产品但仅用于自给自足}。例如,在新中国刚成立时,我国连一辆汽车都不能造,但到2023年,我国已是全球第一大汽车出口大国\cite{HuangXinZhongGuoZhiZaoQiangJinZhuangGuYouDaXiangQiang2024}。这正反映了我国在汽车上的生产力从0到1再不断提高的过程,是可变分工体系的具体表现。

第三类是混合分工体系是前两种分工体系的混合物,即各生产者中至少有一方的专业分工方向不能改变而除此之外其他各方均可以改变其专业分工方向的体系\cite[113]{CaiJiMingCongXiaYiJieZhiLunDaoGuangYiJieZhiLunXiuDingBan2022}。

在马克思的价值理论中,我们只能找到与绝对生产力相符的“劳动生产力”概念,而没有发现相对生产力的痕迹,故马克思的价值理论必然是建立在固定分工体系的假设之上的\cite[58]{LiRenJunJieZhiLiLun2004}。又因为马克思笔下的“社会必要劳动时间”所指的对象是某一特定商品,按照固定分工体系的假设,生产这一特定商品的也只能是某一特定的部门,故形成这一“社会必要劳动时间”的是某一部门而非整个社会。然而,根据广义价值论的定义,社会必要劳动时间还包含了某一特定商品的潜在的生产部门的部门必要劳动时间,也就是包含了实际上并不生产该特定商品的部门在该特定商品上的部门必要劳动时间。这便是广义价值论中的社会必要劳动时间和马克思笔下的“社会必要劳动时间”的本质区别。

\subsection{从社会必要劳动时间到社会平均生产力}

既然我们已经清楚地定义了社会必要劳动时间的含义,由式\ref{jueduishengchanli},我们可以定义:
\begin{definition}
    社会平均生产力是生产某一商品所有部门的绝对生产力的几何平均。
\end{definition}

类似地,这里的社会平均生产力也包含了那些“潜在”生产部门的生产力。

记$\mathit{AP}_1 = \sqrt{q_{11}q_{21}}$和$\mathit{AP}_2 = \sqrt{q_{12}q_{22}}$是商品1、2的社会平均生产力,则类似于相对生产力系数的概念,我们可以定义:
\begin{definition}
    社会平均生产力系数是两种商品的社会平均生产力之比\cite[68]{CaiJiMingCongXiaYiJieZhiLunDaoGuangYiJieZhiLunXiuDingBan2022}。
\end{definition}

记$\mathit{AP}_{2/1} = \sqrt{\frac{q_{12}q_{22}}{q_{11}q_{21}}}$,我们可以重写式\ref{junhengjiaohuanbili}如下\cite[68]{CaiJiMingCongXiaYiJieZhiLunDaoGuangYiJieZhiLunXiuDingBan2022}:
\begin{equation}
    \mathit{R} = \frac{x_2}{x_1} = \sqrt{\frac{q_{12}q_{22}}{q_{11}q_{21}}} = \mathit{AP}_{2/1}
\end{equation}

我们还可以写出平均比较利益率如下:
\begin{equation}
    \mathit{CB}^{\prime}_{1=2} = \sqrt{\mathit{RP}_{1/2}} - 1
\end{equation}

\section{价值决定的一般原理}

\subsection{价值的一般定义}
在前文中,我们已经确定了交换价值,即一种使用价值与另一种使用价值相交换的比例。接下里,我们将从交换价值出发给出价值的定义。

马克思认为,价值是生产商品所耗费的抽象人类劳动\cite[51]{ZhongGongZhongYangMaKeSiEnGeSiLieNingSiDaLinZhuZuoBianYiJuMaKeSiEnGeSiWenJiDi5Juan2009}。然而,这并非价值的定义,而是马克思对价值来源的认识。在笔者看来,马克思对价值的一般定义应当从下面这句话中提取:“因此,在商品的交换关系或交换价值中表现出来的共同东西,也就是商品的价值。”\cite[51]{ZhongGongZhongYangMaKeSiEnGeSiLieNingSiDaLinZhuZuoBianYiJuMaKeSiEnGeSiWenJiDi5Juan2009}从马克思对价值表现形式的分析中也能看出这种定义:当商品的交换处于简单的物物交换阶段时,“一个商品的价值性质通过该商品与另一个商品的关系而显露出来”\cite[65]{ZhongGongZhongYangMaKeSiEnGeSiLieNingSiDaLinZhuZuoBianYiJuMaKeSiEnGeSiWenJiDi5Juan2009}。这里,一个商品与另外一个商品的关系也就是一种使用价值与另一种使用价值之间的关系,也就是说,“一个商品的价值是通过它表现为‘交换价值’而得到独立的表现的”\cite[76]{ZhongGongZhongYangMaKeSiEnGeSiLieNingSiDaLinZhuZuoBianYiJuMaKeSiEnGeSiWenJiDi5Juan2009}。随着物物交换的范围不断扩大,“一个商品例如麻布的价值表现在商品世界的其他无数的元素上。每一个其他的商品体都成为反映麻布价值的镜子”\cite[78]{ZhongGongZhongYangMaKeSiEnGeSiLieNingSiDaLinZhuZuoBianYiJuMaKeSiEnGeSiWenJiDi5Juan2009}。这个时候,“商品世界的一般相对价值形式,室被排挤出商品世界的等价物商品即麻布,获得了一般等价物的性质。”\cite[83]{ZhongGongZhongYangMaKeSiEnGeSiLieNingSiDaLinZhuZuoBianYiJuMaKeSiEnGeSiWenJiDi5Juan2009}。也就是说,其它商品通过麻布表现出来的交换价值,就是商品一般相对价值的体现。最后,随着“等价形式同这种独特商品的自然形式社会地结合在一起,这种独特的商品成了货币商品”\cite[86]{ZhongGongZhongYangMaKeSiEnGeSiLieNingSiDaLinZhuZuoBianYiJuMaKeSiEnGeSiWenJiDi5Juan2009},价值会通过货币形式表现出来,即通过价格的形式表现出来。结合马克思的上述分析,我们终于能够给出价值的一般定义:
\begin{definition}
    价值是调节商品交换价值(或价格)运动的一般规律\cite[6]{CaiJiMingCongXiaYiJieZhiLunDaoGuangYiJieZhiLunXiuDingBan2022}。
\end{definition}

\subsection{单位商品价值的决定}

在对价值概念进行了明确的定义后,我们继续对分工交换的分析。

根据马克思的价值规律,商品交换是等价值的交换。因此,我们记单位商品1、2的价值分别为$V^c_1$和$V^c_2$,则有\cite[70]{CaiJiMingCongXiaYiJieZhiLunDaoGuangYiJieZhiLunXiuDingBan2022}:
\begin{equation}
    \label{dengjiajiaohuan}
    x_1 t_{11} = V^c_1 x_1 = V^c_2 x_2 = x_2 t_{22}
\end{equation}

将其代入均衡交换比例,即式\ref{junhengjiaohuanbili},我们可以得到\cite[70-71]{CaiJiMingCongXiaYiJieZhiLunDaoGuangYiJieZhiLunXiuDingBan2022}:
\begin{equation}
    \begin{cases}
        V^c_1 = \frac{1}{2}\left(t_{11}+t_{22}\sqrt{\frac{t_{11} t_{21}}{t_{22} t_{12}}}\right)=\frac{1}{2}\left(\frac{1}{q_{11}}+\frac{1}{q_{22}}\sqrt{\frac{q_{12}q_{22}}{q_{11}q_{21}}}\right) \\
        V^c_2 =\frac{1}{2}\left(t_{22}+t_{11}\sqrt{\frac{t_{22} t_{12}}{t_{11} t_{21}}}\right) = \frac{1}{2}\left( \frac{1}{q_{22}}+\frac{1}{q_{11}}\sqrt{\frac{q_{11}q_{21}}{q_{12}q_{22}}}\right)
    \end{cases}
\end{equation}

\begin{equation}
    \label{danweishangpingdejiazhi}
    \begin{cases}
        V^c_1 = \frac{t_{11}}{2}\left(1+\sqrt{\frac{t_{21}t_{22}}{t_{11}t_{12}}}\right)=\frac{1}{2q_{11}}\left(1+\sqrt{\frac{q_{11}q_{12}}{q_{22}q_{21}}}\right) \\

        V^c_2 = \frac{t_{22}}{2}\left(1+\sqrt{\frac{t_{11}t_{12}}{t_{21}t_{22}}}\right)=\frac{1}{2q_{22}}\left(1+\sqrt{\frac{q_{21}q_{22}}{q_{12}q_{11}}}\right)
    \end{cases}
\end{equation}

\subsection{综合生产力}

这里,我们发现出现了$q_{21}q_{22}$等不同绝对生产力的交叉项,因此我们引入一个新的概念——综合生产力(comprehensive productivity,cp):
\begin{definition}
    综合生产力是指同一部门在不同商品上的绝对生产力的几何平均\cite[81]{CaiJiMingCongXiaYiJieZhiLunDaoGuangYiJieZhiLunXiuDingBan2022}。
\end{definition}

综合生产力反映了该经济主体在不同商品生产能力上的综合水平\cite[81]{CaiJiMingCongXiaYiJieZhiLunDaoGuangYiJieZhiLunXiuDingBan2022}。我们记生产者1、2的综合生产力分别为$ \mathit{CP}_1 = \sqrt{q_{11}q_{12}} $和$ \mathit{CP}_2 = \sqrt{q_{21}q_{22}} $,同时定义:
\begin{definition}
    综合生产力系数是两部门综合生产力的比率\cite[71]{CaiJiMingCongXiaYiJieZhiLunDaoGuangYiJieZhiLunXiuDingBan2022}。
\end{definition}

我们记生产者1相对于生产者2的综合生产力系数为$ cp_{1/2} $。

在引入比较生产力(Comparative Productivity,CP)分析综合生产力系数的经济学含义时,现有的广义价值论文献出现了一定的矛盾。现有的文献一方面认为“将两部门综合生产力的比率定义为综合生产力系数,其表示两部门在已确定的专业化生产上相比较而言的生产力,即比较生产力。”\cite[264]{CaiJiMingCongGuDianZhengZhiJingJiXueDaoZhongGuoTeSeSheHuiZhuYiZhengZhiJingJiXueJiYuZhongGuoShiJiaoDeZhengZhiJingJiXueYanBianShangCe2023},另一方面又认为“一个生产者与另一个生产者在不同商品上的比较生产力的高低是由两个生产者的相对生产力系数和综合生产力系数两个因素决定的。”\cite[272]{CaiJiMingCongGuDianZhengZhiJingJiXueDaoZhongGuoTeSeSheHuiZhuYiZhengZhiJingJiXueJiYuZhongGuoShiJiaoDeZhengZhiJingJiXueYanBianShangCe2023}

\subsection{比较生产力}

\subsubsection{比较生产力的概念}

首先,广义价值论的提出者借鉴了斯拉法“合成商品”的概念,即把商品1、2视为一种组合商品3,则商品3就是一种抽象意义上的组合商品\cite[293]{CaiJiMingGuangYiJieZhiLun2001}。这种组合商品对于两个生产者而言是同质的东西,因而也就可以比较两个生产者在组合商品上的绝对生产力大小。换句话说,我们实际上是分别把生产者1、2的综合生产力看作是生产者1、2在组合商品上的绝对生产力,为了比较两者的相对大小,便称两者的比值为比较生产力。因此,比较生产力的第一个定义为:
\begin{definition}
    比较生产力是两部门综合生产力的比率\cite[71]{CaiJiMingCongXiaYiJieZhiLunDaoGuangYiJieZhiLunXiuDingBan2022}。
\end{definition}

既然综合生产力反映的是某一生产者的整体生产能力,那么比较生产力作为综合生产力的比率反映的就是两个生产者整体生产能力的大小。例如,发达国家的整体生产能力高于发展中国家的整体生产能力,故$ \mathit{CP}_{\text{发达}/\text{发展}} > 1 $。

其次,比较生产力还有另外一个定义:

\begin{definition}
    比较生产力是一个生产者在一种商品生产上与另一个生产者在另一种商品生产上相比较而言的生产力\cite[82]{CaiJiMingCongXiaYiJieZhiLunDaoGuangYiJieZhiLunXiuDingBan2022}。
\end{definition}

从公式上来看,记$ \mathit{CP}_1 $和$ \mathit{CP}_2 $分别为为生产者1在商品1上的比较生产力,即$q_{11}$的比较生产力,和生产者2在商品1上的比较生产力,即$q_{21}$的比较生产力,则$ \mathit{CP}_1 = \frac{q_{11}}{q_{22}} $,$ \mathit{CP}_2 = \frac{q_{21}}{q_{12}} $。

在此基础上,我们定义:
\begin{definition}
    比较生产力系数是每个生产者在一种商品生产上与另一个生产者在另一种商品生产上相比较而言的生产力的比率。
\end{definition}


从公式上来看,记$\mathit{CP_{1/2}}$为生产者1相对于生产者2在商品1上的比较生产力系数,则:
\begin{equation}
    \mathit{CP}_{1/2} = \frac{\left(\frac{q_{11}}{q_{22}}\right)}{\left(\frac{q_{21}}{q_{12}}\right)} = \frac{q_{11}q_{12}}{q_{21}q_{22}}
\end{equation}

这里要提请读者注意的一点是,无论在哪一商品上计算比较生产力系数,都会得到相同的结果。例如,如果我们测算生产者1相对于生产者2在商品2上的比较生产力系数,则:
\begin{equation}
    \mathit{CP}_{1/2} = \frac{\left(\frac{q_{12}}{q_{21}}\right)}{\left(\frac{q_{22}}{q_{11}}\right)} = \frac{q_{11}q_{12}}{q_{21}q_{22}}
\end{equation}

所以,比较生产力系数的记号只需要区分生产者顺序,而无需区分商品类别。

这里还要提请读者注意的是,按照第二个定义,比较生产力系数是不带根号的。但这并没有本质上的影响,因为是否带根号并不影响比较生产力与1的大小关系。到这里,我们可以不难看出,前文提到的现有文献中的自相矛盾,来源于对比较生产力系数和比较生产力概念的混淆。如果我们说“将两部门综合生产力的比率定义为综合生产力系数,其表示两部门在已确定的专业化生产上相比较而言的生产力\underline{之比},\underline{再去除其根号},即为比较生产力\underline{系数}。”,则这种矛盾便迎刃而解。

\subsubsection{比较生产力与价值决定}

在理清了比较生产力和比较生产力系数的概念之后,我们尚不清楚其实际的经济学含义。为了进一步分析,我们首先分析比较生产力系数对价值决定的影响。

我们将比较生产力系数代入式\ref{danweishangpingdejiazhi}得到:
\begin{equation}
    \begin{cases}
        V_1^c = \frac{1}{2q_{11}} \left( 1 + \mathit{CP}_{1/2} \right) = \frac{t_{11}}{2} \left( 1 + \mathit{CP}_{1/2} \right) \\
        V_2^c = \frac{1}{2q_{22}} \left( 1 + \mathit{CP}_{2/1} \right) = \frac{t_{22}}{2} \left( 1 + \mathit{CP}_{2/1} \right)
    \end{cases}
\end{equation}

不难看出,单位商品的价值取决于绝对生产力(或者单位劳动时间)和比较生产力的大小,如果比较生产力大于1,则单位商品价值大于单位劳动时间。这样来看,劳动价值论就成为比较生产力为1时的一个特例\cite[72]{CaiJiMingCongXiaYiJieZhiLunDaoGuangYiJieZhiLunXiuDingBan2022}。

另外,通过对该式取微分,我们可以得到\cite[93]{CaiJiMingCongXiaYiJieZhiLunDaoGuangYiJieZhiLunXiuDingBan2022}:
\begin{equation}
    \frac{\partial V_1^c}{\partial q_{11}} = \frac{1}{2} \left[ -q_{11}^2 \left( 1 + \mathit{CP}_{1/2} \right) + \frac{\partial \mathit{CP}_{1/2}}{q_{11}\partial q_{11}} \right] < 0
\end{equation}

所以有:

\begin{theorem}
    单位商品价值与绝对生产力负相关,与比较生产力系数正相关\cite[92]{CaiJiMingCongXiaYiJieZhiLunDaoGuangYiJieZhiLunXiuDingBan2022}。
\end{theorem}

现在,我们基于这个结果来分析比较生产力的经济学含义。从直观上看,既然$\mathit{CP}_{1/2} = \frac{q_{11}q_{12}}{q_{21}q_{22}}$,那么如果生产者1拥有更高的综合生产力,则$ \mathit{CP}_{1/2} > 1 $,故可以推断生产者1拥有更高的比较生产力。这样,我们是否可以猜测:生产者1在分工交换的过程中应当会比生产者2更有“优势”,所以说一定量生产者1的劳动生产的商品可以换取生产者2的用更多劳动生产的商品。

我们还可以从比较生产力系数的定义中探索其影响价值的机制。

首先,我们假设生产者1在商品1的生产上存在比较优势,生产者2在商品2的生产上存在比较优势($ \mathit{RP}_{1/2} > 1 $),那么比较生产力系数的定义式中的各项含义如下:
\begin{equation}
    \mathit{CP}_{1/2} = \frac{\left(\frac{q_{11}}{q_{22}}\right)}{\left(\frac{q_{21}}{q_{12}}\right)} = \frac{\left(\frac{\text{生产力1}}{\text{生产力2}}\right)}{\left(\frac{\text{机会成本2}}{\text{机会成本1}}\right)}
\end{equation}

现在,我们假设$ q_{12} $上升,也就是生产者1的机会成本上升。由于$ q_{12} $位于分母的分母上,所以$\mathit{CP}_{1/2} $会上升,进而生产者1生产的单位商品1的价值会上升,同时生产者2生产的单位商品2的价值会下降,也就是说:

\begin{proposition}
    假定其它条件不变,在不改变比较优势的前提下,某个生产者通过提升自己的机会成本,可以增加自己实际生产的单位商品的价值量,减小另外一个生产者实际生产的单位商品价值量。
\end{proposition}

如果$ q_{12} $继续上升,那么生产者1和生产者2的相对生产力系数$ \mathit{RP}_{1/2} = \frac{q_{11}q_{22}}{q_{21}q_{12}} $会不断减小,最终小于1(由于假设生产者1在商品1的生产上拥有比较优势,故原来$ \mathit{RP}_{1/2} > 1 $),即生产者1不再拥有在商品1生产上的比较优势。也就是说:

\begin{proposition}
    假定其它条件不变,某个生产者通过提升自己的机会成本,会减小自己 的比较优势。
\end{proposition}

至此,笔者作出如下猜测:

\begin{conjecture}
    比较生产力系数衡量了一个生产者相对于另一个生产者的相对比较优势大小,而且一个生产者相对于另一个生产者的比较优势越小,比较生产力系数越大。
\end{conjecture}

同时,根据前文的分析,笔者进一步猜测:

\begin{conjecture}
    假定其它条件不变,某个生产者的相对比较优势越小,则其实际生产的单位商品的价值量越高。
\end{conjecture}

同时,笔者注意到平均比较利益率$ \mathit{CB}^{\prime}_{1=2} = \sqrt{\mathit{RP}_{1/2}} - 1 $。也就是说,如果$ q_{12} $上升,$ \mathit{RP}_{1/2} $减小,那么两个生产者进行分工交换时的平均比较利益率$ \mathit{CB}^{\prime}_{1=2} $也会减小。

然而,受限于时间和文章篇幅,笔者在此停止对比较生产力系数的分析,笔者会在将来的研究中对比较生产力系数进行更深入的研究。

\subsection{不同主体的劳动耗费与价值决定}

到此,我们已经分析了单位商品的价值量和价值决定机制。然而,到此为止我们都假设分工交换是发生在生产者之间或是部门之间的,我们的分析对个体、部门、社会没有做出明确的区分。但无论是在政治经济学的分析范式中,还是现实生活中,这三个经济主体之间是截然不同的。因此,广义价值论按照个体、部门、社会的顺序分析了这三个部门的投入(劳动耗费)与价值产出之间的关系。在开始分析之前,笔者要强调:现在我们明确把分工交换视为两个部门之间的分工交换,把之前的所有生产力概念都看作是部门层面的生产力概念。

\subsubsection{单位个别劳动创造的价值量}
首先,我们来分析个体的单位劳动投入所创造的价值量。我们定义:

\begin{definition}
    绝对生产力差别系数是一个生产者在某一商品上的绝对生产力与其所在部门在该商品上的平均生产力之比\cite[73]{CaiJiMingCongXiaYiJieZhiLunDaoGuangYiJieZhiLunXiuDingBan2022}。
\end{definition}

从公式上看,我们记$q_{ij}^k$为生产者$k$在第$i$部门第$j$商品上的绝对生产力差别系数,再记$V_{ijk}^{l}$为生产者$k$在第$i$部门第$j$产品上单位劳动创造的价值量,$q_{ijk}$为该生产者的劳动生产力,则有:
\begin{equation}
    \begin{cases}
        V_{11k}^{l} = q_{11k}V_{1}^{c} = \frac{1}{2}\frac{q_{11k}}{q_{11}}(1+\mathit{CP}_{1/2}) = \frac{1}{2}q_{11}^{k}(1+\mathit{CP}_{1/2}) \\
        V_{22k}^{l} = q_{22k}V_{2}^{c} = \frac{1}{2}\frac{q_{22k}}{q_{22}}(1+\mathit{CP}_{2/1})=\frac{1}{2}q_{22}^{k}(1+\mathit{CP}_{2/1})
    \end{cases}
\end{equation}

如果部门比较生产力上升,那么单位个别劳动创造的价值量显然会上升。但是如果个体绝对生产力$q_{ijk}$上升,则一方面其会使得部门绝对生产力$q_{ij}$提高产生对价值量的负效应,另一方面又会使得部门综合生产力提高产生对价值量的正效应。
对该式作全微分,我们有\cite[96]{CaiJiMingCongXiaYiJieZhiLunDaoGuangYiJieZhiLunXiuDingBan2022}:
\begin{equation}
    \frac{\dif V_{11k}^{l}}{\dif q_{11}^k} = \frac{1}{2} \left[  1 + \mathit{CP}_{1/2} + q_{11}^k \left( 1 + \frac{\dif \mathit{CP}_{1/2}}{\dif q_{11}^k} \right) \right] = \frac{1}{2} + \frac{1}{4} \sqrt{\frac{q_{12} q_{11k}}{q_{22} q_{21} q_{11}^k}} > 0
\end{equation}

所以,

\begin{theorem}
    单位个别劳动创造的价值量与其绝对生产力和部门比较生产力正相关\cite[95]{CaiJiMingCongXiaYiJieZhiLunDaoGuangYiJieZhiLunXiuDingBan2022}。
\end{theorem}

\subsubsection{部门劳动创造的价值量}

首先,部门单位平均劳动创造的价值量为:
\begin{equation}
    \begin{cases}
        V_1^l = q_{11} V_1^c = q_{11}\frac{1}{2q_{11}}\left( 1+\mathit{CP}_{1/2} \right) = \frac{1}{2}\left( 1+\mathit{CP}_{1/2} \right) \\
        V_2^l = q_{22} V_2^c = q_{22}\frac{1}{2q_{22}}\left( 1+\mathit{CP}_{2/1} \right) = \frac{1}{2}\left( 1+\mathit{CP}_{2/1} \right) 
    \end{cases}
\end{equation}

显然,我们有:

\begin{theorem}
    部门单位平均劳动创造的价值量与部门比较生产力正相关\cite[98]{CaiJiMingCongXiaYiJieZhiLunDaoGuangYiJieZhiLunXiuDingBan2022}。
\end{theorem}

其次,部门总劳动创造的价值总量为:
\begin{equation}
    \begin{cases}
        V_1 = T_1q_{11}V_1^c = \frac{T_1}{2}\left( 1 + \mathit{CP}_{1/2} \right) \\
        V_2 = T_2q_{22}V_1^c = \frac{T_2}{2}\left( 1 + \mathit{CP}_{2/1} \right)
    \end{cases}
\end{equation}

显然,我们有:

\begin{theorem}
    部门总劳动创造的价值总量与部门比较生产力正相关\cite[100]{CaiJiMingCongXiaYiJieZhiLunDaoGuangYiJieZhiLunXiuDingBan2022}。
\end{theorem}

\subsubsection{社会总劳动创造的价值量}

假设两部门经济均衡,则$ V_1 = V_2 $,则社会总价值量为:
\begin{equation}
    V = 2V_1 = T_1 + T_1 \mathit{CP}_{1/2}
\end{equation}

回顾式\ref{junhengjiaohuanbili}、\ref{dengjiajiaohuan},我们可以得到\cite[290]{CaiJiMingCongGuDianZhengZhiJingJiXueDaoZhongGuoTeSeSheHuiZhuYiZhengZhiJingJiXueJiYuZhongGuoShiJiaoDeZhengZhiJingJiXueYanBianShangCe2023}:
\begin{equation}
    \frac{T_2}{T_1} = \mathit{CP}_{1/2}
\end{equation}

这说明:

\begin{theorem}
    部门间必要劳动投入比取决于比较生产力系数(综合生产力系数)。
\end{theorem}

进一步推导可以得到:
\begin{equation}
    V = T_1 + T_2 = T
\end{equation}

这意味着:
\begin{theorem}
    社会价值总量等于社会劳动总量\cite[75]{CaiJiMingCongXiaYiJieZhiLunDaoGuangYiJieZhiLunXiuDingBan2022}\footnote{这里假定当期各部门综合生产力保持不变。}。
\end{theorem}

我们还可以考察跨期的社会价值总量。首先,我们定义:
\begin{definition}
    第$t$期的社会总和生产力是$t$期的各部门综合生产力的几何平均\cite[291]{CaiJiMingCongGuDianZhengZhiJingJiXueDaoZhongGuoTeSeSheHuiZhuYiZhengZhiJingJiXueJiYuZhongGuoShiJiaoDeZhengZhiJingJiXueYanBianShangCe2023}。
\end{definition}

从公式上来看,令$\mathit{TP}^t$表第$t$期的社会总和生产力,$CP_{it}$分别表示第$t$期第$i$部门的综合生产力,则第$t$期的社会总和生产力是\cite[291]{CaiJiMingCongGuDianZhengZhiJingJiXueDaoZhongGuoTeSeSheHuiZhuYiZhengZhiJingJiXueJiYuZhongGuoShiJiaoDeZhengZhiJingJiXueYanBianShangCe2023}:
\begin{equation}
    \mathit{TP}^t = \sqrt{\mathit{CP}_1^t \cdot \mathit{CP}_2^t} = \left( q_{11}^t q_{12}^t q_{21}^t q_{22}^t \right)^\frac{1}{4}
\end{equation}

令$g^t$表示第$t$期相对于第$t-1$期的总和生产力增长率,则有\cite[291]{CaiJiMingCongGuDianZhengZhiJingJiXueDaoZhongGuoTeSeSheHuiZhuYiZhengZhiJingJiXueJiYuZhongGuoShiJiaoDeZhengZhiJingJiXueYanBianShangCe2023}:
\begin{equation}
    g^t = \frac{\mathit{TP}^t - \mathit{TP}^{t-1}}{\mathit{TP}^{t-1}} = \left( \frac{q_{11}^t q_{12}^t q_{21}^t q_{22}^t}{q_{11}^{t-1} q_{12}^{t-1} q_{21}^{t-1} q_{22}^{t-1}} \right)^\frac{1}{4} - 1
\end{equation}

再令$m$为劳动力增长率,那么最后我们可以推导得到,社会价值总量的增长率$G$为:
\begin{equation}
    G = \left( 1+m \right) \left( 1+g \right) - 1 \approx m + g
\end{equation}

\section{总结}

总而言之,我们可以将广义价值论中的各种生产力概念及其参与价值决定的机制总结为如下表格:

\begin{sidewaystable}[!h]
  \centering
  \caption{各种生产力与价值决定机制}
  \label{table:GVT}
  \begin{tabularx}{\textheight}{|c|>{\centering}p{3cm}|c|>{\centering\arraybackslash}X|>{\centering\arraybackslash}X|>{\centering\arraybackslash}X|} % 调整列宽为纵向高度
    \toprule
    生产力类型    & 具体变量    & 定义    & 经济学含义    & 单位商品价值决定机制    & 劳动价值决定机制\\ 
    \midrule
    
    绝对生产力    & 部门绝对生产力($q_{ij}$)    & $q_{ij}=\frac{1}{t_{ij}}$    & 部门生产使用价值的劳动效率    & 与单位商品价值负相关    & 与单位个别劳动创造价值正相关 \\ 
    \hline
    
    \multirow{2}{*}{相对生产力}    & 部门相对生产力($\mathit{RP}_i$)    & $\mathit{RP}_i=\frac{q_{i1}}{q_{i2}}$    & 绝对生产力之比    & 相对量,无作用    & \multirow{2}{*}{无作用} \\ 
    \cline{2-5}
    & 相对生产力系数($\mathit{RP}_{1/2}$)    & $\mathit{RP}_{1/2} = \frac{\mathit{RP}_1}{\mathit{RP}_2}$    & 绝对生产力的差别程度    & 决定分工方向和比较利益率 & \\ 
    \cline{1-5}

    \multirow{2}{*}{平均生产力}    & 社会平均生产力($\mathit{AP}_i$)    & $\mathit{AP}_i = \sqrt{q_{1j}q_{2j}}$     & 某商品的社会平均绝对生产力    & 相对量,无作用 & \multirow{2}{*}{无作用} \\ 
    \cline{2-5}
    & 社会平均生产力系数($\mathit{AP}_{1/2}$)    & $\mathit{AP}_{1/2} = \frac{\mathit{AP}_1}{\mathit{AP}_2}$    & 两种商品的社会平均绝对生产能力之比    & 决定均衡交换比例 & \\ 
    \cline{1-5}

    \multirow{2}{*}{综合生产力} & 部门综合生产力($cp_i$)    & $cp_i = \sqrt{q_{i1}q_{i2}}$     & 某一部门的综合生产能力    & 相对量,无作用 & \multirow{2}{*}{无作用} \\ 
    \cline{2-5}
    & 综合生产力系数($cp_{1/2}$)    & $cp_{1/2} = \frac{\mathit{cp}_1}{\mathit{cp}_2}$    & 部门的综合生产能力之比    & \multirow{2}{*}{与单位商品价值正相关} & \\ 
    \cline{1-4}

    \multirow{2}{*}{比较生产力} & 比较生产力系数($\mathit{CP}_{1/2}$)    & $\mathit{CP}_{1/2} = \frac{\mathit{CP}_1}{\mathit{CP}_2}$    & 部门的比较生产能力之比    & & 无作用 \\ 
    \cline{2-6}
    & 比较生产力($\mathit{CP}_i$)    & $\mathit{CP}_i = \frac{q_{ij}}{q_{mn}}$     & 某一部门在一种商品上的绝对生产力与另一部门在另一种商品上的绝对生产力之比    & 相对量,无作用     & 与单位个别劳动、部门劳动创造的价值量正相关;与部门劳动创造的价值总量正相关\\ 
    \bottomrule
  \end{tabularx}
\end{sidewaystable}

