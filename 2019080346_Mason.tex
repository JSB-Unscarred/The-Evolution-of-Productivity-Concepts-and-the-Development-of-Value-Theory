% !TeX encoding = UTF-8
% !TeX program = xelatex
% !TeX spellcheck = en_US



\documentclass[degree=bachelor]{thuthesis}
  % 学位 degree:
  %   doctor | master | bachelor | postdoc
  % 学位类型 degree-type:
  %   academic(默认)| professional
  % 语言 language
  %   chinese(默认)| english
  % 字体库 fontset
  %   windows | mac | fandol | ubuntu
  % 建议终版使用 Windows 平台的字体编译


% 论文基本配置,加载宏包等全局配置
% !TeX root = ./thuthesis-example.tex

% 论文基本信息配置

\thusetup{
  %******************************
  % 注意:
  %   1. 配置里面不要出现空行
  %   2. 不需要的配置信息可以删除
  %   3. 建议先阅读文档中所有关于选项的说明
  %******************************
  %
  % 输出格式
  %   选择打印版(print)或用于提交的电子版(electronic),前者会插入空白页以便直接双面打印
  %
  % output = electronic,
  output = print,
  %
  % 格式类型
  %   默认为论文(thesis),也可以设置为开题报告(proposal)
  % thesis-type = proposal,
  %
  % 标题
  %   可使用“\\”命令手动控制换行
  %
  title  = {价值理论的发展与生产力概念的演进},
  % title* = {},
  %
  % 学科门类
  %   1. 学术型
  %      - 中文
  %        需注明所属的学科门类,例如:
  %        哲学、经济学、法学、教育学、文学、历史学、理学、工学、农学、医学、
  %        军事学、管理学、艺术学
  %      - 英文
  %        博士:Doctor of Philosophy
  %        硕士:
  %          哲学、文学、历史学、法学、教育学、艺术学门类,公共管理学科
  %          填写“Master of Arts“,其它填写“Master of Science”
  %   2. 专业型
  %      直接填写专业学位的名称,例如:
  %      教育博士、工程硕士等
  %      Doctor of Education, Master of Engineering
  %   3. 本科生不需要填写
  %
  % degree-category  = {},
  % degree-category* = {},
  %
  % 培养单位
  %   填写所属院系的全名
  %
  department = {社科学院},
  %
  % 学科
  %   1. 研究生学术型学位,获得一级学科授权的学科填写一级学科名称,其他填写二级学科名称
  %   2. 本科生填写专业名称,第二学位论文需标注“(第二学位)”
  %
  discipline  = {经济学},
  % discipline* = {},
  %
  % 专业领域
  %   1. 设置专业领域的专业学位类别,填写相应专业领域名称
  %   2. 2019 级及之前工程硕士学位论文,在 `engineering-field` 填写相应工程领域名称
  %   3. 其他专业学位类别的学位论文无需此信息
  %
  % professional-field  = {计算机技术},
  % professional-field* = {Computer Technology},
  %
  % 姓名
  %
  author  = {陈海翔},
  % author* = {},
  %
  % 学号
  % 仅当书写开题报告时需要(同时设置 `thesis-type = proposal')
  %
  % student-id = {},
  %
  % 指导教师
  %   中文姓名和职称之间以英文逗号“,”分开,下同
  %
  supervisor  = {蔡继明, 教授},
  % supervisor* = {},
  %
  % 副指导教师
  %
  associate-supervisor  = {李帮喜, 副教授},
  % associate-supervisor* = {},
  %
  % 联合指导教师
  %
  % co-supervisor  = {某某, 教授},
  % co-supervisor* = {Professor Mou Moumou},
  %
  % 日期
  %   使用 ISO 格式;默认为当前时间
  %
  date = {2025-06-05},
  %
  % 是否在中文封面后的空白页生成书脊(默认 false)
  %
  include-spine = false,
  %
  % 密级和年限
  %   秘密, 机密, 绝密
  %
  % secret-level = {秘密},
  % secret-year  = {10},
  %
  % 博士后专有部分
  %
  % clc                = {分类号},
  % udc                = {UDC},
  % id                 = {编号},
  % discipline-level-1 = {计算机科学与技术},  % 流动站(一级学科)名称
  % discipline-level-2 = {系统结构},          % 专业(二级学科)名称
  % start-date         = {2011-07-01},        % 研究工作起始时间
}

% 载入所需的宏包

% 定理类环境宏包
\usepackage{amsthm}
% 也可以使用 ntheorem
% \usepackage[amsmath,thmmarks,hyperref]{ntheorem}

% 斜杠分数线
\usepackage{xfrac}

\thusetup{
  %
  % 数学字体
  % math-style = GB,  % GB | ISO | TeX
  math-font  = xits,  % stix | xits | libertinus
}

% 可以使用 nomencl 生成符号和缩略语说明
% \usepackage{nomencl}
% \makenomenclature

% 插入图片
\usepackage{graphicx} 

% 绘图
% \usepackage{tikz}

%表格旋转
\usepackage{rotating}

% 表格加脚注
\usepackage{threeparttable}

% 表格中支持跨行
\usepackage{multirow}

% 固定宽度的表格。
\usepackage{tabularx}

% 跨页表格
% \usepackage{longtable}

% 算法
% \usepackage{algorithm}
% \usepackage{algorithmic}

% 量和单位
% \usepackage{siunitx}

% 参考文献使用 BibTeX + natbib 宏包
% 顺序编码制
%\usepackage[sort]{natbib}
%\bibliographystyle{thuthesis-numeric}

% 著者-出版年制
% \usepackage{natbib}
% \bibliographystyle{thuthesis-author-year}

% 生命科学学院要求使用 Cell 参考文献格式(2023 年以前使用 author-date 格式)
% \usepackage{natbib}
% \bibliographystyle{cell}

% 本科生参考文献的著录格式
% \usepackage[sort]{natbib}
% \bibliographystyle{thuthesis-bachelor}

% 参考文献使用 BibLaTeX 宏包
% \usepackage[style=thuthesis-numeric]{biblatex}
% \usepackage[style=thuthesis-author-year]{biblatex}
\usepackage[style=gb7714-2015]{biblatex}
% \usepackage[style=apa]{biblatex}
% \usepackage[style=mla-new]{biblatex}
% 声明 BibLaTeX 的数据库
\addbibresource{ref/refs.bib}
\addbibresource{ref/refs_translation.bib}


% 定义所有的图片文件在 figures 子目录下
\graphicspath{{figures/}}

% 数学命令
\makeatletter
\newcommand\dif{%  % 微分符号
  \mathop{}\!%
  \ifthu@math@style@TeX
    d%
  \else
    \mathrm{d}%
  \fi
}
\makeatother

% hyperref 宏包在最后调用
\usepackage{hyperref}



\begin{document}

% 封面
\maketitle

% 学位论文指导小组、公开评阅人和答辩委员会名单
% 本科生不需要
% \input{data/committee}

% 使用授权的说明
% 本科生开题报告不需要
% \copyrightpage
% 将签字扫描后授权文件 scan-copyright.pdf 替换原始页面
% \copyrightpage[file=scan-copyright.pdf]

\frontmatter
% % !TeX root = ../thuthesis-example.tex

% 中英文摘要和关键字

\begin{abstract}
  论文的摘要是对论文研究内容和成果的高度概括。
  摘要应对论文所研究的问题及其研究目的进行描述,对研究方法和过程进行简单介绍,对研究成果和所得结论进行概括。
  摘要应具有独立性和自明性,其内容应包含与论文全文同等量的主要信息。
  使读者即使不阅读全文,通过摘要就能了解论文的总体内容和主要成果。

  论文摘要的书写应力求精确、简明。
  切忌写成对论文书写内容进行提要的形式,尤其要避免“第 1 章……;第 2 章……;……”这种或类似的陈述方式。

  关键词是为了文献标引工作、用以表示全文主要内容信息的单词或术语。
  关键词不超过 5 个,每个关键词中间用分号分隔。

  % 关键词用“英文逗号”分隔,输出时会自动处理为正确的分隔符
  \thusetup{
    keywords = {关键词 1, 关键词 2, 关键词 3, 关键词 4, 关键词 5},
  }
\end{abstract}

\begin{abstract*}
  An abstract of a dissertation is a summary and extraction of research work and contributions.
  Included in an abstract should be description of research topic and research objective, brief introduction to methodology and research process, and summary of conclusion and contributions of the research.
  An abstract should be characterized by independence and clarity and carry identical information with the dissertation.
  It should be such that the general idea and major contributions of the dissertation are conveyed without reading the dissertation.

  An abstract should be concise and to the point.
  It is a misunderstanding to make an abstract an outline of the dissertation and words “the first chapter”, “the second chapter” and the like should be avoided in the abstract.

  Keywords are terms used in a dissertation for indexing, reflecting core information of the dissertation.
  An abstract may contain a maximum of 5 keywords, with semi-colons used in between to separate one another.

  % Use comma as separator when inputting
  \thusetup{
    keywords* = {keyword 1, keyword 2, keyword 3, keyword 4, keyword 5},
  }
\end{abstract*}


% 目录
\tableofcontents

% 符号对照表
% % !TeX root = ../thuthesis-example.tex

\begin{denotation}[3cm]
  \item[PI] 聚酰亚胺
  \item[MPI] 聚酰亚胺模型化合物,N-苯基邻苯酰亚胺
  \item[PBI] 聚苯并咪唑
  \item[MPBI] 聚苯并咪唑模型化合物,N-苯基苯并咪唑
  \item[PY] 聚吡咙
  \item[PMDA-BDA] 均苯四酸二酐与联苯四胺合成的聚吡咙薄膜
  \item[MPY] 聚吡咙模型化合物
  \item[As-PPT] 聚苯基不对称三嗪
  \item[MAsPPT] 聚苯基不对称三嗪单模型化合物,3,5,6-三苯基-1,2,4-三嗪
  \item[DMAsPPT] 聚苯基不对称三嗪双模型化合物(水解实验模型化合物)
  \item[S-PPT] 聚苯基对称三嗪
  \item[MSPPT] 聚苯基对称三嗪模型化合物,2,4,6-三苯基-1,3,5-三嗪
  \item[PPQ] 聚苯基喹噁啉
  \item[MPPQ] 聚苯基喹噁啉模型化合物,3,4-二苯基苯并二嗪
  \item[HMPI] 聚酰亚胺模型化合物的质子化产物
  \item[HMPY] 聚吡咙模型化合物的质子化产物
  \item[HMPBI] 聚苯并咪唑模型化合物的质子化产物
  \item[HMAsPPT] 聚苯基不对称三嗪模型化合物的质子化产物
  \item[HMSPPT] 聚苯基对称三嗪模型化合物的质子化产物
  \item[HMPPQ] 聚苯基喹噁啉模型化合物的质子化产物
  \item[PDT] 热分解温度
  \item[HPLC] 高效液相色谱(High Performance Liquid Chromatography)
  \item[HPCE] 高效毛细管电泳色谱(High Performance Capillary lectrophoresis)
  \item[LC-MS] 液相色谱-质谱联用(Liquid chromatography-Mass Spectrum)
  \item[TIC] 总离子浓度(Total Ion Content)
  \item[\textit{ab initio}] 基于第一原理的量子化学计算方法,常称从头算法
  \item[DFT] 密度泛函理论(Density Functional Theory)
  \item[$E_a$] 化学反应的活化能(Activation Energy)
  \item[ZPE] 零点振动能(Zero Vibration Energy)
  \item[PES] 势能面(Potential Energy Surface)
  \item[TS] 过渡态(Transition State)
  \item[TST] 过渡态理论(Transition State Theory)
  \item[$\increment G^\neq$] 活化自由能(Activation Free Energy)
  \item[$\kappa$] 传输系数(Transmission Coefficient)
  \item[IRC] 内禀反应坐标(Intrinsic Reaction Coordinates)
  \item[$\nu_i$] 虚频(Imaginary Frequency)
  \item[ONIOM] 分层算法(Our own N-layered Integrated molecular Orbital and molecular Mechanics)
  \item[SCF] 自洽场(Self-Consistent Field)
  \item[SCRF] 自洽反应场(Self-Consistent Reaction Field)
\end{denotation}



% 也可以使用 nomencl 宏包,需要在导言区
% \usepackage{nomencl}
% \makenomenclature

% 在这里输出符号说明
% \printnomenclature[3cm]

% 在正文中的任意为都可以标题
% \nomenclature{PI}{聚酰亚胺}
% \nomenclature{MPI}{聚酰亚胺模型化合物,N-苯基邻苯酰亚胺}
% \nomenclature{PBI}{聚苯并咪唑}
% \nomenclature{MPBI}{聚苯并咪唑模型化合物,N-苯基苯并咪唑}
% \nomenclature{PY}{聚吡咙}
% \nomenclature{PMDA-BDA}{均苯四酸二酐与联苯四胺合成的聚吡咙薄膜}
% \nomenclature{MPY}{聚吡咙模型化合物}
% \nomenclature{As-PPT}{聚苯基不对称三嗪}
% \nomenclature{MAsPPT}{聚苯基不对称三嗪单模型化合物,3,5,6-三苯基-1,2,4-三嗪}
% \nomenclature{DMAsPPT}{聚苯基不对称三嗪双模型化合物(水解实验模型化合物)}
% \nomenclature{S-PPT}{聚苯基对称三嗪}
% \nomenclature{MSPPT}{聚苯基对称三嗪模型化合物,2,4,6-三苯基-1,3,5-三嗪}
% \nomenclature{PPQ}{聚苯基喹噁啉}
% \nomenclature{MPPQ}{聚苯基喹噁啉模型化合物,3,4-二苯基苯并二嗪}
% \nomenclature{HMPI}{聚酰亚胺模型化合物的质子化产物}
% \nomenclature{HMPY}{聚吡咙模型化合物的质子化产物}
% \nomenclature{HMPBI}{聚苯并咪唑模型化合物的质子化产物}
% \nomenclature{HMAsPPT}{聚苯基不对称三嗪模型化合物的质子化产物}
% \nomenclature{HMSPPT}{聚苯基对称三嗪模型化合物的质子化产物}
% \nomenclature{HMPPQ}{聚苯基喹噁啉模型化合物的质子化产物}
% \nomenclature{PDT}{热分解温度}
% \nomenclature{HPLC}{高效液相色谱(High Performance Liquid Chromatography)}
% \nomenclature{HPCE}{高效毛细管电泳色谱(High Performance Capillary lectrophoresis)}
% \nomenclature{LC-MS}{液相色谱-质谱联用(Liquid chromatography-Mass Spectrum)}
% \nomenclature{TIC}{总离子浓度(Total Ion Content)}
% \nomenclature{\textit{ab initio}}{基于第一原理的量子化学计算方法,常称从头算法}
% \nomenclature{DFT}{密度泛函理论(Density Functional Theory)}
% \nomenclature{$E_a$}{化学反应的活化能(Activation Energy)}
% \nomenclature{ZPE}{零点振动能(Zero Vibration Energy)}
% \nomenclature{PES}{势能面(Potential Energy Surface)}
% \nomenclature{TS}{过渡态(Transition State)}
% \nomenclature{TST}{过渡态理论(Transition State Theory)}
% \nomenclature{$\increment G^\neq$}{活化自由能(Activation Free Energy)}
% \nomenclature{$\kappa$}{传输系数(Transmission Coefficient)}
% \nomenclature{IRC}{内禀反应坐标(Intrinsic Reaction Coordinates)}
% \nomenclature{$\nu_i$}{虚频(Imaginary Frequency)}
% \nomenclature{ONIOM}{分层算法(Our own N-layered Integrated molecular Orbital and molecular Mechanics)}
% \nomenclature{SCF}{自洽场(Self-Consistent Field)}
% \nomenclature{SCRF}{自洽反应场(Self-Consistent Reaction Field)}


% \listoffigures           % 插图清单
% \listoftables            % 附表清单

% 正文部分
\mainmatter
%% !TeX root = ../2019080346_Mason.tex

\chapter{引言}

\section{研究背景及意义}

新质生产力作为中国特色社会主义经济思想的重要创新成果,其具有丰富的理论内涵与重要的实践价值。该概念最早由习近平总书记在2023年新时代推动东北全面振兴座谈会上提出\cite{YinBoGuLaoLaoBaWoDongBeiDeChongYaoShiMingFenLiPuXieDongBeiQuanMianZhenXingXinPianZhang2023}。而后,习近平总书记又在不同场合多次提到要发展新质生产力,2024年\cite[17]{LiQiangZhengFuGongZuoBaoGao2024Nian3Yue5RiZaiDiShiSiJieQuanGuoRenMinDaiBiaoDaHuiDiErCiHuiYiShang2024}、2025年\cite[19]{LiQiangZhengFuGongZuoBaoGao2025Nian3Yue5RiZaiDiShiSiJieQuanGuoRenMinDaiBiaoDaHuiDiSanCiHuiYiShang2025}政府工作报告和中共二十届三中全会\cite[8]{ZhongGuoGongChanDangDiErShiJieZhongYangWeiYuanHuiDiSanCiQuanTiHuiYiGongBao2024}也强调要发展新质生产力。在中央的重视下,新质生产力很快成为当前经济发展和经济学研究的主旋律。

根据笔者的文献调研,当前的相关文献主要通过与旧质生产力的对比来阐释“新质”,从马克思的著作中找到与生产力概念相关的原文来阐释“生产力”。在“新质”方面,从理论内涵上看,新质生产力是一种由更多生产要素的高质量组合,以关键性颠覆性技术突破为特点的一种生产力\cite[141-142]{GaoFanXinZhiShengChanLiDeTiChuLuoJiDuoWeiNeiHanJiShiDaiYiYi2023}\cite[1-2]{ZhouWenLunXinZhiShengChanLiNeiHanTeZhengYuChongYaoZhaoLiDian2023};从发展过程来看,新质生产力是一个相对于旧质生产力的概念,是对旧质生产力的更新换代。从蒸汽机到电动机再到计算机,每一种技术都代表了各个时期的新质生产力\cite[28]{CaiJiMingXinZhiShengChanLiDeFaZhanDuiJieZhiChuangZaoHeJingJiZengChangDeGongXian2024};从现实角度来看,当今世界正经历新一轮科技革命\cite[06]{XiJinPingJiaKuaiJianSheKeJiQiangGuoShiXianGaoShuiPingKeJiZiLiZiQiang2022},战略新兴产业和未来产业是新质生产力的主要载体\cite[9]{XiJinPingJingJiSiXiangYanJiuZhongXinXinZhiShengChanLiDeNeiHanTeZhengHeFaZhanChongDian2024}。在“生产力”方面,基于马克思将生产力定义为具体劳动效率的经典论述\cite[59]{ZhongGongZhongYangMaKeSiEnGeSiLieNingSiDaLinZhuZuoBianYiJuMaKeSiEnGeSiWenJiDi5Juan2009},目前的文献一致肯定了新质生产力创造使用价值的能力及其对经济发展的推动作用\cite[16]{CaiJiMingXinZhiShengChanLiCanYuJieZhiChuangZaoDeLiLunTanTaoHeShiJianYingYong2024}\cite{YangYuZhenXinZhiShengChanLiLiLunDuiMaKeSiShengChanLiShengChanGuanXiLiLunDeShouZhengHeChuangXin2025}\cite{XieFuShengMaKeSiDeShengChanLiLiLunYuFaZhanXinZhiShengChanLi2024}\cite{HuYingZaiLunXinZhiShengChanLiDeNeiHanTeZhengYuXingChengLuJingYiMaKeSiShengChanLiLiLunWeiShiJiao2024},但只有少数文献基于广义价值论分析了新质生产力参与价值创造的原理\cite{CaiJiMingXinZhiShengChanLiDeFaZhanDuiJieZhiChuangZaoHeJingJiZengChangDeGongXian2024}\cite{CaiJiMingXinZhiShengChanLiCanYuJieZhiChuangZaoDeLiLunTanTaoHeShiJianYingYong2024}。在笔者看来,这一问题的原因是绝大部分现有的文献都遵循了马克斯等量劳动创造等量价值的观点\cite[60]{ZhongGongZhongYangMaKeSiEnGeSiLieNingSiDaLinZhuZuoBianYiJuMaKeSiEnGeSiWenJiDi5Juan2009}

然而,谷书堂教授提出的“价值总量之谜”\cite[6-7]{GuShuTangQiuJieJieZhiZongLiangZhiMiLiangTiaoSiLuDeBiJiao2002}却又暗示着新质生产力的不断涌现推动了社会价值总量的上升。具体而言,国内生产总值(GDP)是一定时期内社会价值量的总和\cite[107]{ChenDeDiGuoFangJingJiDaCiDian2001}\cite[659]{BaoLuo*SaMouErSenJingJiXueDiShiJiuBan2012},这意味着在投入劳动总量基本不变的情况下,以不变价格衡量的GDP(实际GDP)也会保持基本不变,而这显然与我国改革开放以来的实践事实相矛盾。目前来看,唯有将马克思“劳动生产力与价值量成正比”原理的适用范围从单个生产者推广到部门和全社会的广义价值论,能比较合理、严谨地解释这一谜题\cite{CaiJiMingJiShuJinBuJingJiZengChangYuJieZhiZongLiangZhiMiJiYuGuangYiJieZhiLunDeJieShi2019},因此,笔者希望用广义价值论对经济思想史上的”生产力”作一次系统的梳理,以期为新质生产力研究中的“生产力”概念作一个新的阐释。

在此基础上,笔者又发现现有的文献并没有对广义价值论中的各种生产力概念与各种价值理论中的生产力概念进行比较分析,而价值理论又是经济学体系的理论基础\cite[118]{CaiJiMingCongGuDianZhengZhiJingJiXueDaoZhongGuoTeSeSheHuiZhuYiZhengZhiJingJiXueJiYuZhongGuoShiJiaoDeZhengZhiJingJiXueYanBianShangCe2023}。鉴于建立中国特色社会主义政治经济学的目标要求我们积极汲取西方经济学的合理成分\cite[81]{ChengEnFuChongJianZhongGuoJingJiXueChaoYueMaKeSiYuXiFangJingJiXue2000},笔者打算着眼于生产力参与价值创造的原理,从广义价值论出发,系统梳理各种生产力概念,并将其与不同价值理论中的生产力概念进行对比分析。这一方面有助于对生产力理论,特别是新质生产力理论的进一步研究,以更好地指导发展新质生产力的实践;另一方面也有助于广义价值论的进一步发展,为中国特色社会主义政治经济学大厦的构建添砖加瓦。


\section{研究思路}

本文从新质生产力对价值决定的现实作用与传统劳动价值论的矛盾出发,首先介绍价值理论的最新成果——广义价值论的基本原理,而后通过广义价值论框架系统地对经济思想史进行梳理。本文首先回顾了古典政治经济学,在完成了对代表性经济学家的价值理论和生产力概念的描述后,笔者总结了古典政治经济学的基本倾向是把劳动作为衡量交换价值的尺度,并揭示了经济学对价值的认识发展的过程,最后揭示了作为价值源泉的各种生产要素与作为价值尺度的支配劳动之间的对立统一。继而,本文又指出新古典经济学偏重需求侧分析而忽视供给侧系统性关联,斯拉法体系虽隐含更多的生产力概念却未明确生产力与价值量的动态关系。在此批判性梳理基础上,本文根据广义价值论价值论中的绝对生产力、相对生产力、比较生产力和社会总和生产力提出了对新质生产力的四维理论内涵:在绝对生产力维度表现为技术革命驱动的效率跃升;在相对生产力维度重塑动态比较优势;在比较生产力维度重构价值创造机制;在社会总和生产力维度实现社会总价值量的跃升。

\section{文章结构}
第一章 \quad 引言

第二章 \quad 广义价值论的基本原理

该部分主要介绍广义价值论的基本原理并系统整理各种生产力的概念及其与价值创造的联系。

第三章 \quad 古典政治经济学中的生产力概念与价值理论

该部分主要对比分析亚当斯密(重点)、李嘉图(重点)、马尔萨斯、约翰穆勒的价值理论及生产力概念。

第四章 \quad 新古典经济学中的生产力概念与价值理论

该部分主要分析两种边际价值论和马歇尔的均衡价值论及各自理论中的生产力概念。

第五章 \quad 斯拉法价值论

该部分主要介绍斯拉法的价值理论,并浅析了其理论中的生产力概念。

第六章 \quad 价值理论演进与新质生产力理论初探

该部分对价值理论和生产力概念进行全面回顾,并试着对新质生产力的多维内涵进行探讨。

第七章 \quad 结论

该部分总结全文的研究成果,并针对本文的局限展望未来研究方向。

% !TeX root = ../2019080346_Mason.tex

\chapter{广义价值论的基本原理}

\section{广义价值论简介}

蔡继明在1985年\cite{CaiJiMingBiJiaoLiYiShuoYuLaoDongJieZhiLun1985}首次阐述了广义价值论的思想。该理论吸收了李嘉图国际贸易理论中的比较优势原理,并将其应用范围从国际贸易延伸至普遍的分工交换领域,使价值决定成为分工交换的内在机制,揭示了机会成本对价值决定的核心作用。由此,经济思想史中的各种价值理论成为一种特例被整合到了统一的框架之中,广义价值论因此得名\cite[221]{LiRenJunJieZhiLiLun2004}。

下文将阐述广义价值论的基本原理。

\section{分工交换}

广义价值论是建立在对社会分工交换的认识之上的。

\subsection{比较利益是社会分工和交换产生的条件}

广义价值论认为:第一,分工与交换本质上是同一经济现象的两个侧面,二者并无因果关联;第二,比较利益(Comparative Benefit, $\mathit{CB}$)的存在是分工与交换能够进行的根本原因。比较利益是指一个生产者通过交换所获收益同其自给自足状态下的收益之差,或者说是该生产者从交换所得收益与其放弃的产品所付出的机会成本(Opportuinity Cost,$\mathit{OC}$)之差。这里的比较利益既可以是劳动量的节约,也能是效用的提升。只有当双方的比较利益同时为正时,社会分工和交换才有可能出现并维持\cite[32]{CaiJiMingLunFenGongYuJiaoHuanDeQiYuanHeJiaoHuanBiLiDeQueDingGuangYiJieZhiLunGangShang1999}。

事实上,广义价值论正是建立在这样一个基本公理之上:
\begin{axiom}
    \label{Bijiaoliyishishehuifengonghejiaohuanchanshengdetiaojian}
    比较利益是社会分工和交换产生的条件。
\end{axiom}

为了进一步揭示比较利益的来源,需要引入几种生产力概念。

\subsection{绝对生产力与相对生产力}

为了引入生产力的概念,首先需要阐释使用价值。按照马克思的观点,使用价值来源于商品的有用性,其就是商品体本身,是构成财富的物质的内容,\cite[48-49]{ZhongGongZhongYangMaKeSiEnGeSiLieNingSiDaLinZhuZuoBianYiJuMaKeSiEnGeSiWenJiDi5Juan2009}。在此基础上,我们可以给出绝对生产力(Absolute Productivity)的定义:

\begin{definition}
    绝对生产力是指单位劳动耗费所生产的使用价值量\cite[47]{CaiJiMingCongXiaYiJieZhiLunDaoGuangYiJieZhiLunXiuDingBan2022}
\end{definition}

令$q_{ij}$表示生产者$i$在耗费单位劳动所生产产品$j$的使用价值量,也就是生产者$i$的绝对生产力;$t_{ij}$表示生产者$i$在生产单位使用价值的产品$j$所必需要耗费的劳动\footnote{既包括活劳动也包括物化劳动,物化劳动本身不能创造价值而只能转移价值,但可以使活劳动取得自乘的效果\cite{ChengEnFuXinDeHuoLaoDongJieZhiYiYuanLunLaoDongJieZhiLiLunDeDangDaiTuoZhan2001}}量(用劳动时间来衡量\cite[51]{ZhongGongZhongYangMaKeSiEnGeSiLieNingSiDaLinZhuZuoBianYiJuMaKeSiEnGeSiWenJiDi5Juan2009}),则有:
\begin{equation}
    \label{jueduishengchanli}
    q_{ij}=\frac{1}{t_{ij}}
\end{equation}

马克思认为社会必要劳动时间是在社会平均生产条件下制造单位使用价值所必需的劳动时间\cite[52]{ZhongGongZhongYangMaKeSiEnGeSiLieNingSiDaLinZhuZuoBianYiJuMaKeSiEnGeSiWenJiDi5Juan2009}。类似地,我们也可以称$t_{ij}$为个体必要劳动时间。如果我们把生产者i替换为一个部门,则我们也可以称之为部门必要劳动时间。那么,个体必要劳动时间该如何转化为部门必要劳动时间呢?从数值上看,部门必要劳动时间是个体必要劳动时间以使用价值产量为权重的加权平均\cite[53]{LinGangGuanYuSheHuiBiYaoLaoDongShiJianYiJiLaoDongShengChanLuYuJieZhiLiangGuanXiWenTiDeTanTao2005};从机制上看,这种转换是某一特定部门内部在市场上的相互竞争带来的。

基于此,我们可以定义绝对优势(Absolute Advantage)的概念:

\begin{definition}
    绝对优势是指一个生产者在特定产品上拥有绝对优势是指该生产者在特定产品上拥有较高的绝对生产力。
\end{definition}

绝对生产力也就是马克思所认为的“劳动生产力”。马克思认为“生产力当然始终是有用的、具体的劳动的生产力,它事实上只决定有目的的生产活动在一定时间内的效率”\cite[59]{ZhongGongZhongYangMaKeSiEnGeSiLieNingSiDaLinZhuZuoBianYiJuMaKeSiEnGeSiWenJiDi5Juan2009},这和绝对生产力的定义是相符的。因此,那些影响“劳动生产力”的因素也就是影响绝对生产力的因素,这些因素包括:“工人的平均熟练程度,科学的发展水平和它在工艺上应用的程度,生产过程的社会结合,生产资料的规模和效能,以及自然条件”\cite[53]{ZhongGongZhongYangMaKeSiEnGeSiLieNingSiDaLinZhuZuoBianYiJuMaKeSiEnGeSiWenJiDi5Juan2009}。

由于不同的使用价值是异质的,所以不同产品对应的绝对生产力有着不同的单位,是不能相互进行比较的。但是,借助于相对生产力(Relative Productivity,$\mathit{RP}$)的概念却可以进行间接的比较。具体来说:

\begin{definition}
    相对生产力是指一个生产者在不同商品上的绝对生产力之比\cite[48]{CaiJiMingCongXiaYiJieZhiLunDaoGuangYiJieZhiLunXiuDingBan2022}。
\end{definition}

从公式上看,假设市场上有商品1、2,生产者1、2,则生产者1的相对生产力$\mathit{RP}_1$和生产者2的相对生产力$\mathit{RP}_2$分别为
\begin{equation}
    \begin{cases}
        \mathit{RP}_1=\frac{q_{11}}{q_{12}}\\
        \mathit{RP}_2=\frac{q_{21}}{q_{22}}
    \end{cases}
\end{equation}

然而,相对生产力的概念并没有消除使用价值的异质性,因此我们不能直接比较两个生产者的相对生产力。然而,我们仍然希望能分析一个生产者的比较优势(Comparitive Advantage),也就是:

\begin{definition}
    比较优势是指一个生产者生产一种产品的绝对生产力与自身生产另一种产品的绝对生产力之间的相对大小\cite[49]{CaiJiMingCongXiaYiJieZhiLunDaoGuangYiJieZhiLunXiuDingBan2022}。
\end{definition}

为了进行这样的分析,我们需要定义相对生产力系数(Relative Productivity Coefficient)的概念如下:

\begin{definition}
    相对生产力系数是指一个生产者的相对生产力与另一个生产者的相对生产力之比\cite[48]
    {CaiJiMingCongXiaYiJieZhiLunDaoGuangYiJieZhiLunXiuDingBan2022},即
    \begin{equation}
        \mathit{RP}_{1/2} = \frac{q_{11}/q_{12}}{q_{21}/q_{22}} = \frac{q_{11}q_{22}}{q_{21}q_{12}}
    \end{equation}
\end{definition}

相对生产力系数在分子分母上同时出现了两种使用价值的乘积,将反映使用价值异质性的量纲约分后,使用价值的异质性已经被消除,且相对生产力系数是无量纲的,其大小直接反映了生产者的比较优势,具体如表\ref{table:RP_{1/2}}\footnote{引号代表相对大小}所示。

\begin{table}
    \centering
    \caption{相对生产力系数的含义\cite[48-49]{CaiJiMingCongXiaYiJieZhiLunDaoGuangYiJieZhiLunXiuDingBan2022}}
    \label{table:RP_{1/2}}
    \begin{tabular}{|l|l|}
    \hline
        $\mathit{RP}_{1/2}<1$ & $q_{11} \quad "<" \quad q_{12}; \quad q_{21} \quad ">" \quad q_{22}$ \\ \hline
        $\mathit{RP}_{1/2}=1$ & $q_{11} \quad "=" \quad q_{12}; \quad q_{21} \quad "=" \quad q_{22}$ \\ \hline
        $\mathit{RP}_{1/2}>1$ & $q_{11} \quad ">" \quad q_{12}; \quad q_{21} \quad "<" \quad q_{22}$ \\ \hline
    \end{tabular}
\end{table}

\subsection{比较优势是分工交换的充要条件}

现有的文献使用了几何的方法来证明比较优势是分工交换的充要条件。接下来,笔者将试着用代数的方式给出更为严谨的证明。

我们用使用价值来衡量比较利益\cite[63]{CaiJiMingGuangYiJieZhiLun2001},令$T_1$、$T_2$分别为生产者1、2为了交换而投入的劳动量,则生产者1、2的比较利益分别为\footnote{原文中这里有笔误}\footnote{这里假设了线性生产可能性边界,如果假设为非线性生产可能性边界也不会改变结论,只是增加了分析难度\cite[285]{LiRenJunJieZhiLiLun2004}。}
\begin{equation}
    \mathit{CB}_1 = q_{22}T_2 - q_{12}T_1 ; \quad \mathit{CB}_2 = q_{11}T_1 - q_{21}T_2  
\end{equation}

则如果我们可以证明当$\mathit{RP}_{1/2} \neq 1$时,两个生产者能够通过分工交换获得正的比较利益,就可以推断出相对生产力差别决定的比较优势带来了比较利益。下面,我们证明当$RP_{1/2} > 1$时,生产者1、2可以通过分工交换(生产者1生产产品1而生产者2生产产品2)获得正的比较利益。这等价于证明:

\begin{proposition}
    已知正数 $ q_{11}, q_{12}, q_{21}, q_{22} $ 满足 $\frac{q_{11}q_{22}}{q_{12}q_{21}} > 1$,存在正数 $T_1,T_2 > 0$ 使得以下两个不等式同时成立:
    $$
        \begin{cases}
            q_{22}T_2 - q_{12}T_1 > 0, \\
            q_{11}T_1 - q_{21}T_2 > 0.
        \end{cases}
    $$    
\end{proposition}

\begin{proof}
    令 $ r = \frac{T_2}{T_1} $,则 $ r > 0 $,可将原不等式改写为:
    $$
    \begin{cases}
        q_{22}r > q_{12} \quad \Rightarrow \quad r > \frac{q_{12}}{q_{22}}, \\
        q_{11} > q_{21}r \quad \Rightarrow \quad r < \frac{q_{11}}{q_{21}}.
    \end{cases}
    $$

    由 $\frac{q_{11}q_{22}}{q_{12}q_{21}} > 1$ ,可得
    $$
        \frac{q_{11}}{q_{21}} > \frac{q_{12}}{q_{22}}.
    $$

    因此区间 $\left( \frac{q_{12}}{q_{22}}, \frac{q_{11}}{q_{21}} \right)$ 非空,存在 $ r $ 满足
    $$
        \frac{q_{12}}{q_{22}} < r < \frac{q_{11}}{q_{21}}.
    $$

    具体地, $r = \frac{1}{2}\left( \frac{q_{12}}{q_{22}} + \frac{q_{11}}{q_{21}} \right)$ 必是一个解。

    取$T_1 = 1$,则$T_2 =  r T_1$,则已找到正数$T_1,T_2 > 0$。
\end{proof}

这样,根据前文的公理\ref{Bijiaoliyishishehuifengonghejiaohuanchanshengdetiaojian},我们得到了比较优势是分工交换的充分条件。同样,我们也可以证明比较优势是分工交换的必要条件如下:

\begin{proposition}
    已知正数 $ q_{11}, q_{12}, q_{21}, q_{22} $,若存在正数 $ T_1, T_2 > 0 $ 使得以下两个不等式同时成立:
    $$
        \begin{cases}
            q_{22}T_2 - q_{12}T_1 > 0, \\
            q_{11}T_1 - q_{21}T_2 > 0,
        \end{cases}
    $$
    则必有 $\displaystyle \frac{q_{11}q_{22}}{q_{12}q_{21}} > 1$。
\end{proposition}

\begin{proof}
    令 $ r = \frac{T_2}{T_1} $,由 $ T_1, T_2 > 0 $ 知 $ r > 0 $。将原不等式改写为:
    $$
        \begin{cases}
            q_{22}r > q_{12} \quad \Rightarrow \quad r > \frac{q_{12}}{q_{22}}, \\
            q_{11} > q_{21}r \quad \Rightarrow \quad r < \frac{q_{11}}{q_{21}}.
        \end{cases}
    $$

    由于存在 $ r > 0 $ 满足 $\frac{q_{12}}{q_{22}} < r < \frac{q_{11}}{q_{21}}$,这表明区间 $\left( \frac{q_{12}}{q_{22}}, \frac{q_{11}}{q_{21}} \right)$ 非空,因此有:
    $$
        \frac{q_{11}}{q_{21}} > \frac{q_{12}}{q_{22}}.
    $$

    将不等式两边同乘以正数 $ q_{21}q_{22} $,得:
    $$
        q_{11}q_{22} > q_{12}q_{21}.
    $$

    因此,$\displaystyle \frac{q_{11}q_{22}}{q_{12}q_{21}} > 1$,命题得证。

\end{proof}

这样,我们就得到了比较优势和社会分工交换之间的关系如下:

\begin{theorem}
    比较优势是社会分工交换的充要条件。
\end{theorem}

在接下来的分析中,我们会引入商品交换价值的概念,由于我们已经得到:相对生产力系数与$1$的大小关系决定了分工交换的方向,所以相对生产力系数也就决定了商品交换价值的单位。

\subsection{均衡交换比例的确定}

以上分析表明,比较优势是社会分工与交换得以形成的根本原因。随之而来的问题是,不同商品之间的交换比例如何决定?按照马克思的定义,确定这种交换比例也就确定了商品的交换价值\cite[49]{ZhongGongZhongYangMaKeSiEnGeSiLieNingSiDaLinZhuZuoBianYiJuMaKeSiEnGeSiWenJiDi5Juan2009}。

为了分析交换比例,我们引入记号$r_{2/1}=\frac{x_2}{x_1}$代表商品1、2的市场交换比例,同时仍然假定生产者1用商品1交换由生产者2生产的商品2。根据公理\ref{Bijiaoliyishishehuifengonghejiaohuanchanshengdetiaojian},我们可以得到交换必须满足的条件:对于生产者1来说,生产者1用$q_{11}$单位的商品1所换到的商品2的量必须大于其机会成本$q_{12}$,而且要小于生产者2为换取商品1而付出的机会成本$q_{21}$;对于生产者2来说,生产者2用$q_{22}$单位的商品2换取到的商品1的量必须大于其机会成本$q_{21}$,而且要小于生产者1为换取商品2而付出的机会成本$q_{12}$\cite[64]{CaiJiMingCongXiaYiJieZhiLunDaoGuangYiJieZhiLunXiuDingBan2022}。若用公式表达,则有\cite[64]{CaiJiMingCongXiaYiJieZhiLunDaoGuangYiJieZhiLunXiuDingBan2022}:
\begin{equation}
    \label{jiaohuantiaojian}
    \frac{q_{12}}{q_{11}} = \frac{t_{11}}{t_{12}} < r_{2/1} < \frac{t_{21}}{t_{22}} = \frac{q_{22}}{q_{21}}
\end{equation}

原则上,任何满足式\ref{jiaohuantiaojian}的交换比例都是可以达到的,但是交换比例的变动也意味着比较利益在生产者1、2之间的分配比例变动,故只有当交换比例落在一个合理值上或围绕某个合理值波动时,交换才能长久地进行下去。那么,什么样的值才是合理的呢?广义价值论的缔造者提出了下述公理(来自于对人的行为假设\cite[413]{LiRenJunGuangYiJieZhiLunDeLuoJiYuZhengLun2009}):
\begin{axiom}
    当市场处于供求均衡状态时,两个生产者的交换比例满足比较利益率均等的原则\cite[65]{CaiJiMingCongXiaYiJieZhiLunDaoGuangYiJieZhiLunXiuDingBan2022}。
\end{axiom}

其中,
\begin{definition}
    比较利益率是比较利益与机会成本的比率\cite[65]{CaiJiMingCongXiaYiJieZhiLunDaoGuangYiJieZhiLunXiuDingBan2022}。
\end{definition}

为了分析的简便,我们用时间成本来表示比较利益率,有:
\begin{equation}
    \label{bijiaoliyilv}
    \mathit{CB}^{\prime}_1 = \frac{x_2 t_{12} - x_1 t_{11}}{x_1 t_{11}}; \quad \mathit{CB}^{\prime}_2 = \frac{x_2 t_{21} - x_1 t_{22}}{x_1 t_{22}}
\end{equation}

若我们用$r^{\mathit{min}}_{2/1}$和$r^{\mathit{max}}_{2/1}$分别表示最低和最高交换比例$\mathit{R}_{2/1}$表示均衡交换比例;$\mathit{CB}^{\prime}_i$表示生产者$i$的比较利益率,$\mathit{CB}^{\prime}_{i=j}$表示平均比较利益率,则我们可以得到\cite[66]{CaiJiMingCongXiaYiJieZhiLunDaoGuangYiJieZhiLunXiuDingBan2022}:

\begin{equation}
    \begin{cases}
        0 = \frac{q_{11}}{q_{12}}r^{\mathit{min}}_{2/1} - 1 < \mathit{CB}^{\prime}_1 < \frac{q_{11}}{q_{12}}r^{\mathit{max}}_{2/1} - 1 = \mathit{RP}_{1/2} - 1 \\
        0 = \frac{q_{22}}{q_{12}r^{\mathit{max}}_{2/1}} - 1 < \mathit{CB}^{\prime}_2 < \frac{q_{22}}{q_{12}r^{\mathit{min}}_{2/1}} - 1 = \mathit{RP}_{1/2} - 1
    \end{cases}
\end{equation}

在均衡状态下:
\begin{equation}
    \mathit{CB}^{\prime}_1 = \mathit{CB}^{\prime}_2 = \mathit{CB}^{\prime}_{i=j}
\end{equation}

代入式\ref{bijiaoliyilv},我们可以得到:
\begin{equation}
    \label{junhengjiaohuanbili}
    \mathit{R}_{2/1} = \sqrt{\frac{t_{11}t_{21}}{t_{12}t_{22}}} = \sqrt{\frac{q_{12}q_{22}}{q_{11}q_{21}}}
\end{equation}

\section{社会平均生产力}

观察上式,我们发现$\sqrt{t_{11}t_{21}}$和$\sqrt{t_{12}t_{22}}$分别是两个生产者(部门)生产商品1和商品2的个体(部门)必要劳动时间的几何平均。因此,我们可以接着定义:
\begin{definition}
    某一商品的社会必要劳动时间是生产某一商品所有的部门必要劳动时间的几何平均\cite[260]{CaiJiMingCongGuDianZhengZhiJingJiXueDaoZhongGuoTeSeSheHuiZhuYiZhengZhiJingJiXueJiYuZhongGuoShiJiaoDeZhengZhiJingJiXueYanBianShangCe2023}。
\end{definition}

然而,广义价值论中的社会必要劳动时间概念和马克思笔下的“社会必要劳动时间”有着本质区别。接下来笔者将指出:马克思笔下的“社会必要劳动时间”在广义价值论的体系中只能算是部门必要劳动时间,而且这种差别源自对分工体系的不同认识。

\subsection{不同的分工体系与“社会必要劳动时间”}

分工体系的存在密切依赖于劳动的异质性特征。​​ 劳动异质性不仅体现为不同劳动者在生产力水平上的相对差距,也体现在他们所掌握的生产力种类的差异上。根据生产者能否调整其专业化分工方向,可进一步将分工体系划分为三大类型\cite[111]{CaiJiMingCongXiaYiJieZhiLunDaoGuangYiJieZhiLunXiuDingBan2022}。

第一类是不变分工体系,也就是生产者分工方向唯一且固定不可改变的分工体系\cite[55-56]{CaiJiMingGuangYiJieZhiLun2001}。在这类分工体系中,某一生产者只能生产特定的商品,而在其它商品上的生产力为0。例如,一个经济部门甲依山而居,可以种植苹果;另一个经济部门乙是海岛居民,可以出海捕鱼。那么甲在海鱼上的生产力为0,乙在水果上的生产力为0,这就是一个典型的不变分工体系。

第二类是可变分工体系,也就是是生产者分工方向可以改变的分工体系\cite[55-56]{CaiJiMingGuangYiJieZhiLun2001},在这类分工体系中,分工的方向取决于相对生产力系数$\mathit{RP}_{1/2}$的取值,生产者会生产自己占有比较优势的商品,但也具备生产其它商品的潜在的能力\footnote{或者同时还生产其它产品但仅用于自给自足}。例如,在新中国刚成立时,我国连一辆汽车都不能造,但到2023年,我国已是全球第一大汽车出口大国\cite{HuangXinZhongGuoZhiZaoQiangJinZhuangGuYouDaXiangQiang2024}。这正反映了我国在汽车上的生产力从0到1再不断提高的过程,是可变分工体系的具体表现。

第三类是混合分工体系本质上是前两类体系的复合形态,其特征表现为:​在全体生产者中,至少存在一方的分工方向固定不变,而其余各方均具备调整分工方向的能力\cite[113]{CaiJiMingCongXiaYiJieZhiLunDaoGuangYiJieZhiLunXiuDingBan2022}。

马克思在研究一国之内的商品交换时,默认了生产某一特定商品的只能是某一特定的部门。因此,按照不变分工体系的假设,这一特定商品的“社会必要劳动时间”指的是商品生产部门而非整个社会的必要劳动时间。然而,根据广义价值论的定义,社会必要劳动时间还包含了可以生产这一特定商品的、潜在的生产部门的部门必要劳动时间,也就是包含了实际上并不生产该特定商品的部门在该特定商品上的部门必要劳动时间。这便是广义价值论中的社会必要劳动时间概念和马克思笔下的“社会必要劳动时间”的本质区别。

\subsection{从社会必要劳动时间到社会平均生产力}

既然我们已经清楚地定义了社会必要劳动时间的含义,由式\ref{jueduishengchanli},我们可以定义:
\begin{definition}
    社会平均生产力是生产某一商品所有部门的绝对生产力的几何平均。
\end{definition}

类似地,这里的社会平均生产力也包含了那些“潜在”生产部门的生产力。

记$\mathit{AP}_1 = \sqrt{q_{11}q_{21}}$和$\mathit{AP}_2 = \sqrt{q_{12}q_{22}}$是商品1、2的社会平均生产力,则类似于相对生产力系数的概念,我们可以定义:
\begin{definition}
    社会平均生产力系数是两种商品的社会平均生产力之比\cite[68]{CaiJiMingCongXiaYiJieZhiLunDaoGuangYiJieZhiLunXiuDingBan2022}。
\end{definition}

记$\mathit{AP}_{2/1} = \sqrt{\frac{q_{12}q_{22}}{q_{11}q_{21}}}$,我们可以重写式\ref{junhengjiaohuanbili}如下\cite[68]{CaiJiMingCongXiaYiJieZhiLunDaoGuangYiJieZhiLunXiuDingBan2022}:
\begin{equation}
    \mathit{R} = \frac{x_2}{x_1} = \sqrt{\frac{q_{12}q_{22}}{q_{11}q_{21}}} = \mathit{AP}_{2/1}
\end{equation}

我们还可以写出平均比较利益率如下:
\begin{equation}
    \mathit{CB}^{\prime}_{1=2} = \sqrt{\mathit{RP}_{1/2}} - 1
\end{equation}

\section{价值决定的一般原理}

\subsection{价值的一般定义}

在前文中,我们已经确定了交换价值,即一种使用价值与另一种使用价值相交换的比例。现在,笔者先给出广义价值论中价值的定义,并将在本文第三章的总结部分探讨这一定义的发展历程。

\begin{definition}
    价值是调节商品交换价值(或价格)运动的一般规律\cite[6]{CaiJiMingCongXiaYiJieZhiLunDaoGuangYiJieZhiLunXiuDingBan2022}。
\end{definition}

\subsection{单位商品价值的决定}

在对价值概念进行了明确的定义后,我们继续对分工交换的分析。

根据马克思的价值规律,商品交换是等价值的交换。因此,我们记单位商品1、2的价值分别为$V^c_1$和$V^c_2$,则有\cite[70]{CaiJiMingCongXiaYiJieZhiLunDaoGuangYiJieZhiLunXiuDingBan2022}:
\begin{equation}
    \label{dengjiajiaohuan}
    x_1 t_{11} = V^c_1 x_1 = V^c_2 x_2 = x_2 t_{22}
\end{equation}

将其代入均衡交换比例,即式\ref{junhengjiaohuanbili},我们可以得到\cite[70-71]{CaiJiMingCongXiaYiJieZhiLunDaoGuangYiJieZhiLunXiuDingBan2022}:
\begin{equation}
    \begin{cases}
        V^c_1 = \frac{1}{2}\left(t_{11}+t_{22}\sqrt{\frac{t_{11} t_{21}}{t_{22} t_{12}}}\right)=\frac{1}{2}\left(\frac{1}{q_{11}}+\frac{1}{q_{22}}\sqrt{\frac{q_{12}q_{22}}{q_{11}q_{21}}}\right) \\
        V^c_2 =\frac{1}{2}\left(t_{22}+t_{11}\sqrt{\frac{t_{22} t_{12}}{t_{11} t_{21}}}\right) = \frac{1}{2}\left( \frac{1}{q_{22}}+\frac{1}{q_{11}}\sqrt{\frac{q_{11}q_{21}}{q_{12}q_{22}}}\right)
    \end{cases}
\end{equation}

\begin{equation}
    \label{danweishangpingdejiazhi}
    \begin{cases}
        V^c_1 = \frac{t_{11}}{2}\left(1+\sqrt{\frac{t_{21}t_{22}}{t_{11}t_{12}}}\right)=\frac{1}{2q_{11}}\left(1+\sqrt{\frac{q_{11}q_{12}}{q_{22}q_{21}}}\right) \\

        V^c_2 = \frac{t_{22}}{2}\left(1+\sqrt{\frac{t_{11}t_{12}}{t_{21}t_{22}}}\right)=\frac{1}{2q_{22}}\left(1+\sqrt{\frac{q_{21}q_{22}}{q_{12}q_{11}}}\right)
    \end{cases}
\end{equation}

\subsection{综合生产力}

这里,我们发现出现了$q_{21}q_{22}$等不同绝对生产力的交叉项,因此我们引入一个新的概念——综合生产力(comprehensive productivity,$cp$):
\begin{definition}
    综合生产力是指同一部门在不同商品上的绝对生产力的几何平均\cite[81]{CaiJiMingCongXiaYiJieZhiLunDaoGuangYiJieZhiLunXiuDingBan2022}。
\end{definition}

一个生产者的综合生产力反映了其在不同商品生产能力上的整体水平\cite[81]{CaiJiMingCongXiaYiJieZhiLunDaoGuangYiJieZhiLunXiuDingBan2022}。我们记生产者1、2的综合生产力分别为$ \mathit{cp}_1 = \sqrt{q_{11}q_{12}} $和$ \mathit{cp}_2 = \sqrt{q_{21}q_{22}} $。接着我们可以定义:

\begin{definition}
    综合生产力系数是两部门综合生产力的比率\cite[71]{CaiJiMingCongXiaYiJieZhiLunDaoGuangYiJieZhiLunXiuDingBan2022}。
\end{definition}

由于综合生产力系数在价值决定中起了决定性的作用,因此广义价值论给综合生产力系数赋予了专门的名称——比较生产力。


\subsection{比较生产力}

\subsubsection{比较生产力的概念}

首先,广义价值论的提出者借鉴了斯拉法“合成商品”的概念,即把商品1、2视为一种组合商品3,则商品3就是一种抽象意义上的组合商品\cite[293]{CaiJiMingGuangYiJieZhiLun2001}。这种组合商品对于两个生产者而言是同质的东西,因而也就可以比较两个生产者在组合商品上的绝对生产力大小。换句话说,我们实际上是分别把生产者1、2的综合生产力看作是生产者1、2在组合商品上的绝对生产力,为了比较两者的相对大小,便称两者的比值为比较生产力。因此,比较生产力的定义为:
\begin{definition}
    比较生产力是两部门综合生产力的比率,表示两部门在已确定的专业化生产上相比较而言的生产力\cite[264]{CaiJiMingCongGuDianZhengZhiJingJiXueDaoZhongGuoTeSeSheHuiZhuYiZhengZhiJingJiXueJiYuZhongGuoShiJiaoDeZhengZhiJingJiXueYanBianShangCe2023}。
\end{definition}

从公式上来看,记$ \mathit{CP}_{1/2} $为生产者1相对于生产者2的比较生产力,则
\begin{equation}
    \mathit{CP}_{1/2} = \frac{\sqrt{q_{11} q_{12}}}{\sqrt{q_{21}q_{22}}} = \sqrt{\frac{q_{11} q_{12}}{q_{21}q_{22}}}
\end{equation}

由于比较生产力实际上也就是综合生产力系数\cite[264]{CaiJiMingCongGuDianZhengZhiJingJiXueDaoZhongGuoTeSeSheHuiZhuYiZhengZhiJingJiXueJiYuZhongGuoShiJiaoDeZhengZhiJingJiXueYanBianShangCe2023},而且综合生产力反映的是某一生产者的整体生产能力,那么比较生产力作为综合生产力的比率反映的就是两个生产者整体生产能力的相对大小。例如,发达国家的整体生产能力高于发展中国家的整体生产能力,故$ \mathit{CP}_{\text{发达}/\text{发展}} > 1 $。

进一步看,我们可以将比较生产力写成如下形式:
\begin{equation}
    \mathit{CP}_{1/2} = \sqrt{\frac{q_{11}q_{12}}{q_{21}q_{22}}} = \sqrt{\frac{\left(\frac{q_{11}}{q_{22}}\right)}{\left(\frac{q_{21}}{q_{12}}\right)}} 
\end{equation}

这样我们就可以看到比较生产力是$ \sqrt{\sfrac{q_{11}}{q_{22}}} $和$\sqrt{\sfrac{q_{21}}{q_{22}}}$的比值,所以我们也可以认为的另一层含义是在比较优势已经确定的前提下,一个生产者在一种产品生产上与另一个生产者在另一种产品生产上相比较而言的生产力。

\subsubsection{比较生产力与价值决定}

在理清了比较生产力和比较生产力系数的概念之后,我们尚不清楚其实际的经济学含义。为了进一步分析,我们首先分析比较生产力系数对价值决定的影响。

我们将比较生产力系数代入式\ref{danweishangpingdejiazhi}得到:
\begin{equation}
    \begin{cases}
        V_1^c = \frac{1}{2q_{11}} \left( 1 + \mathit{CP}_{1/2} \right) = \frac{t_{11}}{2} \left( 1 + \mathit{CP}_{1/2} \right) \\
        V_2^c = \frac{1}{2q_{22}} \left( 1 + \mathit{CP}_{2/1} \right) = \frac{t_{22}}{2} \left( 1 + \mathit{CP}_{2/1} \right)
    \end{cases}
\end{equation}

不难看出,单位商品的价值取决于绝对生产力(或者单位劳动时间)和比较生产力的大小,如果比较生产力大于1,则单位商品价值大于单位劳动时间。这样来看,劳动价值论就成为比较生产力为1时的一个特例\cite[72]{CaiJiMingCongXiaYiJieZhiLunDaoGuangYiJieZhiLunXiuDingBan2022}。

另外,通过对该式取微分,我们可以得到\cite[93]{CaiJiMingCongXiaYiJieZhiLunDaoGuangYiJieZhiLunXiuDingBan2022}:
\begin{equation}
    \frac{\partial V_1^c}{\partial q_{11}} = \frac{1}{2} \left[ -q_{11}^2 \left( 1 + \mathit{CP}_{1/2} \right) + \frac{\partial \mathit{CP}_{1/2}}{q_{11}\partial q_{11}} \right] < 0
\end{equation}

所以有:

\begin{theorem}
    单位商品价值与绝对生产力负相关,与比较生产力正相关\cite[92]{CaiJiMingCongXiaYiJieZhiLunDaoGuangYiJieZhiLunXiuDingBan2022}。
\end{theorem}

现在,我们基于这个结果来分析比较生产力的经济学含义。从直观上看,既然$\mathit{CP}_{1/2} = \frac{q_{11}q_{12}}{q_{21}q_{22}}$,那么如果生产者1拥有更高的综合生产力,则$ \mathit{CP}_{1/2} > 1 $,故可以推断生产者1拥有更高的比较生产力。这样,我们是否可以猜测:生产者1在分工交换的过程中应当会比生产者2更有“优势”,所以说一定量生产者1的劳动生产的商品可以换取生产者2的用更多劳动生产的商品。

我们还可以从比较生产力的定义中探索其影响价值的机制。

首先,我们假设生产者1在商品1的生产上存在比较优势,生产者2在商品2的生产上存在比较优势($ \mathit{RP}_{1/2} > 1 $),那么比较生产力的定义式中的各项含义如下:
\begin{equation}
    \mathit{CP}_{1/2} = \frac{\left(\frac{q_{11}}{q_{22}}\right)}{\left(\frac{q_{21}}{q_{12}}\right)} = \frac{\left(\frac{\text{生产力1}}{\text{生产力2}}\right)}{\left(\frac{\text{机会成本2}}{\text{机会成本1}}\right)}
\end{equation}

现在,我们假设$ q_{12} $上升,也就是生产者1的机会成本上升。由于$ q_{12} $位于分母的分母上,所以$\mathit{CP}_{1/2} $会上升,进而生产者1生产的单位商品1的价值会上升,同时生产者2生产的单位商品2的价值会下降,也就是说:

\begin{proposition}
    假定其它条件不变,在不改变比较优势的前提下,某个生产者通过提升自己的机会成本,可以增加自己实际生产的单位商品的价值量,减小另外一个生产者实际生产的单位商品价值量。
\end{proposition}

如果$ q_{12} $继续上升,那么生产者1和生产者2的相对生产力系数$ \mathit{RP}_{1/2} = \frac{q_{11}q_{22}}{q_{21}q_{12}} $会不断减小,最终小于1(由于假设生产者1在商品1的生产上拥有比较优势,故原来$ \mathit{RP}_{1/2} > 1 $),即生产者1不再拥有在商品1生产上的比较优势。也就是说:

\begin{proposition}
    假定其它条件不变,某个生产者通过提升自己的机会成本,会减小自己的比较优势。
\end{proposition}

至此,笔者作出如下猜测:

\begin{conjecture}
    比较生产力衡量了一个生产者相对于另一个生产者的相对比较优势大小,而且一个生产者相对于另一个生产者的比较优势越小,比较生产力系数越大。
\end{conjecture}

同时,根据前文的分析,笔者进一步猜测:

\begin{conjecture}
    假定其它条件不变,某个生产者的相对比较优势越小,则其实际生产的单位商品的价值量越高。
\end{conjecture}

同时,笔者注意到平均比较利益率$ \mathit{CB}^{\prime}_{1=2} = \sqrt{\mathit{RP}_{1/2}} - 1 $。也就是说,如果$ q_{12} $上升,$ \mathit{RP}_{1/2} $减小,那么两个生产者进行分工交换时的平均比较利益率$ \mathit{CB}^{\prime}_{1=2} $也会减小。

然而,受限于时间和文章篇幅,笔者在此停止对比较生产力的分析,笔者会在将来的研究中对比较生产力系数进行更深入的研究。

\subsection{不同主体的劳动耗费与价值决定}

到此,我们已经分析了单位商品的价值量和价值决定机制。然而,到此为止我们都假设分工交换是发生在生产者之间或是部门之间的,我们的分析对个体、部门、社会没有做出明确的区分。但无论是在政治经济学的分析范式中,还是现实生活中,这三个经济主体之间是截然不同的。因此,广义价值论按照个体、部门、社会的顺序分析了这三个部门的投入(劳动耗费)与价值产出之间的关系。在开始分析之前,笔者要强调:现在我们明确把分工交换视为两个部门之间的分工交换,把之前的所有生产力概念都看作是部门层面的生产力概念。

\subsubsection{单位个别劳动创造的价值量}
首先,我们来分析个体的单位劳动投入所创造的价值量。我们定义:

\begin{definition}
    绝对生产力差别系数是一个生产者在某一商品上的绝对生产力与其所在部门在该商品上的平均生产力之比\cite[73]{CaiJiMingCongXiaYiJieZhiLunDaoGuangYiJieZhiLunXiuDingBan2022}。
\end{definition}

从公式上看,我们记$q_{ij}^k$为生产者$k$在第$i$部门第$j$商品上的绝对生产力差别系数,$q_{ijk}$为该生产者的劳动生产力,再记$V_{ijk}^{l}$为生产者$k$在第$i$部门第$j$产品上单位劳动创造的价值量,则我们有:
\begin{equation}
    \begin{cases}
        V_{11k}^{l} = q_{11k}V_{1}^{c} = \frac{1}{2}\frac{q_{11k}}{q_{11}}(1+\mathit{CP}_{1/2}) = \frac{1}{2}q_{11}^{k}(1+\mathit{CP}_{1/2}) \\
        V_{22k}^{l} = q_{22k}V_{2}^{c} = \frac{1}{2}\frac{q_{22k}}{q_{22}}(1+\mathit{CP}_{2/1})=\frac{1}{2}q_{22}^{k}(1+\mathit{CP}_{2/1})
    \end{cases}
\end{equation}

如果部门比较生产力上升,那么单位个别劳动创造的价值量显然会上升。但是如果个体绝对生产力$q_{ijk}$上升,则一方面其会使得部门绝对生产力$q_{ij}$提高产生对价值量的负效应,另一方面又会使得部门综合生产力提高产生对价值量的正效应。
对该式作全微分,我们有\cite[96]{CaiJiMingCongXiaYiJieZhiLunDaoGuangYiJieZhiLunXiuDingBan2022}:
\begin{equation}
    \frac{\dif V_{11k}^{l}}{\dif q_{11}^k} = \frac{1}{2} \left[  1 + \mathit{CP}_{1/2} + q_{11}^k \left( 1 + \frac{\dif \mathit{CP}_{1/2}}{\dif q_{11}^k} \right) \right] = \frac{1}{2} + \frac{1}{4} \sqrt{\frac{q_{12} q_{11k}}{q_{22} q_{21} q_{11}^k}} > 0
\end{equation}

所以,

\begin{theorem}
    单位个别劳动创造的价值量与其绝对生产力和部门比较生产力正相关\cite[95]{CaiJiMingCongXiaYiJieZhiLunDaoGuangYiJieZhiLunXiuDingBan2022}。
\end{theorem}

\subsubsection{部门劳动创造的价值量}

首先,部门单位平均劳动创造的价值量为:
\begin{equation}
    \begin{cases}
        V_1^l = q_{11} V_1^c = q_{11}\frac{1}{2q_{11}}\left( 1+\mathit{CP}_{1/2} \right) = \frac{1}{2}\left( 1+\mathit{CP}_{1/2} \right) \\
        V_2^l = q_{22} V_2^c = q_{22}\frac{1}{2q_{22}}\left( 1+\mathit{CP}_{2/1} \right) = \frac{1}{2}\left( 1+\mathit{CP}_{2/1} \right) 
    \end{cases}
\end{equation}

显然,我们有:

\begin{theorem}
    部门单位平均劳动创造的价值量与部门比较生产力正相关\cite[98]{CaiJiMingCongXiaYiJieZhiLunDaoGuangYiJieZhiLunXiuDingBan2022}。
\end{theorem}

其次,部门总劳动创造的价值量为:
\begin{equation}
    \begin{cases}
        V_1 = T_1q_{11}V_1^c = \frac{T_1}{2}\left( 1 + \mathit{CP}_{1/2} \right) \\
        V_2 = T_2q_{22}V_1^c = \frac{T_2}{2}\left( 1 + \mathit{CP}_{2/1} \right)
    \end{cases}
\end{equation}

显然,我们有:
\begin{theorem}
    部门总劳动创造的价值量与部门比较生产力正相关\cite[100]{CaiJiMingCongXiaYiJieZhiLunDaoGuangYiJieZhiLunXiuDingBan2022}。
\end{theorem}

\subsubsection{社会总劳动创造的价值量}

假设两部门经济均衡,则$ V_1 = V_2 $,则社会总价值量为:
\begin{equation}
    V = 2V_1 = T_1 + T_1 \mathit{CP}_{1/2}
\end{equation}

回顾式\ref{junhengjiaohuanbili}、\ref{dengjiajiaohuan},我们可以得到\cite[290]{CaiJiMingCongGuDianZhengZhiJingJiXueDaoZhongGuoTeSeSheHuiZhuYiZhengZhiJingJiXueJiYuZhongGuoShiJiaoDeZhengZhiJingJiXueYanBianShangCe2023}:
\begin{equation}
    \frac{T_2}{T_1} = \mathit{CP}_{1/2}
\end{equation}

这说明:

\begin{theorem}
    部门间必要劳动投入比取决于比较生产力系数(综合生产力系数)。
\end{theorem}

进一步推导可以得到:
\begin{equation}
    V = T_1 + T_2 = T
\end{equation}

这意味着:
\begin{theorem}
    社会价值总量等于社会劳动总量\cite[75]{CaiJiMingCongXiaYiJieZhiLunDaoGuangYiJieZhiLunXiuDingBan2022}\footnote{这里假定当期各部门综合生产力保持不变。}。
\end{theorem}

我们还可以考察跨期的社会价值总量。首先,我们定义:
\begin{definition}
    第$t$期的社会总和生产力是$t$期的各部门综合生产力的几何平均\cite[291]{CaiJiMingCongGuDianZhengZhiJingJiXueDaoZhongGuoTeSeSheHuiZhuYiZhengZhiJingJiXueJiYuZhongGuoShiJiaoDeZhengZhiJingJiXueYanBianShangCe2023}。
\end{definition}

从公式上来看,令$\mathit{TP}^t$表第$t$期的社会总和生产力,$CP_{it}$分别表示第$t$期第$i$部门的综合生产力,则第$t$期的社会总和生产力是\cite[291]{CaiJiMingCongGuDianZhengZhiJingJiXueDaoZhongGuoTeSeSheHuiZhuYiZhengZhiJingJiXueJiYuZhongGuoShiJiaoDeZhengZhiJingJiXueYanBianShangCe2023}:
\begin{equation}
    \mathit{TP}^t = \sqrt{\mathit{CP}_1^t \cdot \mathit{CP}_2^t} = \left( q_{11}^t q_{12}^t q_{21}^t q_{22}^t \right)^\frac{1}{4}
\end{equation}

令$g^t$表示第$t$期相对于第$t-1$期的总和生产力增长率,则有\cite[291]{CaiJiMingCongGuDianZhengZhiJingJiXueDaoZhongGuoTeSeSheHuiZhuYiZhengZhiJingJiXueJiYuZhongGuoShiJiaoDeZhengZhiJingJiXueYanBianShangCe2023}:
\begin{equation}
    g^t = \frac{\mathit{TP}^t - \mathit{TP}^{t-1}}{\mathit{TP}^{t-1}} = \left( \frac{q_{11}^t q_{12}^t q_{21}^t q_{22}^t}{q_{11}^{t-1} q_{12}^{t-1} q_{21}^{t-1} q_{22}^{t-1}} \right)^\frac{1}{4} - 1
\end{equation}

再令$m$为劳动力增长率,那么最后我们可以推导得到,社会价值总量的增长率$G$为:
\begin{equation}
    G = \left( 1+m \right) \left( 1+g \right) - 1 \approx m + g
\end{equation}

所以,我们可以得到:
\begin{theorem}
    社会价值总量与社会总和生产力正相关\cite[291]{CaiJiMingCongXiaYiJieZhiLunDaoGuangYiJieZhiLunXiuDingBan2022}。
\end{theorem}

\section{总结}

总而言之,我们可以将广义价值论中的各种生产力概念及其参与价值决定的机制总结为如下表格:

\begin{sidewaystable}[!h]
  \centering
  \caption{各种生产力与价值决定机制}
  \label{table:GVT}
  \begin{tabularx}{\textheight}{|c|>{\centering}p{3cm}|c|>{\centering\arraybackslash}X|>{\centering\arraybackslash}X|>{\centering\arraybackslash}X|} % 调整列宽为纵向高度
    \toprule
    生产力类型    & 具体变量    & 定义    & 经济学含义    & 商品价值决定机制    & 劳动价值决定机制\\ 
    \midrule
    
    绝对生产力    & 部门绝对生产力($q_{ij}$)    & $q_{ij}=\frac{1}{t_{ij}}$    & 部门生产使用价值的劳动效率    & 与单位商品价值负相关    & 与单位个别劳动创造价值正相关 \\ 
    \hline
    
    \multirow{2}{*}{相对生产力}    & 部门相对生产力($\mathit{RP}_i$)    & $\mathit{RP}_i=\frac{q_{i1}}{q_{i2}}$    & 绝对生产力之比    & 相对量,无作用    & \multirow{2}{*}{无作用} \\ 
    \cline{2-5}
    & 部门1相对于部门2的相对生产力系数($\mathit{RP}_{1/2}$)    & $\mathit{RP}_{1/2} = \frac{\mathit{RP}_1}{\mathit{RP}_2}$    & 绝对生产力的差别程度    & 决定分工方向和比较利益率 & \\ 
    \cline{1-5}

    \multirow{2}{*}{平均生产力}    & 社会平均生产力($\mathit{AP}_i$)    & $\mathit{AP}_i = \sqrt{q_{1j}q_{2j}}$     & 某商品的社会平均绝对生产力    & 无作用 & \multirow{2}{*}{无作用} \\ 
    \cline{2-5}
    & 社会平均生产力系数($\mathit{AP}_{1/2}$)    & $\mathit{AP}_{1/2} = \frac{\mathit{AP}_1}{\mathit{AP}_2}$    & 两种商品的社会平均绝对生产能力之比    & 决定均衡交换比例 & \\ 
    \cline{1-6}

    \multirow{2}{*}{综合生产力} & 部门综合生产力($cp_i$)    & $cp_i = \sqrt{q_{i1}q_{i2}}$     & 某一部门的综合生产能力    & 通过比较生产力起作用 & 通过比较生产力起作用 \\ 
    \cline{1-6}
    
    比较生产力 & 部门1相对于部门2的比较生产力($\mathit{CP}_{1/2}$)    & $\mathit{CP}_{1/2} = \frac{\mathit{cp}_1}{\mathit{cp}_2}$    & 相对于部门2来说,部门1综合生产力的大小    & 与单位商品价值正相关 &  与单位个别劳动、部门劳动创造的价值量正相关;与部门劳动创造的价值总量正相关\\ 
    \cline{1-6}

    社会总和生产力 & 第$t$期的社会总和生产力($\mathit{TP}_t$) & $\mathit{TP}_t = \left( \prod_{i=1}^{n} \mathit{CP}_{it} \right) ^ { \sfrac{1}{n}} $ & 第$t$期整个社会的综合生产力 & 无作用 & 与每一期社会总劳动创造的价值量正相关 \\
    \bottomrule
  \end{tabularx}
\end{sidewaystable}


%% !TeX root = ../2019080346_Mason.tex

\chapter{古典政治经济学中的生产力概念与价值理论}

在完成了对广义价值论的介绍后,笔者将对古典政治经济学中的生产力概念和价值理论进行梳理。

\section{古典政治经济学的界定}

在开始梳理之前,我们首先对古典政治经济学的范围做出定义。

经济学界对古典政治经济学有很多种界定,马克思是最早使用“古典”一词来泛指一段时期的政治经济学理论的\cite[7]{YueHan*MeiNaDe*KaiEnSiJiuYeLiXiHeHuoBiTongLunChongYiBen2021}。他认为“古典政治经济学在英国从威廉·配第开始,到李嘉图结束,在法国从阿吉尔贝尔开始,到西斯蒙第结束”\cite[56]{QiaEr*MaKeSiZhengZhiJingJiXuePiPanYingWenBan2022}。除了马克思以外,还有许多经济学家也对古典政治经济学作了口径不一的划分\cite[5-8]{CaiJiMingCongGuDianZhengZhiJingJiXueDaoZhongGuoTeSeSheHuiZhuYiZhengZhiJingJiXueJiYuZhongGuoShiJiaoDeZhengZhiJingJiXueYanBianShangCe2023}。由于当前学界主流的观点\cite[56]{QiaEr*MaKeSiZhengZhiJingJiXuePiPanYingWenBan2022}\cite[45]{ChenDaiSunCongGuDianJingJiXuePaiDaoMaKeSiRuoGanZhuYaoXueShuoFaZhanLueLun2014}\cite[12]{CaiJiMingCongGuDianZhengZhiJingJiXueDaoZhongGuoTeSeSheHuiZhuYiZhengZhiJingJiXueJiYuZhongGuoShiJiaoDeZhengZhiJingJiXueYanBianShangCe2023}认为古典政治经济学的基本倾向是劳动价值论,同时马克思本人也赞同劳动价值论,
故笔者在本文中将马克思也一并列入古典政治经济学的范畴。在参考了大量的文献后,笔者进一步从古典政治经济学家中选择亚当$\cdot$斯密(Adam  Smith)、大卫$\cdot$李嘉图(David Ricardo)、马克思(Karl Heinrich Marx)、马尔萨斯(Thomas Robert Malthus)、萨伊(Jean-Baptiste Say)及约翰$\cdot$斯图亚特$\cdot$穆勒(John Stuart Mill)及作为古典政治经济学的代表,依次展开分析。

\section{亚当·斯密}
 
亚当·斯密在1776年发表的《国富论》中系统地阐释了古典政治经济学的基本思想\cite[120]{CaiJiMingCongGuDianZhengZhiJingJiXueDaoZhongGuoTeSeSheHuiZhuYiZhengZhiJingJiXueJiYuZhongGuoShiJiaoDeZhengZhiJingJiXueYanBianShangCe2023}\cite[90]{YanZhiJieXiFangJingJiXueShuoShiJiaoChengDiErBan2013}。接下来,笔者将首先介绍亚当·斯密笔下的价值理论,然后梳理其生产力概念。

\subsection{亚当·斯密的价值理论}

斯密是第一个明确地区分使用价值和交换价值的经济学家\cite[122]{CaiJiMingCongGuDianZhengZhiJingJiXueDaoZhongGuoTeSeSheHuiZhuYiZhengZhiJingJiXueJiYuZhongGuoShiJiaoDeZhengZhiJingJiXueYanBianShangCe2023}。他认为使用价值指的是某个商品对使用者的效用,而交换价值指的是某个商品对其他商品的购买力\cite[24]{YaDang*SiMiGuoFuLun2015}应当说,这种区分是自然而严谨的,直到现在经济学界仍然在沿用这一区分。

接下来,斯密提出了自己在《国富论》中"为要探讨支配商品交换价值原则"\cite[24]{YaDang*SiMiGuoFuLun2015}而必须回答的三个问题,这三个问题中的概念构成了斯密价值理论的体系,笔者对斯密价值理论的介绍也将从这些概念入手。为了让斯密的表述更加清晰,笔者把这三个问题重新表述如下:

1.“什么是交换价值的尺度?”\cite[24]{YaDang*SiMiGuoFuLun2015}

2.“真实价格是由什么构成的?”\cite[24]{YaDang*SiMiGuoFuLun2015}

3.“市场价格围绕自然价格波动的原因是什么?”\cite[24]{YaDang*SiMiGuoFuLun2015}

下面,我们按照这三个问题的顺序依次进行分析。

\subsubsection{交换价值的尺度}

首先,斯密认为“劳动是衡量一切商品交换价值的真实尺度”\Cite[25]{YaDang*SiMiGuoFuLun2015}。具体而言,随着社会分工的深入发展,个体所需的物品中仅小部分可通过自身劳动供给,剩下的绝大部分都依赖他人的劳动。所以,个体的财富状况取决于其可支配的社会劳动量\cite[25]{YaDang*SiMiGuoFuLun2015}。这也就是说,一个人拥有的商品的价值,等于这件商品能够支配他人劳动的数量\cite[25]{YaDang*SiMiGuoFuLun2015}。斯密又进一步解释了选用劳动作为尺度的原因:“只有用劳动作标准,才能在一切时代和一切地方比较各种商品的价值”\cite[31]{YaDang*SiMiGuoFuLun2015}。这里,笔者想强调斯密的意思是说交换价值的尺度不仅是劳动,而且是商品可以支配的、他人(社会)的劳动。

但斯密又说:“一切商品的价值,通常不是按劳动估定的。”\cite[26]{YaDang*SiMiGuoFuLun2015}这是因为劳动存在异质性:“它们的不同困难程度和精巧程度,也必须加以考虑”\cite[26]{YaDang*SiMiGuoFuLun2015}。进而,斯密认为通过市场的议价行为可以“消除”劳动的异质性,让不同质的劳动可以通过商品交换的方式实现相互交换,于是,劳动交换这一抽象的概念取得了商品交换这一具体的形式\cite[26]{YaDang*SiMiGuoFuLun2015}。随着货币的出现,物物交换演进为商品与货币的交换,商品取得了货币价格的形式。这种以货币价格为尺度的价格被斯密称作是“名义价格”,而商品的真实价格还是以购买的劳动为尺度的\cite[28]{YaDang*SiMiGuoFuLun2015}。

根据以上内容,笔者认为,斯密笔下的“交换价值”与广义价值论中的交换价值相同,即不同商品之间的交换比例。而当货币出现并被广泛使用之后,交换价值又转换为名义价格。这里的名义价格,应当就是我们可以观察到的市场价格\cite[293]{YueSeFu*XiongBiTeJingJiFenXiShiDi1Juan2017}。

\subsection{真实价格的构成}

但是,斯密没有给出“价值”的明确定义,只是说“同一真实价格的价值,往往相等”\cite[28]{YaDang*SiMiGuoFuLun2015}。有学者认为,斯密并没有从交换价值中抽象出价值的概念\cite[71]{ChenDaiSunCongGuDianJingJiXuePaiDaoMaKeSiRuoGanZhuYaoXueShuoFaZhanLueLun2014},笔者也认同这一观点。在此基础上,笔者进一步认为斯密的意思是“交换价值”在货币出现后转换为“名义价格”,而“真实价格”是“名义价格”围绕波动的中心\cite[52]{YaDang*SiMiGuoFuLun2015}。因此,斯密笔下的“真实价格”才符合广义价值论中对价值的定义。

一般认为,斯密在第一篇第六章中提出了两种价值论\cite[97]{YanZhiJieXiFangJingJiXueShuoShiJiaoChengDiErBan2013}\cite[126]{CaiJiMingCongGuDianZhengZhiJingJiXueDaoZhongGuoTeSeSheHuiZhuYiZhengZhiJingJiXueJiYuZhongGuoShiJiaoDeZhengZhiJingJiXueYanBianShangCe2023}。首先,斯密认为“在资本累积和土地私有尚未发生以前的初期野蛮社会,获取各种物品所需要的劳动量之间的比例,似乎是各种物品相互交换的唯一标准。”\cite[41]{YaDang*SiMiGuoFuLun2015}于是我们可以说,单要素的劳动价值论是亚当斯密提出的第一种价值论。

而当资本积累到一定程度时,则“劳动者对原材料增加的价值”就可以被分解为两个部分,一部分是劳动者的工资,另一部分是雇主的利润\cite[42]{YaDang*SiMiGuoFuLun2015}。当然,衡量这三个组成部分价值的尺度,仍然是其能够支配的、他人(社会)的劳动\cite[43-44]{YaDang*SiMiGuoFuLun2015}。由于每一期的劳动可以被下一年再次投入生产,所以全社会每年的劳动产物能够买的劳动量将远超当年实际耗费的劳动量\cite[48]{YaDang*SiMiGuoFuLun2015}。于是我们又可以说,多要素的劳动价值论是亚当斯密提出的第二种价值论,也就是说耗费的劳动量不能再单独决定能够买的劳动量了,由于其它要素的参与,单位消耗的劳动量可以购买更多的劳动量\cite[138]{CaiJiMingCongGuDianDaoXianDaiZhengZhiJingJiXueGaiNianDeYanBianJianPingXinZhengZhiJingJiXueDeFaZhan2012}。事实上,这种不等关系对应着广义价值论中两个生产者的单位耗费劳动因综合生产力的相对大小不同而能购买不等量劳动的结论\cite[294]{CaiJiMingCongGuDianZhengZhiJingJiXueDaoZhongGuoTeSeSheHuiZhuYiZhengZhiJingJiXueJiYuZhongGuoShiJiaoDeZhengZhiJingJiXueYanBianShangCe2023}。

\subsection{对以上内容的两方面争议}

在此基础上,有许多经济学家对斯密的观点提出了批评,这些批评主要集中在两个方面:价值决定和价值尺度。

从价值决定的角度看,部分经济思想史学者指出,亚当·斯密的价值理论存在逻辑矛盾:斯密一方面提出了劳动价值论的观点,另一方面又有生产费用论的成分\cite[136]{CaiJiMingCongGuDianZhengZhiJingJiXueDaoZhongGuoTeSeSheHuiZhuYiZhengZhiJingJiXueJiYuZhongGuoShiJiaoDeZhengZhiJingJiXueYanBianShangCe2023}\cite[294]{YueSeFu*XiongBiTeJingJiFenXiShiDi1Juan2017}\cite[47]{ZhongGongZhongYangMaKeSiEnGeSiLieNingSiDaLinZhuZuoBianYiJuMaKeSiEnGeSiQuanJiDi26Juan1972}。但如前文所述,笔者更支持这样一种观点:斯密的价值理论是基于不同历史阶段的多要素价值论\cite[136]{CaiJiMingCongGuDianZhengZhiJingJiXueDaoZhongGuoTeSeSheHuiZhuYiZhengZhiJingJiXueJiYuZhongGuoShiJiaoDeZhengZhiJingJiXueYanBianShangCe2023},即当生产要素只有劳动时,商品的价值全部由劳动单一决定,此时斯密的价值理论是单要素劳动价值论;而当生产要素包括了土地和资本时,土地、资本、劳动都会影响商品的价值,此时斯密的价值理论是多要素劳动价值论\cite{peachAdamSmithsLabor2020}\cite[21]{MaKe*BuLaoGeJingJiLiLunDeHuiGu2018}\cite[70-71]{meekStudiesLaborTheory1973}。

从价值尺度的角度看,李嘉图批评斯密同时提出了耗费劳动尺度说和购买劳动尺度说\cite[7]{DaWei*LiJiaTuZhengZhiJingJiXueJiFuShuiYuanLi2021}。但如前文所述,斯密始终是把能购买的劳动作为尺度的\cite[142]{CaiJiMingCongGuDianZhengZhiJingJiXueDaoZhongGuoTeSeSheHuiZhuYiZhengZhiJingJiXueJiYuZhongGuoShiJiaoDeZhengZhiJingJiXueYanBianShangCe2023}\cite[63]{meekStudiesLaborTheory1973}。斯密是把价值尺度和价值决定分开考察的\cite[73]{ChenDaiSunCongGuDianJingJiXuePaiDaoMaKeSiRuoGanZhuYaoXueShuoFaZhanLueLun2014},他首先提出了衡量“交换价值”的外在尺度是可以购买的劳动\cite{rodriguezherreraAdamSmithsConcept2016},接着再进一步提出在土地尚未私有、资本尚未积累时,决定“真实价格”的是商品生产耗费的劳动。这时,“真实价格”的尺度和耗费的劳动是统一的。而在土地私有、资本累积的发达社会,包括直接耗费的劳动在内的多种要素共同决定了“真实价格”。这时,“真实价格”的尺度和耗费的劳动便不再统一,但参与商品生产的总劳动量(包括直接和间接)仍然等于商品能购买的总劳动量。\cite[67-69]{CaiJiMingLunHaoFeiDeLaoDongYuGouMaiDeLaoDongZaiJieZhiLiLunZhongDeZuoYong2022}

关于价值尺度的问题笔者将在本章末尾的总结部分进行深入的分析。

接下来,我们来分析斯密的第三个问题。

\subsection{市场价格围绕自然价格波动的原因}

首先,我们有必要对“自然价格”的概念进行阐释。按照斯密的观点,商品生产所花费的地租、工资和利润构成了商品的“自然价格”,而“自然价格”又“恰恰相当于其价值”\cite[49]{YaDang*SiMiGuoFuLun2015}。这里,斯密略去了对价值向生产价格的转化问题的分析\footnote{即由于竞争使剩余价值在各生产部门资本家之间按资本量平均分配,剩余价值转化为平均利润,价值转化生产价格的过程\cite{XieFuShengXiFangXueZheGuanYuMaKeSiJieZhiZhuanXingLiLunYanJiuShuPing2000}。}
,而直接得出了类似马克思笔下生产价格的“自然价格”概念。

斯密指出,商品的市场价格会受到供给和有效需求的支配而围绕着商品自然价格上下波动,这里的有效需求指的是愿意支付商品“自然价格”者的需求\cite[50]{YaDang*SiMiGuoFuLun2015}。于是,长期的商品价格会在自然价格处达到均衡\cite[50]{YaDang*SiMiGuoFuLun2015}。另外,斯密还指出商品的市场价格会受到货币本身价值变动的影响\cite[28-31]{YaDang*SiMiGuoFuLun2015}。

至此,我们已经回答了斯密提出的三个问题,基本梳理了斯密的价值理论体系如下图所示。

接下来,笔者将梳理斯密笔下的生产力概念。

\begin{figure}
    \centering
    \caption{亚当·斯密的价值理论}
    \label{figures:AdamSmith_Value_Theory}
    \includegraphics[width=\textwidth]{figures/AdamSmith_Value_Theory.pdf}
\end{figure}

\subsection{亚当·斯密的生产力概念}

在《国富论》的开篇,斯密指出分工是对劳动生产力的提升作用最大的因素\cite[3]{YaDang*SiMiGuoFuLun2015}。这是因为通过分工有以下三个优点:第一,劳动者能够熟能生巧;第二,分工可以避免被劳动者切换工作时的时间损失;第三,借助简化劳动的机械,一个劳动者可以完成多个劳动者的工作。\cite[6]{YaDang*SiMiGuoFuLun2015}这里,我们还可以看到斯密笔下的分工不仅是简单的通过劳动力排列组合的方面,而且包括了劳动技能提升和要素使用的方面。

如此来看,斯密笔下的“劳动生产力”对应的是广义价值论中的绝对生产力概念。纵观《国富论》全文,我们也找不到其它的生产力概念。

\subsection{生产力与价值的关系}

按照广义价值论的分析,绝对生产力与单位商品的价值是负相关的关系。虽然斯密认为资本和土地要素的投入会影响商品的价值量,但这些投入都可以被视为是物化劳动的投入,所以绝对生产力与单位商品的价值仍然是负相关的关系。另外,由于斯密认为耗费的劳动和购买的劳动可以不相等,所以我们也不难据此推出绝对生产力和单位劳动的价值量正相关的结论。


\section{大卫·李嘉图}

大卫·李嘉图是英国产业革命时代的经济学家,继承和发展了斯密价值理论中的单一劳动价值论的部分,对价值决定于劳动时间的原理作了比较透彻的表述与发展\cite[iv]{DaWei*LiJiaTuZhengZhiJingJiXueJiFuShuiYuanLi2021}。在这一部分,笔者将探讨李嘉图理论体系中的生产力概念和价值理论。

\subsection{大卫·李嘉图的价值理论}

在《政治经济学及赋税原理》第一章中,李嘉图对他的价值理论做了比斯密更为严谨和系统的介绍,但他同样没有区分开交换价值和价值。

李嘉图首先继承了斯密对使用价值和价值的区分,然后指出商品的交换价值有两个来源——一是来源于商品的稀有性,二是来源于制造商品所必需的劳动量\cite[5-6]{DaWei*LiJiaTuZhengZhiJingJiXueJiFuShuiYuanLi2021}。对于那些劳动无法增加其数量的商品,例如一些稀有的雕像和图画,它们的价值只由其稀少性决定;对于另外那些数量可以通过劳动增加的商品,其价值“几乎完全取决于各商品上所费的相对劳动量”。\cite[6]{DaWei*LiJiaTuZhengZhiJingJiXueJiFuShuiYuanLi2021}但由于前种商品在市场上只占极小一部分,所以李嘉图将分析的重点放在了后者上。

其中,“相对劳动量”指的不是耗费的一切劳动量,而是指耗费的边际劳动量\cite[16]{LiRenJunJieZhiLiLun2004}。李嘉图指出:规定一切商品的交换价值的不是在优越的生产条件下的生产者所耗费的较少量的劳动,而是“不享有这种便利的人进行生产时所必须投入的较大量劳动”\cite[58]{DaWei*LiJiaTuZhengZhiJingJiXueJiFuShuiYuanLi2021}。也就是说,在李嘉图看来,商品价值量是由边际生产条件(即最劣等生产环境)下耗费的必要劳动量来调节的\cite[9]{ChenZhenYuLunSheHuiBiYaoLaoDongShiJianXueShuoCongGuDianXuePaiDaoMaKeSiDeFaZhan1990}\footnote{马克思则认为在工业部门,商品的价值是由部门平均生产条件下所必需的劳动来决定的\cite[16]{LiRenJunJieZhiLiLun2004}。}。

另外,李嘉图又指出商品的价值“不取决于付给这种劳动的报酬的多少”\cite[5]{DaWei*LiJiaTuZhengZhiJingJiXueJiFuShuiYuanLi2021},工资的高低只会影响价值的分配而不会影响价值的决定。具体来说,李嘉图认为工资的高低对商品生产者的利润来说是很重要的,工资高低和利润高低成反比。但因为不同行业的工资是一致的,工资的升降在不同行业间也是同步的,所以工资的高低不会影响商品之间的相对价值\cite[19]{DaWei*LiJiaTuZhengZhiJingJiXueJiFuShuiYuanLi2021}。

总的来说,李嘉图价值理论的基本观点是劳动价值论——价值是由耗费的劳动决定的\cite[142]{CaiJiMingCongGuDianZhengZhiJingJiXueDaoZhongGuoTeSeSheHuiZhuYiZhengZhiJingJiXueJiYuZhongGuoShiJiaoDeZhengZhiJingJiXueYanBianShangCe2023}。

和斯密一样,李嘉图也认为生产商品所耗费劳动的异质性可以通过市场消除。在李嘉图看来,商品的相对价值会在市场交换中形成且“估价的尺度一经形成就很少发生变动”,劳动的异质性实际上已经体现在商品的价格比例中了,所以市场“消除”了劳动的异质性\cite[13-14]{DaWei*LiJiaTuZhengZhiJingJiXueJiFuShuiYuanLi2021}。

随后,李嘉图又指出资本的使用不违背劳动价值论。李嘉图首先把资本看作是物化劳动,他指出:影响商品交换价值的“不仅是指投在商品的直接生产过程中的劳动,而且也包括投在实现该种劳动所需要的一切器具或机器上的劳动”\cite[17]{DaWei*LiJiaTuZhengZhiJingJiXueJiFuShuiYuanLi2021}。因此,无论是商品直接耗费劳动的减少还是通过资本间接耗费劳动的减少,都会使得商品的相对价值下降\cite[18]{DaWei*LiJiaTuZhengZhiJingJiXueJiFuShuiYuanLi2021}。

在此基础上,李嘉图又进一步研究了资本变化情况下的价值决定。首先,李嘉图认识到具有不同固定资本和流动资本比例的资本对价值决定会产生影响。具体而言,使用固定资本占比较高的资本生产的商品的价值会高于使用流动资本占比较高的资本生产的商品的价值。李嘉图举例说,假定有两个人各雇佣100人工作1年,其中一人是织造业者,选择制造机器;另外一人是农场主,选择栽种谷物。那么根据价值由耗费劳动决定的基本观点,在第一年年末时机器的价值等于谷物的价值。在第二年,这两人继续各雇佣100人工作1年,机器的所有者制造棉织品,另外一人仍然选择栽种谷物。那么在第二年末,棉织品+棉织机和谷物(两年累积)的价值仍应是相等的。但是,由于一般利润率\footnote{即等量资本带来等量回报形成的利润率}的存在,拥有机器的两人实际上是把第一年的利润加入到了各自的资本中,而栽种谷物的一人把第一年的利润消费掉了,所以棉织品+棉织机的价值应该要受到补偿,也就是把第一年所应得的利润一并计入第二年的资本额内,并依据这个资本额计算第二年的利润。于是,棉织品+棉织机的价值实际上会大于谷物的价值\cite[24-25]{DaWei*LiJiaTuZhengZhiJingJiXueJiFuShuiYuanLi2021}\cite[119]{YanZhiJieXiFangJingJiXueShuoShiJiaoChengDiErBan2013}。

从数值上看,假设每个劳动者一年的工资是50镑,一般利润率为10\%,那么我们可以得到表格如下\cite[19]{LiRenJunJieZhiLiLun2004}:

\begin{table}[!h]
    \caption{资本构成不同对价值决定的影响}
    \begin{tabularx}{\textwidth}{|>{\centering\arraybackslash}p{1.2cm}|>{\centering\arraybackslash}X|>{\centering\arraybackslash}X|>{\centering\arraybackslash}X|>{\centering\arraybackslash}X|>{\centering\arraybackslash}X|>{\centering\arraybackslash}X|}
    \toprule
        & \multicolumn{3}{c|}{织造业者}                                                                   & \multicolumn{3}{c|}{农场主}                                                        \\ \hline
        & 固定资本 & 流动资本        & 商品价值                         & 固定资本 & 流动资本        & 商品价值             \\ \hline
    第一年 & $0$    & $100 \times 50=5000$ & $5000 \times 1.1=5500$             & $0$    & $100 \times 50=5000$ & $5000 \times 1.1=5500$ \\ \hline
    第二年 & $5500$ & $100 \times 50=5000$ & $(5500+5000) \times 1.1 -5500=6050$ & $0$    & $100 \times 50=5000$ & $5000 \times 1.1=5500$ \\ \bottomrule
    \end{tabularx}
\end{table}

接着,李嘉图又察觉到工资波动会对使用不同固定资本和流动资本比例的资本生产的商品的价值产生不同的影响\cite[117-119]{YanZhiJieXiFangJingJiXueShuoShiJiaoChengDiErBan2013}。具体而言,工资涨落使得那些使用固定资本占比较高的资本所生产的商品价值跌落,而那些使用流动资本占比较高的的资本所生产的商品价值上升。

沿用前面的例子,假设每个劳动者第一年的工资是50镑,第一年的一般利润率为10\%;第二年的一般利润率因工资上涨而下降为9\%,那么我们可以得到表格如下\cite[21]{LiRenJunJieZhiLiLun2004}:

\begin{table}[!h]
    \caption{工资变动对价值决定的影响}
    \begin{tabularx}{\textwidth}{|>{\centering\arraybackslash}p{1.2cm}|>{\centering\arraybackslash}X|>{\centering\arraybackslash}X|>{\centering\arraybackslash}X|>{\centering\arraybackslash}X|>{\centering\arraybackslash}X|>{\centering\arraybackslash}X|}
    \toprule
        & \multicolumn{3}{c|}{织造业者}                                                                   & \multicolumn{3}{c|}{农场主}                                                        \\ \hline
        & 固定资本 & 流动资本        & 商品价值                         & 固定资本 & 流动资本        & 商品价值             \\ \hline
    第一年 & $0$    & $100 \times 50=5000$ & $5000 \times 1.1=5500$             & $0$    & $100 \times 50=5000$ & $5000 \times 1.1=5500$ \\ \hline
    第二年 & $5500$ & $100 \times 50.46=5046$ & $(5500+5046) \times 1.09 -5500=5995$ & $0$    & $100 \times 50.46=5000$ & $5046 \times 1.09=5500$ \\ \bottomrule
    \end{tabularx}
\end{table}

最后,李嘉图还意识到资本周转速度的差异会对商品的价值产生影响。具体而言,使用周转周期长的资本生产的商品的价值会高于使用周转周期短的资本生产的商品。假定有甲、乙两个资本家,甲资本家用1000镑雇佣20个工人生产一年得到一批半成品,第二年再用1000镑雇佣20人对半成品进行加工,第二年末完成生产;乙资本家则在1年中用2000镑雇佣40个工人生产商品,第一年年末就送上市场\cite[27-28]{DaWei*LiJiaTuZhengZhiJingJiXueJiFuShuiYuanLi2021}。再假定一般利润率为10\%,那我们可以得到表格如下\cite[20]{LiRenJunJieZhiLiLun2004}:

\begin{table}[!h]
    \caption{资本周转差异对价值决定的影响}
    \begin{tabularx}{\textwidth}{|>{\centering\arraybackslash}p{1.2cm}|>{\centering\arraybackslash}X|>{\centering\arraybackslash}X|>{\centering\arraybackslash}X|>{\centering\arraybackslash}X|>{\centering\arraybackslash}X|>{\centering\arraybackslash}X|}
    \toprule
        & \multicolumn{3}{c|}{甲}                                                                   & \multicolumn{3}{c|}{乙}                                                        \\ \hline
        & 固定资本 & 流动资本        & 商品价值                         & 固定资本 & 流动资本        & 商品价值             \\ \hline
    第一年 & $0$    & $20 \times 50=1000$ & $1000 \times 1.1=1100$             & $0$    & $40 \times 50=2000$ & $2000 \times 1.1=2200$ \\ \hline
    第二年 & $110000$ & $20 \times 50=1000$ & $(1100+1000) \times 1.1 =2310$ & $0$    & $0 $ & $0$ \\ \bottomrule
    \end{tabularx}
\end{table}

如果再考虑到资本周转时间的差异,那么工资变动会进一步影响商品的价值决定。对此,李嘉图总结道:商品价值对工资涨落的敏感性取决于固定资本在全部资本中的占比;那些固定资本占比高或是资本周转时间长的商品的相对价值会因为工资变动而跌落,反之则会上涨\cite[26]{DaWei*LiJiaTuZhengZhiJingJiXueJiFuShuiYuanLi2021}。

虽然以上的结论实际上已经违背了“价值是由耗费的劳动决定的”这一基本观点,但李嘉图认为“商品价值变动的这一原因的影响是比较小的。工资上涨到使利润跌落百分之一时,在前述假定情况下生产出来的商品的相对价值只会发生百分之一的变动”\cite[26]{DaWei*LiJiaTuZhengZhiJingJiXueJiFuShuiYuanLi2021}。由此,李嘉图的劳动价值论也被人戏称为“93\%的劳动价值论”\cite{georgej.stiglerRicardo93Labor1958}。

\subsection{大卫·李嘉图的生产力概念}

李嘉图的著作中没有出现明确的生产力概念,我们只能从他的文字中总结出他对生产力的看法。例如,他在分析黄金的价值变动时说:如果“由于发现了更丰饶的新矿山,或是由于更有利地使用机器,用较少的劳动量就可以获得一定量的黄金,那么,我就有理由说,黄金相对于其他商品的价值发生变动的原因,是它的生产已经比较便利,或获得时所必需的劳动量已经减少”\cite[11]{DaWei*LiJiaTuZhengZhiJingJiXueJiFuShuiYuanLi2021}。李嘉图在别处分析生产力的改进时也是指商品生产所必需的劳动量的减少,所以笔者认为李嘉图的理论中也只有绝对生产力的概念。

另一方面,李嘉图也注意到商品的价值取决于“相对劳动量”而不是绝对劳动量\cite[5]{DaWei*LiJiaTuZhengZhiJingJiXueJiFuShuiYuanLi2021},“如果生产其他商品所需的劳动有所增减,那么我们已经说过,这种情形一定会立即造成其相对价值的变动。”\cite[21]{DaWei*LiJiaTuZhengZhiJingJiXueJiFuShuiYuanLi2021}由此看来,李嘉图其实具备了提出比较生产力的潜力,只不过李嘉图并没有把研究重点放在生产力上。

由于李嘉图也没有区分部门、个体的绝对生产力变动带来的影响,笔者在此先不探讨李嘉图学说中生产力与价值的关系,而是先介绍与李嘉图价值理论相似的马克思的价值理论,并在马克思的理论基础上对绝对生产力与价值的关系做出进一步分析。

最后,李嘉图在解释国际贸易时讨论了国家的比较优势,引入了机会成本,因此实际上提出了相对生产力的概念。李嘉图构建了一个英国葡萄牙两国贸易模型:英国生产单位毛呢需投入100人$\cdot$年的​劳动量,等值葡萄酒则需120人$\cdot$年;葡萄牙生产单位毛呢需要90人$\cdot$年的劳动量,等值葡萄酒仅需要80人$\cdot$年;如果两国可以进行贸易,那么葡萄牙即使能够以90人$\cdot$年的劳动量生产毛呢,但仍会选择从需要100人$\cdot$年的​劳动量生产毛呢的英国进口;因为对葡萄牙来说,把资本全部用于生产葡萄酒然后同英国交换得到的毛呢将多于把资本投入本国毛呢生产而得到的毛呢\cite[111-112]{DaWei*LiJiaTuZhengZhiJingJiXueJiFuShuiYuanLi2021}。在这个例子中,尽管葡萄牙在两种商品的生产商占据绝对优势,但由于葡萄牙在葡萄酒的生产商具有比较优势,英国在生产毛呢商具有比较优势,所以双方仍能从国际贸易中获得好处(节约劳动)。这实际上就是广义价值论中的比较优势原理在国际贸易中的运用。所以,笔者认为李嘉图事实上已经提出了相对生产力的概念,并事实上借助了相对生产力系数来判断交换双方的比较优势。

但是,李嘉图没能把比较优势原理运用到一国内的商品交换过程中,他认为支配国内商品交换的价值法则不适用于国际的商品交换,反之亦然\cite[110]{DaWei*LiJiaTuZhengZhiJingJiXueJiFuShuiYuanLi2021}。事实上,广义价值论正是吸收了李嘉图的思想,把这一原理拓展到了一般的商品交换过程之中\cite[157]{CaiJiMingCongGuDianZhengZhiJingJiXueDaoZhongGuoTeSeSheHuiZhuYiZhengZhiJingJiXueJiYuZhongGuoShiJiaoDeZhengZhiJingJiXueYanBianShangCe2023}。

\section{卡尔·马克思}

一般认为,马克思继承和发展了李嘉图的劳动价值论\cite[347]{YueSeFu*XiongBiTeJingJiFenXiShiDi2Juan2017}\cite[84]{ChenDaiSunCongGuDianJingJiXuePaiDaoMaKeSiRuoGanZhuYaoXueShuoFaZhanLueLun2014}。

\subsection{马克思的价值理论}

从价值理论来看,马克思首先区分了交换价值和价值:价值是交换价值的基础,交换价值是价值的表现形态\cite[86-88]{ChenDaiSunCongGuDianJingJiXuePaiDaoMaKeSiRuoGanZhuYaoXueShuoFaZhanLueLun2014}。根本上来说,李嘉图之所以不能做出这种区分,是因为李嘉图把劳动价值论“仅仅是作为一种假设”,“用来说明相对价格(能观察到的市场价格)的实际长期正常状态”\cite[348]{YueSeFu*XiongBiTeJingJiFenXiShiDi2Juan2017},于是李嘉图遇到了“商品价值量取决于社会必要劳动时间与等量资本获取等量利润”的矛盾\cite[144]{CaiJiMingCongGuDianZhengZhiJingJiXueDaoZhongGuoTeSeSheHuiZhuYiZhengZhiJingJiXueJiYuZhongGuoShiJiaoDeZhengZhiJingJiXueYanBianShangCe2023}\cite[21-28]{DaWei*LiJiaTuZhengZhiJingJiXueJiFuShuiYuanLi2021};而马克思则是把劳动看成价值的实质——价值就是凝结的劳动本身,于是马克思遇到了价值向生产价格的转形问题\cite[348-350]{YueSeFu*XiongBiTeJingJiFenXiShiDi2Juan2017}\cite[159]{CaiJiMingCongGuDianZhengZhiJingJiXueDaoZhongGuoTeSeSheHuiZhuYiZhengZhiJingJiXueJiYuZhongGuoShiJiaoDeZhengZhiJingJiXueYanBianShangCe2023}。

再次,马克思通过区分劳动与劳动力解释“资本与劳动交换”的问题。马克思认为工人让渡的是劳动力使用权而非劳动本身。当劳动力作为特殊商品进入生产领域,其使用过程(即活劳动)创造的价值超越自身价值(工资),形成被资本家无偿占有的价值剩余,由此消解了资本交换与价值规律的逻辑悖论。\cite[615,581-606]{ZhongGongZhongYangMaKeSiEnGeSiLieNingSiDaLinZhuZuoBianYiJuMaKeSiEnGeSiWenJiDi5Juan2009}\cite[157-158]{CaiJiMingCongGuDianZhengZhiJingJiXueDaoZhongGuoTeSeSheHuiZhuYiZhengZhiJingJiXueJiYuZhongGuoShiJiaoDeZhengZhiJingJiXueYanBianShangCe2023}\cite[348]{YueSeFu*XiongBiTeJingJiFenXiShiDi2Juan2017}。

总的来说,马克思完善了劳动价值论,其价值理论可以简单地表述为:商品的价值由生产商品的社会必要劳动时间决定\cite[51-52]{ZhongGongZhongYangMaKeSiEnGeSiLieNingSiDaLinZhuZuoBianYiJuMaKeSiEnGeSiWenJiDi5Juan2009}。

\subsection{马克思的生产力理论}

马克思虽然对生产力有不同的提法,但各种提法的本质是清晰且一贯的\cite{YangQiaoYuShengChanLiGaiNianCongSiMiDaoMaKeSiDeSiXiangPuXi2013}\cite{DingXiaoPingZhengQueLiJieMaKeSiZhuYiDeShengChanLiGaiNian2021}——“生产力当然始终是有用的、具体的劳动的生产力,它事实上只决定有目的的生产活动在一定时间内的效率”\cite[59]{ZhongGongZhongYangMaKeSiEnGeSiLieNingSiDaLinZhuZuoBianYiJuMaKeSiEnGeSiWenJiDi5Juan2009}——这和广义价值论中绝生产力的定义是相符的。

事实上,由于马克思把价值看成是“凝结的劳动”\cite[51]{ZhongGongZhongYangMaKeSiEnGeSiLieNingSiDaLinZhuZuoBianYiJuMaKeSiEnGeSiWenJiDi5Juan2009},所以马克思反倒失去了提出“相对生产力”的潜力。

\subsection{生产力与价值的关系}

基于绝对生产力不同维度,马克思针对生产力与价值的关系给出了三个对立统一的命题\cite[273]{CaiJiMingCongGuDianZhengZhiJingJiXueDaoZhongGuoTeSeSheHuiZhuYiZhengZhiJingJiXueJiYuZhongGuoShiJiaoDeZhengZhiJingJiXueYanBianShangCe2023}。

第一,劳动生产力与价值量负相关。马克思指出,“劳动生产力越高,生产一种物品所需要的劳动时间就越少,凝结在该物品中的劳动量就越小,该物品的价值就越小”\cite[53]{ZhongGongZhongYangMaKeSiEnGeSiLieNingSiDaLinZhuZuoBianYiJuMaKeSiEnGeSiWenJiDi5Juan2009}。这里的“劳动生产力”应当是指广义价值论中的部门绝对生产力,而这里的“价值量”应当是指单位商品的价值量\cite[273]{CaiJiMingCongGuDianZhengZhiJingJiXueDaoZhongGuoTeSeSheHuiZhuYiZhengZhiJingJiXueJiYuZhongGuoShiJiaoDeZhengZhiJingJiXueYanBianShangCe2023}。该命题与广义价值论的判断是一致的。但马克思还说单位商品的价值量和劳动生产力成反比\cite[53-54]{ZhongGongZhongYangMaKeSiEnGeSiLieNingSiDaLinZhuZuoBianYiJuMaKeSiEnGeSiWenJiDi5Juan2009},这就与广义价值论的判断出现矛盾。在广义价值论中,部门绝对生产力的提高会带动部门综合生产力提高,进而驱动部门比较生产力系数发生变化,最终导致单位商品价值量的降幅总是比部门绝对生产力的提高幅度更小,所以部门绝对生产力与单位商品价值量只能构成负相关而不是反比的关系\cite[274, 282]{CaiJiMingCongGuDianZhengZhiJingJiXueDaoZhongGuoTeSeSheHuiZhuYiZhengZhiJingJiXueJiYuZhongGuoShiJiaoDeZhengZhiJingJiXueYanBianShangCe2023}。出现这种矛盾的根本原因,是马克思没有考虑部门综合生产能力对价值决定的影响。

第二,劳动生产力与价值量正相关。马克思指出:“生产力特别高的劳动起了自乘的劳动的作用,$\cdots$,在同样的时间内,它所创造的价值比同种社会平均劳动要多。”\cite[370]{ZhongGongZhongYangMaKeSiEnGeSiLieNingSiDaLinZhuZuoBianYiJuMaKeSiEnGeSiWenJiDi5Juan2009}此处的“劳动生产力”应当是指广义价值论中的个别绝对生产力,而这里的“价值量”应当是指单个生产者在单位劳动时间内所创造的价值量\cite[273]{CaiJiMingCongGuDianZhengZhiJingJiXueDaoZhongGuoTeSeSheHuiZhuYiZhengZhiJingJiXueJiYuZhongGuoShiJiaoDeZhengZhiJingJiXueYanBianShangCe2023}。该命题与广义价值论的判断也是一致的。

第三,劳动生产力与价值量不相关。马克思指出,“不管生产力发生了什么变化,同一劳动在同样的时间内提供的价值量总是相同的”\cite[60]{ZhongGongZhongYangMaKeSiEnGeSiLieNingSiDaLinZhuZuoBianYiJuMaKeSiEnGeSiWenJiDi5Juan2009}。这里的“劳动生产力”应当是指部门绝对生产力,而这里的“价值量”应当是指部门总劳动创造的价值量\cite[274]{CaiJiMingCongGuDianZhengZhiJingJiXueDaoZhongGuoTeSeSheHuiZhuYiZhengZhiJingJiXueJiYuZhongGuoShiJiaoDeZhengZhiJingJiXueYanBianShangCe2023}。该命题与广义价值论的判断出现了矛盾。出现这种矛盾的根本原因,是马克思否认了非劳动要素对价值决定的作用\cite[274]{CaiJiMingCongGuDianZhengZhiJingJiXueDaoZhongGuoTeSeSheHuiZhuYiZhengZhiJingJiXueJiYuZhongGuoShiJiaoDeZhengZhiJingJiXueYanBianShangCe2023}。

总的来说,马克思并没有提出绝对生产力以外的概念,并且不承认非劳动要素对价值决定的作用,但他区分了个体和部门的绝对生产力变化对价值的不同影响,将单一劳动价值论发展到了极致。

\section{马尔萨斯和萨伊}

至此,我们完成了对李嘉图和马克思的介绍。不难看出,他们的经济思想主要以继承和发展斯密的单要素劳动价值论为主。接下来,笔者将介绍继承和发展了斯密多要素价值论的马尔萨斯、萨伊的价值理论并分析其生产力概念。笔者之所以把这两位经济学家放在一起,一方面是因为两人生活在同一个时代\cite[132,140]{YanZhiJieXiFangJingJiXueShuoShiJiaoChengDiErBan2013},另一方面是因为一般认为他们都按照各自的理解继承了斯密的多要素价值论\cite[169]{CaiJiMingCongGuDianZhengZhiJingJiXueDaoZhongGuoTeSeSheHuiZhuYiZhengZhiJingJiXueJiYuZhongGuoShiJiaoDeZhengZhiJingJiXueYanBianShangCe2023}。

\subsection{马尔萨斯的价值理论}

马尔萨斯在区分使用价值和交换价值的基础上对交换价值作了进一步细分,他认为“价值”一词可以被分解为三个方面:第一是使用价值,即物品的效用;第二是名义交换价值,即以货币表示的价值(如果还没有出现货币,则商品的交换价值通过其它任何一种商品表现出来\cite[32]{BiLuo*SiLaFaDaWeiLiJiaTuQuanJiDi2JuanMaErSaSiZhengZhiJingJiXueYuanLiPingZhu2013});第三是实际交换价值,即必需品、享用品和劳动的价值\cite[42]{BiLuo*SiLaFaDaWeiLiJiaTuQuanJiDi2JuanMaErSaSiZhengZhiJingJiXueYuanLiPingZhu2013}。马尔萨斯重点研究的是“实际交换价值”。

然而,值得指出的是,马尔萨斯并没有把价值从价格中抽象出来,他认为“任何时间与地点的商品的自然价值\footnote{笔者认为这里就是指“实际交换价值”}”是“商品处于自然或一般状态下由原始生产成本或供应条件决定的估价”\cite[132]{MaErSaSiZhengZhiJingJiXueDingYi2023}。换句话说,马尔萨斯的“价值”或者“实际交换价值”在广义价值论中对应的是交换价值的概念而非价值概念,马尔萨斯并没有揭开价值的面纱。

马尔萨斯虽然认为“实际交换价值”就是“估价”,但他意识到用贵金属或是其它商品来衡量这一“估价”是很困难的\cite[98]{BiLuo*SiLaFaDaWeiLiJiaTuQuanJiDi2JuanMaErSaSiZhengZhiJingJiXueYuanLiPingZhu2013},
所以他认为应当用商品能支配的劳动而非生产耗费的劳动作为“实际交换价值”的尺度\footnote{事实上马尔萨斯认为用单一的商品能支配的劳动作为价值尺度都是不够精准的,马尔萨斯认为在某些情况下,由谷物和劳动两种尺度组合成的新的尺度优于任何一个单独的尺度\cite[98-105]{BiLuo*SiLaFaDaWeiLiJiaTuQuanJiDi2JuanMaErSaSiZhengZhiJingJiXueYuanLiPingZhu2013}。}\cite[60-82, 92-97]{BiLuo*SiLaFaDaWeiLiJiaTuQuanJiDi2JuanMaErSaSiZhengZhiJingJiXueYuanLiPingZhu2013}\cite[133]{MaErSaSiZhengZhiJingJiXueDingYi2023}。这是因为:第一,“用价值中的绝大部分来进行交换的是,生产性或非生产性的劳动”\cite[92]{BiLuo*SiLaFaDaWeiLiJiaTuQuanJiDi2JuanMaErSaSiZhengZhiJingJiXueYuanLiPingZhu2013};第二,“只有与劳动交换的商品的价值,能够表达商品对社会需要和爱好的配合程度,能够表达同消费者的愿望与人数对照下,商品供给的丰裕程度。”\cite[92]{BiLuo*SiLaFaDaWeiLiJiaTuQuanJiDi2JuanMaErSaSiZhengZhiJingJiXueYuanLiPingZhu2013};第三,“资本的积累,以及其增加财富和人口的效能...取决于其换取劳动的力量”\cite[93]{BiLuo*SiLaFaDaWeiLiJiaTuQuanJiDi2JuanMaErSaSiZhengZhiJingJiXueYuanLiPingZhu2013}。最后,用商品生产耗费的劳动作为价值尺度则存在很多例外\cite[60-82]{BiLuo*SiLaFaDaWeiLiJiaTuQuanJiDi2JuanMaErSaSiZhengZhiJingJiXueYuanLiPingZhu2013},既然例外如此之多,例外反倒成了法则\cite[171]{CaiJiMingCongGuDianZhengZhiJingJiXueDaoZhongGuoTeSeSheHuiZhuYiZhengZhiJingJiXueJiYuZhongGuoShiJiaoDeZhengZhiJingJiXueYanBianShangCe2023}。

接着,马尔萨斯又提出一切商品的实际交换价值取决于市场的供给和需求的相对关系\cite[43]{BiLuo*SiLaFaDaWeiLiJiaTuQuanJiDi2JuanMaErSaSiZhengZhiJingJiXueYuanLiPingZhu2013}。这里,需求指的是“购买的力量和愿望的结合”\cite[43]{BiLuo*SiLaFaDaWeiLiJiaTuQuanJiDi2JuanMaErSaSiZhengZhiJingJiXueYuanLiPingZhu2013},供给指的是“商品的生产和卖出商品的意向的结合”\cite[43]{BiLuo*SiLaFaDaWeiLiJiaTuQuanJiDi2JuanMaErSaSiZhengZhiJingJiXueYuanLiPingZhu2013}。在此基础上,马尔萨斯又进一步指出尽管商品的生产成本是商品供给的必要条件,对商品价格有很大的影响\cite[50,55]{BiLuo*SiLaFaDaWeiLiJiaTuQuanJiDi2JuanMaErSaSiZhengZhiJingJiXueYuanLiPingZhu2013},但生产成本本身还是由供求法则决定的,所以商品的实际交换价值根本上是由供求法则来决定的\cite[59]{BiLuo*SiLaFaDaWeiLiJiaTuQuanJiDi2JuanMaErSaSiZhengZhiJingJiXueYuanLiPingZhu2013}。

总的来说,马尔萨斯把商品所能支配的劳动作为衡量“实际交换价值”的尺度,并认为是供需关系决定了商品的“实际交换价值”。

\subsection{马尔萨斯的生产力理论}

马尔萨斯与其所处时代的其他经济学家一样,并没有提出绝对生产力以外的生产力概念。马尔萨斯实际上是在分析财富的增长时间接地分析了生产力。

马尔萨斯指出,财富和价值有着根本性的区别,财富的多少“部分取决于产品的数量,部分取决于产品对社会的需要和力量的适应”\cite[292]{BiLuo*SiLaFaDaWeiLiJiaTuQuanJiDi2JuanMaErSaSiZhengZhiJingJiXueYuanLiPingZhu2013}。光看这段话,似乎可以认为马尔萨斯的“财富”与使用价值是相等的概念,但是笔者认为,马尔萨斯的“财富”指的应当是全社会商品的价值总量。因为马尔萨斯在后文中又指出“商品的价值,也就是人们为了取得这些商品所愿意作出的牺牲,可以说是任何数量的财富存在的唯一原因”\cite[292]{BiLuo*SiLaFaDaWeiLiJiaTuQuanJiDi2JuanMaErSaSiZhengZhiJingJiXueYuanLiPingZhu2013},以及“...,对财富的不断增长,...,就非有对商品需求的不断增长的配合不可”\cite[355]{BiLuo*SiLaFaDaWeiLiJiaTuQuanJiDi2JuanMaErSaSiZhengZhiJingJiXueYuanLiPingZhu2013}。马尔萨斯之所以认为财富和价值有着根本性的区别,还是因为他没能从交换价值中抽象出价值的概念,而交换价值是某一个比例,只有大小而无数量,所以马尔萨斯必须引入“财富”来衡量价值的数量或大小。

紧接着,马尔萨斯分析了“财富增长的直接原因”,并指出“资本的积累、土地的肥力和节省劳动的发明”是提高供给的三大原因\cite[355]{BiLuo*SiLaFaDaWeiLiJiaTuQuanJiDi2JuanMaErSaSiZhengZhiJingJiXueYuanLiPingZhu2013}。但这还不够,马尔萨斯还强调“生产力与分配手段结合的必要性”\cite[356]{BiLuo*SiLaFaDaWeiLiJiaTuQuanJiDi2JuanMaErSaSiZhengZhiJingJiXueYuanLiPingZhu2013}:从正面看,“...,单是生产力,不论巨大到什么程度是不足以保证财富在适当的程度上的增长的”,为了让生产力的提升充分发挥作用,还需要“这样一种情况的产品分配,和产品对消费者的需要的这样一种情况的适应,从而使全部产品的交换价值\footnote{这里可以看到因马尔萨斯没能区分价值和交换价值而产生的谬误}不断提高”\cite[356]{BiLuo*SiLaFaDaWeiLiJiaTuQuanJiDi2JuanMaErSaSiZhengZhiJingJiXueYuanLiPingZhu2013};从反面看,“假使一个国家所有的公路和运河都被破坏,其产品的分配手段根本受到阻碍,其产品的整个价值将显著下降”\cite[357]{BiLuo*SiLaFaDaWeiLiJiaTuQuanJiDi2JuanMaErSaSiZhengZhiJingJiXueYuanLiPingZhu2013}。

\subsection{马尔萨斯的生产力与价值的关系}

根据以上的分析,我们不难看出马尔萨斯笔下的生产力是绝对生产力,且马尔萨斯认为当绝对生产力能适应消费者的需求时,绝对生产力的提升与全社会商品的总价值量正相关;当绝对生产力不能适应消费者的需求时,绝对生产力的提升与全社会商品的总价值量不一定相关。

另一方面,马尔萨斯也间接地指出了绝对生产力与单位商品价值量之间的关系。他说“用改进的机器,在同样成本下取得同样质量的更多的商品时,财富与价值之间的区别是明显的;然而,即使就这里的情况说,这一增益量的拥有者,也只是从消费看来,而不是从交换看来,比前富裕”\cite[291]{BiLuo*SiLaFaDaWeiLiJiaTuQuanJiDi2JuanMaErSaSiZhengZhiJingJiXueYuanLiPingZhu2013}。不难看出,马尔萨斯认为绝对生产力与单个生产者单位劳动创造的价值量是不相关的。这与广义价值论的判断是不同的,其原因是马尔萨斯采取了供求价值论,商品的数量增加会影响商品的供求关系。最后,在供求价值论下,绝对生产力与单位商品的价值量仍应是负相关的,这一结论与广义价值论的判断是一致的。

\subsection{萨伊的价值理论}

萨伊把使用价值和交换价值放到一起来分析价值,形成了一种综合了生产要素论、生产费用论和效用论的价值论\footnote{原文认为还有“供求价值论”,但笔者认为萨伊并没有引入“供求价值论”。}\cite[138]{YanZhiJieXiFangJingJiXueShuoShiJiaoChengDiErBan2013}。

萨伊认为,“物品满足人类需要的内在力量叫做效用”,效用构成了商品价值的基础\cite[59]{SaYiZhengZhiJingJiXueGaiLunCaiFuDeShengChanFenPeiHeXiaoFei2020}。萨伊进一步指出,价格是测量价值的尺度,价值又是测量效用的尺度\cite[60]{SaYiZhengZhiJingJiXueGaiLunCaiFuDeShengChanFenPeiHeXiaoFei2020}。

在此基础上,萨伊又认为价值或者效用是劳动、资本和自然力共同创造的\cite[78]{SaYiZhengZhiJingJiXueGaiLunCaiFuDeShengChanFenPeiHeXiaoFei2020},劳动、资本和自然力是价值的三大来源。接着,萨伊又指出了价值的决定方式。首先,萨伊指出商品的价值决定于“在生产方面所作的努力”\cite[351]{SaYiZhengZhiJingJiXueGaiLunCaiFuDeShengChanFenPeiHeXiaoFei2020},这种努力该如何衡量?萨伊又进一步指出:“使生产力有价值的,乃是创造那需要所从以产生效用的能力。这个价值的大小和这件物品在生产事业中所提供的合作的重要性成比例,而就各个产品说,这个价值构成所谓的生产费用”\cite[352]{SaYiZhengZhiJingJiXueGaiLunCaiFuDeShengChanFenPeiHeXiaoFei2020}。也就是说,生产费用衡量了生产商品所付出的努力,进而决定了商品的价值。这里,萨伊强调了“生产劳动的市值,是基于许多产品相比较的价值”\cite[352]{SaYiZhengZhiJingJiXueGaiLunCaiFuDeShengChanFenPeiHeXiaoFei2020},也就是说,生产费用决定于某一生产要素潜在的创造最大效用的能力。萨伊还引用了一个例子来阐释这一思想:生产一件生产费用为四法郎而售价为三法郎的物品的生产力的价值不是三法郎而是四法郎。因为既然该生产力的生产费用为四法郎,那么其本来能够创造四法郎的价值。但是在这种情况下,该生产力仅仅创造了三法郎的价值\cite[352]{SaYiZhengZhiJingJiXueGaiLunCaiFuDeShengChanFenPeiHeXiaoFei2020}。

前文提到,萨伊认为价格是衡量价值的尺度。那价格是如何决定的呢?萨伊指出价格受到供求法则的支配——“在一定的时间和地点,一种货物的价格,随着需求的增加与供给的减少而成比例地上升;反过来也是一样。换句话说,物价的上升和需求成正比例,但和供给成反比例。”\cite[256]{SaYiZhengZhiJingJiXueGaiLunCaiFuDeShengChanFenPeiHeXiaoFei2020}

然而,萨伊并没有指出由生产费用决定的价值是如何转化为受供求关系支配的价格的,这种空缺使得萨伊的价值理论看上去比较混乱。例如,有的学者认为萨伊的价值概念有三重内涵:第一是要素成本意义上的价值,指的是获取商品的必要代价,体现为劳动、资本、土地的生产要素支出;第二是市场交易价值,即市价,指的是供求关系调节形成的即时交换比率;第三是真正的价值,指的是来自生产费用的物品的效用\cite[138]{YanZhiJieXiFangJingJiXueShuoShiJiaoChengDiErBan2013}\cite[175]{CaiJiMingCongGuDianZhengZhiJingJiXueDaoZhongGuoTeSeSheHuiZhuYiZhengZhiJingJiXueJiYuZhongGuoShiJiaoDeZhengZhiJingJiXueYanBianShangCe2023}。在笔者看来,萨伊的价值理论没有那么复杂。如前所述,萨伊认为价值的基础是效用,价值的来源是劳动、资本和土地的协作生产,价值又决定于商品生产过程中所使用的这三种要素的生产费用,而生产费用又取决于某生产力能创造的最大效用。所以,萨伊的价值理论是效用论和生产费用论的综合。不过在解释价格和价值之间的关系时,萨伊遇到了和马克思“转型问题”类似的价值向价格转化的问题。

\subsection{萨伊的生产力理论}

不难推断,持效用价值论观点的萨伊没有提出绝对生产力以外的生产力概念,他认为:“所谓生产,不是创造物质,而是创造效用。”\cite[60]{SaYiZhengZhiJingJiXueGaiLunCaiFuDeShengChanFenPeiHeXiaoFei2020}萨伊进一步用“劳力”来衡量生产的效率,也就是衡量绝对生产力的大小,他指出:“我所说的劳力,是指任何一种劳动工作时所进行的继续不断的动作,或在从事任何一种劳动工作的某一部分时所进行的继续不断的动作。”\cite[90]{SaYiZhengZhiJingJiXueGaiLunCaiFuDeShengChanFenPeiHeXiaoFei2020}这里的“劳力”又可以按照价值来源的三要素被分为:人的劳力、自然的劳力和机器的劳力\cite[90]{SaYiZhengZhiJingJiXueGaiLunCaiFuDeShengChanFenPeiHeXiaoFei2020}。而绝对生产力的提升,“在于减少生产同一数量产品所需的劳力,或与此类似,在于扩大同一数量人力所能获得的产品量”\cite[91]{SaYiZhengZhiJingJiXueGaiLunCaiFuDeShengChanFenPeiHeXiaoFei2020}

\subsection{生产力与价值的关系}

广义价值论关于绝对生产力与单位商品价值量负相关的判断和萨伊的判断是一致的——“由于人类所掌握的生产手段实际上变得更有力量,所以创造出来的产品在数量上总是增加,而在价值上总是成比例地减少。”\cite[369]{SaYiZhengZhiJingJiXueGaiLunCaiFuDeShengChanFenPeiHeXiaoFei2020}这里的“价值”应当是指单位商品的价值量。

当然,由于萨伊没有统一价值的来源——效用和价值的决定——生产费用,以上的结论显然是不严谨的。

\section{约翰·穆勒}

 穆勒被尊称为古典经济学的集大成者,对各种观点的调和折中是其经济思想公认的特征\cite[165]{YanZhiJieXiFangJingJiXueShuoShiJiaoChengDiErBan2013}\cite[176-178]{CaiJiMingCongGuDianZhengZhiJingJiXueDaoZhongGuoTeSeSheHuiZhuYiZhengZhiJingJiXueJiYuZhongGuoShiJiaoDeZhengZhiJingJiXueYanBianShangCe2023}。
 
 \subsection{约翰·穆勒的价值理论}

 穆勒继承了斯密对交换价值和使用价值的区分,并将其研究重点放在了交换价值上。和斯密一样,穆勒认为商品的价值“是指它的一般购买力,即拥有这一物品对于一般可购商品所具有的支配力”,而价格则是用货币表示的价值\cite[493]{YueHan*MuLeZhengZhiJingJiXueYuanLiJiQiZaiSheHuiZheXueShangDeRuoGanYingYongShangJuan1991}。

 穆勒强调,“价值是一个相对的术语”\cite[495]{YueHan*MuLeZhengZhiJingJiXueYuanLiJiQiZaiSheHuiZheXueShangDeRuoGanYingYongShangJuan1991}。由于穆勒笔下的价值就是交换价值,其尺度是可以用其交换到的其它商品或货币,所以不可能出现所有商品的价值一起提高的情况。另外,由于价值是相对的概念,所以穆勒认为:“价值尺度是以任何两种物品与其相比较,我们即可推知这些物品相互价值的某种物品。”\cite[102]{YueHan*MuLeZhengZhiJingJiXueYuanLiJiQiZaiSheHuiZheXueShangDeRuoGanYingYongXiaJuan1991}从这个定义出发,“任何商品在一定的时间和地点都可以用作价值尺度;因为如果我们知道两种物品各自与任何第三种物品相交换的比率,就总是可以推知这两种物品相互交换的比率”\cite[102]{YueHan*MuLeZhengZhiJingJiXueYuanLiJiQiZaiSheHuiZheXueShangDeRuoGanYingYongXiaJuan1991}。但是穆勒自己也意识到,这种定义下的价值尺度的并不是斯密、马克思或马尔萨斯所讨论的价值尺度。“政治经济学家所寻求的,$\cdots$是同一物品在不同的时间和地点的价值尺度”\cite[103]{YueHan*MuLeZhengZhiJingJiXueYuanLiJiQiZaiSheHuiZheXueShangDeRuoGanYingYongXiaJuan1991};政治经济学家们是希望能找到这样一种媒介,“$\cdots$一种商品只须同这个媒介相比较,而不必额外地同任何其他特定的商品相比较,它的价值就可以确定”\cite[103]{YueHan*MuLeZhengZhiJingJiXueYuanLiJiQiZaiSheHuiZheXueShangDeRuoGanYingYongXiaJuan1991}。显然,“政治经济学家所寻求的”那种价值尺度是和穆勒所认为的价值的相对性在根本上矛盾的,所以穆勒认为价值尺度是不存在的\cite[104]{YueHan*MuLeZhengZhiJingJiXueYuanLiJiQiZaiSheHuiZheXueShangDeRuoGanYingYongXiaJuan1991}。但穆勒没有完全否定以往的政治经济学家,他又指出:“把这个概念称为生产费用的尺度,或许更为恰当”\footnote{这实际上也反映了穆勒的价值理论中生产费用论的倾向。}\cite[104]{YueHan*MuLeZhengZhiJingJiXueYuanLiJiQiZaiSheHuiZheXueShangDeRuoGanYingYongXiaJuan1991}。但由于“生产费用不变的商品是没有的”,所以生产费用尺度实际上“同交换价值尺度一样是不存在的”\cite[105]{YueHan*MuLeZhengZhiJingJiXueYuanLiJiQiZaiSheHuiZheXueShangDeRuoGanYingYongXiaJuan1991}。最后,尽管在价值尺度方面取得进展,穆勒却意识到了许多古典政治经济学犯了混淆价值尺度和价值决定的错误,他指出:“我们不应该把价值尺度的概念同价值规定者或决定价值的原理相混淆。”\cite[107]{YueHan*MuLeZhengZhiJingJiXueYuanLiJiQiZaiSheHuiZheXueShangDeRuoGanYingYongXiaJuan1991}笔者认为,在区分价值尺度和价值决定一点上,穆勒是相当值得肯定的。

 穆勒还认为,商品要有价值,一方面必须具备一定的效用,另一方面必须是稀缺的\footnote{也就是在获得上存在困难。}\cite[499]{YueHan*MuLeZhengZhiJingJiXueYuanLiJiQiZaiSheHuiZheXueShangDeRuoGanYingYongShangJuan1991}。而且商品的价值是由竞争决定的,一方面“买主力求贱买,卖主则力求贵卖”,另一方面“质量相同的物品,在同一市场上不能有两种价格\footnote{根据上下文,这里也可以理解为“价值”}”\cite[497]{YueHan*MuLeZhengZhiJingJiXueYuanLiJiQiZaiSheHuiZheXueShangDeRuoGanYingYongShangJuan1991}。穆勒又将价值(交换价值)区分为暂时价值(市场价值)和永久价值(自然价值),前者是指受到供求法则支配的经常变动的价值,后者是指前者波动的中心\cite[2]{YueHan*MuLeZhengZhiJingJiXueYuanLiJiQiZaiSheHuiZheXueShangDeRuoGanYingYongXiaJuan1991}。为了研究自然价值,穆勒进一步把商品分成三类:第一类是供给绝对有限,其数量不能任意增加的商品,例如古代雕塑、特殊土壤、气候才能生产的葡萄酒等\cite[502]{YueHan*MuLeZhengZhiJingJiXueYuanLiJiQiZaiSheHuiZheXueShangDeRuoGanYingYongShangJuan1991};第二类是用劳动和资本可以任意增加的商品,如大多数的工业产品\cite[502,532]{YueHan*MuLeZhengZhiJingJiXueYuanLiJiQiZaiSheHuiZheXueShangDeRuoGanYingYongShangJuan1991};第三类是“用劳动和费用可以无限量地增加,但不能以固定数量的劳动和费用无限量地增加”,例如农产品\cite[502]{YueHan*MuLeZhengZhiJingJiXueYuanLiJiQiZaiSheHuiZheXueShangDeRuoGanYingYongShangJuan1991}。“这类商品不是有一种生产费用,而是有几种生产费用”\cite[532]{YueHan*MuLeZhengZhiJingJiXueYuanLiJiQiZaiSheHuiZheXueShangDeRuoGanYingYongShangJuan1991}。这三类商品有着不同的价值决定法则。

 第一类商品的价值由供求法则决定。“需求增加,则价值上升;需求减少,则价值降低。另一方面,供给减少,则价值上升,供给增加,则价值下降。价值的上升或降低将继续下去,直到需求和供给再度平衡为止。”\cite[506-507]{YueHan*MuLeZhengZhiJingJiXueYuanLiJiQiZaiSheHuiZheXueShangDeRuoGanYingYongShangJuan1991}

 第二类商品的价值由生产费用决定。穆勒认为,这类商品都有一个“最低价值”,这个“最低价值”是由生产费用加上通常的利润构成的,\cite[510-511]{YueHan*MuLeZhengZhiJingJiXueYuanLiJiQiZaiSheHuiZheXueShangDeRuoGanYingYongShangJuan1991}。而这里的生产费用指的是劳动数量\cite[517]{YueHan*MuLeZhengZhiJingJiXueYuanLiJiQiZaiSheHuiZheXueShangDeRuoGanYingYongShangJuan1991}。值得注意的是,这里的“劳动数量”包括生产过程中所投入的资本,因为资本也是由先前的劳动创造出来的\cite[517]{YueHan*MuLeZhengZhiJingJiXueYuanLiJiQiZaiSheHuiZheXueShangDeRuoGanYingYongShangJuan1991}。另外,这里的“劳动数量”也不是用劳动工资来衡量的,因为劳动工资的升降,会同时影响所有商品,所以不会改变商品之间的相对价值\cite[519-520]{YueHan*MuLeZhengZhiJingJiXueYuanLiJiQiZaiSheHuiZheXueShangDeRuoGanYingYongShangJuan1991}。但是,如果不同行业之间的工资有差异,那么这种差异就会“改变不同商品的相对生产费用”,并影响商品的价值\cite[521]{YueHan*MuLeZhengZhiJingJiXueYuanLiJiQiZaiSheHuiZheXueShangDeRuoGanYingYongShangJuan1991}。穆勒还指出了价值由生产费用决定和由供求法则决定这两种决定方式之间的关系,他认为对于第二类商品,其价值“并非取决于需求和供给;相反地,需求和供给取决于价值”\cite[515]{YueHan*MuLeZhengZhiJingJiXueYuanLiJiQiZaiSheHuiZheXueShangDeRuoGanYingYongShangJuan1991}。

第三类商品的价值由“以最大费用生产并运至市场的那部分供应量的生产费用”\cite[535]{YueHan*MuLeZhengZhiJingJiXueYuanLiJiQiZaiSheHuiZheXueShangDeRuoGanYingYongShangJuan1991}决定。这里的“以最大费用生产”应当是指最劣等土地的生产费用。由于这里的最劣等土地是指那些仅能产生普通利润而不能产生地租的土地\footnote{这说明穆勒否认绝对地租的存在\cite[176]{YanZhiJieXiFangJingJiXueShuoShiJiaoChengDiErBan2013}},所以穆勒进一步认为地租不是这类商品生产费用的组成部分,也就不影响这类商品的价值\cite[596]{YueHan*MuLeZhengZhiJingJiXueYuanLiJiQiZaiSheHuiZheXueShangDeRuoGanYingYongShangJuan1991}。最后,我们可以推断第三类商品价值生产费用决定和供求法则决定这两种决定方式之间的关系和第二类商品是一致的。

注意,以上的分析都是建立在不同行业的资本、劳动投入占比相同而且占用资本时间相同的前提之下的,如果这个前提不成立,那么按照等量资本带来等量利润的原则,商品将不仅仅按照生产所需的劳动数量比例进行交换,而且一般利润的每一次升降都会影响价值\cite[526-527]{YueHan*MuLeZhengZhiJingJiXueYuanLiJiQiZaiSheHuiZheXueShangDeRuoGanYingYongShangJuan1991}。可见,穆勒也没能较好地解释“转型问题”,而只是强行增加了价值决定的法则。

总的来说,穆勒价值理论的最大特点就是折衷,他通过对各种商品进行分类,将劳动价值论、供需价值论、生产费用价值论糅合在了一起。另外,正如熊彼特所言,穆勒自己的贡献是“充分发展了供给与需求分析”\cite[359]{YueSeFu*XiongBiTeJingJiFenXiShiDi2Juan2017}。他成功地将商品价值由生产费用决定和由供需法则决定这两种决定方式统一在了一起。

 \subsection{约翰·穆勒的生产力概念}

穆勒认为,劳动生产的是效用\cite[60]{YueHan*MuLeZhengZhiJingJiXueYuanLiJiQiZaiSheHuiZheXueShangDeRuoGanYingYongShangJuan1991}。效用有三类:第一类是“固定和体现在外界物体中的效用”\cite[62]{YueHan*MuLeZhengZhiJingJiXueYuanLiJiQiZaiSheHuiZheXueShangDeRuoGanYingYongShangJuan1991},即平常认为的物的效用;第二类是“固定和体现在人身上的效用”,也就是“对自己和别人有用的品质”\cite[62]{YueHan*MuLeZhengZhiJingJiXueYuanLiJiQiZaiSheHuiZheXueShangDeRuoGanYingYongShangJuan1991},笔者理解为是人力资本;第三类则是“存在于所提供地服务中”的效用\cite[62]{YueHan*MuLeZhengZhiJingJiXueYuanLiJiQiZaiSheHuiZheXueShangDeRuoGanYingYongShangJuan1991}。据此,我们不难推测穆勒和前文提到的其他经济学家一样,认识到了绝对生产力的存在。

接着,穆勒总结了五条绝对生产力提高的原因:第一是有利的自然条件;第二是较大的劳动干劲;第三是较高的技能和知识;第四是较高的整个社会的知识水平和相互信任程度;第五是较高的安全感。\cite[123-135]{YueHan*MuLeZhengZhiJingJiXueYuanLiJiQiZaiSheHuiZheXueShangDeRuoGanYingYongShangJuan1991}

但如前文所述,穆勒还强调了价值概念的相对性,穆勒讨论并否认了“所有商品价值一起提高”的可能性,这就使他拥有了探究比较生产力的潜力。假使所有商品的绝对生产力都提高,那么某一商品的价值变化就要考虑其它商品的绝对生产力变化之幅度。遗憾的是,穆勒并没有在此深究。

\subsection{生产力与价值的关系}

根据穆勒对价值决定的理解,我们不难推断:如果资本、劳动投入占比不变且占用资本时间不变,那么绝对生产力提高会使单位商品的生产费用下降,进而使单位商品的价值下降。

\section{总结}

\subsection{劳动作为衡量交换价值的尺度}

如前所述,亚当·斯密在国富论中首次提出了“交换价值\footnote{这里应当是价“价值”而非“交换价值;这里笔者先按照几位古典政治经济学家的原文使用“交换价值”一词;交换价值和价值的关系在后文中阐述。}的真实尺度”问题,并指出劳动是这一问题的答案。而后李嘉图、马克思、马尔萨斯都认为劳动是衡量交换价值的尺度。值得注意的是,马克思认为古典政治经济学的根本倾向是劳动价值论,但这一论断忽视了古典政治经济学不同理论之间的差异性\footnote{由于学界对古典政治经济学的划分不尽相同,对古典政治经济学家思想的理解也不尽相同,所以有学者认为古典政治经济学所得出的结论是“多要素供求价值论”\cite[179]{CaiJiMingCongGuDianZhengZhiJingJiXueDaoZhongGuoTeSeSheHuiZhuYiZhengZhiJingJiXueJiYuZhongGuoShiJiaoDeZhengZhiJingJiXueYanBianShangCe2023}。}。根据笔者对古典政治经济学的认识,笔者认为更有说服力的结论是:古典政治经济学的基本倾向是把劳动作为衡量交换价值的尺度;这里的价值尺度,指的是衡量商品价值量的社会标准,强调价值的社会属性。

一方面,古典政治经济学家意识到,在社会分工建立起来后,一个生产者或许因为不能生产某一需要的商品,或许因为通过交换得到的商品更好更便宜,所以会愿意用自己的劳动生产一些不愿自己消费的商品来交换他物。然而,商品都具有作为使用价值的异质性,为了与其它商品进行交换,商品的拥有者会根据在这一商品上花费的劳动\footnote{包括活劳动和物化劳动}与他人进行交换。这是因为这个生产者为了生产者这一商品所投入的就是自己的劳动,也只能知道自己为了这一商品投入了多少劳动。而那些与他进行商品交换的生产者,也是以同样的考量与他进行交换。因此,劳动成为了商品交换时所依据的尺度。换言之,劳动成为了衡量交换价值的尺度。\cite[1016-1017]{ZhongGongZhongYangMaKeSiEnGeSiLieNingSiDaLinZhuZuoBianYiJuMaKeSiEnGeSiWenJiDi7Juan2009}\cite[25]{YaDang*SiMiGuoFuLun2015}\cite[133]{MaErSaSiZhengZhiJingJiXueDingYi2023}

另一方面,交换价值会随着时间和地点的不同上下波动,具有偶然性和相对性。因此,古典政治经济学家所能找到的不变的、普遍的交换价值的尺度,只有劳动。\cite[49-51]{ZhongGongZhongYangMaKeSiEnGeSiLieNingSiDaLinZhuZuoBianYiJuMaKeSiEnGeSiWenJiDi5Juan2009}\cite[28-30]{YaDang*SiMiGuoFuLun2015}

\subsection{从交换价值到价值}

然而,交换价值是一种使用价值同另一种使用价值的比例,其本身没有单位,也没有数量大小上的意义。所以前文中古典政治经济学家使用“交换价值的尺度”这一说法是不合理的——一个没有大小的量怎么能被“衡量”呢?事实上,古典政治经济学家所衡量的其实不是交换价值,而是价值。只是在政治经济学发展的早期阶段,经济学家们还没有能力从交换价值中抽象出价值的概念。直到马克思首次区分了交换价值和价值,价值的面纱才被揭开\cite[86-87]{ChenDaiSunCongGuDianJingJiXuePaiDaoMaKeSiRuoGanZhuYaoXueShuoFaZhanLueLun2014}。

笔者认为,政治经济学逐步区分交换价值和价值的过程可以从两个角度来理解。

首先,从价值实体的角度来看,假设某种商品可以和多种商品进行交换,那么这一商品就具有了许多种交换价值,这些交换价值之间必然是可以相互替代的。所以,正如马克思所说:“第一,同一种商品的各种有效的交换价值表示一个等同的东西。第二,交换价值只能是可以与它相区别的某种内容的表现方式,‘表现形式’。”\cite[49]{ZhongGongZhongYangMaKeSiEnGeSiLieNingSiDaLinZhuZuoBianYiJuMaKeSiEnGeSiWenJiDi5Juan2009}也就是说,这些不同的交换价值都可以被化为“一种等量的共同的东西”\cite[49]{ZhongGongZhongYangMaKeSiEnGeSiLieNingSiDaLinZhuZuoBianYiJuMaKeSiEnGeSiWenJiDi5Juan2009}。而且,这种共同东西既不是使用价值——使用价值具有异质性,也不是交换价值——交换价值仅仅是一个比例,没有大小,也无法被“衡量”。进而,马克思把这种共同的东西称为价值\cite[50]{ZhongGongZhongYangMaKeSiEnGeSiLieNingSiDaLinZhuZuoBianYiJuMaKeSiEnGeSiWenJiDi5Juan2009}。值得注意的是,尽管马克思已经能从交换价值中抽象出价值来,但是笔者却不认同马克思推演的逻辑。马克思在意识到不同的交换价值是在衡量一个既不是使用价值也不是交换价值的“共同的东西”之后,就说“如果把商品体的使用价值撇开,商品体就只剩下一个属性,即劳动产品这个属性”\cite[50-51]{ZhongGongZhongYangMaKeSiEnGeSiLieNingSiDaLinZhuZuoBianYiJuMaKeSiEnGeSiWenJiDi5Juan2009},再把具体劳动的成分撇开,那么剩下的是“无差别的人类劳动的单纯凝结”,这种凝结的抽象劳动,就是商品的价值实体\cite[51]{ZhongGongZhongYangMaKeSiEnGeSiLieNingSiDaLinZhuZuoBianYiJuMaKeSiEnGeSiWenJiDi5Juan2009}。这里的问题主要在于:若主张使用价值的异质性使其不能决定价值,则同理可证劳动的异质性亦不具备价值决定功能;若承认具体劳动可抽象为无差别人类劳动,则此抽象逻辑同样可以将使用价值抽象为无差别的效用\cite[84]{CaiJiMingLunJieZhiJueDingYuJieZhiFenPeiDeTongYi2003}。也就是说,将商品的价值和“无差别的人类劳动的单纯凝结”等同起来并不是一个不证自明的过程,马克思的论证逻辑是存在问题的。但是,笔者认为马克思将价值从交换价值中抽象出来的逻辑过程是正确的,价值确实是一种既不同于使用价值也不同于交换价值的,客观存在的实体。

正因如此,笔者进一步认为前文中不同经济学家对“交换价值的尺度”的争论实际上是对“价值尺度”的争论。交换价值本身只是两种使用价值的比例,没有“大小”的概念,因此也无法被“衡量”。而价值作为一个实体,理论上是可以被衡量的。不同经济学家所争论的,正是价值应该用什么尺度来衡量,或者说价值的单位究竟是什么。

其次,人们对价值认识的演进和商品交换的发展是两个相反的过程。在商品交换的早期阶段,商品的交换价值表现为两个使用价值的比例关系\cite[49]{ZhongGongZhongYangMaKeSiEnGeSiLieNingSiDaLinZhuZuoBianYiJuMaKeSiEnGeSiWenJiDi5Juan2009},也就是简单的交换价值形式\footnote{在资本论中,马克思用的是“简单价值形式”一词,但由于马克思没有绝对严谨地区分价值和交换价值\cite[37]{ZhongGongZhongYangMaKeSiEnGeSiLieNingSiDaLinZhuZuoBianYiJuMaKeSiEnGeSiWenJiDi8Juan2009},所以这里笔者采取了蔡继明教授的更严谨的提法\cite[145]{CaiJiMingJieZhiZhengLunHuiGuYuZhanWang2008}。};随着交换范围的扩大,简单的交换价值形式演化为扩大的交换价值形式;当一般等价物出现后,所有的商品都借助一般等价物进行交换,交换价值就发展为一般的形式;当货币产生后,交换价值便取得价格的形式。在价格形式出现后,人们发现尽管市场价格受供求波动影响,但从长期来看其始终会围绕着一个相对稳定的轴心运动——这个轴心,或者说调节价格运动的规律,被亚当·斯密称为"自然价格"、马尔萨斯谓之"自然价值",最终在政治经济学发展中凝练为“价值”概念。至此,价值作为价格运动规律的本质规定性得以确立。\cite[145]{CaiJiMingJieZhiZhengLunHuiGuYuZhanWang2008}\footnote{事实上马克思也把价值的这种内涵以价值作用的方式表述了出来\cite[199]{ZhongGongZhongYangMaKeSiEnGeSiLieNingSiDaLinZhuZuoBianYiJuMaKeSiEnGeSiWenJiDi7Juan2009}。}随着人们进一步认识到价格只是交换价值的一种形式,价值的内涵也进一步一般化为调节交换价值的规律。

正如马克思所言:“对人类生活形式的思索,从而对这些形式的科学分析,总是采取同实际发展相反的道路。这种思索是从事后开始的,就是说,是从发展过程的完成的结果开始的。$\cdots$因此,只有商品价格的分析才导致价值量的决定,只有商品共同的货币表现才导致商品的价值性质的确定。”\cite[93]{ZhongGongZhongYangMaKeSiEnGeSiLieNingSiDaLinZhuZuoBianYiJuMaKeSiEnGeSiWenJiDi5Juan2009}在现实层面,价值研究确实肇始于对价格现象的经验观察;在理论层面,支配交换价值的规律作为价值的内在本质,也符合人类认知从现象到本质的渐进过程,体现了历史发展的内在连贯性。

上述对价值实体认知的演进,自然引向价值决定机制的核心争论——究竟何种劳动形态构成价值尺度?这需要从价值形成与价值决定的关系维度展开分析。

\subsection{价值尺度和价值源泉的对立统一}

前文的论述表明,把劳动作为价值的尺度是古典政治经济学的基本倾向\footnote{值得一提的是,作为新古典价值论代表人物的马歇尔也有用支配的劳动作为价值尺度的倾向\cite{perskyMarshallsNeoClassicalLaborValues1999}}。但正如笔者在亚当·斯密相关章节中探讨的学术争议所示,古典政治经济学家们在该用商品生产所耗费的劳动还是用商品所能支配的劳动来作为价值尺度的问题上产生了严重的分歧。总的来说,劳动价值论派的李嘉图、马克思认为应当用耗费的劳动作为价值尺度,而多要素价值论派的斯密、马尔萨斯则认为应当用支配的劳动作为价值尺度。

\subsubsection{商品所能支配的劳动作为商品价值的尺度}

在笔者看来,用商品所能支配的劳动作为价值尺度无疑是更合适的。

首先,正如马克思所说:“价值的对象性只能在商品同商品的社会关系中表现出来”\cite[61]{ZhongGongZhongYangMaKeSiEnGeSiLieNingSiDaLinZhuZuoBianYiJuMaKeSiEnGeSiWenJiDi5Juan2009},价值本身具有社会属性。

恩格斯曾在《<资本论>第三册增补》中举了一个例子来论证应当用商品生产所耗费的劳动作为商品价值的尺度\cite[1015-1018]{ZhongGongZhongYangMaKeSiEnGeSiLieNingSiDaLinZhuZuoBianYiJuMaKeSiEnGeSiWenJiDi7Juan2009}。恩格斯说,在商品经济发展的初期,进行商品交换的主要是劳动的农民。这些农民借助自己家庭的帮助,在自己的田地上进行农业、畜牧业和手工业的生产,并拿除必需品之外剩下的剩余产品同其它农民家庭进行交换。这些农民之所以进行交换,并非因为自己不会生产这些物品\footnote{这一对可变分工体系的认识与广义价值论的假设有异曲同工之妙,可惜恩格斯并没有把可变分工体系推广到一般的商品经济。},而是因为得不到原料或者因为通过交换得到的物品要更好或更便宜。这些农民在生产用于交换的产品时——无论是为了补偿工具还是为了加工原料——所耗费的只有自己的劳动。于是,恩格斯得出用耗费的劳动作为价值尺度结论:“在这里,花在这些产品上的劳动时间不仅对于互相交换的产品量的数量规定来说是唯一合适的尺度;在这里,也根本不可能有别的尺度。”\cite[1016]{ZhongGongZhongYangMaKeSiEnGeSiLieNingSiDaLinZhuZuoBianYiJuMaKeSiEnGeSiWenJiDi7Juan2009}但如果我们进一步深入思考,会发现恩格斯的这一结论是不能成立的。

正如恩格斯所言,农民之所以选择交换,不是因为自己不能生产,而是因为受到资源约束或者因为通过交换能节约自己的劳动。当农民在用劳动作为尺度衡量是否要进行交换的时候,他衡量的是“如果不进行交换,为了得到同样的产品我要多耗费多少劳动?”以及“为了得到同样的产品,如果进行交换我能省下多少劳动?”于是,这些“多耗费的劳动”或者“省下的劳动”就是这个农民眼中,通过交换得到的商品的价值。这里笔者想提请读者注意,农民是通过自己的劳动衡量了别人生产的商品的价值。换句话说,假设有农民A、B,分别花费劳动$L_A$和$L_B$生产商品$C_A$和$C_B$,现在农民A想要通过交换得到商品$C_B$,于是农民A会衡量“多耗费的劳动”或者“省下的劳动”。我们记农民A若生产商品$C_B$比农民B生产商品$C_B$多耗费的劳动为$\Delta L_A$。此时,$\Delta L_A$所衡量的,正是农民B生产的商品$C_B$的价值。同样,农民B也是以同样的逻辑用$\Delta L_B$衡量了商品$C_A$的价值\footnote{当然,这里所指的各种劳动量都可以按照马克思所述的“抽象劳动”概念来理解。}。至此,笔者的分析应当说是忠于恩格斯的分析逻辑的。接下来,根据马克思劳动价值论的等价交换原则,要让上述交换能够长久地持续,农民A必须拿出在农民B看来有足够价值的商品与农民B进行交换;所以要让交换持续,$\Delta L_A = \Delta L_B$必须成立。再根据马克思劳动价值论所认为的“等量劳动带来等量价值”的价值决定原理,农民A必须付出与$\Delta L_B$相等的劳动$L_A$才能创造出$\Delta L_B$的价值,所以我们得到$\Delta L_B = L_A$。也就是说,商品$C_B$实际上能够支配$L_A$量的劳动;进而我们可以说,商品$C_B$的价值是由能够支配$L_A$量的劳动来衡量的;这等价于说商品所能支配的劳动是商品价值的尺度。同样地,我们可以得到$\Delta L_A = L_B$,所以最终有$L_B =\Delta L_A = \Delta L_B = L_A$。以上的分析表明,按照恩格斯分析的逻辑,在简单交换中商品所能支配的劳动是商品价值的尺度;并且,用商品能够支配的劳动作为价值的尺度与马克思的劳动价值论并没有发生任何结论上的冲突。

在成熟的商品经济中,商品能够支配的劳动仍然是价值的尺度。马克思在《资本论》第一卷的第三章中指出,货币的作用之一是价值尺度\cite[114-124]{ZhongGongZhongYangMaKeSiEnGeSiLieNingSiDaLinZhuZuoBianYiJuMaKeSiEnGeSiWenJiDi5Juan2009}。而一种贵金属,例如金,为什么可以成为货币?马克思认为这是因为金本身是劳动产品,因而具有潜在可变的价值\cite[118]{ZhongGongZhongYangMaKeSiEnGeSiLieNingSiDaLinZhuZuoBianYiJuMaKeSiEnGeSiWenJiDi5Juan2009}。那么,如果我们用金来衡量某一件商品的价值,实际上是把由商品生产者耗费的劳动与金的生产者所耗费的抽象劳动进行了比较与折算。于是这件商品的价值不仅是被金衡量出来了,而且是本质上被金的生产者所耗费的抽象劳动衡量出来了。可见,在成熟的商品经济中,把商品能够支配的劳动作为商品价值的尺度也是符合马克思分析的逻辑和结论的。但笔者也注意到,马克思曾在《哲学贫困中》中批评“把用商品中所包含的劳动量来衡量的商品价值和用“劳动价值”来衡量的商品价值混为一谈”的行为\cite[97]{ZhongGongZhongYangMaKeSiEnGeSiLieNingSiDaLinZhuZuoBianYiJuMaKeSiEnGeSiQuanJiDi4Juan1958}。对此,笔者认为马克思所批评的主要是“用‘劳动价值’作为价值尺度”的观点,因为在马克思看来,商品所能支配的劳动就是指“劳动者的报酬”,而如果用劳动者的报酬作为价值尺度,那就相当于把生产费用作为价值尺度,会出现生产费用取决于价值而价值又用生产费用衡量的循环论证问题\cite[98]{ZhongGongZhongYangMaKeSiEnGeSiLieNingSiDaLinZhuZuoBianYiJuMaKeSiEnGeSiQuanJiDi4Juan1958}。\cite[5]{ZhangLeiShengMaKeSiLaoDongJieZhiLunYanJiuDeLiShiZhengTiXing2015}所以,笔者想要强调的是,笔者所指的支配劳动绝不是指劳动者的报酬,而是指一般的、平均的社会劳动。

事实上,如果从广义价值论的角度来看,耗费的劳动是具体的特殊的个别劳动,而支配的劳动是将不同的、具体的个别劳动折算得到的一般的、平均的社会劳动。

最后,笔者想要强调的是,将不同的、具体的个别劳动折算得到的一般的、平均的社会劳动的方式在不同的价值理论中是不一样的。例如马克思认为劳动本身具有二重性:“由于人隶属于机器或由于极端的分工,各种不同的劳动逐渐趋于一致”\cite[96]{ZhongGongZhongYangMaKeSiEnGeSiLieNingSiDaLinZhuZuoBianYiJuMaKeSiEnGeSiQuanJiDi4Juan1958};“在使用机器的企业中,这个工人的劳动和那个工人的劳动几乎没有什么差别;工人彼此间的区别,只是他们在劳动中所化的时间不等。”\cite[97]{ZhongGongZhongYangMaKeSiEnGeSiLieNingSiDaLinZhuZuoBianYiJuMaKeSiEnGeSiQuanJiDi4Juan1958}也就是说,商品耗费的劳动本身就具有抽象劳动的属性;而斯密等经济学家则认是市场上的讨价还价消除了劳动的异质性。目前,经济学界对此还没有一个统一的认识,所以笔者认为在这一点上仍然存在研究的空间\footnote{笔者认为不同折算方式的本质,是对“等价交换”概念的不同理解,可参考《论耗费的劳动与购买的劳动在价值理论中的作用》\cite[69]{CaiJiMingLunHaoFeiDeLaoDongYuGouMaiDeLaoDongZaiJieZhiLiLunZhongDeZuoYong2022}。}。不过,无论是哪种转换方式,都强调了价值的社会属性。因此从社会属性的角度出发,不难看出用支配的劳动比耗费的劳动作为价值尺度更能表现价值的社会属性。

\subsubsection{参与价值决定的所有要素作为价值源泉}

前文提到,古典政治经济学的基本倾向是把劳动作为价值尺度,但所有的古典政治经济学家都承认耗费的劳动作为一种价值源泉,在价值决定的过程中发挥了重要作用。事实上,从古典学派内部诸多学者到新古典经济学派,直至广义价值论体系,均承认非劳动要素同样参与价值决定过程,因而具备价值源泉属性。

笔者将从以下几方面来论证应当把参与价值决定的所有要素作为价值源泉。

首先,承认参与价值决定的所有要素作为价值源泉并不是指非劳动要素可以直接替代劳动决定价值,而是指与劳动共同决定价值,只不过二者的作用或贡献都是用购买劳动来折算和表现的。按照广义价值论的观点,非劳动要素是通过影响特定生产者劳动的绝对生产力,进而影响特定生产者相对于交换对手而言的比较生产力,最终参与价值决定的。

其次,从经济现实来看,中共十九届四中全会指出,要“健全劳动、资本、土地、知识、技术、管理、数据等生产要素由市场评价贡献、按贡献决定报酬的机制”\cite[39]{ZhongGuoGongChanDangDiShiJiuJieZhongYangWeiYuanHuiDiSiCiQuanTiHuiYiWenJianHuiBian2019}。笔者认为,这里的“贡献”指的就是各生产要素对价值创造的贡献——倘若我们承认上述六种非劳动生产要素对提升绝对生产力有帮助,根据前文和广义价值论的基本原理,那实际上就是承认了非劳动生产要素对价值创造的贡献\footnote{关于这方面的讨论,参见(蔡继明,2023)\cite[26-56]{CaiJiMingCongGuDianZhengZhiJingJiXueDaoZhongGuoTeSeSheHuiZhuYiZhengZhiJingJiXueJiYuZhongGuoShiJiaoDeZhengZhiJingJiXueYanBianXiaCe2023}}。笔者认为,十九届四中全会把按生产要素贡献分配的分配制度纳入社会主义基本经济制度的范畴\cite[5]{XieFuZhanWanShanJiBenJingJiZhiDuTuiJinGuoJiaZhiLiTiXiXianDaiHuaXueXiGuanCheZhongGongShiJiuJieSiZhongQuanHuiJingShenBiTan2020},一方面是因为经济现实的变化让各种生产要素作为价值源泉的身份越来越凸显。例如,十九届四中全会首次将“数据”增列为生产要素,体现了现代经济增长的新特征\cite[6]{CaiJiMingLunShuJuYaoSuAnGongXianCanYuFenPeiDeJieZhiJiChuJiYuGuangYiJieZhiLunDeShiJiao2023}\cite[5]{XieFuZhanWanShanJiBenJingJiZhiDuTuiJinGuoJiaZhiLiTiXiXianDaiHuaXueXiGuanCheZhongGongShiJiuJieSiZhongQuanHuiJingShenBiTan2020}。另一方面是因为将按生产要素贡献分配确定社会主义基本经济制度,体现了收入分配制度尊重知识、尊重人才、尊重创新的导向,可以更好地解放和发展社会生产力、推动经济高质量发展\cite[4-5]{XieFuZhanWanShanJiBenJingJiZhiDuTuiJinGuoJiaZhiLiTiXiXianDaiHuaXueXiGuanCheZhongGongShiJiuJieSiZhongQuanHuiJingShenBiTan2020}。

\subsubsection{价值源泉向价值因素的转化}

基于广义价值论的视角,价值形成过程应区分为两个阶段:在生产阶段,各种生产要素通过劳动过程将具体化、私人化的劳动转化为特定商品的使用价值。此时,各种价值源泉仅转化为潜在的价值因素,但尚未形成可量化的价值实体。因此笔者把这一过程称为价值源泉向价值因素转化的过程。在交换阶段,通过社会化的市场竞争和供需双方的博弈,具体劳动才真正实现向抽象劳动的转化,价值实体才在质上最终形成,在量上决定下来。因此笔者把这一过程称为价值决定的过程。

笔者区分价值源泉向价值因素的转化过程和和价值的决定过程有以下三个原因。 首先,包括广义价值论在内的许多价值理论都认为价值决定是在商品交换时完成的。也就是说,在商品被生产出来后到交换完成之前的这段过程中,商品的价值的质还没有形成,商品的价值的量也没有决定下来。一方面,商品价值的决定是一个社会的过程,而只有商品交换是在社会中完成的;在完成交换之前,商品耗费的劳动只是具体的、私人的劳动,只有在商品交换完成后,这些耗费的劳动才能转化为抽象的、社会的劳动,社会意义上的价值才能出现。因此,为了区分商品的生产和交换对价值的不同影响,笔者把价值形成划归到商品的生产过程,把价值决定划归到商品的交换过程。其次,这种区分和规定具有普适性。如果我们考虑商品的使用价值,那么其形成正是在商品的生产阶段,其决定是在商品的消费阶段,其尺度是消费者的效用;显然,在商品被生产出来之后和被消费之前,其使用价值只是潜在的、没有决定也没有表现出来的,只有当消费者真正消费商品时,商品的使用价值才被效用衡量而决定下来。上述对使用价值形成和决定的描述也适用于对效用价值论的分析。最后,这种区分是对马克思主义政治经济学的一种继承。尽管马克思不承认非劳动要素也参与价值形成,但他在早期\footnote{如前文所述,马克思在后期认为耗费的劳动本身包含了抽象劳动,因此商品的价值在生产过程就已经形成了。}的著作中认为:“商品的价值由生产成本即劳动决定,是通过竞争的作用实现的,只有这样,商品的价值才能最终由生产该商品所耗费的劳动来决定。”\cite[3]{ZhangLeiShengMaKeSiLaoDongJieZhiLunYanJiuDeLiShiZhengTiXing2015}。 这就是说,马克思也意识到商品的价值在生产完成后,交换完成前处于一种潜在的状态,只有当商品在充满竞争的市场上完成了交换,商品的价值才真正决定下来。换句话说,马克思也意识到了价值源泉向价值因素的转化和价值决定之间的区别,只是他没有将其诉诸文字。

\subsubsection{价值决定——价值源泉和价值尺度的对立统一}

总的来说,耗费的劳动作为价值的源泉实现价值形成,支配的劳动作为价值尺度实现价值衡量。而价值的源泉和价值的尺度在商品交换的过程中实现了对立统一,商品的价值被决定下来。

首先,从微观的层面考察,价值源泉与价值尺度的对立统一并不体现为量上的完全等同,而在于两者可通过特定机制实现折算转换。广义价值论揭示了一个核心规律:部门间耗费劳动与支配劳动的比值等于其比较生产力的相对水平。具体表现为,当某部门比较生产力高于另一部门时,该部门能以较少耗费劳动置换对方较多劳动量,反之则需付出更多劳动;当两部门比较生产力之比等于1时,此时耗费劳动量恰与支配劳动量相等,这种特殊状态正对应着马克思劳动价值论中关于价值决定的基本论断。需要特别说明的是,广义价值论之所以强调比较生产力对价值决定的作用,在于其理论框架不仅考察劳动要素,还纳入非劳动生产要素对绝对生产力的实际影响,这类要素同样会作用于比较生产力的形成过程。

其次,从宏观总量的角度来说,由于耗费的劳动总量等于支配的劳动总量等于全社会价值总量\cite[71-72]{CaiJiMingLunHaoFeiDeLaoDongYuGouMaiDeLaoDongZaiJieZhiLiLunZhongDeZuoYong2022},所以作为价值源泉的耗费劳动和作为价值尺度的支配劳动是对立统一的。对于商品经济来说,任何支配的劳动都必然是要耗费的劳动——价值实体只能取决于耗费的劳动;任何耗费的劳动也都必然是要被支配的劳动——正如商品是用于交换的产品,自己耗费的劳动也是为了支配他人劳动而付出的劳动。在斯密的价值理论中,价值源泉和价值尺度是对立统一的。斯密认为,由每个国家全年总劳动耗费所得的所有商品一定可以被分解为工资、利润和地租三个部分。而这三个部分又可以用劳动作为价值尺度进行衡量从而被折算为所有商品能支配的劳动。因此,从一个国家特定时期的总量上看,支配的劳动等于耗费的劳动。\cite[41-48]{YaDang*SiMiGuoFuLun2015}在马克思的劳动价值论中,价值源泉和价值尺度也是对立统一的。马克思指出过两种不同含义的社会必要劳动\cite[29]{CaiJiMingBiYaoLaoDongIHeBiYaoLaoDongIIGongTongJueDingJieZhi1995}\cite[19]{GuShuTangDuiJieZhiJueDingHeJieZhiGuiLuDeZaiTanTao1982}:第一种含义的社会必要劳动时间是指“在现有的社会正常的生产条件下制造某种使用价值所需要的劳动时间”\cite[52]{ZhongGongZhongYangMaKeSiEnGeSiLieNingSiDaLinZhuZuoBianYiJuMaKeSiEnGeSiWenJiDi5Juan2009},第二种含义的社会必要劳动时间则是指“既然社会要满足需要,并为此目的而生产某种物品,它就必须为这种物品进行支付”,“所以,社会购买这些物品的方法,就是把它所能利用的劳动时间的一部分用来生产这些物品,也就是说,用该社会所能支配的劳动时间的一定量来购买这些物品”\cite[208]{ZhongGongZhongYangMaKeSiEnGeSiLieNingSiDaLinZhuZuoBianYiJuMaKeSiEnGeSiWenJiDi7Juan2009};或者说是“当时社会平均生产条件下生产市场上这种商品的社会必需总量所必要的劳动时间”\cite[722]{ZhongGongZhongYangMaKeSiEnGeSiLieNingSiDaLinZhuZuoBianYiJuMaKeSiEnGeSiWenJiDi7Juan2009}。如果我们把第一种含义的社会必要劳动看作在商品生产中耗费的劳动,把第二种含义的社会必要劳动时间看作为满足社会对某种商品的需求而被商品所支配的总劳动。那么前者指的代表供给,是价值源泉;后者代表需求,是价值尺度。当两者总量相等时,也就是供求平衡而均衡价格的以确定时,全社会所有商品的两种含义的社会必要劳动时间总量也必然是相等的\cite[72]{CaiJiMingLunHaoFeiDeLaoDongYuGouMaiDeLaoDongZaiJieZhiLiLunZhongDeZuoYong2022}。
%% !TeX root = ../2019080346_Mason.tex

\chapter{新古典经济学中的生产力概念与价值理论}

\section{新古典经济学的界定}

"新古典"这一术语最早由T. 凡勃伦(T. Veblen)用于指称经济学家马歇尔及其经济学理论\cite[9382]{macmillanpublishersltdNewPalgraveDictionary2018}。当今经济学界通常将19世纪70年代至20世纪30年代的欧洲主流经济思想统称为新古典经济学\cite[241]{YanZhiJieXiFangJingJiXueShuoShiJiaoChengDiErBan2013}。在思想特征上,新古典经济学采用边际分析与均衡分析方法,并侧重于需求侧分析。从价值理论上看,新古典经济学包含了采用边际分析方法建立的边际价值论和马歇尔综合而成的新古典价值论\cite[181-184]{CaiJiMingCongGuDianZhengZhiJingJiXueDaoZhongGuoTeSeSheHuiZhuYiZhengZhiJingJiXueJiYuZhongGuoShiJiaoDeZhengZhiJingJiXueYanBianShangCe2023}。

在本章中,笔者将依次介绍这两个价值理论及其生产力概念。

\section{边际价值论及其生产力概念}

一般认为,边际价值论包含了边际效用价值论和边际生产力价值论\cite[181]{CaiJiMingCongGuDianZhengZhiJingJiXueDaoZhongGuoTeSeSheHuiZhuYiZhengZhiJingJiXueJiYuZhongGuoShiJiaoDeZhengZhiJingJiXueYanBianShangCe2023}。

\subsection{边际效用价值论}

19世纪70年代初期,英国的杰文斯、奥地利的门格尔和法国的瓦尔拉斯几乎同时而又各自独立地发现了边际效用递减原理,并以此为基础建立了边际效用价值论\cite[iii]{r.d.c.BuLaiKeJingJiXueDeBianJiGeMingShuoMingHePingJie2020}\cite[242]{YanZhiJieXiFangJingJiXueShuoShiJiaoChengDiErBan2013}。

杰文斯认为经济学应以消费理论为起点,因为消费是人类生活的目的和经济的动力,而生产仅是满足消费的手段。由于消费旨在追求快乐并避免痛苦,杰文斯用“引起快乐”和“避免痛苦”来定义效用,进而提出经济学的作用在于最大化人们从物品中获得的效用。他进一步指出,物品的效用遵循“最后效用程度递减”法则(即边际效用递减法则)。此外,杰文斯认为商品价值应由其边际效用强度决定,且商品间的交换价值取决于两者边际效用强度之比的倒数。最后,杰文斯还试图把劳动引入商品交换的过程,认为商品交换也会受到边际“生产成本”的制约。但是,杰文斯所说的生产成本并非指实际劳动耗费量及其产品,而是指人对劳动的心理感受,所以杰文斯实际上并没有提出边际效用价值论以外的价值论。\cite[125-136,142-149]{YanZhiJieCongBianJiGeMingDaoKaiEnSiGeMing2022}\cite[52-131]{SiTanLi*JieWenSiZhengZhiJingJiXueLiLun1984}

门格尔认为财货的价值性质源于其稀缺性及对人生命与福利的意义;价值则是人对这种意义的主观判断,其大小取决于财货支配量与人的愿望的对比关系。为探究价值的数量与尺度,门格尔指出,不同欲望种类及同一欲望的不同强度(随满足而递减)均影响价值量,且价值量呈递减趋势——欲望种类越次要、强度越弱,物品价值量越低。然而,由于商品可能满足不同种类或强度的欲望,门格尔提出以"最小欲望满足"作为统一尺度:当失去一定量物品时,人们必然放弃当下意义最小的欲望,因此该部分物品的价值由最小意义欲望决定。\cite[167-172]{YanZhiJieCongBianJiGeMingDaoKaiEnSiGeMing2022}\cite[52-80]{QiaEr*MenGeErGuoMinJingJiXueYuanLi2024}

瓦尔拉斯则把价格和价值混为一谈。首先,他认为物品的稀少性原理使得市场交换成为一个普遍现象,进而市场价格也成为一个普遍现象。而价格又取决于市场供给和需求的均衡——供求双方就商品的边际效用进行讨价还价。瓦尔拉斯承认供给和需求都是价值决定的因素,并在对供求关系的研究中引申出了边际效用的概念。总的来说,瓦尔拉斯的分析更强调需求和供给在价格形成中的相互制约以及各种商品价格决定之间的相互依赖,并最终建立了一般均衡理论。\cite[181-188]{YanZhiJieCongBianJiGeMingDaoKaiEnSiGeMing2022}\cite[139-147]{LaiAng*WaErLaSiChunCuiJingJiXueYaoYi1989}

\subsection{生产力与价值的关系}

如前所述,边际效用价值论只关注了需求侧对价值决定的作用,而几乎完全忽视了供给侧的作用。因此,边际效用价值论中的价值不仅难以用一般的的尺度来度量,而且与生产力失去了联系。事实上,杰文斯、门格尔和瓦尔拉斯几乎都默认商品已经被生产出来了,而不关注商品是如何被生产出来的。所以,在边际效用价值论中我们基本看不到与生产力有关的概念,更不用说对生产力和价值关系的分析了。

\subsection{边际生产力价值论}

边际生产力价值论的发展历史较边际效用论更长一些。马尔萨斯和李嘉图的地租理论中首次出现了边际生产力的概念——资本和劳动的边际收益会等于土地的边际收益。而后,朗菲尔德(Longfield)提出利润是由物质资本的边际生产力决定的;屠能(von Thünen)在同一时期把边际生产力运用到了对工资和资本的分析上,但耽误了很久才发表文章,没有产生什么影响。杰文斯也用边际生产力来解释利率,但是把工资解释为支付租金和利息后的剩余部分。总而言之,上述的这些分析都没能将边际原理推广到所有的生产要素上。最后是美国经济学家克拉克(J.B. Clark)和马歇尔在1890年左右各自独立地完成了推广的工作。\cite[8222-8224]{macmillanpublishersltdNewPalgraveDictionary2018}

克拉克在边际效用价值论的基础上作了一些修正。首先,他认为价值是一个社会性的概念,尽管商品的效用是个人的心理现象,但决定价值的应当是商品的“实际社会效用”(也就是对社会的边际效用)。接着,克拉克要从数量上衡量价值;克拉克首先考察了社会劳动或资本为获得某种享受所愿意支付的代价,他认为这种代价就是劳动或资本的边际产品,前者是社会劳动的反效用,表现为工资;后者是社会资本的忍欲,表现为利息(率\footnote{引文的原文中没有“率”字,但笔者认为资本的边际生产力应当与资本的单位成本而不是总成本有关。});两者都服从边际产品递减的规律。\cite[309-316]{YanZhiJieCongBianJiGeMingDaoKaiEnSiGeMing2022}这样,克拉克实际上得到了生产要素——劳动和资本的成本,进而就可以根据每个生产者生产一定商品所消耗的生产要素数量,确定这些商品的价值\cite[311]{KeLaKeCaiFuDeFenPei1983}。那么,商品的边际效用决定的价值如何同生产要素成本决定的价值统一起来呢?克拉克指出,“实际效用是一切财富所共有的要素,这个要素可以使用社会的反效用来衡量。许多种类的享受,都可以使用劳动所带来的损失来衡量。”\cite[xv]{KeLaKeCaiFuDeFenPei1983}这就从单位上把商品的效用和生产要素的成本(工资、利息率)统一起来了;再借助的市场交换,就可以在数量上把决定生产成本的负效用和来自享受消费的正效用统一起来\cite[331-333]{KeLaKeCaiFuDeFenPei1983}。

不难看出,克拉克所谓的边际生产力价值论实际上是利用边际生产力的概念把边际效用价值论、生产费用价值论\footnote{晏智杰\cite[312]{YanZhiJieCongBianJiGeMingDaoKaiEnSiGeMing2022}认为克拉克的边际生产力价值论还综合了供求价值论,但笔者认为克拉克的理论中没有出现供求价值论。}综合到了一起。

\subsection{生产力与价值的关系}

从上述边际生产力价值论的含义上看,“边际生产力”一词中的“生产力”毫无疑问指的是绝对生产力。克拉克也对这一“生产力”的变动作出了分析。一方面,克拉克指出如果某一个行业出现某种发明使得该行业生产的商品的生产成本降低,“那么,这个团体所生产的物品的标准价值,开始时一定突然下跌,然后就稳定一段时间,再后由于第二个发明的出现又重新下跌”\cite[361]{KeLaKeCaiFuDeFenPei1983}。这实际上符合广义价值论关于绝对生产力与单位商品价值负相关的判断。另一方面,“生产力”的增加会使得“工资增高,利息总额增大”\cite[241]{KeLaKeCaiFuDeFenPei1983},进而还会给企业家带来暂时的利润\cite[352]{KeLaKeCaiFuDeFenPei1983}。这实际上不仅符合广义价值论关于单位个别劳动创造的价值量与其绝对生产力正相关的判断。

然而,克拉克又指出,“一个团体内劳动和资本所特有的生产力,是由两个力量决定的。一个是产品的价格,这要看这种产品的总产量是多少。另一个是产品中由一个单位的劳动(或一个单位的资本)所生产的部分,这要看这个团体里劳动和资本在数量上的对比是怎样的。”\cite[252]{KeLaKeCaiFuDeFenPei1983}例如,如果工人是在资本非常充裕的工厂里工作,那么可以归功于一个单位劳动的产量就很大;而且由于此时产品的总量还不大,单位产品的价格就很高。所以,在这种情况下劳动的生产力就很高。现在,如果许多工人涌入这家工厂,那么由于资本的数量没有变化和边际生产力递减的规律,单个工人的产量就会下降;而且由于产品的总量上升,单位产品的价格就会下降。所以,劳动的边际生产力就会下降。这样,如果劳动和资本可以自由流动的话,各个行业的劳动和资本的边际\footnote{原文中没有“边际”一词,但笔者认为这里的意思就是边际生产力。}生产力将会趋于一致\cite[254]{KeLaKeCaiFuDeFenPei1983}。

这里笔者注意到:第一,某个部门的资本和劳动的边际生产力大小对资本和劳动流动的影响是相对于其它部门或是整个社会来说的,单独说某个部门的资本和劳动的边际生产力大小没有意义;第二,决定资本和劳动力成本(工资和利息率)的是整个社会的资本和劳动力边际生产力,所以单个部门如果有比较高的资本和劳动力生产力,就可以获得超额利润\cite[255]{KeLaKeCaiFuDeFenPei1983}。所以笔者认为,这里的“边际生产力”对应着广义价值论中的比较生产力概念,只不过广义价值论中的比较生产力比较的是某一部门在一种商品上的绝对生产力与另一部门在另一种商品上的绝对生产力,而边际生产力价值论中劳动和和资本的“边际生产力”比较的是某一部门的某一要素在一种商品上的(边际)绝对生产力与另一部门的同种要素在另一种商品上的(边际)绝对生产力。类似地,单个部门如果有比较高的资本和劳动力生产力,就可以获得超额利润的命题对应着广义价值论中部门劳动创造的价值总量与部门比较生产力正相关的判断。然而,由于这里没有量化的分析,所以边际生产力价值论中没有与广义价值论中比较生产力与单位商品价值量正相关的判断对应的命题\footnote{而且引入单位商品的价值会出现循环论证。}。

\section{新古典价值论及其生产力概念}

一般认为,马歇尔在19世纪末将边际效用价值论、边际生产力价值论、供求论和生产成本论综合起来,创建了新古典经济学\cite[295]{YanZhiJieXiFangJingJiXueShuoShiJiaoChengDiErBan2013}\cite[340]{YanZhiJieCongBianJiGeMingDaoKaiEnSiGeMing2022}\cite[i]{MaXieErJingJiXueYuanLi2019}\cite[183-184]{CaiJiMingCongGuDianZhengZhiJingJiXueDaoZhongGuoTeSeSheHuiZhuYiZhengZhiJingJiXueJiYuZhongGuoShiJiaoDeZhengZhiJingJiXueYanBianShangCe2023}。马歇尔的价值理论强调供给和需求的均衡,因此也被称为供求均衡价值论\cite[390]{YanZhiJieCongBianJiGeMingDaoKaiEnSiGeMing2022}。接下来笔者将对此进行介绍。

首先,马歇尔认为商品的价值就是交换价值,由货币表示的价格又代表着商品的一般交换价值\cite[86-87]{MaXieErJingJiXueYuanLi2019}。这里笔者想要指出:有很多学者批评马歇尔的理论中没有价值的概念\cite[299]{YanZhiJieXiFangJingJiXueShuoShiJiaoChengDiErBan2013}\cite[v]{MaXieErJingJiXueYuanLi2019},但笔者认为,马歇尔研究的对象就是价格运动的规律,按照价值的一般定义,均衡价值论也应当是一种价值理论而非“价格理论”,只是马歇尔对价值的认识可能存在错误罢了。事实上,笔者还注意到马歇尔定义了“实际价值”的概念,也即“一定数量的产品行将购买的生活必需品、安逸品和奢侈品的数量”\cite[720]{MaXieErJingJiXueYuanLi2019};同时,马歇尔又说价格是商品“与一般物品比较时的交换价值的代表”\cite[87]{MaXieErJingJiXueYuanLi2019},而且“货币的实际价值用劳动来衡量比用商品来衡量较好”\cite[87]{MaXieErJingJiXueYuanLi2019}。这里的实际价值,难道不是古典政治经济学所想要探讨的“价值”概念吗?

马歇尔认为价值取决于供求力量的均衡,两者缺一不可。在此基础上,马歇尔又指出了不同时间条件下价值决定的区别:在极短时间(如一天)内出现的暂时均衡价格取决于需求,因为供给在极短时间内无法变化;在短时间(通常是一年)内的短期均衡价格取决于供求双方对等的相互作用,因为双方在短期内均可作出调整;在长期内出现的长期市场均衡价格取决于供给,因为供给在长期内可以有很大的增长\cite[299-300]{YanZhiJieXiFangJingJiXueShuoShiJiaoChengDiErBan2013}\cite[390]{YanZhiJieCongBianJiGeMingDaoKaiEnSiGeMing2022}。同时,马歇尔所指的均衡是局部均衡,即在研究某种商品的价值决定时假定其它商品的价值不变卖不考虑其他商品供求关系的变动对该商品价值的影响\cite[391]{YanZhiJieCongBianJiGeMingDaoKaiEnSiGeMing2022}。

 在需求方面,马歇尔继承了边际效用价值论的观点,认为需求价格决定于边际效用递减律;在供给方面,马歇尔实际上单独提出了边际生产力的概念\cite[8223]{macmillanpublishersltdNewPalgraveDictionary2018},认为供给价格决定于生产费用,而生产费用又决定于劳动和资本,前者是边际负产品,后者是等待或边际负效用\cite[391]{YanZhiJieCongBianJiGeMingDaoKaiEnSiGeMing2022}。与边际效用或生产力价值论不同的是,马歇尔既不认为边际效用单独可以决定价值,也不认为由边际生产力决定要素价格单独可以决定价值,马歇尔始终强调两者的相互作用。

最后,值得一提的是,以萨缪尔森为代表的新古典综合派将李嘉图研究国际贸易时所用的机会成本纳入了供求价值论,一方面,其认为完全市场上商品的均衡价格(价值\footnote{同样,这里我们认为萨缪尔森研究的对象也是价格运动的规律,所以其笔下的均衡价格也就是价值。})等于其机会成本\cite[228]{BaoLuo*SaMouErSenJingJiXueDiShiJiuBan2012}。但是,这里的机会成本指的是市场上其它出价者的出价\cite[228-229]{BaoLuo*SaMouErSenJingJiXueDiShiJiuBan2012},所以其本质上只是供求均衡价值论的另一种说法而已。另一方面,萨缪尔森将机会成本的概念用于分析生产可能性边界\footnote{表示在技术知识和可投入品数量既定的条件下,一个经济体所能有效率地得到的最大产量\cite[22]{BaoLuo*SaMouErSenJingJiXueDiShiJiuBan2012}。},并详细地阐述了比较优势原理——“如果各国专门生产和出口其生产成本相对低地产品,就会从贸易中获益。”\cite[581]{BaoLuo*SaMouErSenJingJiXueDiShiJiuBan2012}。但是,萨缪尔森也把机会成本的概念和比较优势原理运用在了国际贸易体系内,而没能扩展到一般的商品交换过程。

 \subsection{生产力与价值的关系}

 由于新古典价值论强调均衡分析,因此我们很难单从供给入手分析生产力与价值的关系。事实上,新古典价值论本身存在循环论证的问题:“新古典价值论在讨论产品市场均衡价格时预先假定要素价格已经存在,由此才能导出由成本曲线构成的供给曲线;而在讨论要素市场时,又假定产品价格已经存在,由此才能形成由要素边际收益形成的要素需求曲线。”\cite[186]{CaiJiMingCongGuDianZhengZhiJingJiXueDaoZhongGuoTeSeSheHuiZhuYiZhengZhiJingJiXueJiYuZhongGuoShiJiaoDeZhengZhiJingJiXueYanBianShangCe2023}因此,如果我们要分析供给端的生产力对价值的影响,那么就要假定产品价格已经存在;那既然产品价格已经存在,产品的价值已经确定了,那就无法分析价值的“变化”。这个问题不光存在于新古典价值论,持供求价值论观点的马尔萨斯的论证也出现了相同的问题。

由前所述,新古典综合派引入了机会成本的概念,这就为新古典价值论引入了相对生产力的概念。但由于其研究的仍然是国际贸易,所以在相对生产力这一概念上,新古典综合派和李嘉图的理解是相同的。
%% !TeX root = ../2019080346_Mason.tex

\chapter{斯拉法价值论}

\section{引言}

斯拉法在1960年的《用商品生产商品》一书中提出了一种独具特色的价值理论。斯拉法一方面批判了新古典价值论的边际分析范式,另一方面又不同于马克思的劳动价值论,强调价值决定和剩余分配是同时进行的。接下来,笔者将试着梳理斯拉法的理论框架。

\section{斯拉法的价值理论}

\subsection{为维持生存的生产}

斯拉法首先分析了为维持生存的生产体系,即产出品仅能补偿投入品,生产规模无法扩大的简单再生产体系\cite[4]{SiLaFaYongShangPinShengChanShangPinJingJiLiLunPiPanXuLun1963}。他首先举了一个包含小麦、铁和猪这三种商品的三部门生产体系的例子\cite[5]{SiLaFaYongShangPinShengChanShangPinJingJiLiLunPiPanXuLun1963}:
\begin{equation*}
    \begin{aligned}
    240\text{夸特小麦}&+&12\text{吨铁} &+&18\text{头猪} &\rightarrow  450\text{夸特小麦} \\
    90\text{夸特小麦}&+&6\text{吨铁} &+&12\text{头猪} &\rightarrow  21\text{吨铁} \\
    120\text{夸特小麦}&+&3\text{吨铁} &+&30\text{头猪} &\rightarrow  60\text{头猪} \\
    \text{总计:}450\text{夸特小麦}&&21\text{吨铁}&&60\text{头猪}& \\
    \end{aligned}
\end{equation*}

可见,这里三个商品的产出和生产这些商品所需要的投入是相等的,这个体系刚好可以维持生存。斯拉法指出:“这里有唯一的一套交换价值,如果市场采用这些交换价值,会使产品的原来分配复原,使生产过程能够反复进行。”\cite[5]{SiLaFaYongShangPinShengChanShangPinJingJiLiLunPiPanXuLun1963}。把上面的生产体系视为一组方程,我们可以求得这套交换价值必须是\cite[5]{SiLaFaYongShangPinShengChanShangPinJingJiLiLunPiPanXuLun1963}:
\begin{equation*}
    10\text{夸特小麦}=1\text{吨铁}=2\text{只猪}
\end{equation*}

接着,斯拉法把上述思路扩展到了一般的情形\cite[5-6]{SiLaFaYongShangPinShengChanShangPinJingJiLiLunPiPanXuLun1963}\cite[189-190]{CaiJiMingCongGuDianZhengZhiJingJiXueDaoZhongGuoTeSeSheHuiZhuYiZhengZhiJingJiXueJiYuZhongGuoShiJiaoDeZhengZhiJingJiXueYanBianShangCe2023}。具体来说,假设一个生产体系中有$n$个行业,每个行业对应一个商品。每个商品都需要由这一体系中的其它商品按照一个固定技术系数线性地被生产出来。也即存在一个技术矩阵$\bm{A}$:
\begin{equation}
    \bm{A} =
    \begin{pmatrix}
    a_{11} & \cdots & a_{1n} \\
    \vdots & \ddots & \vdots \\
    a_{n1} & \cdots & a_{nn} \\
    \end{pmatrix}
\end{equation}

其中,$a_{ij}$代表第$j$行业使用第$i$行业商品的技术系数,即每单位商品$j$需要$a_{ij}$单位的商品$i$,在仅能维持生存的生产体系中,$a_{ij}$必须满足:
\begin{equation}
    \sum_{i=1}^{n} a_{ij} = 1,\qquad \sum_{j=1}^{n}a_{ij} = 1
\end{equation}

令该生产体系中的价值向量为$\bm{p} = \left( p_1, p_2, \cdots, p_n \right)^\top$,则$\bm{p}$和$\bm{A}$满足:
\begin{equation}
    \label{weichishengcun}
    \bm{A} \bm{p} = \bm{p}
\end{equation}

如果再令某一产品为计价物,即价格为$1$,那么就会有$n-1$个未知数,就可以解出价值向量$\bm{p}$来。

\subsection{具有剩余的生产}

如果这一生产体系中出现剩余,也就是\cite[190]{CaiJiMingCongGuDianZhengZhiJingJiXueDaoZhongGuoTeSeSheHuiZhuYiZhengZhiJingJiXueJiYuZhongGuoShiJiaoDeZhengZhiJingJiXueYanBianShangCe2023}:
\begin{equation}
    \sum_{i=1}^{n} a_{ij} < 1,\qquad \sum_{j=1}^{n}a_{ij} < 1
\end{equation}

那么我们便只有$n-1$个未知数而有$n$个独立方程,这一体系就会出现矛盾。斯拉法的解决方法是引入一个对所有生产部门统一的利润率$r$来分配剩余\cite[7]{SiLaFaYongShangPinShengChanShangPinJingJiLiLunPiPanXuLun1963}。具体而言,式\ref{weichishengcun}变为:
\begin{equation}
    \bm{A} \bm{p} \left( 1 + r \right)= \bm{p}
\end{equation}

可以看到,在斯拉法价值论中,商品的价值和利润率就是同时决定的,这就避免了新古典价值论中循环论证的逻辑矛盾。

斯拉法还强调在上述分析中,我们都认为工资仅由工人的必需生存用品组成\cite[11]{SiLaFaYongShangPinShengChanShangPinJingJiLiLunPiPanXuLun1963}也就是说,当我们在确定技术矩阵$\bm{A}$中的技术系数时,我们实际上已经把工人工资中的“生存工资”部分包括进去了。但是当经济中出现剩余时,工资将不再仅是一个维持简单再生产的水平,“工资可以包括一部分剩余产品”\cite[11]{SiLaFaYongShangPinShengChanShangPinJingJiLiLunPiPanXuLun1963},所以我们可以在体系中进一步加入劳动和工资。具体来说,令$\bm{l} = \left( l_1, l_2, \cdots, l_n \right)^\top$代表生产体系中的劳动投入向量,其中$l_i$代表投入到第$i$行业的劳动量;令$w$代表对所有生产部门统一的工资率,那么式\ref{weichishengcun}就变为:
\begin{equation}
    \bm{A} \bm{p} \left( 1 + r \right) + \bm{l} w = \bm{p}
\end{equation}

至此,我们不仅可以得到生产体系中各商品的价值,还可以得到生产体系中的分配关系。

\subsection{生产体系的劳动还原}

斯拉法通过将其生产体系的循环生产还原为劳动,反映了劳动在斯拉法价值论中的独特地位。这里的“还原”指的是“用一系列劳动量来代替所使用的各种生产资料,每一劳动量都有适合于它的‘时期’。”\cite[37]{SiLaFaYongShangPinShengChanShangPinJingJiLiLunPiPanXuLun1963}

对式\ref{weichishengcun}进行变换,得到:
\begin{equation}
    \bm{l}\left[ \bm{I}_n - \bm{A} \left( 1+r \right) \right]^{-1} = \bm{p}
\end{equation}

再根据\footnote{$\left[ \bm{I}_n - \bm{A} \left( 1+r \right) \right]^{-1}$是一个性质足够好的矩阵\cite[89]{pasinettiLecturesTheoryProduction1977}}$\left[ \bm{I}_n - \bm{A} \left( 1+r \right) \right]^{-1} = \bm{I}_n + \left( 1+r \right) \bm{l} \bm{A} w + \left( 1+r \right)^2 \bm{l} \bm{A}^2 w + \cdots $,可以得到:
\begin{equation}
    \bm{p} = \bm{l}w + \left( 1+r \right)\bm{l}\bm{A}w + \left( 1+r \right)^2 \bm{l}\bm{A}^2w + \cdots
\end{equation}

因此,每种商品的价值都可以被视为工资和利润从无穷远的过去进行积累的结果\cite[37-38]{SiLaFaYongShangPinShengChanShangPinJingJiLiLunPiPanXuLun1963}\cite[193]{CaiJiMingCongGuDianZhengZhiJingJiXueDaoZhongGuoTeSeSheHuiZhuYiZhengZhiJingJiXueJiYuZhongGuoShiJiaoDeZhengZhiJingJiXueYanBianShangCe2023}。

如果$r=0$,那么有:
\begin{equation}
    \bm{p} = \bm{l}w + \bm{l}\bm{A}w + \bm{l}\bm{A}^2w + \cdots
\end{equation}

这意味着整个生产体系的价值本质上都是无穷期工资的叠加,所以可以被完全还原为不同时期的劳动量的叠加。

\subsection{其他部分}

为了研究经济剩余的分配,斯拉法又进一步区分了基本和非基本商品,提出了标准商品和标准体系的概念,还研究了一个部门可以生产多个产品的联合生产体系。但是,笔者认为这些部分只是前述价值理论的推广,因此在此按下不表。

\section{生产力与价值的关系}

很遗憾,斯拉法的价值理论虽然很有新意,但斯拉法通篇没有提到生产力的概念。斯拉法注重研究分配(工资和利润率)的变化对价值的影响,而没有考虑生产力变化对价值的影响。倘若我们硬要对此进行分析,恐怕只能对矩阵$\bm{A}$的某一行的系数施加幅扰动,然后用数值的方法模拟该行对应商品价值的变化。但是这样的探究不仅很难得出一般的结论,也没有太大的意义。

然而,我们从斯拉法的价值决定方法中不难推断,一个商品的价值不仅取决于本部门在该商品上的(绝对)生产力,也取决于其它部门在其它商品上的(绝对)生产力。而广义价值论中的比较生产力有着极其相似的定义。因此,斯拉法价值论中的这种价值决定方法实际上暗示了比较生产力在价值决定中的作用。
% !TeX root = ../2019080346_Mason.tex
\chapter{总结与讨论}

\section{交换价值、价值、价值尺度和价值决定}

\subsection{劳动作为衡量交换价值的尺度}

如前所述,亚当·斯密在国富论中首次提出了“交换价值\footnote{这里应当是价“价值”而非“交换价值;这里笔者先按照几位古典政治经济学家的原文使用“交换价值”一词;交换价值和价值的关系在后文中阐述。}的真实尺度”问题,并指出劳动是这一问题的答案。而后李嘉图、马克思、马尔萨斯都认为劳动是衡量交换价值的尺度。值得注意的是,马克思认为古典政治经济学的根本倾向是劳动价值论,但这一论断忽视了古典政治经济学不同理论之间的差异性\footnote{由于学界对古典政治经济学的划分不尽相同,对古典政治经济学家思想的理解也不尽相同,所以有学者认为古典政治经济学所得出的结论是“多要素供求价值论”\cite[179]{CaiJiMingCongGuDianZhengZhiJingJiXueDaoZhongGuoTeSeSheHuiZhuYiZhengZhiJingJiXueJiYuZhongGuoShiJiaoDeZhengZhiJingJiXueYanBianShangCe2023}。}。根据笔者对古典政治经济学的认识,笔者认为更有说服力的结论是:古典政治经济学的基本倾向是把劳动作为衡量交换价值的尺度。

一方面,古典政治经济学家意识到,在社会分工建立起来后,一个生产者或许因为不能生产某一需要的商品,或许因为通过交换得到的商品更好更便宜,所以会愿意用自己的劳动生产一些不愿自己消费的商品来交换他物。然而,商品都具有作为使用价值的异质性,为了与其它商品进行交换,商品的拥有者会根据在这一商品上花费的劳动\footnote{包括活劳动和物化劳动}与他人进行交换。这是因为这个生产者为了生产者这一商品所投入的就是自己的劳动,也只能知道自己为了这一商品投入了多少劳动。而那些与他进行商品交换的生产者,也是以同样的考量与他进行交换。因此,劳动成为了商品交换时所依据的尺度。换言之,劳动成为了衡量交换价值的尺度。\cite[1016-1017]{ZhongGongZhongYangMaKeSiEnGeSiLieNingSiDaLinZhuZuoBianYiJuMaKeSiEnGeSiWenJiDi7Juan2009}\cite[25]{YaDang*SiMiGuoFuLun2015}\cite[133]{MaErSaSiZhengZhiJingJiXueDingYi2023}

另一方面,交换价值会随着时间和地点的不同上下波动,具有偶然性和相对性。因此,古典政治经济学家所能找到的不变的、普遍的交换价值的尺度,只有劳动。\cite[49-51]{ZhongGongZhongYangMaKeSiEnGeSiLieNingSiDaLinZhuZuoBianYiJuMaKeSiEnGeSiWenJiDi5Juan2009}\cite[28-30]{YaDang*SiMiGuoFuLun2015}

\subsection{从交换价值到价值}

然而,交换价值是一种使用价值同另一种使用价值的比例,其本身没有单位,也没有数量大小上的意义。所以前文中古典政治经济学家使用“交换价值的尺度”这一说法是不合理的——一个没有大小的量怎么能被“衡量”呢?事实上,古典政治经济学家所衡量的其实不是交换价值,而是价值。只是在政治经济学发展的早期阶段,经济学家们还没有能力从交换价值中抽象出价值的概念。直到马克思首次区分了交换价值和价值,价值的面纱才被揭开\cite[86-87]{ChenDaiSunCongGuDianJingJiXuePaiDaoMaKeSiRuoGanZhuYaoXueShuoFaZhanLueLun2014}。

笔者认为,政治经济学逐步区分交换价值和价值的过程可以从两个角度来理解。

首先,从价值实体的角度来看,假设某种商品可以和多种商品进行交换,那么这一商品就具有了许多种交换价值,这些交换价值必然也是可以相互替代的交换价值。所以,“第一,同一种商品的各种有效的交换价值表示一个等同的东西。第二,交换价值只能是可以与它相区别的某种内容的表现方式,‘表现形式’。”\cite[49]{ZhongGongZhongYangMaKeSiEnGeSiLieNingSiDaLinZhuZuoBianYiJuMaKeSiEnGeSiWenJiDi5Juan2009}也就是说,这些不同的交换价值都可以被化为“一种等量的共同的东西”\cite[49]{ZhongGongZhongYangMaKeSiEnGeSiLieNingSiDaLinZhuZuoBianYiJuMaKeSiEnGeSiWenJiDi5Juan2009}。而且,这种共同东西既不是使用价值——使用价值具有异质性,也不是交换价值——交换价值仅仅是一个比例,没有大小,也无法被“衡量”。进而,马克思把这种共同的东西称为价值\cite[50]{ZhongGongZhongYangMaKeSiEnGeSiLieNingSiDaLinZhuZuoBianYiJuMaKeSiEnGeSiWenJiDi5Juan2009}。值得注意的是,尽管马克思已经能从交换价值中抽象出价值来,但是笔者却不认同马克思推演的逻辑。马克思在意识到不同的交换价值是在衡量一个既不是使用价值也不是交换价值的“共同的东西”之后,就说“如果把商品体的使用价值撇开,商品体就只剩下一个属性,即劳动产品这个属性”\cite[50-51]{ZhongGongZhongYangMaKeSiEnGeSiLieNingSiDaLinZhuZuoBianYiJuMaKeSiEnGeSiWenJiDi5Juan2009},再把具体劳动的成分撇开,那么剩下的“只是无差别的人类劳动的单纯凝结”,这种凝结的抽象劳动,就是商品的价值\cite[51]{ZhongGongZhongYangMaKeSiEnGeSiLieNingSiDaLinZhuZuoBianYiJuMaKeSiEnGeSiWenJiDi5Juan2009}。这里的问题主要在于:如果认为使用价值是异质的而不能作为价值决定的因素,那么,劳动也是异质的, 同样不能作为价值决定的因素;如果认为各种具体的异质的劳动可以抽象为无差别的一般人类劳动,那么,这一抽象过程同样适用于各种异质的使用价值。或者说,当人们将具体劳动抽象为无差别的一般人类劳动的同时,事实上也就把各种使用价值抽象为一般的使用价值即效用\cite[84]{CaiJiMingLunJieZhiJueDingYuJieZhiFenPeiDeTongYi2003}。也就是说,将商品的价值和“无差别的人类劳动的单纯凝结”等同起来并不是一个不证自明的过程,马克思的论证逻辑是存在问题的。但是,笔者认为马克思将价值从交换价值中抽象出来的逻辑过程是正确的,价值确实是一种既不同于使用价值也不同于交换价值的,客观存在的实体。

正因如此,笔者进一步认为前文中不同经济学家对“交换价值的尺度”的争论实际上是对“价值尺度”的争论。交换价值本身只是两种使用价值的比例,没有“大小”的概念,因此也无法被“衡量”。而价值作为一个实体,理论上是可以被衡量的。不同经济学家所争论的,正是价值应该用什么尺度来衡量,或者说价值的单位究竟是什么。

其次,人们对价值认识的演进和商品交换的发展是两个相反的过程。在商品交换的早期阶段,商品的“交换价值首先表现为一种使用价值同另一种使用价值相交换的量的关系或比例”\cite[49]{ZhongGongZhongYangMaKeSiEnGeSiLieNingSiDaLinZhuZuoBianYiJuMaKeSiEnGeSiWenJiDi5Juan2009},也就是简单的交换价值形式\footnote{在资本论中,马克思用的是“简单价值形式”一词,但由于马克思没有绝对严谨地区分价值和交换价值\cite[37]{ZhongGongZhongYangMaKeSiEnGeSiLieNingSiDaLinZhuZuoBianYiJuMaKeSiEnGeSiWenJiDi8Juan2009},所以这里笔者采取了蔡继明教授的更严谨的提法\cite[145]{CaiJiMingJieZhiZhengLunHuiGuYuZhanWang2008}。};随着交换范围的扩大,简单的交换价值形式发展为扩大的交换价值形式;当一般等价物出现后,所有使用价值都以一般等价物为媒介而进行交换,交换价值就发展为一般的形式;当货币产生后,交换价值便取得价格这种形式。尽管市场价格受供求波动影响,但长期观察显示其始终会围绕着一个相对稳定的轴心运动——这个轴心,或者说调节价格运动的规律,被亚当·斯密称为"自然价格"、马尔萨斯谓之"自然价值",最终在政治经济学发展中凝练为“价值”概念。这样,价值作为调节价格运动的规律这一特定的内涵便确定下来。\cite[145]{CaiJiMingJieZhiZhengLunHuiGuYuZhanWang2008}\footnote{事实上马克思也把价值的这种内涵以价值作用的方式表述了出来\cite[199]{ZhongGongZhongYangMaKeSiEnGeSiLieNingSiDaLinZhuZuoBianYiJuMaKeSiEnGeSiWenJiDi7Juan2009}。}随着人们进一步认识到价格只是交换价值的一种形式,价值的内涵也进一步一般化为调节交换价值的规律。

正如马克思所言:“对人类生活形式的思索,从而对这些形式的科学分析,总是采取同实际发展相反的道路。这种思索是从事后开始的,就是说,是从发展过程的完成的结果开始的。$\cdots$因此,只有商品价格的分析才导致价值量的决定,只有商品共同的货币表现才导致商品的价值性质的确定。”\cite[93]{ZhongGongZhongYangMaKeSiEnGeSiLieNingSiDaLinZhuZuoBianYiJuMaKeSiEnGeSiWenJiDi5Juan2009}在现实层面,价值研究确实肇始于对价格现象的经验观察;在理论层面,支配交换价值的规律作为价值的内在本质,也符合人类认知从现象到本质的渐进过程,体现了历史发展的内在连贯性。

\subsection{价值尺度和价值决定的对立统一}

前文的论述表明,把劳动作为价值的尺度是古典政治经济学的基本倾向。但正如笔者在亚当·斯密相关章节中探讨的学术争议所示,古典政治经济学家们在该用商品生产所耗费的劳动还是用商品所能支配的劳动来作为价值尺度的问题上产生了严重的分歧。总的来说,劳动价值论派的李嘉图、马克思认为应当用耗费的劳动作为价值尺度,而多要素价值论派的斯密、马尔萨斯则认为应当用支配的劳动作为价值尺度。事实上,作为新古典价值论代表人物的马歇尔也有用支配的劳动作为价值尺度的倾向\cite{perskyMarshallsNeoClassicalLaborValues1999}。

\subsubsection{商品所能支配的劳动作为商品价值的尺度}
在笔者看来,用商品所能支配的劳动作为价值尺度无疑是更合适的。

首先,正如马克思所说:“价值的对象性只能在商品同商品的社会关系中表现出来”\cite[61]{ZhongGongZhongYangMaKeSiEnGeSiLieNingSiDaLinZhuZuoBianYiJuMaKeSiEnGeSiWenJiDi5Juan2009},价值本身具有社会属性。

恩格斯曾在《<资本论>第三册增补》中举了一个例子来论证应当用商品生产所耗费的劳动作为商品价值的尺度\cite[1015-1018]{ZhongGongZhongYangMaKeSiEnGeSiLieNingSiDaLinZhuZuoBianYiJuMaKeSiEnGeSiWenJiDi7Juan2009}。恩格斯说,在商品经济发展的初期,进行商品交换的主要是劳动的农民。这些农民借助自己家庭的帮助,在自己的田地上进行农业、畜牧业和手工业的生产,并拿除必需品之外剩下的剩余产品同其它农民家庭进行交换。这些农民之所以进行交换,并非因为自己不会生产这些物品\footnote{这一对可变分工体系的认识与广义价值论的假设有异曲同工之妙,可惜恩格斯并没有把可变分工体系推广到一般的商品经济。},而是因为得不到原料或者因为买到的物品要更好或更便宜。他们在生产那些用于交换的产品时所耗费的,只有他们自己的劳动;无论是为了补偿工具、为生产和加工原料所花费的,只有他们自己的劳动力。于是,恩格斯得出用耗费的劳动作为价值尺度结论:“因此,如果不按照花费在他们这些产品上的劳动的比例,他们又能怎样用这些产品同其他从事劳动的生产者的产品进行交换呢?在这里,花在这些产品上的劳动时间不仅对于互相交换的产品量的数量规定来说是唯一合适的尺度;在这里,也根本不可能有别的尺度。”\cite[1016]{ZhongGongZhongYangMaKeSiEnGeSiLieNingSiDaLinZhuZuoBianYiJuMaKeSiEnGeSiWenJiDi7Juan2009}但如果我们进一步深入思考,会发现恩格斯的这一结论是不能成立的。

正如恩格斯所言,农民之所以选择交换,不是因为自己不能生产,而是因为得不到原料或者因为买到的物品要好得多或便宜得多。当农民在用劳动作为尺度衡量是否要进行交换的时候,他衡量的是“如果不进行交换,为了得到同样的产品我要多耗费多少劳动?”以及“为了得到同样的产品,如果进行交换我能省下多少劳动?”于是,这些“多耗费的劳动”或者“省下的劳动”就是这个农民眼中,通过交换得到的商品的价值。这里笔者想提请读者注意,农民是通过自己的劳动衡量了别人生产的商品的价值。换句话说,假设有农民A、B,分别花费劳动$L_A$和$L_B$生产商品$C_A$和$C_B$,现在农民A想要通过交换得到商品$C_B$,于是农民A会衡量“多耗费的劳动”或者“省下的劳动”,记为$\Delta L_A$。此时,$\Delta L_A$所衡量的,正是农民B生产的商品$C_B$的价值。同样,农民B也是以同样的逻辑用$\Delta L_B$衡量了商品$C_A$的价值\footnote{当然,这里所指的各种劳动量都可以按照马克思所述的“抽象劳动”概念来理解。}。至此,笔者的分析应当说是忠于恩格斯的分析逻辑的。接下来,根据马克思劳动价值论的等价交换原则,要让上述交换能够长久地持续,农民A必须拿出在农民B看来有足够价值的商品与农民B进行交换;所以要让交换持续,$\Delta L_A = \Delta L_B$必须成立。再根据马克思劳动价值论所认为的“等量劳动带来等量价值”的价值决定原理,农民A必须付出与$\Delta L_B$相等的劳动$L_A$才能创造出$\Delta L_B$的价值,所以我们得到$\Delta L_B = L_A$。也就是说,商品$C_B$实际上能够支配$L_A$量的劳动;进而我们可以说,商品$C_B$的价值是由能够支配$L_A$量的劳动来衡量的;这等价于说商品所能支配的劳动是商品价值的尺度。同样地,我们可以得到$\Delta L_A = L_B$,所以最终有$L_B =\Delta L_A = \Delta L_B = L_A$。以上的分析表明,按照恩格斯分析的逻辑,在简单交换中商品所能支配的劳动是商品价值的尺度;并且,用商品能够支配的劳动作为价值的尺度与马克思的劳动价值论并没有发生任何结论上的冲突。

在成熟的商品经济中,商品能够支配的劳动仍然是价值的尺度。马克思在《资本论》第一卷的第三章中指出,货币的作用之一是价值尺度\cite[114-124]{ZhongGongZhongYangMaKeSiEnGeSiLieNingSiDaLinZhuZuoBianYiJuMaKeSiEnGeSiWenJiDi5Juan2009}。而一种贵金属,例如金,为什么可以成为货币?马克思指出:“金能够充当价值尺度,只是因为它本身是劳动产品,因而使潜在可变的价值。”\cite[118]{ZhongGongZhongYangMaKeSiEnGeSiLieNingSiDaLinZhuZuoBianYiJuMaKeSiEnGeSiWenJiDi5Juan2009}那么,如果我们用金来衡量某一件商品的价值,实际上是把由商品生产者耗费的劳动与金的生产者所耗费的抽象劳动进行了比较与折算。于是这件商品的价值不仅是被金衡量出来了,而且是本质上被金的生产者所耗费的抽象劳动衡量出来了。可见,在成熟的商品经济中,把商品能够支配的劳动作为商品价值的尺度也是符合马克思分析的逻辑和结论的。

最后,笔者想要强调的是,如何消除劳动的异质性在不同的价值理论中仍然是不一样的。例如马克思认为劳动本身具有二重性,商品的劳动本身具有抽象劳动的属性,斯密等经济学家则认为市场上的讨价还价消除了劳动的异质性。目前,经济学界对此还没有一个统一的认识,所以笔者认为在这一点上仍然存在研究的空间\footnote{消除劳动的异质性问题的本质,是如何理解“等价交换”\cite[69]{CaiJiMingLunHaoFeiDeLaoDongYuGouMaiDeLaoDongZaiJieZhiLiLunZhongDeZuoYong2022}。}。不过,无论是哪种消除方式,都是强调了价值的社会属性。从社会属性的角度出发,不难看出用支配的劳动比耗费的劳动作为价值尺度更能表现价值的社会属性。

\subsubsection{价值尺度和价值决定以劳动为基础的对立统一}



%% !TeX root = ../2019080346_Mason.tex
\chapter{新质生产力}

\section{引言}

在完成了对广义价值论以及经济思想史上比较重要的价值理论的梳理后,笔者将在这一章以此为基础,试着对新质生产力进行一些探讨。

\section{新质生产力的价值基础}

\subsection{传统的劳动价值论难以支撑新质生产力的发展}
% \input{data/test.tex}
% 其他部分
\backmatter

% 插图和附表清单
% 本科生的插图索引和表格索引需要移至正文之后、参考文献前
% \listoffiguresandtables  % 插图和附表清单(仅限研究生)
% \listoffigures           % 插图清单
% \listoftables            % 附表清单

% 参考文献
% \bibliography{ref/refs}    % 参考文献使用 BibTeX 编译
\printbibliography       % 参考文献使用 BibLaTeX 编译

% 附录
% 本科生需要将附录放到声明之后,个人简历之前
% \appendix
% % !TeX root = ../thuthesis-example.tex

\begin{survey}
\label{cha:survey}

\title{Title of the Survey}
\maketitle


\tableofcontents


本科生的外文资料调研阅读报告。


\section{Figures and Tables}

\subsection{Figures}

An example figure in appendix (Figure~\ref{fig:appendix-survey-figure}).

\begin{figure}
  \centering
  \includegraphics[width=0.6\linewidth]{example-image-a.pdf}
  \caption{Example figure in appendix}
  \label{fig:appendix-survey-figure}
\end{figure}


\subsection{Tables}

An example table in appendix (Table~\ref{tab:appendix-survey-table}).

\begin{table}
  \centering
  \caption{Example table in appendix}
  \begin{tabular}{ll}
    \toprule
    File name       & Description                                         \\
    \midrule
    thuthesis.dtx   & The source file including documentation and comments \\
    thuthesis.cls   & The template file                                   \\
    thuthesis-*.bst & BibTeX styles                                       \\
    thuthesis-*.bbx & BibLaTeX styles for bibliographies                  \\
    thuthesis-*.cbx & BibLaTeX styles for citations                       \\
    \bottomrule
  \end{tabular}
  \label{tab:appendix-survey-table}
\end{table}


\section{Equations}

An example equation in appendix (Equation~\eqref{eq:appendix-survey-equation}).
\begin{equation}
  \frac{1}{2 \uppi \symup{i}} \int_\gamma f = \sum_{k=1}^m n(\gamma; a_k) \mathscr{R}(f; a_k)
  \label{eq:appendix-survey-equation}
\end{equation}


\section{Citations}

Example\cite{dupont1974bone} citations\cite{merkt1995rotational} in appendix
\cite{dupont1974bone,merkt1995rotational}.


% 默认使用正文的参考文献样式;
% 如果使用 BibTeX,可以切换为其他兼容 natbib 的 BibTeX 样式。
\bibliographystyle{unsrtnat}
% \bibliographystyle{IEEEtranN}

% 默认使用正文的参考文献 .bib 数据库;
% 如果使用 BibTeX,可以改为指定数据库,如 \bibliography{ref/refs}。
\printbibliography

\end{survey}
       % 本科生:外文资料的调研阅读报告
% % !TeX root = ../2019080346_Mason.tex
\appendix

\begin{translation}
\label{cha:translation}

\title{马歇尔的新古典劳动价值论}
\maketitle

\tableofcontents

\section{引言}

随着边际效用理论的兴起,19世纪末的经济学家们拥有了一套解释交换价值的理论,却未能建立一套真正的价值理论。例如,威廉斯坦利杰文斯(William Stanley Jevons)曾主张将“价值”概念从经济学中剔除,仅保留对交换比率的研究(Jevons 1879)。时至今日,许多一般均衡经济学家仍持有类似观点。然而,无论是当时还是现在,经济学家在分析经济随时间的变化或跨国比较时,始终需要一种更广义的“价值”概念。他们希望回答诸如“商品的实际价值如何随时代变迁或地域差异而变动”这类根本问题。

边际主义者们沿袭了数百年的经济思想传统(如洛克1696/1991),深知货币并非衡量价值的可靠尺度——与其他商品一样,货币的购买力会因时空不同而波动。这一认知引发了对“货币一般交换价值”度量标准的探索。经济学家们借鉴古典文献中关于“标准商品计价”(tabular standard)的讨论,提出了多种通过加权平均商品价格构建价格指数的方案。令人意外的是,这一领域的先驱竟是杰文斯本人(Jevons 1893)\footnote{关于探索货币一般交换价值的最全面的论述可见于Walsh(1901)。另见Persky(1998)。}。

阿尔弗雷德马歇尔(Alfred Marshall)深度参与了这场关于价值度量的讨论。他不仅贡献了链式指数(chain indexes)的构建方法(Marshall 1887/1925),还提出:在某些场景下,以基本工资率为基准定义的“劳动价值”(labor-values)比价格指数调整后的“实际价值”(real values)更具优势。遗憾的是,马歇尔对此的论述零散而简略,远未形成完整的理论体系。

本文旨在梳理马歇尔关于劳动价值的论述及其实际应用,并尝试重构其新古典主义框架下的理论逻辑。

\section{马歇尔的价值概念}

“价值”并非马歇尔理论的核心概念。正因如此,人们很容易忽略一个事实:马歇尔在《经济学原理》中至少提出了四种不同的价值定义。

\subsection{交换价值}

在《原理》第二章,马歇尔先以亚当斯密的名言开篇:“‘价值’一词有两种不同含义,有时指特定物品的效用,有时指占有该物品所能换取的购买力”(斯密1776/1937,第28页;马歇尔1920,第61页)。但他随即补充道:“经验表明,将‘价值’用于前一种含义并不明智。”紧接着,他给出了交换价值的经典定义——某物在特定时空下的交换价值即“该时空下用此物可换得的他物数量”(马歇尔1920,第61页)。

\subsection{价格作为一般购买力的衡量}

马歇尔指出,商品与货币的交换价值即价格。他承认货币购买力可能波动,但建议“本书将忽略货币一般购买力的潜在变化”,从而将价格视为衡量商品“相对于一般商品的交换价值”的尺度(马歇尔1920,第62页)。

\subsection{实际价值}

在第六卷中,马歇尔将“实际价值”定义为“一定量产品所能换取的生活必需品、舒适品和奢侈品的数量”(马歇尔1920,第632页)。此时他显然主张通过价格指数对货币购买力进行平减处理。他提到,对于某些用途,用生产者商品构建指数可能更优,但未具体说明适用场景。

\subsection{劳动价值}

在讨论价值之初,马歇尔便指出:“对于某些目的,用劳动而非商品来衡量货币的实际价值更为恰当”(马歇尔1920,第62页)。在第六卷第十二章“经济进步的一般影响”中,他进一步分析了多种基础商品的劳动价值(以劳动时间衡量)的历史波动。

\subsection{如何理解这四重定义?}

与多数新古典学者一样,马歇尔无意构建一般均衡体系下的交换价值矩阵。为开展局部均衡分析,他选择以货币价格定义价值,但承认当货币价值变动时需引入调整机制。问题在于:何种基准商品(numéraire)能稳定衡量价值?

马歇尔认为,短期内可用消费品或生产资料篮子衡量货币购买力,但转向历史分析时,他选择以支配的劳动(labor commanded)定义的劳动价值为尺度。这一跳跃令人费解——它需要比马歇尔本人提供的更系统的理论支撑。

\section{替代性理论依据}

马歇尔对“支配的劳动”(labor commanded)价值尺度的支持初看令人困惑:为何一位新古典边际主义者要复兴这一早期古典思想的遗存?人们很容易想象,马歇尔本应追随杰文斯的建议,直接拒绝寻找固定价值尺度的可能性。究竟何种理由能解释他对劳动价值尺度的包容?

以下三种可能性值得探讨:

马歇尔对支配的劳动价值尺度的支持初看起来似乎令人费解。为何一个新古典边际主义者要复兴这种早期古典思想的遗物?人们很容易想象马歇尔会遵循杰文斯的建议,直接拒绝寻找固定价值尺度的可能性。对于马歇尔容忍劳动支配标准的做法,可以提出哪些解释?

这里存在三种可能性。首先,马歇尔在此处(正如他在其他场合)试图强调古典与新古典思想的连续性。他仅仅是将劳动价值作为旧体系的残余保留下来。第二,在历史研究中,可能难以获得全面的价格与数量数据来构建完整的价格指数。因此,若必须选择单一的价值尺度,最直接的答案就是选取容易获得的核心价格。马歇尔可能认为基础工资率同时满足这两个条件。第三,马歇尔有严肃的新古典主义——甚至边际主义——理由来选择劳动作为价值尺度。

\subsection{古典思想遗产}

或许可以理解,秉持“连续性原则”和马歇尔的著名格言‘自然无飞跃’(Natura non facit saltum)的马歇尔,会对其思想前辈表现出高度尊重。正如约瑟夫熊彼特所观察到的,马歇尔的理论结构“不必要地堆砌着李嘉图式遗产,这些遗产获得的重视程度与其操作重要性完全不成比例”(Schumpeter 1954, p. 837)。熊彼特指出,因此“少数英国作家和大多数非英国作家”开始将马歇尔的著作视为古典经济学与边际效用学派的综合——或许是一种略显牵强的综合。熊彼特强烈反对这一观点,并令人信服地论证马歇尔本质上完全是新古典主义者。熊彼特是正确的,但当然劳动价值可能是那些遗产之一。尽管如此,若果真如此,奇怪的是马歇尔从未将其对劳动价值虽属随意但仍存在的讨论与任何古典经济学家或传统联系起来。

即使马歇尔愿意,他也很难援引大卫李嘉图作为支持。毕竟,李嘉图断然拒绝劳动支配价值,转而支持劳动凝结价值。同样,约翰斯图亚特穆勒从理论和实证角度都认为劳动支配标准存在缺陷。劳动支配价值远未被古典作家普遍接受\footnote{关于古典经济学对劳动投入论与劳动支配论价值理论的早期论述,可参见威廉罗雪尔(Wilhelm Roscher, 1882)与阿尔伯特惠特克(Albert Whitaker, 1904/1968)的著作。}。在著名古典经济学家中,只有斯密和后来的托马斯马尔萨斯认真论证过劳动支配作为价值尺度的主张。

尽管斯密从未完全解决其价值理论,但他始终认为劳动支配是合适的价值尺度。他秉持这一信念的深层原因是:在所有交易商品中,唯有劳动与劳动者明确且(可推测)恒定的牺牲相关联:

“等量劳动,在任何时间和地点,对劳动者而言都可说是等值的。在劳动者通常的健康、体力和精神状态下,在通常的技能和熟练程度中,他必然总是牺牲相同程度的安逸、自由和幸福。无论他换取的商品数量多少,他支付的价格必然总是相同的……因此,唯有劳动——其自身价值永不变化——才是能够在任何时空下对所有商品价值进行估算和比较的终极真实标准。”(Smith 1776/1937, p.33)

斯密的论点显然基于对普通劳动者实际经验的主观看法。就此而言,它可能吸引马歇尔这样的新古典主义者,但几乎无法获得古典经济学家的普遍认同。

甚至马尔萨斯最初也摒弃了斯密的劳动支配价值。在其《政治经济学原理》第一版中,他提出了一个将农业工资与谷物价格平均的权宜价值尺度。只有在与李嘉图进行大量辩论后,马尔萨斯才宣布其新信念——斯密的简单劳动支配方法实际上是正确的。其《原理》第二版的修订主要就是出于这种观念转变\footnote{见“Advertisement to the Second Edition” (Malthus 1836/1989, pp. 9-12)。}。马尔萨斯对此的论证最为有力。在评论英美工人工资差异时,他总结道:

“(美国劳动者)并非为他所得支付更多,而是因他所付获得更多;除非我们想以产品数量作为价值尺度(这将导致最荒谬且无法解决的困难),否则我们必须用劳动者付出的劳动来衡量他在美国所获物品的价值。”(Malthus 1836/1989, Vol. II, p.104)

马尔萨斯的立场与斯密一样,指向了主观(负)效用价值理论。几乎所有其他古典经济学家都否认劳动者主观经验的核心地位。例如,李嘉图将工资视为系统中的另一种价格。在讨论了贵金属和谷物价值的常见波动后,李嘉图继续写道:

“劳动的价值难道不是同样易变吗?它不仅像所有其他事物一样受供求比例影响(这种比例随社会状况的每次变化而不断改变),还受食物和其他必需品的价格波动影响——劳动者的工资正是花费在这些物品上。”(Ricardo 1821/1951, Vol I, p. 15)

对李嘉图而言,只要工资相对于其他商品上涨或下跌,劳动支配就无法成为恒定标准。

这自然将他推回自己的劳动凝结理论。当马尔萨斯指责他用成本标准替代价值尺度时,李嘉图接受了这个指控:

“我认为商品的真实价值与其生产成本是同一回事,两种商品的相对生产成本大致与其各自从始至终投入的劳动量成比例。”(Ricardo 1951, Vol. II, p. 35)

李嘉图以供给成本为导向的价值理论,几乎不可能成为马歇尔劳动支配价值的基础。

同样,当J.S.穆勒构建其价值思想时,直接拒绝了斯密和马尔萨斯的观点。他呼应李嘉图的观点:

“如果美国劳动者一日劳动能购买的普通消费品是英国的两倍,那么坚持认为两国劳动价值相同、其他物品价值不同,不过是徒劳的诡辩。在这种情况下,可以正确地说——无论对市场还是劳动者自身而言——美国劳动的价值是英国的两倍。”(Mill 1929, p. 567)

对穆勒而言(我们还可以将约翰麦克库洛赫和西尼尔加入此列\footnote{在重要晚期古典经济学家中,托马斯图克(Thomas Tooke)是这一共识的例外。在其鸿篇巨制《价格史》中,他断言:“……如今可以——也理应——视作公认结论(尽管仍存争议)的是:相较于谷物,普通日常劳动的货币价格是衡量贵金属价值的更优标准”(Tooke 1838/1928, p. 56)。这一论点在图克的核心主张中具有关键作用,即尽管谷物价格跌幅更大,但白银的实际价值在十八世纪大部分时期持续下降。}),劳动的主观负效用与劳动交换价值毫无关系。
将马歇尔对劳动支配价值的使用简单归类为“古典遗产”是困难的。古典经济学家在恰当价值尺度问题上存在根本分歧。相反,我们更有理由推断马歇尔对斯密价值理论中的主观主义要素抱有真正共鸣。在此解读下,马歇尔对劳动价值的使用绝非对古典图腾的盲目接受,而是对相互冲突的古典主题的有意识选择\footnote{马歇尔关于“真实成本体现为代价的主观负效用”的观点,早先已被迈因特(Myint 1948, p. 133)与斯密相联系,并与李嘉图形成对照。感谢一位审稿人提供此文献来源。}。

\subsection{简单计价物(Simple Numeraire)}

选择工资作为计价标准的理由可能源于工资数据的相对可得性。然而自亚当斯密以来,经济学家就意识到界定恒定特征与质量的劳动极其困难。实际上,斯密明确指出工资数据难以标准化。在论证谷物价格与基础工资高度相关后,他得出结论:实践中最佳选择是以谷物价格为计价标准(Smith 1776/1937, p. 38)。

马歇尔从未提出基于可得性的工资率论证。相反,他指出历史工资的错误估算极易导致误导性劳动价值。他特别批评某研究者将“人口中较优越阶层的工资视为整体代表”(Marshall 1920, p. 675)。若马歇尔唯一的考虑是数据可得性,最自然的选择应是选取基础商品(如谷物)价格,用以代表整体消费组合。这种做法确有充分先例。

\subsection{新古典方案}
新古典主义方案试图将边际分析运用于“真实世界”\footnote{关于边际效用学派与新古典主义者之间的区别,参见Myint(1948)的论述。}的实用经济学,而非形式主义的机械套用。在马歇尔对新古典方案的整体承诺下,我们可以运用何种理论论证来合理化其对劳动支配价值的使用?

首先,马歇尔明确认为对一般价值衡量标准的普遍需求具有合理性。这种工具对历史研究与国际比较不可或缺。作为新古典主义者,马歇尔不能像某些边际主义者那样耸肩回避对这种尺度工具的实际需求。

如前所述,马歇尔对斯密与马尔萨斯理论中隐含的主观主义立场抱以同情。尽管包含多重主题,马歇尔的论述始终围绕新古典作家共有的主观边际效用理论展开。对他而言,劳动负效用是明确定义的核心概念。

新古典主义者马歇尔也未必会反对斯密与马尔萨斯进行的跨期与跨人际效用比较。当然,他承认此处存在严重问题:并非所有个体都相同,有人可能更健康或更偏好工作等。但马歇尔并不排斥效用平均化,他认为将劳动负效用曲线视为基本给定并无内在困难\footnote{参见马歇尔(Marshall 1920, pp. 18-20)对个体差异与群体福利的讨论。马歇尔得出结论认为,对于足够大规模的群体,其内部差异将会相互抵消。该方法与其对代表性个体或企业的运用完全一致。}。

但我们仍要追问:为何马歇尔关注劳动负效用而非直接关注商品效用?他明确做出这种选择体现在对劳动价值的使用。但原因何在?为何以努力而非获得的享受作为更恰当的衡量标准?答案可能在于马歇尔深信物质消费面临显著的边际效用递减,因此货币的边际效用随收入递减。马歇尔明确指出:“一先令对富人产生的愉悦或满足感小于对穷人”(Marshall 1920,p. 19)。这意味着“获得的享受量”不能简单地以消费品数量衡量。因为若社会收入分配改变,商品价值必然随之变化。

随着时间推移,技术进步带来劳动生产率提升。因此我们可以预期大多数劳动者将享受越来越多的消费。马歇尔坚信这种普遍进步。但如果这些劳动者对后续消费增量的边际评价递减,那么固定商品组合的效用价值必然下降。因此,这样的商品篮子几乎不可能作为长期不变的价值尺度。

当然,马歇尔不会否认技术进步允许产出数量增长。为此,使用商品篮子的指数可能是最有用的工具。这种指数与马歇尔的“真实价值”概念一致。但此类指数无法捕捉人口体验的价值变化,因为其消费价值尺度在消费能力提升时也在改变。

在此背景下,新古典主义者马歇尔很可能被劳动价值吸引。遵循斯密的基本论证,马歇尔指出,劳动负效用作为主观价值标准,具有跨越时空的恒定性。。若想了解某商品价值如何变化,最佳方法就是将其价格与劳动价格比较。

这种解读为马歇尔本人零散且令人困惑的劳动价值论述提供了最清晰的合理化。特别值得注意的是,包含前文引述的关键文本(第二节第258页)的完整句子写道:“但如果发明极大地增强了人类对自然的掌控力,那么对某些目的而言,以劳动而非商品衡量货币的真实价值更为恰当”(Marshall 1920, p. 62)。马歇尔明确指出,劳动价值在生产率发生显著变化时最有用——实际上可能仅在此类情况下有用\footnote{应当指出,马歇尔将这一基本观点延伸至其货币政策主张中。根据彼得格罗尼维根的研究,马歇尔曾认真考虑在《货币、信用与商业》一书中加入如下论述:​“统一购买力标准应基于所需劳动量而非所获效用量:因此货币供给应如此调控,以使标准质量单位劳动的平均报酬作为一般购买力的计量单位”(格罗尼维根(Groenewegen),1995,721)。 

正如本文一位审稿人所指出的,这一主张很可能源于马歇尔(Marshall)试图制定应对工资黏性的货币政策。马歇尔认为,长期通缩有助于实际工资增长,从而提升劳动者福利。在马歇尔看来,长期通缩的分配效应比长期通胀更为有利(Laidler 1990, pp. 62-63)。

这种将“工资黏性/宏观经济学”视角解读马歇尔以工资作为基本购买力单位的做法,与正文自然推导出的“不变负效用”解读形成互补。无论我们还是马歇尔都无需在两者间做取舍。但必须强调的是,马歇尔对劳动价值计量法的承诺绝非非对称性的——即不仅限于生产率增长情景,同样适用于生产率衰退情形。参见正文下一段落。}。 

若劳动价值在生产率提升时期有效,则在生产率下降时期也应适用。对新古典主义者马歇尔而言,这种下降最可能发生在国家经历“人口对生存资料压力持续增加”之时。此类马尔萨斯式状况必然导致“人民贫困化”与“初级产品劳动价值的上升”(Marshall 1920, p. 632)。

\section{马歇尔论证的形式化}

若上述解读准确反映马歇尔思想,他最终将其价值尺度与相对标准脱钩,转而建立在假定恒定的劳动边际负效用这一绝对标准之上。

假设存在具有明确效用函数(跨时空恒定)的代表性工人\footnote{从马歇尔(Marshall)的著作中无法明确他主张采用何种工资率作为计价基准。他仅笼统提及“特定种类的劳动”。在实证分析中,他始终未具体说明所使用的工资标准。推测其思路与亚当斯密(Adam Smith)相似,应指主要从事体力劳动的低技能工人。​这种选择在低技能劳动者占劳动力主体时具有合理性。然而,当半技术工人和技术工人在劳动力中的比例持续扩大时(如当前美国经济扩张期所示),低技能工资率的相关性将面临挑战——此时“平均劳动者”的边际负效用已不等同于低技能工人的边际负效用。尤其当低技能工资增速显著落后于技术工人工资增长时,这一观察结论更具说服力。}。马歇尔试图论证劳动边际负效用的恒定性。但在何种情况下这种主张成立?正如马歇尔本人强调,特定类型劳动的边际负效用随工时长度变化。比较不同国家或时期时,工时通常并不相同。实际上,我们不禁要问:马歇尔在关于劳动强度恒定性的各种暗示中,为何回避了自己的结论——“劳动边际负效用通常随劳动量增加而增强”(Marshall 1920, p. 141)?若边际负效用随日/周工时变化,那么即使个体基本相似,不同历史时期的边际小时价值也会变化。

马歇尔对劳动供给的分析(1920年,数学附录注释X,p. 843)假设总效用实质上是消费产生的效用$U$与劳动负效用$V$之差。则最大化$U-V$的一阶条件为:
\begin{equation}
    \label{eq1}
    \frac{\dif v}{\dif l} = \left( \frac{\dif u}{\dif m} \right) * w
\end{equation}

其中$l$是劳动时间,$m$是名义收入,$w$是名义工资率。若马歇尔寻求与消费中商品边际效用成比例的价值衡量标准,则在其效用最大化描述中,只需找到与下式成比例的尺度:
\begin{equation}
    \label{eq2}
    \left( \frac{\dif u}{\dif m} \right) * p_i
\end{equation}

其中$p_i$是商品$i$的名义价格。若构建支配的劳动价值尺度,则取:
\begin{equation}
    \label{eq3}
    \frac{p_i}{w} = \left( \frac{\dif u}{\dif m} \right) * \frac{p_i}{\frac{\dif v}{\dif l}}
\end{equation}

若跨期(或跨空间)劳动者工时保持不变,则$\sfrac{\dif v}{\dif l}$为常数,劳动边际负效用不变化,此时劳动支配标准将与式\ref{eq2}成比例。

但如前所述,该结论仅在个体劳动供给曲线对工资变化无弹性时成立。若工时随工资变化,劳动支配标准的增速将快于或慢于应有水平。在深入分析这种偏差前,首先考虑传统价格指数框架下的相同问题。

我们的目标仍是估计式\ref{eq2}中的价值。此时的估计值为:
\begin{equation}
    \label{eq4}
    \frac{p_i}{P}
\end{equation}

其中$P$是价格指数。马歇尔称此类比值为真实价值。只要$\sfrac{\dif u}{\dif m}$与$\sfrac{1}{P}$成比例——即仅价格变化影响名义收入的边际效用——该式就成立。但马歇尔的基本观点是:无论价格如何变化,当劳动者获得更多购买力并消费更多商品时,收入边际效用下降将影响商品价值。令经价格调整的实际购买力$\sfrac{m}{P}$为$y$,则:
\begin{equation}
    \label{eq5}
    \frac{\dif u}{\dif m} = \left( \frac{\dif u}{\dif y} \right) * \left( \frac{\dif y}{\dif m} \right) = \frac{\frac{\dif u}{\dif y}}{P}
\end{equation}

其中$\left(\sfrac{\dif u}{\dif y}\right)$是经价格调整购买力的边际效用。因此:
\begin{equation}
    \label{eq6}
    \frac{p_i}{P} = \left( \frac{\dif u}{\dif m} \right) * \frac{p_i}{\frac{\dif u}{\dif y}}
\end{equation}

式\ref{eq6}右侧与式\ref{eq3}右侧相似,但$\sfrac{\dif v}{\dif l}$被替换为$\sfrac{\dif u}{\dif y}$。

秉承马歇尔的新古典方法,价尺度问题可归结为在式\ref{eq6}与式\ref{eq3}之间的选择——即在他的劳动价值与真实价值之间。由于$\sfrac{\dif v}{\dif l}$与$\sfrac{\dif u}{\dif y}$都可能跨时空变化,两种估计都不完美。但我们可以追问(正如马歇尔本人隐含的:$\sfrac{\dif v}{\dif l}$与$\sfrac{\dif u}{\dif y}$何者更易随时间空间变化?

将式\ref{eq5}代入式\ref{eq1}得:
\begin{equation}
    \label{eq7}
    \frac{w}{P} = \frac{\frac{\dif v}{\dif l}}{\frac{\dif u}{\dif y}}
\end{equation}

考虑马歇尔认为最适用劳动价值的情形:技术显著改进、生产率提升。此时预期$\sfrac{w}{P}$上升且$\sfrac{\dif u}{\dif y}$下降,但$\sfrac{\dif v}{\dif l}$可能依劳动供给弹性向任意方向移动。历史上,工资上升通常伴随工时下降。假设工时减少导致$\sfrac{\dif v}{\dif l}$下降。由于$\sfrac{w}{P}$上升,可知$\frac{\dif v}{\dif l}$的降幅必然小于$\sfrac{\dif u}{\dif y}$的降幅。此时我们可以较有把握地断言,马歇尔劳动价值的偏差小于其真实价值。

上述论证本身颇具说服力。我们也希望将其视为对马歇尔立场的合理化或形式化。但必须承认,这种主张可能略显牵强。马歇尔明确指出:对特定类型的固定劳动力,工资上升将带来更多工时。在若干次要限制条件下,他总结道:“大体而言,劳动者群体的努力程度将随报酬增减而升降”(Marshall 1920, p. 142)。他迫切希望建立劳动供给的正常向上倾斜曲线,这正是达成该目标所需的假设。但这意味着实际工资上升将导致$\sfrac{\dif v}{\dif l}$增加(尽管增速必然放缓)。此时我们无法确知$\sfrac{\dif u}{\dif y}$的降幅是否绝对大于$\sfrac{\dif v}{\dif l}$的增幅。

尽管如此,马歇尔注意到工时的长期下降趋势。在讨论“经济进步的普遍影响”时,他明确指出:

“除最高等级外,所有工种的劳动者较之以往更重视休闲,更不耐过度劳累导致的疲劳;总体而言,他们可能较过去更不愿为获取眼前奢侈品而忍受超长工时带来的持续增加的‘负效用’。”(Marshall 1920, pp. 680-81)

在重构马歇尔关于工时对工资敏感性的观点时,或许最佳结论是:他认为短期工资上升会增加工时,但存在暗示表明长期中他看到工时与工资率的反向关系。而后者正是马歇尔认为最适用劳动价值的情形——即足以产生重大技术进步的长时期。

\section{结论}

把支配劳动作为价值尺度在本质上比把劳动耗费作为价值尺度更契合新古典经济学。劳动支配方法在强调价值尺度的主观性时,并未对商品价值的源泉提供解释,也未曾言及劳动与资本间收入分配的正当性。它仅提供对结果的测量,而非价值判断。相比之下,劳动投入理论似乎不可避免地暗示对收入分配的批判。或许这正是马歇尔能轻易采用劳动支配价值作为尺度,同时援引新古典供需框架进行解释的原因。

剑桥学派对劳动支配价值的兴趣在后续发展中续写篇章。约翰梅纳德凯恩斯在《通论》中明确拒绝使用价格指数,转而建议采用基于总工资额与技能调整总工时的工资单位。当凯恩斯谈及物价上升时,他意指相对于此工资单位的上升。此处不宜详论凯恩斯的工资理论,但值得注意的是:在将工资单位作为明确的短期计价标准提出时,凯恩斯建议价格指数适用于长期,即“在特定(可能相当宽泛)限度内公开承认不精确与近似性的历史比较”(Keynes 1936, p.43)。

马歇尔最杰出的学生对其核心信息的这种讽刺性反转是否蕴含更深意涵?在《凯恩斯指南》中,阿尔文$\cdot$汉森认为凯恩斯使用价格指数同样可以处理短尺度问题。但他着重强调:长期中采用工资率平减将低估产出,因其未考虑生产率增长。汉森似乎完全忽视了价值理论的论证。这或许正是关键所在。不论好坏,在当代,价值尺度已成为虚幻而几被遗忘的实践。量化尺度已成为核心挑战\footnote{在完成最后这句论述后,我注意到近期有两项大规模的劳动价值测算研究。瑞士联合银行经济研究部(UBS Economic Research, 1997)在全球范围内测算了巨无霸汉堡、小麦和大米的劳动支配价值。达拉斯联邦储备银行(Cox与Alm,1997)的研究则通过对比二十世纪初与当代数据,测算了美国各类商品服务的劳动价值变迁。我相信马歇尔定会赞许此类研究。}。


% 书面翻译的参考文献
% 默认使用正文的参考文献样式;
% 如果使用 BibTeX,可以切换为其他兼容 natbib 的 BibTeX 样式。
% \bibliographystyle{unsrtnat}
% \bibliographystyle{IEEEtranN}

% 默认使用正文的参考文献 .bib 数据库;
% 如果使用 BibTeX,可以改为指定数据库,如 \bibliography{ref/refs}。
\notecite{almTimeWellSpent1997}
\notecite{groenewegenSOARINGEAGLEAlfred1995}
\notecite{hansenGuideKeynes1953}
\notecite{huttEconomistsPublicStudy1990}
\notecite{jevonsMoneyMechanismExchange1893}
\notecite{jevonsTheoryPoliticalEconomy}
\notecite{keynesGeneralTheoryEmployment1936}
\notecite{lockeConsiderationsConsequencesLowering1991}
\notecite{laidlerAlfredMarshallDevelopment1990}
\notecite{malthusPrinciplesPoliticalEconomy1989}
\notecite{marshallPrinciplesEconomics1920}
\notecite{marshallRemediesFluctuationsGeneral1925}
\notecite{milPrinciplesPoliticalEconomy1929}
\notecite{myintTheoriesWelfareEconomics1948}
\notecite{perskyMarshallsNeoClassicalLaborValues1999}
\notecite{ricardoPrinciplesPoliticalEconomy1951}
\notecite{roscherPrinciplesPoliticalEconomy1882}
\notecite{smithWealthNations1937}
\notecite{tookeHistoryPricesState1838}
\notecite{ubseconomicresearchPricesEarningsGlobe1997}
\notecite{walshMeasurementGeneralExchangeValue1901}
\notecite{whitakerHistoryCriticismLabor1968}

\printbibliography

% 书面翻译对应的原文索引
\begin{translation-index}
  \notecite{perskyMarshallsNeoClassicalLaborValues1999}
  \printbibliography 
\end{translation-index}

\end{translation}  % 本科生:外文资料的书面翻译
%% !TeX root = ../thuthesis-example.tex

\chapter{补充内容}

附录是与论文内容密切相关、但编入正文又影响整篇论文编排的条理和逻辑性的资料,例如某些重要的数据表格、计算程序、统计表等,是论文主体的补充内容,可根据需要设置。

附录中的图、表、数学表达式、参考文献等另行编序号,与正文分开,一律用阿拉伯数字编码,
但在数码前冠以附录的序号,例如“图~\ref{fig:appendix-figure}”,
“表~\ref{tab:appendix-table}”,“式\eqref{eq:appendix-equation}”等。


\section{插图}

% 附录中的插图示例(图~\ref{fig:appendix-figure})。

\begin{figure}
  \centering
  \includegraphics[width=0.6\linewidth]{example-image-a.pdf}
  \caption{附录中的图片示例}
  \label{fig:appendix-figure}
\end{figure}


\section{表格}

% 附录中的表格示例(表~\ref{tab:appendix-table})。

\begin{table}
  \centering
  \caption{附录中的表格示例}
  \begin{tabular}{ll}
    \toprule
    文件名          & 描述                         \\
    \midrule
    thuthesis.dtx   & 模板的源文件,包括文档和注释 \\
    thuthesis.cls   & 模板文件                     \\
    thuthesis-*.bst & BibTeX 参考文献表样式文件    \\
    thuthesis-*.bbx & BibLaTeX 参考文献表样式文件  \\
    thuthesis-*.cbx & BibLaTeX 引用样式文件        \\
    \bottomrule
  \end{tabular}
  \label{tab:appendix-table}
\end{table}


\section{数学表达式}

% 附录中的数学表达式示例(式\eqref{eq:appendix-equation})。
\begin{equation}
  \frac{1}{2 \uppi \symup{i}} \int_\gamma f = \sum_{k=1}^m n(\gamma; a_k) \mathscr{R}(f; a_k)
  \label{eq:appendix-equation}
\end{equation}


\section{文献引用}

附录\cite{dupont1974bone}中的参考文献引用\cite{zhengkaiqing1987}示例
\cite{dupont1974bone,zhengkaiqing1987}。

\printbibliography


% 致谢
% % !TeX root = ../thuthesis-example.tex

\begin{acknowledgements}

终于到了写致谢的时候了!从2019年到2025年,我的本科读了整整六年,这六年里充满挑战和成长。回望这六年,回顾这段历程,若非诸位师友亲朋的鼎力支持,我不可能顺利地完成本科的学业。因此,我想在此对大家致以诚挚的谢意。

​首先,我怀着无比崇高的敬意与由衷的感激,向我的毕业设计指导老师——蔡继明老师和李帮喜老师——致以最诚挚的谢意,感谢他们在整个过程中倾注的宝贵时间、无私帮助与无尽耐心。

我特别感谢蔡继明教授对我学术生涯的深远影响。正是蔡老师当年在《政治经济学原理》课堂上鞭辟入里、引人入胜的讲授,点燃了我对政治经济学研究的浓厚兴趣。这份宝贵的兴趣,驱动我在2022年做出了人生中最重要的选择之一——从自动化系转至社会科学学院经济研究所。蔡老师所提出的广义价值论,不仅为我的本科毕业设计构建了坚实而有力的分析框架,更为我指明了未来的研究路径,奠定了我在即将开始的硕士研究生阶段继续深入探索的方向与信心。

我还要衷心感谢李帮喜老师的悉心指导。李老师不仅是我的毕业设计指导者,更是全程悉心陪伴的引路人。从最初构思选题的启发性点拨,到后续涉及设计整体布局、文献搜集整理乃至格式规范调整的各个关键环节,李老师都给予了至关重要且极具建设性的宝贵意见。尤其令人感佩的是,李老师付出大量心血,逐字逐句审阅了我的每一份阶段性文稿(累计近二十份),并提供了具体、精准的修改建议与及时的反馈。没有李老师细致入微、不遗余力的指导与支持,我的毕业设计不可能如此顺利地完成最终定稿。对此,我铭记于心,不胜感激!​

此外,感谢李红军老师在我毕业设计期间给予的深刻启迪与方向指引。李老师勉励我不应仅将毕业设计视为对本科阶段知识的总结,而更应视作开启研究生学术生涯的重要准备。受此启发,我自觉投入更多精力深入研读相关文献,力求拓展研究的广度和深度;同时更加注重行文逻辑的严谨性和论证的严密性,反复推敲与打磨每一个环节。李老师的鼓励始终鞭策着我在毕业设计中投入更多的时间和精力。​另外,李老师还为我撰写了申请研究生的专家推荐信,对我成功通过清华大学硕士研究生的考核起了决定性的作用。对此慷慨提携,我感铭肺腑。

深深感谢自动化系的王红老师。在我曾彷徨于人生抉择的十字路口、陷入短暂迷茫之时,是王红老师以智者般的洞察和师者般的仁心,为我点亮了一盏指引航向的明灯。她那充满力量与温度的鼓励,不仅让我重拾前行的信心,更是拨云见日,使我得以清晰地辨识并坚定地回归到真正属于自己的人生道路上——这份指引,是我大学历程中最为宝贵的财富。与此同时,王红老师亦为我撰写了专家推荐信,在清华大学硕士研究生申请的激烈竞争中,起到了关键作用。

我还想感谢人称“靳妈”的靳卫萍老师。一方面,在《中国宏观经济分析》的课堂上,靳老师以其独到的见解,教会我用逻辑和经济学的原理串联起世界上纷繁复杂的政治、经济事件,让我领略了经济学的强大力量。另一方面,当我深陷健康危机的困扰时,靳老师如慈母般忧心切切,为我寻医问药。这份超越师生之责的倾力相助,极大地缓解了我生理上的苦楚,为我当时的学习和生活提供了最坚实的屏障。倘若说王红老师是一盏明灯,那么靳老师就是一双温暖的手。若说王红老师是我迷茫时的明灯,那么靳老师便是那双在我困顿之际给予鼎力帮助的援手。这份情谊我将永远铭刻于心。​

而后,我要感谢我的家人。是他们在我求学的漫漫长路上,倾尽所能提供了最为坚实的物质保障与精神依靠。尤令我感怀于心的是,在饱含深厚亲情的同时,他们更能以难能可贵的科学态度与冷静思维,在我遇到困难时给予我清晰理性的指引与强大温暖的支持,帮助我一次次安然渡过挑战。我深知,我的每一分艰辛挣扎都牵动着他们的心神;此刻,我亦殷切期盼能将完成毕设、顺利毕业的欢欣传递给他们,与我一同共享这份十八年苦读结出的硕果!​

我还要感谢我的女朋友谢凤婷。在我需要依靠的时候,她总是愿意张开双臂给我最温暖的拥抱;当我感到快要撑不下去时,她的鼓励总能实实在在地给我继续前进的力量;当我骄傲自满时,更是她以难能可贵的清醒与温婉,及时将我拉回正轨。我感谢她毫无保留的付出与给予,也感谢她始终如一的包容与理解。我会永远珍惜这份感情。

最后,我还要感谢来自自动化、社科学院的朋友们,难忘与自动化系的同窗们决战紫荆之巅、激战炼狱小镇,也难忘和社科的同学们纵论时政、针砭时弊。这份友谊是大学生涯馈赠的珍宝,我将长久珍视于心!

行文至此,六载韶华恍如昨日。这段在清华大学的珍贵旅程,因有你们而更为珍贵。在未来的求学路上,我将带着一颗充满感恩的心继续前行!

\end{acknowledgements}


% 声明
% 本科生开题报告不需要
% 以下命令生成空白声明页
% \statement
% 将签字扫描后的声明文件替换原始页面
% \statement[file=figures/Statement_signed.pdf]
% 本科生编译生成的声明页默认不加页脚,插入扫描版时再补上;
% 研究生编译生成时有页眉页脚,插入扫描版时不再重复。
% 也可以手动控制是否加页眉页脚
% \statement[page-style=empty]
% \statement[file=scan-statement.pdf, page-style=plain]

% 个人简历、在学期间完成的相关学术成果
% 本科生可以附个人简历,也可以不附个人简历
% \input{data/resume}

% 指导教师/指导小组评语
% 本科生不需要
% \input{data/comments}

% 答辩委员会决议书
% 本科生不需要
% \input{data/resolution}

% 本科生的综合论文训练记录表(扫描版)
% \record{file=scan-record.pdf}

\end{document}
