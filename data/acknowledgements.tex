% !TeX root = ../thuthesis-example.tex

\begin{acknowledgements}

终于到了写致谢的时候了!从2019年到2025年,我的本科读了整整六年,这六年里充满挑战和成长。回望这六年,回顾这段历程,若非诸位师友亲朋的鼎力支持,我不可能顺利地完成本科的学业。因此,我想在此对大家致以诚挚的谢意。

​首先,我怀着无比崇高的敬意与由衷的感激,向我的毕业设计指导老师——蔡继明老师和李帮喜老师——致以最诚挚的谢意,感谢他们在整个过程中倾注的宝贵时间、无私帮助与无尽耐心。

我特别感谢蔡继明教授对我学术生涯的深远影响。正是蔡老师当年在《政治经济学原理》课堂上鞭辟入里、引人入胜的讲授,点燃了我对政治经济学研究的浓厚兴趣。这份宝贵的兴趣,驱动我在2022年做出了人生中最重要的选择之一——从自动化系转至社会科学学院经济研究所。蔡老师所提出的广义价值论,不仅为我的本科毕业设计构建了坚实而有力的分析框架,更为我指明了未来的研究路径,奠定了我在即将开始的硕士研究生阶段继续深入探索的方向与信心。

我还要衷心感谢李帮喜老师的悉心指导。李老师不仅是我的毕业设计指导者,更是全程悉心陪伴的引路人。从最初构思选题的启发性点拨,到后续涉及设计整体布局、文献搜集整理乃至格式规范调整的各个关键环节,李老师都给予了至关重要且极具建设性的宝贵意见。尤其令人感佩的是,李老师付出大量心血,逐字逐句审阅了我的每一份阶段性文稿(累计近二十份),并提供了具体、精准的修改建议与及时的反馈。没有李老师细致入微、不遗余力的指导与支持,我的毕业设计不可能如此顺利地完成最终定稿。对此,我铭记于心,不胜感激!​

此外,感谢李红军老师在我毕业设计期间给予的深刻启迪与方向指引。李老师勉励我不应仅将毕业设计视为对本科阶段知识的总结,而更应视作开启研究生学术生涯的重要准备。受此启发,我自觉投入更多精力深入研读相关文献,力求拓展研究的广度和深度;同时更加注重行文逻辑的严谨性和论证的严密性,反复推敲与打磨每一个环节。李老师的鼓励始终鞭策着我在毕业设计中投入更多的时间和精力。​另外,李老师还为我撰写了申请研究生的专家推荐信,对我成功通过清华大学硕士研究生的考核起了决定性的作用。对此慷慨提携,我感铭肺腑。

深深感谢自动化系的王红老师。在我曾彷徨于人生抉择的十字路口、陷入短暂迷茫之时,是王红老师以智者般的洞察和师者般的仁心,为我点亮了一盏指引航向的明灯。她那充满力量与温度的鼓励,不仅让我重拾前行的信心,更是拨云见日,使我得以清晰地辨识并坚定地回归到真正属于自己的人生道路上——这份指引,是我大学历程中最为宝贵的财富。与此同时,王红老师亦为我撰写了专家推荐信,在清华大学硕士研究生申请的激烈竞争中,起到了关键作用。

我还想感谢人称“靳妈”的靳卫萍老师。一方面,在《中国宏观经济分析》的课堂上,靳老师以其独到的见解,教会我用逻辑和经济学的原理串联起世界上纷繁复杂的政治、经济事件,让我领略了经济学的强大力量。另一方面,当我深陷健康危机的困扰时,靳老师如慈母般忧心切切,为我寻医问药。这份超越师生之责的倾力相助,极大地缓解了我生理上的苦楚,为我当时的学习和生活提供了最坚实的屏障。倘若说王红老师是一盏明灯,那么靳老师就是一双温暖的手。若说王红老师是我迷茫时的明灯,那么靳老师便是那双在我困顿之际给予鼎力帮助的援手。这份情谊我将永远铭刻于心。​

而后,我要感谢我的家人。是他们在我求学的漫漫长路上,倾尽所能提供了最为坚实的物质保障与精神依靠。尤令我感怀于心的是,在饱含深厚亲情的同时,他们更能以难能可贵的科学态度与冷静思维,在我遇到困难时给予我清晰理性的指引与强大温暖的支持,帮助我一次次安然渡过挑战。我深知,我的每一分艰辛挣扎都牵动着他们的心神;此刻,我亦殷切期盼能将完成毕设、顺利毕业的欢欣传递给他们,与我一同共享这份十八年苦读结出的硕果!​

我还要感谢我的女朋友谢凤婷。在我需要依靠的时候,她总是愿意张开双臂给我最温暖的拥抱;当我感到快要撑不下去时,她的鼓励总能实实在在地给我继续前进的力量;当我骄傲自满时,更是她以难能可贵的清醒与温婉,及时将我拉回正轨。我感谢她毫无保留的付出与给予,也感谢她始终如一的包容与理解。我会永远珍惜这份感情。

最后,我还要感谢来自自动化、社科学院的朋友们,难忘与自动化系的同窗们决战紫荆之巅、激战炼狱小镇,也难忘和社科的同学们纵论时政、针砭时弊。这份友谊是大学生涯馈赠的珍宝,我将长久珍视于心!

行文至此,六载韶华恍如昨日。这段在清华大学的珍贵旅程,因有你们而更为珍贵。在未来的求学路上,我将带着一颗充满感恩的心继续前行!

\end{acknowledgements}
