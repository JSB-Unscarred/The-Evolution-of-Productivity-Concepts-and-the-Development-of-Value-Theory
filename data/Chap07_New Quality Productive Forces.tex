% !TeX root = ../2019080346_Mason.tex
\chapter{新质生产力}

\section{引言}

在完成了对价值理论和生产力概念的梳理后,笔者将在这一章试着对新质生产力进行一些探讨。

新质生产力一词的提出者习近平指出:“新质生产力是创新起主导作用,摆脱传统经济增长方式、生产力发展路径,具有高科技、高效能、高质量特征,符合新发展理念的先进生产力质态。它由技术革命性突破、生产要素创新性配置、产业深度转型升级而催生,以劳动者、劳动资料、劳动对象及其优化组合的跃升为基本内涵,以全要素生产率大幅提升为核心标志,特点是创新,关键在质优,本质是先进生产力。”\cite[515-516]{XiJinPingXiJinPingJingJiWenXuanDiYiJuan2025}在笔者看来,新质生产力是一个内涵丰富的经济学概念,以上这段话只是对新质生产力的一个概括。要全面认识新质生产力,需要从绝对生产力、相对生产力、比较(综合)生产力和社会总和生产力这四个维度入手。

\section{从绝对生产力的维度理解新质生产力}

从绝对生产力的维度来看,新质生产力的本质是绝对生产力。

从本质上看,新质生产力在本质上落脚于生产力\cite[138]{ZhangLinXinZhiShengChanLiDeNeiHanTeZhengLiLunChuangXinYuJieZhiYiYun2023},属于马克思主义生产力的范畴\cite[7]{RenBaoPingXinZhiShengChanLiWenXianZongShuYuYanJiuZhanWang2024}\cite[2-4]{ZhouWenLunXinZhiShengChanLiNeiHanTeZhengYuChongYaoZhaoLiDian2023}\cite[129]{GaoFanXinZhiShengChanLiDeTiChuLuoJiDuoWeiNeiHanJiShiDaiYiYi2023}\cite[1-2]{PuQingPingXiJinPingZongShuJiGuanYuXinZhiShengChanLiChongYaoLunShuDeShengChengLuoJiLiLunChuangXinYuShiDaiJieZhi2023}\cite[15-16]{RenBaoPingShengChanLiXianDaiHuaZhuanXingXingChengXinZhiShengChanLiDeLuoJi2024}。而前文已经指出,根据马克思“生产力当然始终是有用的、具体的劳动的生产力。它事实上只决定有目的的生产活动在一定时间内的效率”\cite[59]{ZhongGongZhongYangMaKeSiEnGeSiLieNingSiDaLinZhuZuoBianYiJuMaKeSiEnGeSiWenJiDi5Juan2009}的观点,马克思政治经济学中的生产力概念属于绝对生产力的范畴。有学者指出,“新质”生产力是传统生产力的质变或跃升\cite[3]{ZhouWenLunXinZhiShengChanLiNeiHanTeZhengYuChongYaoZhaoLiDian2023}\cite[143]{ZhangLinXinZhiShengChanLiDeNeiHanTeZhengLiLunChuangXinYuJieZhiYiYun2023}\cite[52]{XuZhengXinZhiShengChanLiFuNengGaoZhiLiangFaZhanDeNeiZaiLuoJiYuShiJianGouXiang2023}\cite[2]{PuQingPingXiJinPingZongShuJiGuanYuXinZhiShengChanLiChongYaoLunShuDeShengChengLuoJiLiLunChuangXinYuShiDaiJieZhi2023}\cite[13]{RenBaoPingShengChanLiXianDaiHuaZhuanXingXingChengXinZhiShengChanLiDeLuoJi2024}\cite[6]{RenBaoPingXinZhiShengChanLiWenXianZongShuYuYanJiuZhanWang2024}。所以,“新质”生产力在本质维度上是绝对生产力的质变或跃升。

\section{从相对生产力的维度理解新质生产力}

从相对生产力的维度来看,新质生产力是相对生产力重构引发的比较优势转换。

首先,新质生产力是基于可变分工体系\footnote{回顾前文,可变分工体系指的是生产者分工方向可以改变的分工体系}的。正如马克思和恩格斯所言:“一个民族的生产力发展的水平,最明显地表现于该民族分工的发展程度。任何新的生产力,只要它不是迄今已知的生产力单纯的量的扩大(例如,开垦土地),都会引起分工的进一步发展。” \cite[520]{ZhongGongZhongYangMaKeSiEnGeSiLieNingSiDaLinZhuZuoBianYiJuMaKeSiEnGeSiWenJiDi1Juan2009}对于新质生产力来说,一方面如前文所述,新质生产力是传统生产力的质变或跃升,发展新质生产力在理论上必然会引起分工的进一步发展\cite[141]{ZhangLinXinZhiShengChanLiDeNeiHanTeZhengLiLunChuangXinYuJieZhiYiYun2023}。另一方面,发展新质生产力也在实践上要求我们主动推进分工的进一步发展。正如习近平指出,发展新质生产力要“大力推动科技创新”,“科技创新能够催生新产业、新模式、新动能”;发展新质生产力还要“以科技创新推动产业创新”,“科技成果转化为现实生产力,表现形式为催生新产业、推动产业深度转型升级。因此,我们要及时将科技创新成果应用到具体产业和产业链上,改造提升传统产业,培育壮大新兴产业,布局建设未来产业,完善现代化产业体系”。\cite[516]{XiJinPingXiJinPingJingJiWenXuanDiYiJuan2025}这里的“新产业”、“新兴产业”和“未来产业”,本质上都是新的、更深层次的分工。总而言之,新质生产力的概念一定是建立在可变分工体系的基础之上的。

其次,基于可变分工体系的比较优势是一个动态的概念。前文指出,在可变分工体系下我们可以定义相对生产力的概念,进而可以计算相对生产力系数并确定自己的比较优势。同时,比较优势也不是固化的,而是可以通过重构自己的相对生产力而动态变化的\cite[69-70]{LiuLeYiJingTaiBiJiaoYouShiDongTaiHuaDeQuDongLiYuLiShiJingYanJianLunFaZhanXinZhiShengChanLiYuTiShengChanYeLianGongYingLianRenXingNeiYin2025}。例如,韩国在20世纪60年代曾以简单制造业(假发、胶合板、纺织品)为自己的比较优势,而到90年代,其比较优势已经成为汽车、计算机等\cite[43]{westphalIndustrialPolicyExport1990};日本再20世纪50-70年代时以钢铁、化工和造船为比较优势产业,而在70-80年代以汽车和半导体为比较优势产业\cite[278]{itoEastAsianMiracle1994}。

因此,发展新质生产力的核心要义,在于将我国传统依赖劳动力和自然资源形成的比较优势,通过构建以战略性新兴产业和未来产业为主导的现代产业体系\cite[29]{CaiJiMingXinZhiShengChanLiDeFaZhanDuiJieZhiChuangZaoHeJingJiZengChangDeGongXian2024}\cite{WuKeDaLiTuiJinXianDaiHuaChanYeTiXiJianSheJiaKuaiFaZhanXinZhiShengChanLi2024}\cite[18-20]{HuangQunHuiXinZhiShengChanLiXiTongYaoSuTeZhiJieGouChengZaiYuGongNengQuXiang2024},转化为以核心技术为基础的新型比较优势\cite[6]{ZhouWenLunXinZhiShengChanLiNeiHanTeZhengYuChongYaoZhaoLiDian2023}。具体而言,战略性新兴产业包括新一代信息技术、生物技术、新能源、新材料 、高端装备、新能源汽车、绿色环保以及航空航天、海洋装备等产业;未来产业包括类脑智能、量子信息、基因技术、未来网络、深海空天开发、氢能与储能等产业\cite[24]{QuanGuoRenMinDaiBiaoDaHuiZhongHuaRenMinGongHeGuoGuoMinJingJiHeSheHuiFaZhanDiShiSiGeWuNianGuiHuaHe2035NianYuanJingMuBiaoGangYao2021}。

从生产力的角度来看,实现比较优势动态转换的关键在于构建现代化产业体系绝对生产力的双重增速优势:一方面需确保其增速超越国内传统产业升级速率,改变我国内部的相对生产力大小;另一方面要形成具有国际竞争力的持续增长动能,改变国际分工中的相对生产力系数变化。通过这种内外联动的生产力跃升机制,方能引发我国比较优势的转换。

我们也要看到,新质生产力在发展的早期极有属于比较劣势产业,是脆弱的。但是,比较劣势产业生产力的从无到有代表着一国迈出了培育新比较优势的第一步\cite[70]{LiuLeYiJingTaiBiJiaoYouShiDongTaiHuaDeQuDongLiYuLiShiJingYanJianLunFaZhanXinZhiShengChanLiYuTiShengChanYeLianGongYingLianRenXingNeiYin2025},发展比较劣势产业的最终目的是为了使其成为新的比较优势产业。所以,新质生产力发展的初期往往需要一定的产业、贸易政策进行保护。例如,我国的新能源汽车产业就经历了从无到有,从弱小到强大的过程,现在已经成为新能源汽车的最大生产国和销售国。在这个过程中,产业政策发挥了巨大的作用。\cite{WangMingHeWoGuoXinNengYuanQiCheChanYeZhengCeYanJiu2023}

\section{从比较生产力的的维度理解新质生产力}

从比较生产力的的维度来看,新质生产力是价值创造导向的比较生产力跃迁。

\subsection{国际贸易}

前文指出,发展新质生产力的核心要义是比较优势的转换,但这种转换本身不是目的,发展新质生产力的目的是在国际贸易中获得更大的好处,为社会创造更多的价值。广义价值论指出,单位商品的价值与比较生产力正相关,部门劳动创造的价值总量与部门比较生产力正相关。发展新质生产力是在高水平对外开放的背景之下的\cite[518]{XiJinPingXiJinPingJingJiWenXuanDiYiJuan2025},发展新质生产力建立新的比较优势,能够提升我国在国际贸易中的比较生产力,提高我国劳动产出的价值,最大化我国贸易收益\cite[76]{LiuLeYiJingTaiBiJiaoYouShiDongTaiHuaDeQuDongLiYuLiShiJingYanJianLunFaZhanXinZhiShengChanLiYuTiShengChanYeLianGongYingLianRenXingNeiYin2025}。

\subsection{价值创造与生产关系}

倘若我们承认发展新质生产力能够提高我国劳动产出在国际贸易中的价值,我们实际上就已经承认了国际贸易中的耗费劳动作为价值源泉和价值尺度并不在量上绝对相等,而是可以通过一定机制进行折算。如果我们承认这一点,那我们实际上就是承认了非劳动要素在价值创造中的作用。事实上,没有理由把这种认识局限在国际贸易的领域,这一结论完全可以推广到一般的商品交换之中,将这一结论推广到一般的。这种推广不仅能推动中国特色社会主义政治经济学的发展,而且是生产关系适应生产力的必然要求。

新质生产力的发展必然扩大与传统生产力水平的差距\cite[39-40]{ChenZhangHongGuanJingJiBoDongShiZhengFenXiDeYiZhongSiLuShiLunShengChanLiFeiPingHengJieGouYuJingJiBoDong1999},从而引起战略性新兴产业、未来产业与传统产业之间收入差距的扩大\cite[31]{CaiJiMingXinZhiShengChanLiDeFaZhanDuiJieZhiChuangZaoHeJingJiZengChangDeGongXian2024}。上文已经指出,新质生产力的载体产业往往具有较高的比较生产力,这些产业的单位劳动能够创造出更多的价值,因此这些产业与传统产业存在一定的收入差距是合理的。而要判断行业间收入差距是否合理,就要看各行业间的收入差距与各行业间的比较生产力差别是否一致:二者一致就是合理的,应该肯定和保护;二者不一致的部分要缩小和消除\cite[69]{CaiJiMingLongDuanHeJingZhengXingYeDeBiJiaoShengChanLiYuShouRuChaiJuJiYuGuangYiJieZhiLunDeFenXi2014}。而了达到这种分配的合理性,需要在发展新质生产力的同时坚持生产要素按贡献参与分配的原则\cite[58]{CaiJiMingMaKeSiLaoDongShengChanLiYuJieZhiLiangZhengXiangGuanYuanLiDeKuoZhanJiYingYongJianLunXinZhiShengChanLiDeFaZhanYuWoGuoJiBenJingJiZhiDuDeWanShan2025}。

\section{从社会总和生产力的维度理解新质生产力}

从社会总和生产力的维度来看,新质生产力能够驱动社会总价值量的跃迁。

尽管实际GDP作为衡量一个国家全年创造价值总量的指标并不是完美的,但大部分经济学家仍然使用实际GDP来衡量一个国家的整体发展水平和整体福利水平\cite[386]{BaoLuo*SaMouErSenJingJiXueDiShiJiuBan2012}\cite[17-24]{n.GeLiGaoLi*ManKunJingJiXueYuanLiDi7BanHongGuanJingJiFenCe2015}。新质生产力强调技术创新和要素重组,其发展必然能推动我国整体经济发展水平的上升,也必然伴随着实际GDP的上升。但是,正如本文开头所述的“价值总量之谜”,用劳动价值论所提供的绝对生产力视角并不能解释新质生产力对经济发展的作用,更不用说量化评估这种作用了。

解决这一谜题的办法从社会总和生产力的角度看待经济发展。根据广义价值论的判断,社会价值总量与社会总和生产力正相关。新质生产力的发展会推动我国社会总和生产力的提高,从而增加每期的社会价值总量。这一理论判断已得到实证研究的支持:以数字经济为代表的新质生产力形态,正通过提升社会总和生产力显著扩大价值总量。例如,有学者测算,2011-2019年间数据对经济增长率的平均贡献率达到34.46\%\cite[63]{LiuTaoXiongShuJuZiBenGuSuanJiDuiZhongGuoJingJiZengChangDeGongXianJiYuShuJuJieZhiLianDeShiJiao2023};还有学者研究了数据对经济增长贡献的具体机制\cite{CaiJiMingLunShuJuYaoSuAnGongXianCanYuFenPeiDeJieZhiJiChuJiYuGuangYiJieZhiLunDeShiJiao2023}\cite{RenBaoPingShuZiXinZhiShengChanLiTuiDongJingJiGaoZhiLiangFaZhanDeLuoJiYuLuJing2023}。

总而言之,新质生产力能够驱动社会总价值量的跃迁。

\section{总结}

总的来说,新质生产力的本质是绝对生产力的质变跃升。它通过发展战略新兴产业和培育未来产业,促进现代产业体系的形成,重构动态比较优势,改善我国贸易调节。最后,新质生产力还会推动经济理论以及生产关系的进步,塑造了中国式现代化的发展路径。