% !TeX root = ../2019080346_Mason.tex
\chapter{总结与讨论}

\section{价值、价值形成、价值决定和价值尺度}

\subsection{劳动作为衡量交换价值的尺度}

如前所述,亚当·斯密在国富论中首次提出了“交换价值\footnote{这里应当是价“价值”而非“交换价值;这里笔者先按照几位古典政治经济学家的原文使用“交换价值”一词;交换价值和价值的关系在后文中阐述。}的真实尺度”问题,并指出劳动是这一问题的答案。而后李嘉图、马克思、马尔萨斯都认为劳动是衡量交换价值的尺度。值得注意的是,马克思认为古典政治经济学的根本倾向是劳动价值论,但这一论断忽视了古典政治经济学不同理论之间的差异性\footnote{由于学界对古典政治经济学的划分不尽相同,对古典政治经济学家思想的理解也不尽相同,所以有学者认为古典政治经济学所得出的结论是“多要素供求价值论”\cite[179]{CaiJiMingCongGuDianZhengZhiJingJiXueDaoZhongGuoTeSeSheHuiZhuYiZhengZhiJingJiXueJiYuZhongGuoShiJiaoDeZhengZhiJingJiXueYanBianShangCe2023}。}。根据笔者对古典政治经济学的认识,笔者认为更有说服力的结论是:古典政治经济学的基本倾向是把劳动作为衡量交换价值的尺度;这里的价值尺度,指的是衡量商品价值量的社会标准,强调价值的社会属性。

一方面,古典政治经济学家意识到,在社会分工建立起来后,一个生产者或许因为不能生产某一需要的商品,或许因为通过交换得到的商品更好更便宜,所以会愿意用自己的劳动生产一些不愿自己消费的商品来交换他物。然而,商品都具有作为使用价值的异质性,为了与其它商品进行交换,商品的拥有者会根据在这一商品上花费的劳动\footnote{包括活劳动和物化劳动}与他人进行交换。这是因为这个生产者为了生产者这一商品所投入的就是自己的劳动,也只能知道自己为了这一商品投入了多少劳动。而那些与他进行商品交换的生产者,也是以同样的考量与他进行交换。因此,劳动成为了商品交换时所依据的尺度。换言之,劳动成为了衡量交换价值的尺度。\cite[1016-1017]{ZhongGongZhongYangMaKeSiEnGeSiLieNingSiDaLinZhuZuoBianYiJuMaKeSiEnGeSiWenJiDi7Juan2009}\cite[25]{YaDang*SiMiGuoFuLun2015}\cite[133]{MaErSaSiZhengZhiJingJiXueDingYi2023}

另一方面,交换价值会随着时间和地点的不同上下波动,具有偶然性和相对性。因此,古典政治经济学家所能找到的不变的、普遍的交换价值的尺度,只有劳动。\cite[49-51]{ZhongGongZhongYangMaKeSiEnGeSiLieNingSiDaLinZhuZuoBianYiJuMaKeSiEnGeSiWenJiDi5Juan2009}\cite[28-30]{YaDang*SiMiGuoFuLun2015}

\subsection{从交换价值到价值}

然而,交换价值是一种使用价值同另一种使用价值的比例,其本身没有单位,也没有数量大小上的意义。所以前文中古典政治经济学家使用“交换价值的尺度”这一说法是不合理的——一个没有大小的量怎么能被“衡量”呢?事实上,古典政治经济学家所衡量的其实不是交换价值,而是价值。只是在政治经济学发展的早期阶段,经济学家们还没有能力从交换价值中抽象出价值的概念。直到马克思首次区分了交换价值和价值,价值的面纱才被揭开\cite[86-87]{ChenDaiSunCongGuDianJingJiXuePaiDaoMaKeSiRuoGanZhuYaoXueShuoFaZhanLueLun2014}。

笔者认为,政治经济学逐步区分交换价值和价值的过程可以从两个角度来理解。

首先,从价值实体的角度来看,假设某种商品可以和多种商品进行交换,那么这一商品就具有了许多种交换价值,这些交换价值必然也是可以相互替代的交换价值。所以,“第一,同一种商品的各种有效的交换价值表示一个等同的东西。第二,交换价值只能是可以与它相区别的某种内容的表现方式,‘表现形式’。”\cite[49]{ZhongGongZhongYangMaKeSiEnGeSiLieNingSiDaLinZhuZuoBianYiJuMaKeSiEnGeSiWenJiDi5Juan2009}也就是说,这些不同的交换价值都可以被化为“一种等量的共同的东西”\cite[49]{ZhongGongZhongYangMaKeSiEnGeSiLieNingSiDaLinZhuZuoBianYiJuMaKeSiEnGeSiWenJiDi5Juan2009}。而且,这种共同东西既不是使用价值——使用价值具有异质性,也不是交换价值——交换价值仅仅是一个比例,没有大小,也无法被“衡量”。进而,马克思把这种共同的东西称为价值\cite[50]{ZhongGongZhongYangMaKeSiEnGeSiLieNingSiDaLinZhuZuoBianYiJuMaKeSiEnGeSiWenJiDi5Juan2009}。值得注意的是,尽管马克思已经能从交换价值中抽象出价值来,但是笔者却不认同马克思推演的逻辑。马克思在意识到不同的交换价值是在衡量一个既不是使用价值也不是交换价值的“共同的东西”之后,就说“如果把商品体的使用价值撇开,商品体就只剩下一个属性,即劳动产品这个属性”\cite[50-51]{ZhongGongZhongYangMaKeSiEnGeSiLieNingSiDaLinZhuZuoBianYiJuMaKeSiEnGeSiWenJiDi5Juan2009},再把具体劳动的成分撇开,那么剩下的“只是无差别的人类劳动的单纯凝结”,这种凝结的抽象劳动,就是商品的价值\cite[51]{ZhongGongZhongYangMaKeSiEnGeSiLieNingSiDaLinZhuZuoBianYiJuMaKeSiEnGeSiWenJiDi5Juan2009}。这里的问题主要在于:如果认为使用价值是异质的而不能作为价值决定的因素,那么,劳动也是异质的, 同样不能作为价值决定的因素;如果认为各种具体的异质的劳动可以抽象为无差别的一般人类劳动,那么,这一抽象过程同样适用于各种异质的使用价值。或者说,当人们将具体劳动抽象为无差别的一般人类劳动的同时,事实上也就把各种使用价值抽象为一般的使用价值即效用\cite[84]{CaiJiMingLunJieZhiJueDingYuJieZhiFenPeiDeTongYi2003}。也就是说,将商品的价值和“无差别的人类劳动的单纯凝结”等同起来并不是一个不证自明的过程,马克思的论证逻辑是存在问题的。但是,笔者认为马克思将价值从交换价值中抽象出来的逻辑过程是正确的,价值确实是一种既不同于使用价值也不同于交换价值的,客观存在的实体。

正因如此,笔者进一步认为前文中不同经济学家对“交换价值的尺度”的争论实际上是对“价值尺度”的争论。交换价值本身只是两种使用价值的比例,没有“大小”的概念,因此也无法被“衡量”。而价值作为一个实体,理论上是可以被衡量的。不同经济学家所争论的,正是价值应该用什么尺度来衡量,或者说价值的单位究竟是什么。

其次,人们对价值认识的演进和商品交换的发展是两个相反的过程。在商品交换的早期阶段,商品的“交换价值首先表现为一种使用价值同另一种使用价值相交换的量的关系或比例”\cite[49]{ZhongGongZhongYangMaKeSiEnGeSiLieNingSiDaLinZhuZuoBianYiJuMaKeSiEnGeSiWenJiDi5Juan2009},也就是简单的交换价值形式\footnote{在资本论中,马克思用的是“简单价值形式”一词,但由于马克思没有绝对严谨地区分价值和交换价值\cite[37]{ZhongGongZhongYangMaKeSiEnGeSiLieNingSiDaLinZhuZuoBianYiJuMaKeSiEnGeSiWenJiDi8Juan2009},所以这里笔者采取了蔡继明教授的更严谨的提法\cite[145]{CaiJiMingJieZhiZhengLunHuiGuYuZhanWang2008}。};随着交换范围的扩大,简单的交换价值形式发展为扩大的交换价值形式;当一般等价物出现后,所有使用价值都以一般等价物为媒介而进行交换,交换价值就发展为一般的形式;当货币产生后,交换价值便取得价格这种形式。尽管市场价格受供求波动影响,但长期观察显示其始终会围绕着一个相对稳定的轴心运动——这个轴心,或者说调节价格运动的规律,被亚当·斯密称为"自然价格"、马尔萨斯谓之"自然价值",最终在政治经济学发展中凝练为“价值”概念。这样,价值作为调节价格运动的规律这一特定的内涵便确定下来。\cite[145]{CaiJiMingJieZhiZhengLunHuiGuYuZhanWang2008}\footnote{事实上马克思也把价值的这种内涵以价值作用的方式表述了出来\cite[199]{ZhongGongZhongYangMaKeSiEnGeSiLieNingSiDaLinZhuZuoBianYiJuMaKeSiEnGeSiWenJiDi7Juan2009}。}随着人们进一步认识到价格只是交换价值的一种形式,价值的内涵也进一步一般化为调节交换价值的规律。

正如马克思所言:“对人类生活形式的思索,从而对这些形式的科学分析,总是采取同实际发展相反的道路。这种思索是从事后开始的,就是说,是从发展过程的完成的结果开始的。$\cdots$因此,只有商品价格的分析才导致价值量的决定,只有商品共同的货币表现才导致商品的价值性质的确定。”\cite[93]{ZhongGongZhongYangMaKeSiEnGeSiLieNingSiDaLinZhuZuoBianYiJuMaKeSiEnGeSiWenJiDi5Juan2009}在现实层面,价值研究确实肇始于对价格现象的经验观察;在理论层面,支配交换价值的规律作为价值的内在本质,也符合人类认知从现象到本质的渐进过程,体现了历史发展的内在连贯性。

上述对价值实体认知的演进,自然引向价值决定机制的核心争论——究竟何种劳动形态构成价值尺度?这需要从价值形成与价值决定的关系维度展开分析。

\subsection{价值尺度和价值形成的对立统一}

前文的论述表明,把劳动作为价值的尺度是古典政治经济学的基本倾向。但正如笔者在亚当·斯密相关章节中探讨的学术争议所示,古典政治经济学家们在该用商品生产所耗费的劳动还是用商品所能支配的劳动来作为价值尺度的问题上产生了严重的分歧。总的来说,劳动价值论派的李嘉图、马克思认为应当用耗费的劳动作为价值尺度,而多要素价值论派的斯密、马尔萨斯则认为应当用支配的劳动作为价值尺度。事实上,作为新古典价值论代表人物的马歇尔也有用支配的劳动作为价值尺度的倾向\cite{perskyMarshallsNeoClassicalLaborValues1999}。

\subsubsection{商品所能支配的劳动作为商品价值的尺度}
在笔者看来,用商品所能支配的劳动作为价值尺度无疑是更合适的。

首先,正如马克思所说:“价值的对象性只能在商品同商品的社会关系中表现出来”\cite[61]{ZhongGongZhongYangMaKeSiEnGeSiLieNingSiDaLinZhuZuoBianYiJuMaKeSiEnGeSiWenJiDi5Juan2009},价值本身具有社会属性。

恩格斯曾在《<资本论>第三册增补》中举了一个例子来论证应当用商品生产所耗费的劳动作为商品价值的尺度\cite[1015-1018]{ZhongGongZhongYangMaKeSiEnGeSiLieNingSiDaLinZhuZuoBianYiJuMaKeSiEnGeSiWenJiDi7Juan2009}。恩格斯说,在商品经济发展的初期,进行商品交换的主要是劳动的农民。这些农民借助自己家庭的帮助,在自己的田地上进行农业、畜牧业和手工业的生产,并拿除必需品之外剩下的剩余产品同其它农民家庭进行交换。这些农民之所以进行交换,并非因为自己不会生产这些物品\footnote{这一对可变分工体系的认识与广义价值论的假设有异曲同工之妙,可惜恩格斯并没有把可变分工体系推广到一般的商品经济。},而是因为得不到原料或者因为买到的物品要更好或更便宜。他们在生产那些用于交换的产品时所耗费的,只有他们自己的劳动;无论是为了补偿工具、为生产和加工原料所花费的,只有他们自己的劳动力。于是,恩格斯得出用耗费的劳动作为价值尺度结论:“因此,如果不按照花费在他们这些产品上的劳动的比例,他们又能怎样用这些产品同其他从事劳动的生产者的产品进行交换呢?在这里,花在这些产品上的劳动时间不仅对于互相交换的产品量的数量规定来说是唯一合适的尺度;在这里,也根本不可能有别的尺度。”\cite[1016]{ZhongGongZhongYangMaKeSiEnGeSiLieNingSiDaLinZhuZuoBianYiJuMaKeSiEnGeSiWenJiDi7Juan2009}但如果我们进一步深入思考,会发现恩格斯的这一结论是不能成立的。

正如恩格斯所言,农民之所以选择交换,不是因为自己不能生产,而是因为得不到原料或者因为买到的物品要好得多或便宜得多。当农民在用劳动作为尺度衡量是否要进行交换的时候,他衡量的是“如果不进行交换,为了得到同样的产品我要多耗费多少劳动?”以及“为了得到同样的产品,如果进行交换我能省下多少劳动?”于是,这些“多耗费的劳动”或者“省下的劳动”就是这个农民眼中,通过交换得到的商品的价值。这里笔者想提请读者注意,农民是通过自己的劳动衡量了别人生产的商品的价值。换句话说,假设有农民A、B,分别花费劳动$L_A$和$L_B$生产商品$C_A$和$C_B$,现在农民A想要通过交换得到商品$C_B$,于是农民A会衡量“多耗费的劳动”或者“省下的劳动”,记为$\Delta L_A$。此时,$\Delta L_A$所衡量的,正是农民B生产的商品$C_B$的价值。同样,农民B也是以同样的逻辑用$\Delta L_B$衡量了商品$C_A$的价值\footnote{当然,这里所指的各种劳动量都可以按照马克思所述的“抽象劳动”概念来理解。}。至此,笔者的分析应当说是忠于恩格斯的分析逻辑的。接下来,根据马克思劳动价值论的等价交换原则,要让上述交换能够长久地持续,农民A必须拿出在农民B看来有足够价值的商品与农民B进行交换;所以要让交换持续,$\Delta L_A = \Delta L_B$必须成立。再根据马克思劳动价值论所认为的“等量劳动带来等量价值”的价值决定原理,农民A必须付出与$\Delta L_B$相等的劳动$L_A$才能创造出$\Delta L_B$的价值,所以我们得到$\Delta L_B = L_A$。也就是说,商品$C_B$实际上能够支配$L_A$量的劳动;进而我们可以说,商品$C_B$的价值是由能够支配$L_A$量的劳动来衡量的;这等价于说商品所能支配的劳动是商品价值的尺度。同样地,我们可以得到$\Delta L_A = L_B$,所以最终有$L_B =\Delta L_A = \Delta L_B = L_A$。以上的分析表明,按照恩格斯分析的逻辑,在简单交换中商品所能支配的劳动是商品价值的尺度;并且,用商品能够支配的劳动作为价值的尺度与马克思的劳动价值论并没有发生任何结论上的冲突。

在成熟的商品经济中,商品能够支配的劳动仍然是价值的尺度。马克思在《资本论》第一卷的第三章中指出,货币的作用之一是价值尺度\cite[114-124]{ZhongGongZhongYangMaKeSiEnGeSiLieNingSiDaLinZhuZuoBianYiJuMaKeSiEnGeSiWenJiDi5Juan2009}。而一种贵金属,例如金,为什么可以成为货币?马克思指出:“金能够充当价值尺度,只是因为它本身是劳动产品,因而使潜在可变的价值。”\cite[118]{ZhongGongZhongYangMaKeSiEnGeSiLieNingSiDaLinZhuZuoBianYiJuMaKeSiEnGeSiWenJiDi5Juan2009}那么,如果我们用金来衡量某一件商品的价值,实际上是把由商品生产者耗费的劳动与金的生产者所耗费的抽象劳动进行了比较与折算。于是这件商品的价值不仅是被金衡量出来了,而且是本质上被金的生产者所耗费的抽象劳动衡量出来了。可见,在成熟的商品经济中,把商品能够支配的劳动作为商品价值的尺度也是符合马克思分析的逻辑和结论的。但笔者也注意到,马克思曾在《哲学贫困中》中批评道:“他把用商品中所包含的劳动量来衡量的商品价值和用“劳动价值”来衡量的商品价值混为一谈。如果把这两种衡量商品价值的方法搅在一起,那末也就同样可以说,任何一种商品的相对价值都是由它本身所包含的劳动量来衡量的;或者说,商品的相对价值是由它可以购买的劳动量来衡量的; 或者还可以说,商品的相对价值是由可以得到它的那种劳动量来衡量的。但是情况远不是这样。象任何其他的商品价值一样,劳动价值不能作为价值尺度。”\cite[97]{ZhongGongZhongYangMaKeSiEnGeSiLieNingSiDaLinZhuZuoBianYiJuMaKeSiEnGeSiQuanJiDi4Juan1958}对此,笔者认为马克思所批评的主要是“用‘劳动价值’作为价值尺度”的观点,而在马克思看来,商品所能支配的劳动就是指“劳动者的报酬”,而如果用劳动者的报酬作为价值尺度,那就相当于把生产费用作为价值尺度,会出现生产费用取决于价值而价值又用生产费用衡量的循环论证问题\cite[98]{ZhongGongZhongYangMaKeSiEnGeSiLieNingSiDaLinZhuZuoBianYiJuMaKeSiEnGeSiQuanJiDi4Juan1958}。\cite[5]{ZhangLeiShengMaKeSiLaoDongJieZhiLunYanJiuDeLiShiZhengTiXing2015}所以,笔者想要强调的是,笔者所指的支配劳动绝不是指劳动者的报酬,而是指一般的、平均的社会劳动。

事实上,如果从广义价值论的角度来看,耗费的劳动是具体的特殊的个别劳动,而支配的劳动是将不同的、具体的个别劳动折算得到的一般的、平均的社会劳动。

最后,笔者想要强调的是,将不同的、具体的个别劳动折算得到的一般的、平均的社会劳动的方式在不同的价值理论中是不一样的。例如马克思认为劳动本身具有二重性:“由于人隶属于机器或由于极端的分工,各种不同的劳动逐渐趋于一致”\cite[96]{ZhongGongZhongYangMaKeSiEnGeSiLieNingSiDaLinZhuZuoBianYiJuMaKeSiEnGeSiQuanJiDi4Juan1958};“在使用机器的企业中,这个工人的劳动和那个工人的劳动几乎没有什么差别;工人彼此间的区别,只是他们在劳动中所化的时间不等。”\cite[97]{ZhongGongZhongYangMaKeSiEnGeSiLieNingSiDaLinZhuZuoBianYiJuMaKeSiEnGeSiQuanJiDi4Juan1958}也就是说,商品耗费的劳动本身就具有抽象劳动的属性;而斯密等经济学家则认是市场上的讨价还价消除了劳动的异质性。目前,经济学界对此还没有一个统一的认识,所以笔者认为在这一点上仍然存在研究的空间\footnote{笔者认为不同折算方式的本质,是对“等价交换”概念的不同理解,可参考《论耗费的劳动与购买的劳动在价值理论中的作用》\cite[69]{CaiJiMingLunHaoFeiDeLaoDongYuGouMaiDeLaoDongZaiJieZhiLiLunZhongDeZuoYong2022}。}。不过,无论是哪种转换方式,都强调了价值的社会属性。因此从社会属性的角度出发,不难看出用支配的劳动比耗费的劳动作为价值尺度更能表现价值的社会属性。

\subsubsection{价值形成}

前文提到,古典政治经济学的基本倾向是把劳动作为价值尺度,但不可否认的是,所有的古典政治经济学家都把耗费的劳动作为一种重要的生产要素或是价值源泉置于一种特殊的地位。在笔者看来,耗费的劳动无论是被视为生产要素的一种还是唯一的价值源泉,都参与了商品价值的形成过程。按照广义价值论和劳动价值论的观点,我们现在把耗费的劳动视为唯一的价值源泉。

笔者认为,从广义价值论的角度来看,价值形成是指作为价值源泉的具体的、私人的劳动形成潜在的价值的过程,是在商品生产的过程中完成的。而价值的决定则是指价值源泉和价值尺度取得统一的过程,是具体的、私人的劳动转化为抽象的、社会的劳动的过程,是在商品交换的过程中完成的。马克思也认为,商品的价值由生产成本即劳动决定,是通过竞争的作用实现的,只有这样,商品的价值才能最终由生产该商品所耗费的劳动来决定\cite[3]{ZhangLeiShengMaKeSiLaoDongJieZhiLunYanJiuDeLiShiZhengTiXing2015}。 

笔者认为价值形成的结果是潜在的价值的第一个原因,是包括广义价值论在内的许多价值理论都认为价值决定是在商品交换时完成的。也就是说,在商品被生产出来后到交换完成之前的这段过程中,商品的价值既没有表现出来,也没有决定下来。一方面,商品价值的决定是一个社会的过程,而只有商品交换是在社会中完成的;在完成交换之前,商品耗费的劳动只是具体的、私人的劳动,只有在商品交换完成后,这些耗费的劳动才能转化为抽象的、社会的劳动,社会意义上的价值才能出现。因此,为了区分商品的生产和交换对价值的不同影响,笔者把价值形成划归到商品的生产过程,把价值决定划归到商品的交换过程。第二个原因是这种区分和规定具有普适性。如果我们考虑商品的使用价值,那么其形成正是在商品的生产阶段,其决定是在商品的消费阶段,其尺度是消费者的效用;显然,在商品被生产出来之后和被消费之前,其使用价值只是潜在的、没有决定也没有表现出来的,只有当消费者真正消费商品时,商品的使用价值才被效用衡量而决定下来。上述对使用价值形成和决定的描述也适用于对效用价值论的分析。

\subsubsection{价值决定——价值源泉和价值尺度的对立统一}

总的来说,耗费的劳动作为价值的源泉实现价值形成,支配的劳动作为价值尺度实现价值衡量。而价值的源泉和价值的尺度在商品交换的过程中实现了对立统一,商品的价值被决定下来。

首先,从微观的层面考察,价值源泉与价值尺度的对立统一并不体现为量上的完全等同,而在于两者可通过特定机制实现折算转换。广义价值论揭示了一个核心规律:部门间耗费劳动与支配劳动的比值等于其比较生产力的相对水平。具体表现为,当某部门比较生产力高于另一部门时,该部门能以较少耗费劳动置换对方较多劳动量,反之则需付出更多劳动;当两部门比较生产力之比等于1时,此时耗费劳动量恰与支配劳动量相等,这种特殊状态正对应着马克思劳动价值论中关于价值决定的基本论断。需要特别说明的是,广义价值论之所以强调比较生产力对价值决定的作用,在于其理论框架不仅考察劳动要素,还纳入非劳动生产要素对绝对生产力的实际影响,这类要素同样会作用于比较生产力的形成过程。

其次,从宏观总量的角度来说,由于耗费的劳动总量等于支配的劳动总量等于全社会价值总量\cite[71-72]{CaiJiMingLunHaoFeiDeLaoDongYuGouMaiDeLaoDongZaiJieZhiLiLunZhongDeZuoYong2022},所以作为价值源泉的耗费劳动和作为价值尺度的支配劳动是对立统一的。对于商品经济来说,任何支配的劳动都必然是要耗费的劳动——价值实体只能取决于耗费的劳动;任何耗费的劳动也都必然是要被支配的劳动——正如商品是用于交换的产品,自己耗费的劳动也是为了支配他人劳动而付出的劳动。在斯密的价值理论中,价值源泉和价值尺度是对立统一的。斯密认为,由每个国家全年总劳动耗费所得的所有商品一定可以被分解为工资、利润和地租三个部分。而这三个部分又可以用劳动作为价值尺度进行衡量从而被折算为所有商品能支配的劳动。因此,从一个国家特定时期的总量上看,支配的劳动等于耗费的劳动。\cite[41-48]{YaDang*SiMiGuoFuLun2015}在马克思的劳动价值论中,价值源泉和价值尺度也是对立统一的。马克思指出过两种不同含义的社会必要劳动\cite[29]{CaiJiMingBiYaoLaoDongIHeBiYaoLaoDongIIGongTongJueDingJieZhi1995}\cite[19]{GuShuTangDuiJieZhiJueDingHeJieZhiGuiLuDeZaiTanTao1982}:第一种含义的社会必要劳动时间是指“在现有的社会正常的生产条件下制造某种使用价值所需要的劳动时间”\cite[52]{ZhongGongZhongYangMaKeSiEnGeSiLieNingSiDaLinZhuZuoBianYiJuMaKeSiEnGeSiWenJiDi5Juan2009},第二种含义的社会必要劳动时间则是指“既然社会要满足需要,并为此目的而生产某种物品,它就必须为这种物品进行支付”,“所以,社会购买这些物品的方法,就是把它所能利用的劳动时间的一部分用来生产这些物品,也就是说,用该社会所能支配的劳动时间的一定量来购买这些物品”\cite[208]{ZhongGongZhongYangMaKeSiEnGeSiLieNingSiDaLinZhuZuoBianYiJuMaKeSiEnGeSiWenJiDi7Juan2009};或者说是“当时社会平均生产条件下生产市场上这种商品的社会必需总量所必要的劳动时间”\cite[722]{ZhongGongZhongYangMaKeSiEnGeSiLieNingSiDaLinZhuZuoBianYiJuMaKeSiEnGeSiWenJiDi7Juan2009}。如果我们把第一种含义的社会必要劳动看作在商品生产中总耗费的劳动,把第二种含义的社会必要劳动时间看作为满足社会对某种商品的需求而被商品所支配的总劳动。那么前者是耗费的劳动作为价值的源泉进行价值的形成,代表供给;后者是支配的劳动作为价值尺度进行价值的衡量,代表需求。当两者总量相等时,也就是供求平衡而均衡价格的以确定时,全社会所有商品的两种含义的社会必要劳动时间总量也必然是相等的。\cite[72]{CaiJiMingLunHaoFeiDeLaoDongYuGouMaiDeLaoDongZaiJieZhiLiLunZhongDeZuoYong2022}

在完成了对价值理论的总结后,笔者将在接下来的部分中对生产力概念进行梳理。

\section{生产力概念的演进}

分析前文可以发现,生产力概念的演进路程集中体现于分工体系的可变性特征,并取决于价值尺度与价值源泉的对立统一关系。其内在逻辑关系可如下图所示进行阐释。

\begin{figure}
    \includegraphics[width=\textwidth]{figures/The Concept of Productive forces.drawio.pdf}
\end{figure}
