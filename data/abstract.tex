% !TeX root = ../thuthesis-example.tex

% 中英文摘要和关键字

\begin{abstract}

新质生产力作为中国特色社会主义政治经济学的重要创新,其价值创造机制的理论阐释亟需突破传统劳动价值论的局限。本文以广义价值论为分析框架,首先介绍了广义价值论的基本原理,再系统梳理了古典政治经济学中亚当$\cdot$斯密、大卫$\cdot$李嘉图、卡尔$\cdot$马克思、马尔萨斯、萨伊和约翰$\cdot$穆勒及新古典经济学和斯拉法价值论中的价值理论及生产力概念,揭示了不同价值理论下生产力参与价值创造的机制;并对价值尺度、价值源泉及其对立统一的过程——价值决定这三个概念进行了批判性的总结。研究指出:对可变分工体系的认识是提出相对生产力概念的必要条件;对比较生产力的认识决定了对价值源泉和价值价值尺度之间关系的判断。研究还提出新质生产力的四维理论内涵:从绝对生产力的维度来看,新质生产力是技术驱动的劳动效率跃升;从相对生产力的维度来看,新质生产力是相对生产力重构所引发的比较优势转换;从比较生产力的的维度来看,新质生产力是价值创造导向的能力跃迁;从社会总和生产力的维度来看,新质生产力能够驱动社会总价值量的跃迁。本文为构建中国特色政治经济学提供了价值理论层面的创新路径,为发展新质生产力的政策实践奠定了微观机理基础。

  % 关键词用“英文逗号”分隔,输出时会自动处理为正确的分隔符
  \thusetup{
    keywords = {新质生产力,广义价值论,价值理论,生产力, 古典政治经济学, 新古典经济学},
  }
\end{abstract}

\begin{abstract*}

As a significant innovation in the political economy of socialism with Chinese characteristics, the theoretical interpretation of the value creation mechanism of new quality productive forces urgently needs to transcend the limitations of the traditional labor theory of value; this paper employs the Generalized Theory of Value as its analytical framework, first introducing the basic principles of generalized value theory, then systematically reviewing value theories and productive forces concepts in classical political economy—including Adam Smith, David Ricardo, Karl Marx, Malthus, Say, and John Stuart Mill—as well as in neoclassical economics and Sraffian value theory, revealing how productive forces participate in value creation under different value theories, and critically summarizing value measures, sources of value, and their dialectically unified process—value determination. The study indicates that understanding variable division of labor systems is necessary for proposing the concept of relative productive forces, while comprehending comparative productive forces determines judgments about the relationship between value sources and value measures; it further proposes a four-dimensional theoretical connotation of new quality productive forces: from the dimension of absolute productive forces, it represents a technology-driven leap in labor efficiency; from the dimension of relative productive forces, it signifies a transformation of comparative advantages caused by the restructuring of relative productive forces; from the dimension of comparative productive forces, it constitutes a value creation-oriented capability leap; from the dimension of aggregate social productive forces, it drives a leap in total social value creation. This paper provides an innovative pathway for value theory in constructing the political economy of socialism with Chinese characteristics and establishes micro-mechanistic foundations for policy practices in developing new quality productive forces.

% Use comma as separator when inputting

\thusetup{
  keywords* = {New-quality Productive Forces, Generalized Value Theory, Value Theory, Productive Forces, Classical Political Economy, Neoclassical Economics},
}
\end{abstract*}
