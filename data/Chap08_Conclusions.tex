% !TeX root = ../2019080346_Mason.tex

\chapter{结论}

\section{研究成果}

总的来说,笔者在本文中取得了以下三个成果。

第一,重新构建了广义价值论的分析框架,用更严谨的方式重新证明了一些结论,并将广义价值论中各种生产力创造价值的机制系统地总结为一张表格。

第二,系统梳理了古典政治经济学、新古典经济学中的价值论和斯拉法价值论,纠正了许多经济学者对亚当·斯密、马克思、马歇尔等人的误解;回顾了价值概念的提出过程,详细地分析了价值形成、价值决定和价值尺度这三个概念的区别和联系。同时,笔者还在广义价值论的基础上分析了各种价值理论中所出现的生产力概念,并总结了提出各种生产力概念的必备条件。

第三,从绝对生产力、相对生产力、比较生产力和社会总和生产力对应的四个维度解析了新质生产力的概念,阐明了新质生产力和价值创造之间的联系。

\section{局限性与未来研究展望}

本文主要在以下几个方面存在不足。

第一,受限于时间和篇幅,广义价值论的部分结论没有被详细地证明。第二,在回顾价值理论和梳理生产力概念的时候不够深入,引用的外文文献不足,部分经济学家的著作引用不全。第三,对新质生产力的分析浅尝辄止,没有搭建起系统的分析框架。

基于以上不足,笔者认为未来研究的重点是建立对新质生产力的系统性分析框架,可以剖析现有的四种维度之间的相互关系,也可以尝试针对每个维度建立计量指标以推动实证研究,还可以针对每个维度给出发展新质生产力的具体建议。