% !TeX root = ../2019080346_Mason.tex

\chapter{新古典经济学中的生产力概念与价值理论}

\section{新古典经济学的界定}

"新古典"这一术语最早由T. 凡勃伦(T. Veblen)用于指称经济学家马歇尔及其经济学理论\cite[9382]{macmillanpublishersltdNewPalgraveDictionary2018}。当今经济学界通常将19世纪70年代至20世纪30年代的欧洲主流经济思想统称为新古典经济学\cite[241]{YanZhiJieXiFangJingJiXueShuoShiJiaoChengDiErBan2013}。在思想特征上,新古典经济学采用边际分析与均衡分析方法,并侧重于需求侧分析。从价值理论上看,新古典经济学包含了采用边际分析方法建立的边际价值论和马歇尔综合而成的新古典价值论\cite[181-184]{CaiJiMingCongGuDianZhengZhiJingJiXueDaoZhongGuoTeSeSheHuiZhuYiZhengZhiJingJiXueJiYuZhongGuoShiJiaoDeZhengZhiJingJiXueYanBianShangCe2023}。

在本章中,笔者将依次介绍这两个价值理论及其生产力概念。

\section{边际价值论及其生产力概念}

一般认为,边际价值论包含了边际效用价值论和边际生产力价值论\cite[181]{CaiJiMingCongGuDianZhengZhiJingJiXueDaoZhongGuoTeSeSheHuiZhuYiZhengZhiJingJiXueJiYuZhongGuoShiJiaoDeZhengZhiJingJiXueYanBianShangCe2023}。

\subsection{边际效用价值论}

19世纪70年代初期,英国的杰文斯、奥地利的门格尔和法国的瓦尔拉斯几乎同时而又各自独立地发现了边际效用递减原理,并以此为基础建立了边际效用价值论\cite[iii]{r.d.c.BuLaiKeJingJiXueDeBianJiGeMingShuoMingHePingJie2020}\cite[242]{YanZhiJieXiFangJingJiXueShuoShiJiaoChengDiErBan2013}。

杰文斯认为经济学应以消费理论为起点,因为消费是人类生活的目的和经济的动力,而生产仅是满足消费的手段。由于消费旨在追求快乐并避免痛苦,杰文斯用“引起快乐”和“避免痛苦”来定义效用,进而提出经济学的作用在于最大化人们从物品中获得的效用。他进一步指出,物品的效用遵循“最后效用程度递减”法则(即边际效用递减法则)。此外,杰文斯认为商品价值应由其边际效用强度决定,且商品间的交换价值取决于两者边际效用强度之比的倒数。最后,杰文斯还试图把劳动引入商品交换的过程,认为商品交换也会受到边际“生产成本”的制约。但是,杰文斯所说的生产成本并非指实际劳动耗费量及其产品,而是指人对劳动的心理感受,所以杰文斯实际上并没有提出边际效用价值论以外的价值论。\cite[125-136,142-149]{YanZhiJieCongBianJiGeMingDaoKaiEnSiGeMing2022}\cite[52-131]{SiTanLi*JieWenSiZhengZhiJingJiXueLiLun1984}

门格尔认为财货的价值性质源于其稀缺性及对人生命与福利的意义;价值则是人对这种意义的主观判断,其大小取决于财货支配量与人的愿望的对比关系。为探究价值的数量与尺度,门格尔指出,不同欲望种类及同一欲望的不同强度(随满足而递减)均影响价值量,且价值量呈递减趋势——欲望种类越次要、强度越弱,物品价值量越低。然而,由于商品可能满足不同种类或强度的欲望,门格尔提出以"最小欲望满足"作为统一尺度:当失去一定量物品时,人们必然放弃当下意义最小的欲望,因此该部分物品的价值由最小意义欲望决定。\cite[167-172]{YanZhiJieCongBianJiGeMingDaoKaiEnSiGeMing2022}\cite[52-80]{QiaEr*MenGeErGuoMinJingJiXueYuanLi2024}

瓦尔拉斯则把价格和价值混为一谈。首先,他认为物品的稀少性原理使得市场交换成为一个普遍现象,进而市场价格也成为一个普遍现象。而价格又取决于市场供给和需求的均衡——供求双方就商品的边际效用进行讨价还价。瓦尔拉斯承认供给和需求都是价值决定的因素,并在对供求关系的研究中引申出了边际效用的概念。总的来说,瓦尔拉斯的分析更强调需求和供给在价格形成中的相互制约以及各种商品价格决定之间的相互依赖,并最终建立了一般均衡理论。\cite[181-188]{YanZhiJieCongBianJiGeMingDaoKaiEnSiGeMing2022}\cite[139-147]{LaiAng*WaErLaSiChunCuiJingJiXueYaoYi1989}

\subsection{生产力与价值的关系}

如前所述,边际效用价值论只关注了需求侧对价值决定的作用,而几乎完全忽视了供给侧的作用。因此,边际效用价值论中的价值不仅难以用一般的的尺度来度量,而且与生产力失去了联系。事实上,杰文斯、门格尔和瓦尔拉斯几乎都默认商品已经被生产出来了,而不关注商品是如何被生产出来的。所以,在边际效用价值论中我们基本看不到与生产力有关的概念,更不用说对生产力和价值关系的分析了。

\subsection{边际生产力价值论}

边际生产力价值论的发展历史较边际效用论更长一些。马尔萨斯和李嘉图的地租理论中首次出现了边际生产力的概念——资本和劳动的边际收益会等于土地的边际收益。而后,朗菲尔德(Longfield)提出利润是由物质资本的边际生产力决定的;屠能(von Thünen)在同一时期把边际生产力运用到了对工资和资本的分析上,但耽误了很久才发表文章,没有产生什么影响。杰文斯也用边际生产力来解释利率,但是把工资解释为支付租金和利息后的剩余部分。总而言之,上述的这些分析都没能将边际原理推广到所有的生产要素上。最后是美国经济学家克拉克(J.B. Clark)和马歇尔在1890年左右各自独立地完成了推广的工作。\cite[8222-8224]{macmillanpublishersltdNewPalgraveDictionary2018}

克拉克在边际效用价值论的基础上作了一些修正。首先,他认为价值是一个社会性的概念,尽管商品的效用是个人的心理现象,但决定价值的应当是商品的“实际社会效用”(也就是对社会的边际效用)。接着,克拉克要从数量上衡量价值;克拉克首先考察了社会劳动或资本为获得某种享受所愿意支付的代价,他认为这种代价就是劳动或资本的边际产品,前者是社会劳动的反效用,表现为工资;后者是社会资本的忍欲,表现为利息(率\footnote{引文的原文中没有“率”字,但笔者认为资本的边际生产力应当与资本的单位成本而不是总成本有关。});两者都服从边际产品递减的规律。\cite[309-316]{YanZhiJieCongBianJiGeMingDaoKaiEnSiGeMing2022}这样,克拉克实际上得到了生产要素——劳动和资本的成本,进而就可以根据每个生产者生产一定商品所消耗的生产要素数量,确定这些商品的价值\cite[311]{KeLaKeCaiFuDeFenPei1983}。那么,商品的边际效用决定的价值如何同生产要素成本决定的价值统一起来呢?克拉克指出,“实际效用是一切财富所共有的要素,这个要素可以使用社会的反效用来衡量。许多种类的享受,都可以使用劳动所带来的损失来衡量。”\cite[xv]{KeLaKeCaiFuDeFenPei1983}这就从单位上把商品的效用和生产要素的成本(工资、利息率)统一起来了;再借助的市场交换,就可以在数量上把决定生产成本的负效用和来自享受消费的正效用统一起来\cite[331-333]{KeLaKeCaiFuDeFenPei1983}。

不难看出,克拉克所谓的边际生产力价值论实际上是利用边际生产力的概念把边际效用价值论、生产费用价值论\footnote{晏智杰\cite[312]{YanZhiJieCongBianJiGeMingDaoKaiEnSiGeMing2022}认为克拉克的边际生产力价值论还综合了供求价值论,但笔者认为克拉克的理论中没有出现供求价值论。}综合到了一起。

\subsection{生产力与价值的关系}

从上述边际生产力价值论的含义上看,“边际生产力”一词中的“生产力”毫无疑问指的是绝对生产力。克拉克也对这一“生产力”的变动作出了分析。一方面,克拉克指出如果某一个行业出现某种发明使得该行业生产的商品的生产成本降低,“那么,这个团体所生产的物品的标准价值,开始时一定突然下跌,然后就稳定一段时间,再后由于第二个发明的出现又重新下跌”\cite[361]{KeLaKeCaiFuDeFenPei1983}。这实际上符合广义价值论关于绝对生产力与单位商品价值负相关的判断。另一方面,“生产力”的增加会使得“工资增高,利息总额增大”\cite[241]{KeLaKeCaiFuDeFenPei1983},进而还会给企业家带来暂时的利润\cite[352]{KeLaKeCaiFuDeFenPei1983}。这实际上不仅符合广义价值论关于单位个别劳动创造的价值量与其绝对生产力正相关的判断。

然而,克拉克又指出,“一个团体内劳动和资本所特有的生产力,是由两个力量决定的。一个是产品的价格,这要看这种产品的总产量是多少。另一个是产品中由一个单位的劳动(或一个单位的资本)所生产的部分,这要看这个团体里劳动和资本在数量上的对比是怎样的。”\cite[252]{KeLaKeCaiFuDeFenPei1983}例如,如果工人是在资本非常充裕的工厂里工作,那么可以归功于一个单位劳动的产量就很大;而且由于此时产品的总量还不大,单位产品的价格就很高。所以,在这种情况下劳动的生产力就很高。现在,如果许多工人涌入这家工厂,那么由于资本的数量没有变化和边际生产力递减的规律,单个工人的产量就会下降;而且由于产品的总量上升,单位产品的价格就会下降。所以,劳动的边际生产力就会下降。这样,如果劳动和资本可以自由流动的话,各个行业的劳动和资本的边际\footnote{原文中没有“边际”一词,但笔者认为这里的意思就是边际生产力。}生产力将会趋于一致\cite[254]{KeLaKeCaiFuDeFenPei1983}。

这里笔者注意到:第一,某个部门的资本和劳动的边际生产力大小对资本和劳动流动的影响是相对于其它部门或是整个社会来说的,单独说某个部门的资本和劳动的边际生产力大小没有意义;第二,决定资本和劳动力成本(工资和利息率)的是整个社会的资本和劳动力边际生产力,所以单个部门如果有比较高的资本和劳动力生产力,就可以获得超额利润\cite[255]{KeLaKeCaiFuDeFenPei1983}。所以笔者认为,这里的“边际生产力”对应着广义价值论中的比较生产力概念,只不过广义价值论中的比较生产力比较的是某一部门在一种商品上的绝对生产力与另一部门在另一种商品上的绝对生产力,而边际生产力价值论中劳动和和资本的“边际生产力”比较的是某一部门的某一要素在一种商品上的(边际)绝对生产力与另一部门的同种要素在另一种商品上的(边际)绝对生产力。类似地,单个部门如果有比较高的资本和劳动力生产力,就可以获得超额利润的命题对应着广义价值论中部门劳动创造的价值总量与部门比较生产力正相关的判断。然而,由于这里没有量化的分析,所以边际生产力价值论中没有与广义价值论中比较生产力与单位商品价值量正相关的判断对应的命题\footnote{而且引入单位商品的价值会出现循环论证。}。

\section{新古典价值论及其生产力概念}

一般认为,马歇尔在19世纪末将边际效用价值论、边际生产力价值论、供求论和生产成本论综合起来,创建了新古典经济学\cite[295]{YanZhiJieXiFangJingJiXueShuoShiJiaoChengDiErBan2013}\cite[340]{YanZhiJieCongBianJiGeMingDaoKaiEnSiGeMing2022}\cite[i]{MaXieErJingJiXueYuanLi2019}\cite[183-184]{CaiJiMingCongGuDianZhengZhiJingJiXueDaoZhongGuoTeSeSheHuiZhuYiZhengZhiJingJiXueJiYuZhongGuoShiJiaoDeZhengZhiJingJiXueYanBianShangCe2023}。马歇尔的价值理论强调供给和需求的均衡,因此也被称为供求均衡价值论\cite[390]{YanZhiJieCongBianJiGeMingDaoKaiEnSiGeMing2022}。接下来笔者将对此进行介绍。

首先,马歇尔认为商品的价值就是交换价值,由货币表示的价格又代表着商品的一般交换价值\cite[86-87]{MaXieErJingJiXueYuanLi2019}。这里笔者想要指出:有很多学者批评马歇尔的理论中没有价值的概念\cite[299]{YanZhiJieXiFangJingJiXueShuoShiJiaoChengDiErBan2013}\cite[v]{MaXieErJingJiXueYuanLi2019},但笔者认为,马歇尔研究的对象就是价格运动的规律,按照价值的一般定义,均衡价值论也应当是一种价值理论而非“价格理论”,只是马歇尔对价值的认识可能存在错误罢了。

马歇尔认为价值取决于供求力量的均衡,两者缺一不可。在此基础上,马歇尔又指出了不同时间条件下价值决定的区别:在极短时间(如一天)内出现的暂时均衡价格取决于需求,因为供给在极短时间内无法变化;在短时间(通常是一年)内的短期均衡价格取决于供求双方对等的相互作用,因为双方在短期内均可作出调整;在长期内出现的长期市场均衡价格取决于供给,因为供给在长期内可以有很大的增长\cite[299-300]{YanZhiJieXiFangJingJiXueShuoShiJiaoChengDiErBan2013}\cite[390]{YanZhiJieCongBianJiGeMingDaoKaiEnSiGeMing2022}。同时,马歇尔所指的均衡是局部均衡,即在研究某种商品的价值决定时假定其它商品的价值不变卖不考虑其他商品供求关系的变动对该商品价值的影响\cite[391]{YanZhiJieCongBianJiGeMingDaoKaiEnSiGeMing2022}。

 在需求方面,马歇尔继承了边际效用价值论的观点,认为需求价格决定于边际效用递减律;在供给方面,马歇尔实际上单独提出了边际生产力的概念\cite[8223]{macmillanpublishersltdNewPalgraveDictionary2018},认为供给价格决定于生产费用,而生产费用又决定于劳动和资本,前者是边际负产品,后者是等待或边际负效用\cite[391]{YanZhiJieCongBianJiGeMingDaoKaiEnSiGeMing2022}。与边际效用或生产力价值论不同的是,马歇尔既不认为边际效用单独可以决定价值,也不认为由边际生产力决定要素价格单独可以决定价值,马歇尔始终强调两者的相互作用。

 \subsection{生产力与价值的关系}

 由于新古典价值论强调均衡分析,因此我们很难单从供给入手分析生产力与价值的关系。事实上,新古典价值论本身存在循环论证的问题:“新古典价值论在讨论产品市场均衡价格时预先假定要素价格已经存在,由此才能导出由成本曲线构成的供给曲线;而在讨论要素市场时,又假定产品价格已经存在,由此才能形成由要素边际收益形成的要素需求曲线。”\cite[186]{CaiJiMingCongGuDianZhengZhiJingJiXueDaoZhongGuoTeSeSheHuiZhuYiZhengZhiJingJiXueJiYuZhongGuoShiJiaoDeZhengZhiJingJiXueYanBianShangCe2023}因此,如果我们要分析供给端的生产力对价值的影响,那么就要假定产品价格已经存在;那既然产品价格已经存在,产品的价值已经确定了,那就无法分析价值的“变化”。这个问题不光存在于新古典价值论,认同供求价值论的马尔萨斯的论证也出现了相同的问题。