% !TeX root = ../2019080346_Mason.tex

\chapter{古典政治经济学中的生产力概念与价值理论}

在完成了对广义价值论的介绍后,笔者将对古典政治经济学中的生产力概念和价值理论进行梳理。

\section{古典政治经济学的界定}

在开始梳理之前,我们首先对古典政治经济学的范围做出定义。

经济学界对古典政治经济学有很多种界定,马克思是最早使用“古典”一词来泛指一段时期的政治经济学理论的\cite[7]{YueHan*MeiNaDe*KaiEnSiJiuYeLiXiHeHuoBiTongLunChongYiBen2021}。他认为“古典政治经济学在英国从威廉·配第开始,到李嘉图结束,在法国从阿吉尔贝尔开始,到西斯蒙第结束”\cite[41]{ZhongGongZhongYangMaKeSiEnGeSiLieNingSiDaLinZhuZuoBianYiJuMaKeSiEnGeSiQuanJiDi13Juan1962}。除了马克思以外,还有许多经济学家也对古典政治经济学作了口径不一的划分\cite[5-8]{CaiJiMingCongGuDianZhengZhiJingJiXueDaoZhongGuoTeSeSheHuiZhuYiZhengZhiJingJiXueJiYuZhongGuoShiJiaoDeZhengZhiJingJiXueYanBianShangCe2023}。正如马克思所说:“把交换价值归结于劳动时间或相同的社会劳动,是古典政治经济学一个半世纪以上的研究得出的批判性的最后成果。”\cite[41]{ZhongGongZhongYangMaKeSiEnGeSiLieNingSiDaLinZhuZuoBianYiJuMaKeSiEnGeSiQuanJiDi13Juan1962}笔者和学界主流的观点\cite[45]{ChenDaiSunCongGuDianJingJiXuePaiDaoMaKeSiRuoGanZhuYaoXueShuoFaZhanLueLun2014}\cite[12]{CaiJiMingCongGuDianZhengZhiJingJiXueDaoZhongGuoTeSeSheHuiZhuYiZhengZhiJingJiXueJiYuZhongGuoShiJiaoDeZhengZhiJingJiXueYanBianShangCe2023}都认为古典政治经济学的基本倾向是劳动价值论。鉴于马克思本人也赞同劳动价值论,
故笔者在本文中将马克思也一并列入古典政治经济学的范畴。在参考了大量的文献后,笔者进一步从古典政治经济学家中选择亚当·斯密、大卫·李嘉图、詹姆斯·穆勒、马尔萨斯、萨伊、约翰·斯图亚特·穆勒、马克思、斯拉法作为古典政治经济学的代表,依次展开分析。

\section{亚当·斯密}
 
亚当·斯密在1776年发表的《国民财富的性质和原因的研究》中最早建立了古典政治经济学的体系\cite[120]{CaiJiMingCongGuDianZhengZhiJingJiXueDaoZhongGuoTeSeSheHuiZhuYiZhengZhiJingJiXueJiYuZhongGuoShiJiaoDeZhengZhiJingJiXueYanBianShangCe2023}\cite[90]{YanZhiJieXiFangJingJiXueShuoShiJiaoChengDiErBan2013}。接下来,笔者将首先介绍亚当·斯密笔下的价值理论,然后梳理其生产力概念。

\subsection{亚当·斯密的价值理论}

斯密是第一个明确地区分使用价值和交换价值的经济学家\cite[122]{CaiJiMingCongGuDianZhengZhiJingJiXueDaoZhongGuoTeSeSheHuiZhuYiZhengZhiJingJiXueJiYuZhongGuoShiJiaoDeZhengZhiJingJiXueYanBianShangCe2023}。他说:“价值一词有两个不同的含义。它有时表示特定物品的效用,有时又表示由于占有某物而取得的对他种货物的购买力。前者可叫做使用价值,后者可叫做交换价值。”\cite[24]{YaDang*SiMiGuoFuLun2015}应当说,这种区分是自然而严谨的,直到现在经济学界仍然在沿用这一区分。

接下来,斯密提出了自己在《国富论》中"为要探讨支配商品交换价值原则"\cite[24]{YaDang*SiMiGuoFuLun2015}而必须回答的三个问题,这三个问题中的概念构成了斯密价值理论的体系,笔者对斯密价值理论的介绍也将从这些概念入手。这三个问题分别是:

1.“什么是交换价值的真实尺度,换言之,构成一切商品真实价格的,究竟是什么?”\cite[24]{YaDang*SiMiGuoFuLun2015}

2.“构成真实价格的各部分,究竟是什么?”\cite[24]{YaDang*SiMiGuoFuLun2015}

3.“什么情况使上述价格的某些部分或全部,有时高于其自然价格或普通价格,有时又低于其自然价格或普通价格?”\cite[24]{YaDang*SiMiGuoFuLun2015}

我们可以看到,斯密要研究的,是“交换价值”“真实价格”“自然价格(普通价格)”这几个概念,我们可以据此把这三个问题转化为更清晰的表达:

1.“什么是交换价值的尺度?”

2.“真实价格是由什么构成的?”

3.“市场价格围绕自然价格波动的原因是什么?”

下面,我们依次对这三个问题进行回答和分析。

\subsubsection{交换价值的尺度}

首先,斯密认为“劳动是衡量一切商品交换价值的真实尺度”\Cite[25]{YaDang*SiMiGuoFuLun2015}。具体而言,“自分工完全确立以来,各人所需要的物品,仅有极小部分仰给于自己劳动,最大部分却须仰给于他人劳动。所以,他是贫是富,要看他能够支配多少劳动”\cite[25]{YaDang*SiMiGuoFuLun2015}“一个人占有某货物,但不愿自己消费,而愿用以交换他物,对他来说,这货物的价值,等于使他能购买或能支配的劳动量。”\cite[25]{YaDang*SiMiGuoFuLun2015}斯密又解释了选用劳动作为尺度的原因,即“只有用劳动作标准,才能在一切时代和一切地方比较各种商品的价值”\cite[31]{YaDang*SiMiGuoFuLun2015}。这里,笔者想强调斯密的意思是:交换价值的尺度不仅是劳动,而且是商品可以购买或支配的劳动。

但斯密又说:“劳动虽是一切商品交换价值的真实尺度,但一切商品的价值,通常不是按劳动估定的。”\cite[26]{YaDang*SiMiGuoFuLun2015}这是因为不同的劳动存在异质性,“它们的不同困难程度和精巧程度,也必须加以考虑”\cite[26]{YaDang*SiMiGuoFuLun2015}。进而,斯密认为通过市场的议价行为可以“消除”劳动的异质性,让不同质的劳动可以通过商品交换的方式实现相互交换,于是,劳动交换这一抽象的概念取得了商品交换这一具体的形式\cite[26]{YaDang*SiMiGuoFuLun2015}。随着货币的出现,物物交换演进为商品与货币的交换,商品取得了货币价格的形式。这种以货币价格为尺度的价格被斯密称作是“名义价格”,而商品的真实价格还是以购买的劳动为尺度的\cite[28]{YaDang*SiMiGuoFuLun2015}。

根据以上内容,笔者认为,斯密笔下的“交换价值”与广义价值论中的交换价值相同,即不同商品之间的交换比例。而当货币出现并被广泛使用之后,交换价值又转换为名义价格。这里的名义价格,应当就是我们可以观察到的市场价格\cite[293]{YueSeFu*XiongBiTeJingJiFenXiShiDi1Juan2017}。

\subsection{真实价格的构成}

但是,斯密没有给出“价值”的明确定义,只是说“同一真实价格的价值,往往相等”\cite[28]{YaDang*SiMiGuoFuLun2015}。有学者认为,斯密并没有从交换价值中抽象出价值的概念\cite[71]{ChenDaiSunCongGuDianJingJiXuePaiDaoMaKeSiRuoGanZhuYaoXueShuoFaZhanLueLun2014},笔者也认同这一观点。在此基础上,笔者进一步认为斯密的意思是“交换价值”在货币出现后转换为“名义价格”,而“真实价格”是“名义价格”围绕波动的中心\cite[52]{YaDang*SiMiGuoFuLun2015}。因此,斯密笔下的“真实价格”才符合广义价值论中对价值的定义。

一般认为,斯密在第一篇第六章中提出了两种价值论\cite[97]{YanZhiJieXiFangJingJiXueShuoShiJiaoChengDiErBan2013}\cite[126]{CaiJiMingCongGuDianZhengZhiJingJiXueDaoZhongGuoTeSeSheHuiZhuYiZhengZhiJingJiXueJiYuZhongGuoShiJiaoDeZhengZhiJingJiXueYanBianShangCe2023}。首先,斯密认为“在资本累积和土地私有尚未发生以前的初期野蛮社会,获取各种物品所需要的劳动量之间的比例,似乎是各种物品相互交换的唯一标准。”\cite[41]{YaDang*SiMiGuoFuLun2015}于是我们可以说,单要素的劳动价值论是亚当斯密提出的第一种价值论。

而当“资本一经在个别人手中积聚起来”,则“劳动者对原材料增加的价值”“就分为两个部分,其中一部分支付劳动者的工资,另一部分支付雇主的利润”\cite[42]{YaDang*SiMiGuoFuLun2015}。当然,“这三个组成部分各自的真实价值,由各自所能购买或所能支配的劳动量来衡量”\cite[43-44]{YaDang*SiMiGuoFuLun2015}。由于每一期的劳动可以被下一年再次投入生产,所以“社会全部劳动年产物所能购买或支配的劳动量,远远超过这年产物生产制造乃至运输所需要的劳动量”\cite[48]{YaDang*SiMiGuoFuLun2015}。于是我们又可以说,多要素的劳动价值论是亚当斯密提出的第二种价值论,也就是说耗费的劳动量不能再单独决定能够买的劳动量了,由于其它要素的参与,单位消耗的劳动量可以购买更多的劳动量\cite[138]{CaiJiMingCongGuDianDaoXianDaiZhengZhiJingJiXueGaiNianDeYanBianJianPingXinZhengZhiJingJiXueDeFaZhan2012}。这种不等关系对应着广义价值论中两个生产者的单位耗费劳动因综合生产力的相对大小不同而能购买不等量劳动的结论\cite[294]{CaiJiMingCongGuDianZhengZhiJingJiXueDaoZhongGuoTeSeSheHuiZhuYiZhengZhiJingJiXueJiYuZhongGuoShiJiaoDeZhengZhiJingJiXueYanBianShangCe2023}。

\subsection{对以上内容的两方面争议}

在此基础上,有许多经济学家对斯密的观点提出了批评,这些批评主要集中在两个方面:价值决定和价值尺度。

从价值决定的角度看,许多经济学家认为斯密的经济思想中既有劳动价值论的观点又有生产费用论的成分,批评斯密的价值理论是自相矛盾的\cite[136]{CaiJiMingCongGuDianZhengZhiJingJiXueDaoZhongGuoTeSeSheHuiZhuYiZhengZhiJingJiXueJiYuZhongGuoShiJiaoDeZhengZhiJingJiXueYanBianShangCe2023}\cite[294]{YueSeFu*XiongBiTeJingJiFenXiShiDi1Juan2017}。但如前文所述,笔者更支持这样一种观点:斯密的价值理论是一元多要素价值论\cite[136]{CaiJiMingCongGuDianZhengZhiJingJiXueDaoZhongGuoTeSeSheHuiZhuYiZhengZhiJingJiXueJiYuZhongGuoShiJiaoDeZhengZhiJingJiXueYanBianShangCe2023},即当只有劳动一种要素稀缺时价值由劳动单一决定,是单要素劳动价值论;而当资本和土地也成为稀缺要素时劳动的价值决定会受到其它要素的影响,是多要素劳动价值论。

从价值尺度的角度看,李嘉图批评斯密同时提出了耗费劳动尺度说和购买劳动尺度说\cite[7]{DaWei*LiJiaTuZhengZhiJingJiXueJiFuShuiYuanLi2021}。但如前文所述,斯密始终是把能购买的劳动作为尺度的\cite[142]{CaiJiMingCongGuDianZhengZhiJingJiXueDaoZhongGuoTeSeSheHuiZhuYiZhengZhiJingJiXueJiYuZhongGuoShiJiaoDeZhengZhiJingJiXueYanBianShangCe2023}。斯密是把价值尺度和价值决定分开考察的\cite[73]{ChenDaiSunCongGuDianJingJiXuePaiDaoMaKeSiRuoGanZhuYaoXueShuoFaZhanLueLun2014},他首先提出了衡量“交换价值”的外在尺度是可以购买的劳动,接着再进一步提出在土地尚未私有、资本尚未积累时,决定“真实价格”的是商品生产耗费的劳动。这时,“真实价格”的尺度和耗费的劳动是统一的。而在土地私有、资本累积的发达社会,包括直接耗费的劳动在内的多种要素共同决定了“真实价格”。这时,“真实价格”的尺度和耗费的劳动便不再统一,但参与商品生产的总劳动量(包括直接和间接)仍然等于商品能购买的总劳动量。

另外,

接下来,我们来分析斯密的第三个问题。

\subsection{市场价格围绕自然价格波动的原因}

首先,我们有必要对“自然价格”的概念进行阐释。按照斯密的观点,“生产、制造这商品乃至运送这商品到市场所使用的按自然率支付的地租、工资和利润“构成了商品的自然价格,自然价格又“恰恰相当于其价值”\cite[49]{YaDang*SiMiGuoFuLun2015}。这里,斯密略去了对价值向生产价格的转化问题的分析\footnote{即由于竞争使剩余价值在各生产部门资本家之间按资本量平均分配,剩余价值转化为平均利润,价值转化生产价格的过程\cite{XieFuShengXiFangXueZheGuanYuMaKeSiJieZhiZhuanXingLiLunYanJiuShuPing2000}。}
,而直接得出了类似马克思笔下生产价格的“自然价格”概念。

斯密指出,商品的市场价格会受到供给和有效需求的比例支配围绕着自然价格波动,其中,有效需求指的是愿意支付商品自然价格的人的需求\cite[50]{YaDang*SiMiGuoFuLun2015}。于是,长期的商品价格会在自然价格处达到均衡\cite[50]{YaDang*SiMiGuoFuLun2015}。另外,斯密还指出商品的市场价格会受到货币本身价值变动的影响\cite[28-31]{YaDang*SiMiGuoFuLun2015}。

至此,我们已经回答了斯密提出的三个问题,基本梳理了斯密的价值理论体系如下图所示。

接下来,笔者将梳理斯密笔下的生产力概念。

\begin{figure}
    \centering
    \caption{亚当·斯密的价值理论}
    \label{figures:AdamSmith_Value_Theory}
    \includegraphics[width=\textwidth]{figures/AdamSmith_Value_Theory.pdf}
\end{figure}

\subsection{亚当·斯密的生产力概念}

在《国富论》的开篇,斯密指出:“劳动生产力最大的增进,以及运用劳动时所表现的更大的熟练、技巧和判断力,似乎都是分工的结果。”\cite[3]{YaDang*SiMiGuoFuLun2015}这是因为通过分工——“第一,劳动者的技巧因业专而日进;第二,由一种工作转到另一种工作,通常须损失不少时间,有了分工,就可以免除这种损失;第三,许多简化劳动和缩减劳动的机械的发明,使一个人能够做许多人的工作”\cite[6]{YaDang*SiMiGuoFuLun2015}。这里,我们还可以看到斯密笔下的分工不仅是简单的通过劳动力排列组合的方面,而且包括了劳动技能提升和要素使用的方面。

如此来看,斯密笔下的“劳动生产力”对应的是广义价值论中的绝对生产力概念。纵观《国富论》全文,我们也找不到其它的生产力概念。

\subsection{生产力与价值的关系}

按照广义价值论的分析,绝对生产力与单位商品的价值是负相关的关系。虽然斯密认为资本和土地要素的投入会影响商品的价值量,但这些投入都可以被视为是物化劳动的投入,所以绝对生产力与单位商品的价值仍然是负相关的关系。同样,我们也不难推出绝对生产力和单位劳动的价值量正相关的结论。

总而言之,我们可以把亚当·斯密理论体系中的生产力与价值决定机制总结为下表:

\begin{table}
    \caption{亚当·斯密的生产力与价值决定机制}
    \label{table:AdamSmith}
     \begin{tabularx}{\textwidth}{|c<{\centering}|c<{\centering}|X<{\centering}|}
        \toprule
        生产力类型    &具体变量    &价值决定机制 \\ \midrule

        绝对生产力    &“劳动生产力”    &与单位劳动价值正相关 \\ \bottomrule
     \end{tabularx}
\end{table}

\section{大卫·李嘉图}

大卫·李嘉图是英国产业革命时代的经济学家,继承和发展了斯密价值理论中的单一劳动价值论的部分,对价值决定于劳动时间的原理作了比较透彻的表述与发展\cite[iv]{DaWei*LiJiaTuZhengZhiJingJiXueJiFuShuiYuanLi2021}。在这一部分,笔者将探讨李嘉图理论体系中的生产力概念和价值理论。

\subsection{大卫·李嘉图的价值理论}

李嘉图在《政治经济学及赋税原理》第一章中对他的价值理论做了比斯密更为严谨和系统的介绍,但他同样没有区分开交换价值和价值。

李嘉图首先继承了斯密对使用价值和价值的区分,然后指出交换价值有两个来源——一个是商品的稀有性,另一个是获取商品时所必需的劳动量\cite[5-6]{DaWei*LiJiaTuZhengZhiJingJiXueJiFuShuiYuanLi2021}。但由于那些单由稀有性决定价值的商品在市场上只占极小一部分,而那些“数量可以由人类劳动增加、生产可以不受限制地进行竞争地商品”占了绝大多数,所以李嘉图将分析的重点放在了后者上,并指出“决定这一商品交换另一商品时所应付出的数量的尺度,几乎完全取决于各商品上所费的相对劳动量”\cite[6]{DaWei*LiJiaTuZhengZhiJingJiXueJiFuShuiYuanLi2021}。

李嘉图同样探讨了劳动的异质性。李嘉图指出,他关注的只是商品相对价值的变动,而商品的相对价值会在市场交换中形成且“估价的尺度一经形成就很少发生变动”,所以“比较同一商品在不同时期的价值时,我们无需考虑这种商品所需劳动的相对熟练程度和强度”,即市场“消除”了劳动的异质性\cite[13-14]{DaWei*LiJiaTuZhengZhiJingJiXueJiFuShuiYuanLi2021}。

随后,李嘉图又指出影响商品交换价值的“不仅是指投在商品的直接生产过程中的劳动,而且也包括投在实现该种劳动所需要的一切器具或机器上的劳动”\cite[17]{DaWei*LiJiaTuZhengZhiJingJiXueJiFuShuiYuanLi2021},而且“劳动使用的节约必然会使商品的相对价值下降,无论这种节约是发生在制造这种商品本身所需的劳动方面,还是发生在构造协助生产这种商品的资本所需的劳动方面。”\cite[18]{DaWei*LiJiaTuZhengZhiJingJiXueJiFuShuiYuanLi2021}

据此,我们不难总结出李嘉图认为商品的价值是由耗费的劳动决定的。

\subsection{大卫·李嘉图的生产力概念}

李嘉图的著作中没有出现明确的生产力概念,我们只能从他的文字中总结出他对生产力的看法。例如,他在分析黄金的价值变动时说:如果“由于发现了更丰饶的新矿山,或是由于更有利地使用机器,用较少的劳动量就可以获得一定量的黄金,那么,我就有理由说,黄金相对于其他商品的价值发生变动的原因,是它的生产已经比较便利,或获得时所必需的劳动量已经减少”\cite[11]{DaWei*LiJiaTuZhengZhiJingJiXueJiFuShuiYuanLi2021}。李嘉图在别处分析生产力的改进时也是指商品生产时劳动量的减少,所以笔者认为李嘉图的理论中也只有绝对生产力的概念。

由于李嘉图也没有区分部门、个体的绝对生产力变动带来的影响,笔者在此先不探讨生产力与价值的关系,而是先介绍与李嘉图价值理论相似的马克思的价值理论,并在马克思的理论基础上对生产力与价值的关系做出进一步分析。

\section{卡尔·马克思}

一般认为,马克思继承和发展了李嘉图的劳动价值论\cite[347]{YueSeFu*XiongBiTeJingJiFenXiShiDi2Juan2017}\cite[84]{ChenDaiSunCongGuDianJingJiXuePaiDaoMaKeSiRuoGanZhuYaoXueShuoFaZhanLueLun2014}。

\subsection{马克思的价值理论}

从价值理论来看,马克思首先区分了交换价值和价值:价值是交换价值的基础,交换价值是价值的表现形态\cite[86-88]{ChenDaiSunCongGuDianJingJiXuePaiDaoMaKeSiRuoGanZhuYaoXueShuoFaZhanLueLun2014}。根本上来说,李嘉图之所以不能做出这种区分,是因为李嘉图把劳动价值论“仅仅是作为一种假设”,“用来说明相对价格(能观察到的市场价格)的实际长期正常状态”\cite[348]{YueSeFu*XiongBiTeJingJiFenXiShiDi2Juan2017},于是李嘉图遇到了“等量劳动创造等量价值和等量资本获得等量利润的矛盾”\cite[144]{CaiJiMingCongGuDianZhengZhiJingJiXueDaoZhongGuoTeSeSheHuiZhuYiZhengZhiJingJiXueJiYuZhongGuoShiJiaoDeZhengZhiJingJiXueYanBianShangCe2023}\cite[21-28]{DaWei*LiJiaTuZhengZhiJingJiXueJiFuShuiYuanLi2021};而马克思则是把劳动看成价值的实质——价值就是凝结的劳动本身,于是马克思遇到了价值向生产价格的转形问题\cite[348-350]{YueSeFu*XiongBiTeJingJiFenXiShiDi2Juan2017}\cite[159]{CaiJiMingCongGuDianZhengZhiJingJiXueDaoZhongGuoTeSeSheHuiZhuYiZhengZhiJingJiXueJiYuZhongGuoShiJiaoDeZhengZhiJingJiXueYanBianShangCe2023}。

再次,马克思区分了劳动与劳动力:工人出卖的是劳动力而不是劳动,劳动力的使用(劳动)创造的超过劳动力价值的价值(剩余价值)被资本家无偿占有,解决了“资本与劳动力交换违反劳动价值论”的矛盾\cite[615,581-606]{ZhongGongZhongYangMaKeSiEnGeSiLieNingSiDaLinZhuZuoBianYiJuMaKeSiEnGeSiWenJiDi5Juan2009}\cite[157-158]{CaiJiMingCongGuDianZhengZhiJingJiXueDaoZhongGuoTeSeSheHuiZhuYiZhengZhiJingJiXueJiYuZhongGuoShiJiaoDeZhengZhiJingJiXueYanBianShangCe2023}\cite[348]{YueSeFu*XiongBiTeJingJiFenXiShiDi2Juan2017}。

总的来说,马克思完善了劳动价值论,其价值理论可以简单地表述为:商品的价值由生产商品的社会必要劳动时间决定\cite[51-52]{ZhongGongZhongYangMaKeSiEnGeSiLieNingSiDaLinZhuZuoBianYiJuMaKeSiEnGeSiWenJiDi5Juan2009}。

\subsection{马克思的生产力理论}

马克思虽然对生产力有不同的提法,但各种提法的本质是清晰且一贯的\cite{YangQiaoYuShengChanLiGaiNianCongSiMiDaoMaKeSiDeSiXiangPuXi2013}\cite{DingXiaoPingZhengQueLiJieMaKeSiZhuYiDeShengChanLiGaiNian2021}——“生产力当然始终是有用的、具体的劳动的生产力,它事实上只决定有目的的生产活动在一定时间内的效率”\cite[59]{ZhongGongZhongYangMaKeSiEnGeSiLieNingSiDaLinZhuZuoBianYiJuMaKeSiEnGeSiWenJiDi5Juan2009}——这和广义价值论中绝生产力的定义是相符的。

\subsection{生产力与价值的关系}

在此基础上,马克思根据绝对生产力的描述对象不同,针对生产力与价值的关系给出了三个看似相互矛盾实则内在统一的命题\cite[273]{CaiJiMingCongGuDianZhengZhiJingJiXueDaoZhongGuoTeSeSheHuiZhuYiZhengZhiJingJiXueJiYuZhongGuoShiJiaoDeZhengZhiJingJiXueYanBianShangCe2023}。

第一,劳动生产力与加质量负相关。马克思指出,“劳动生产力越高,生产一种物品所需要的劳动时间就越少,凝结在该物品中的劳动量就越小,该物品的价值就越小”\cite[53]{ZhongGongZhongYangMaKeSiEnGeSiLieNingSiDaLinZhuZuoBianYiJuMaKeSiEnGeSiWenJiDi5Juan2009}。这里的“劳动生产力”应当是指广义价值论中的部门绝对生产力,而这里的“价值量”应当是指单位商品的价值量\cite[273]{CaiJiMingCongGuDianZhengZhiJingJiXueDaoZhongGuoTeSeSheHuiZhuYiZhengZhiJingJiXueJiYuZhongGuoShiJiaoDeZhengZhiJingJiXueYanBianShangCe2023}。该命题与广义价值论的判断是一致的。

第二,劳动生产力与价值量正相关。马克思指出:“生产力特别高的劳动起了自乘的劳动的作用,或者说,在同样的时间内,它所创造的价值比同种社会平均劳动要多。”\cite[370]{ZhongGongZhongYangMaKeSiEnGeSiLieNingSiDaLinZhuZuoBianYiJuMaKeSiEnGeSiWenJiDi5Juan2009}这里的“劳动生产力”应当是指广义价值论中的个别绝对生产力,而这里的“价值量”应当是指单个生产者在单位劳动时间内所创造的价值总量\cite[273]{CaiJiMingCongGuDianZhengZhiJingJiXueDaoZhongGuoTeSeSheHuiZhuYiZhengZhiJingJiXueJiYuZhongGuoShiJiaoDeZhengZhiJingJiXueYanBianShangCe2023}。该命题与广义价值论的判断也是一致的。

第三,劳动生产力与价值量不相关。马克思指出,“不管生产力发生了什么变化,同一劳动在同样的时间内提供的价值量总是相同的”\cite[60]{ZhongGongZhongYangMaKeSiEnGeSiLieNingSiDaLinZhuZuoBianYiJuMaKeSiEnGeSiWenJiDi5Juan2009}。这里的“劳动生产力”应当是指部门绝对生产力,而这里的“价值量”应当是指部门商品价值总量\cite[274]{CaiJiMingCongGuDianZhengZhiJingJiXueDaoZhongGuoTeSeSheHuiZhuYiZhengZhiJingJiXueJiYuZhongGuoShiJiaoDeZhengZhiJingJiXueYanBianShangCe2023}。该命题与广义价值论的判断出现了矛盾。出现这种矛盾的根本原因,是马克思否认了非劳动要素对价值决定的影响\cite[274]{CaiJiMingCongGuDianZhengZhiJingJiXueDaoZhongGuoTeSeSheHuiZhuYiZhengZhiJingJiXueJiYuZhongGuoShiJiaoDeZhengZhiJingJiXueYanBianShangCe2023}。