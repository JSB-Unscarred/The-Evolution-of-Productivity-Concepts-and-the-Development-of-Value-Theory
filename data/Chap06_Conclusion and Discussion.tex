% !TeX root = ../2019080346_Mason.tex
\chapter{总结与讨论}

\section{交换价值、价值、价值决定和价值尺度}

\subsection{劳动作为衡量交换价值的尺度}

如前所述,亚当·斯密在国富论中首次提出了“交换价\footnote{这里应当是价“价值”而非“交换价值;这里笔者先按照几位古典政治经济学家的原文使用“交换价值”一词;交换价值和价值的关系在后文中阐述。}值的真实尺度”问题,并指出劳动是这一问题的答案。而后李嘉图、马克思、马尔萨斯都认为劳动是衡量交换价值的尺度。值得注意的是,马克思认为古典政治经济学的根本倾向是劳动价值论,但这一论断忽视了古典政治经济学不同理论之间的差异性\footnote{由于学界对古典政治经济学的划分不尽相同,对古典政治经济学家思想的理解也不尽相同,所以有学者认为古典政治经济学所得出的结论是“多要素供求价值论”\cite[179]{CaiJiMingCongGuDianZhengZhiJingJiXueDaoZhongGuoTeSeSheHuiZhuYiZhengZhiJingJiXueJiYuZhongGuoShiJiaoDeZhengZhiJingJiXueYanBianShangCe2023}。}。根据笔者对古典政治经济学的认识,笔者认为更有说服力的结论是:古典政治经济学的基本倾向是把劳动作为衡量交换价值的尺度。

一方面,古典政治经济学家意识到,在社会分工建立起来后,一个生产者或许因为不能生产某一需要的商品,或许因为通过交换得到的商品更好更便宜,所以会愿意用自己的劳动生产一些不愿自己消费的商品来交换他物。然而,商品都具有作为使用价值的异质性,为了与其它商品进行交换,商品的拥有者会根据在这一商品上花费的劳动\footnote{包括活劳动和物化劳动}与他人进行交换。这是因为这个生产者为了生产者这一商品所投入的就是自己的劳动,也只能知道自己为了这一商品投入了多少劳动。而那些与他进行商品交换的生产者,也是以同样的考量与他进行交换。因此,劳动成为了商品交换时所依据的尺度。换言之,劳动成为了衡量交换价值的尺度。\cite[1016-1017]{ZhongGongZhongYangMaKeSiEnGeSiLieNingSiDaLinZhuZuoBianYiJuMaKeSiEnGeSiWenJiDi7Juan2009}\cite[25]{YaDang*SiMiGuoFuLun2015}

另一方面,交换价值会随着时间和地点的不同上下波动,具有偶然性和相对性。因此,古典政治经济学家所能找到的不变的、普遍的交换价值的尺度,只有劳动。\cite[49-51]{ZhongGongZhongYangMaKeSiEnGeSiLieNingSiDaLinZhuZuoBianYiJuMaKeSiEnGeSiWenJiDi5Juan2009}\cite[28-30]{YaDang*SiMiGuoFuLun2015}

\subsection{从交换价值到价值}

然而,交换价值是一种使用价值同另一种使用价值的比例,其本身没有单位,也没有数量大小上的意义。所以前文中古典政治经济学家使用“交换价值的尺度”这一说法是不合理的——一个没有大小的量怎么能被“衡量”呢?事实上,古典政治经济学家所衡量的其实不是交换价值,而是价值。只是在政治经济学发展的早期阶段,经济学家们还没有能力从交换价值中抽象出价值的概念。直到马克思首次区分了交换价值和价值,价值的面纱才被揭开。

笔者认为,政治经济学逐步区分交换价值和价值的过程可以从两个角度来理解。

首先,从价值实体的角度来看,假设某种商品可以和多种商品进行交换,那么这一商品就具有了许多种交换价值,这些交换价值必然也是可以相互替代的交换价值。所以,“第一,同一种商品的各种有效的交换价值表示一个等同的东西。第二,交换价值只能是可以与它相区别的某种内容的表现方式,‘表现形式’。”\cite[49]{ZhongGongZhongYangMaKeSiEnGeSiLieNingSiDaLinZhuZuoBianYiJuMaKeSiEnGeSiWenJiDi5Juan2009}也就是说,这些不同的交换价值都可以被化为“一种等量的共同的东西”\cite[49]{ZhongGongZhongYangMaKeSiEnGeSiLieNingSiDaLinZhuZuoBianYiJuMaKeSiEnGeSiWenJiDi5Juan2009}。而且,这种共同东西既不是使用价值——使用价值具有异质性,也不是交换价值——交换价值仅仅是一个比例,没有大小的,无法被“衡量”。进而,马克思把这种共同的东西称为价值\cite[50]{ZhongGongZhongYangMaKeSiEnGeSiLieNingSiDaLinZhuZuoBianYiJuMaKeSiEnGeSiWenJiDi5Juan2009}。值得注意的是,尽管马克思已经能从交换价值中抽象出价值来,但是笔者却不认同马克思推演的逻辑。马克思在意识到不同的交换价值是在衡量一个既不是使用价值也不是交换价值的“共同的东西”之后,就说“如果把商品体的使用价值撇开,商品体就只剩下一个属性,即劳动产品这个属性”\cite[50-51]{ZhongGongZhongYangMaKeSiEnGeSiLieNingSiDaLinZhuZuoBianYiJuMaKeSiEnGeSiWenJiDi5Juan2009},再把具体劳动的成分撇开,那么剩下的“只是无差别的人类劳动的单纯凝结”,这种凝结的抽象劳动,就是商品的价值\cite[51]{ZhongGongZhongYangMaKeSiEnGeSiLieNingSiDaLinZhuZuoBianYiJuMaKeSiEnGeSiWenJiDi5Juan2009}。这里的问题主要在于:如果认为使用价值是异质的而不能作为价值决定的因素,那么,劳动也是异质的, 同样不能作为价值决定的因素;如果认为各种具体的异质的劳动可以抽象为无差别的一般人类劳动,那么,这一抽象过程同样适用于各种异质的使用价值。或者说,当人们将具体劳动抽象为无差别的一般人类劳动的同时,事实上也就把各种使用价值抽象为一般的使用价值即效用\cite[84]{CaiJiMingLunJieZhiJueDingYuJieZhiFenPeiDeTongYi2003}。也就是说,将商品的价值和“无差别的人类劳动的单纯凝结”等同起来并不是一个不证自明的过程,马克思的论证逻辑是存在问题的。但是,笔者认为马克思将价值从交换价值中抽象出来的逻辑过程是正确的,价值确实是一种既不同于使用价值也不同于交换价值的,客观存在的实体。

正因如此,笔者进一步认为前文中不同经济学家对“交换价值的尺度”的争论实际上是对“价值尺度”的争论。交换价值本身只是两种使用价值的比例,没有“大小”的概念,因此也无法被“衡量”。而价值作为一个实体,理论上是可以被衡量的。不同经济学家所争论的,正是用什么尺度来衡量价值,或者说价值的单位究竟是什么。

其次,人们对价值认识的演进和商品交换的发展是两个相反的过程。在商品交换的早期阶段,商品的“交换价值首先表现为一种使用价值同另一种使用价值相交换的量的关系或比例”\cite[49]{ZhongGongZhongYangMaKeSiEnGeSiLieNingSiDaLinZhuZuoBianYiJuMaKeSiEnGeSiWenJiDi5Juan2009},也就是简单的交换价值形式\footnote{在资本论中,马克思用的是“简单价值形式”一词,但由于马克思没有绝对严谨地区分价值和交换价值\cite[37]{ZhongGongZhongYangMaKeSiEnGeSiLieNingSiDaLinZhuZuoBianYiJuMaKeSiEnGeSiWenJiDi8Juan2009},所以这里笔者采取了蔡继明教授的更严谨的提法\cite[145]{CaiJiMingJieZhiZhengLunHuiGuYuZhanWang2008}。};随着交换范围的扩大,简单的交换价值形式发展为扩大的交换价值形式;当一般等价物出现后,所有使用价值都以一般等价物为媒介而进行交换,交换价值就发展为一般的形式;当货币产生后,交换价值便取得价格这种形式。尽管市场价格受供求波动影响,但长期观察显示其始终会围绕着一个相对稳定的轴心运动——这个轴心,或者说调节价格运动的规律,被亚当·斯密称为"自然价格"、马尔萨斯谓之"自然价值",最终在政治经济学发展中凝练为"价值"概念这样,价值作为调节价格运动的规律这一特定的内涵便确定下来。\cite[145]{CaiJiMingJieZhiZhengLunHuiGuYuZhanWang2008}随着人们进一步认识到价格只是交换价值的一种形式,价值的内涵也进一步一般化为调节交换价值的规律。

正如马克思所言:“对人类生活形式的思索,从而对这些形式的科学分析,总是采取同实际发展相反的道路。这种思索是从事后开始的,就是说,是从发展过程的完成的结果开始的。$\cdots$因此,只有商品价格的分析才导致价值量的决定,只有商品共同的货币表现才导致商品的价值性质的确定。”\cite[93]{ZhongGongZhongYangMaKeSiEnGeSiLieNingSiDaLinZhuZuoBianYiJuMaKeSiEnGeSiWenJiDi5Juan2009}人们对价值的研究,的确是从对价格的研究开始的。


\section{三种价值理论}

