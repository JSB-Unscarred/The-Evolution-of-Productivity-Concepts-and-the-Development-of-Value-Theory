% !TeX root = ../2019080346_Mason.tex
\appendix

\begin{translation}
\label{cha:translation}

\title{马歇尔的新古典劳动价值论}
\maketitle

\tableofcontents

\section{引言}

随着边际效用理论的兴起,19世纪末的经济学家们拥有了一套解释交换价值的理论,却未能建立一套真正的价值理论。例如,威廉斯坦利杰文斯(William Stanley Jevons)曾主张将“价值”概念从经济学中剔除,仅保留对交换比率的研究(Jevons 1879)。时至今日,许多一般均衡经济学家仍持有类似观点。然而,无论是当时还是现在,经济学家在分析经济随时间的变化或跨国比较时,始终需要一种更广义的“价值”概念。他们希望回答诸如“商品的实际价值如何随时代变迁或地域差异而变动”这类根本问题。

边际主义者们沿袭了数百年的经济思想传统(如洛克1696/1991),深知货币并非衡量价值的可靠尺度——与其他商品一样,货币的购买力会因时空不同而波动。这一认知引发了对“货币一般交换价值”度量标准的探索。经济学家们借鉴古典文献中关于“标准商品计价”(tabular standard)的讨论,提出了多种通过加权平均商品价格构建价格指数的方案。令人意外的是,这一领域的先驱竟是杰文斯本人(Jevons 1893)\footnote{关于探索货币一般交换价值的最全面的论述可见于Walsh(1901)。另见Persky(1998)。}。

阿尔弗雷德马歇尔(Alfred Marshall)深度参与了这场关于价值度量的讨论。他不仅贡献了链式指数(chain indexes)的构建方法(Marshall 1887/1925),还提出:在某些场景下,以基本工资率为基准定义的“劳动价值”(labor-values)比价格指数调整后的“实际价值”(real values)更具优势。遗憾的是,马歇尔对此的论述零散而简略,远未形成完整的理论体系。

本文旨在梳理马歇尔关于劳动价值的论述及其实际应用,并尝试重构其新古典主义框架下的理论逻辑。

\section{马歇尔的价值概念}

“价值”并非马歇尔理论的核心概念。正因如此,人们很容易忽略一个事实:马歇尔在《经济学原理》中至少提出了四种不同的价值定义。

\subsection{交换价值}

在《原理》第二章,马歇尔先以亚当斯密的名言开篇:“‘价值’一词有两种不同含义,有时指特定物品的效用,有时指占有该物品所能换取的购买力”(斯密1776/1937,第28页;马歇尔1920,第61页)。但他随即补充道:“经验表明,将‘价值’用于前一种含义并不明智。”紧接着,他给出了交换价值的经典定义——某物在特定时空下的交换价值即“该时空下用此物可换得的他物数量”(马歇尔1920,第61页)。

\subsection{价格作为一般购买力的衡量}

马歇尔指出,商品与货币的交换价值即价格。他承认货币购买力可能波动,但建议“本书将忽略货币一般购买力的潜在变化”,从而将价格视为衡量商品“相对于一般商品的交换价值”的尺度(马歇尔1920,第62页)。

\subsection{实际价值}

在第六卷中,马歇尔将“实际价值”定义为“一定量产品所能换取的生活必需品、舒适品和奢侈品的数量”(马歇尔1920,第632页)。此时他显然主张通过价格指数对货币购买力进行平减处理。他提到,对于某些用途,用生产者商品构建指数可能更优,但未具体说明适用场景。

\subsection{劳动价值}

在讨论价值之初,马歇尔便指出:“对于某些目的,用劳动而非商品来衡量货币的实际价值更为恰当”(马歇尔1920,第62页)。在第六卷第十二章“经济进步的一般影响”中,他进一步分析了多种基础商品的劳动价值(以劳动时间衡量)的历史波动。

\subsection{如何理解这四重定义?}

与多数新古典学者一样,马歇尔无意构建一般均衡体系下的交换价值矩阵。为开展局部均衡分析,他选择以货币价格定义价值,但承认当货币价值变动时需引入调整机制。问题在于:何种基准商品(numéraire)能稳定衡量价值?

马歇尔认为,短期内可用消费品或生产资料篮子衡量货币购买力,但转向历史分析时,他选择以支配的劳动(labor commanded)定义的劳动价值为尺度。这一跳跃令人费解——它需要比马歇尔本人提供的更系统的理论支撑。

\section{替代性理论依据}

马歇尔对“支配的劳动”(labor commanded)价值尺度的支持初看令人困惑:为何一位新古典边际主义者要复兴这一早期古典思想的遗存?人们很容易想象,马歇尔本应追随杰文斯的建议,直接拒绝寻找固定价值尺度的可能性。究竟何种理由能解释他对劳动价值尺度的包容?

以下三种可能性值得探讨:

马歇尔对支配的劳动价值尺度的支持初看起来似乎令人费解。为何一个新古典边际主义者要复兴这种早期古典思想的遗物?人们很容易想象马歇尔会遵循杰文斯的建议,直接拒绝寻找固定价值尺度的可能性。对于马歇尔容忍劳动支配标准的做法,可以提出哪些解释?

这里存在三种可能性。首先,马歇尔在此处(正如他在其他场合)试图强调古典与新古典思想的连续性。他仅仅是将劳动价值作为旧体系的残余保留下来。第二,在历史研究中,可能难以获得全面的价格与数量数据来构建完整的价格指数。因此,若必须选择单一的价值尺度,最直接的答案就是选取容易获得的核心价格。马歇尔可能认为基础工资率同时满足这两个条件。第三,马歇尔有严肃的新古典主义——甚至边际主义——理由来选择劳动作为价值尺度。

\subsection{古典思想遗产}

或许可以理解,秉持“连续性原则”和马歇尔的著名格言‘自然无飞跃’(Natura non facit saltum)的马歇尔,会对其思想前辈表现出高度尊重。正如约瑟夫熊彼特所观察到的,马歇尔的理论结构“不必要地堆砌着李嘉图式遗产,这些遗产获得的重视程度与其操作重要性完全不成比例”(Schumpeter 1954, p. 837)。熊彼特指出,因此“少数英国作家和大多数非英国作家”开始将马歇尔的著作视为古典经济学与边际效用学派的综合——或许是一种略显牵强的综合。熊彼特强烈反对这一观点,并令人信服地论证马歇尔本质上完全是新古典主义者。熊彼特是正确的,但当然劳动价值可能是那些遗产之一。尽管如此,若果真如此,奇怪的是马歇尔从未将其对劳动价值虽属随意但仍存在的讨论与任何古典经济学家或传统联系起来。

即使马歇尔愿意,他也很难援引大卫李嘉图作为支持。毕竟,李嘉图断然拒绝劳动支配价值,转而支持劳动凝结价值。同样,约翰斯图亚特穆勒从理论和实证角度都认为劳动支配标准存在缺陷。劳动支配价值远未被古典作家普遍接受\footnote{关于古典经济学对劳动投入论与劳动支配论价值理论的早期论述,可参见威廉罗雪尔(Wilhelm Roscher, 1882)与阿尔伯特惠特克(Albert Whitaker, 1904/1968)的著作。}。在著名古典经济学家中,只有斯密和后来的托马斯马尔萨斯认真论证过劳动支配作为价值尺度的主张。

尽管斯密从未完全解决其价值理论,但他始终认为劳动支配是合适的价值尺度。他秉持这一信念的深层原因是:在所有交易商品中,唯有劳动与劳动者明确且(可推测)恒定的牺牲相关联:

“等量劳动,在任何时间和地点,对劳动者而言都可说是等值的。在劳动者通常的健康、体力和精神状态下,在通常的技能和熟练程度中,他必然总是牺牲相同程度的安逸、自由和幸福。无论他换取的商品数量多少,他支付的价格必然总是相同的……因此,唯有劳动——其自身价值永不变化——才是能够在任何时空下对所有商品价值进行估算和比较的终极真实标准。”(Smith 1776/1937, p.33)

斯密的论点显然基于对普通劳动者实际经验的主观看法。就此而言,它可能吸引马歇尔这样的新古典主义者,但几乎无法获得古典经济学家的普遍认同。

甚至马尔萨斯最初也摒弃了斯密的劳动支配价值。在其《政治经济学原理》第一版中,他提出了一个将农业工资与谷物价格平均的权宜价值尺度。只有在与李嘉图进行大量辩论后,马尔萨斯才宣布其新信念——斯密的简单劳动支配方法实际上是正确的。其《原理》第二版的修订主要就是出于这种观念转变\footnote{见“Advertisement to the Second Edition” (Malthus 1836/1989, pp. 9-12)。}。马尔萨斯对此的论证最为有力。在评论英美工人工资差异时,他总结道:

“(美国劳动者)并非为他所得支付更多,而是因他所付获得更多;除非我们想以产品数量作为价值尺度(这将导致最荒谬且无法解决的困难),否则我们必须用劳动者付出的劳动来衡量他在美国所获物品的价值。”(Malthus 1836/1989, Vol. II, p.104)

马尔萨斯的立场与斯密一样,指向了主观(负)效用价值理论。几乎所有其他古典经济学家都否认劳动者主观经验的核心地位。例如,李嘉图将工资视为系统中的另一种价格。在讨论了贵金属和谷物价值的常见波动后,李嘉图继续写道:

“劳动的价值难道不是同样易变吗?它不仅像所有其他事物一样受供求比例影响(这种比例随社会状况的每次变化而不断改变),还受食物和其他必需品的价格波动影响——劳动者的工资正是花费在这些物品上。”(Ricardo 1821/1951, Vol I, p. 15)

对李嘉图而言,只要工资相对于其他商品上涨或下跌,劳动支配就无法成为恒定标准。

这自然将他推回自己的劳动凝结理论。当马尔萨斯指责他用成本标准替代价值尺度时,李嘉图接受了这个指控:

“我认为商品的真实价值与其生产成本是同一回事,两种商品的相对生产成本大致与其各自从始至终投入的劳动量成比例。”(Ricardo 1951, Vol. II, p. 35)

李嘉图以供给成本为导向的价值理论,几乎不可能成为马歇尔劳动支配价值的基础。

同样,当J.S.穆勒构建其价值思想时,直接拒绝了斯密和马尔萨斯的观点。他呼应李嘉图的观点:

“如果美国劳动者一日劳动能购买的普通消费品是英国的两倍,那么坚持认为两国劳动价值相同、其他物品价值不同,不过是徒劳的诡辩。在这种情况下,可以正确地说——无论对市场还是劳动者自身而言——美国劳动的价值是英国的两倍。”(Mill 1929, p. 567)

对穆勒而言(我们还可以将约翰麦克库洛赫和西尼尔加入此列\footnote{在重要晚期古典经济学家中,托马斯图克(Thomas Tooke)是这一共识的例外。在其鸿篇巨制《价格史》中,他断言:“……如今可以——也理应——视作公认结论(尽管仍存争议)的是:相较于谷物,普通日常劳动的货币价格是衡量贵金属价值的更优标准”(Tooke 1838/1928, p. 56)。这一论点在图克的核心主张中具有关键作用,即尽管谷物价格跌幅更大,但白银的实际价值在十八世纪大部分时期持续下降。}),劳动的主观负效用与劳动交换价值毫无关系。
将马歇尔对劳动支配价值的使用简单归类为“古典遗产”是困难的。古典经济学家在恰当价值尺度问题上存在根本分歧。相反,我们更有理由推断马歇尔对斯密价值理论中的主观主义要素抱有真正共鸣。在此解读下,马歇尔对劳动价值的使用绝非对古典图腾的盲目接受,而是对相互冲突的古典主题的有意识选择\footnote{马歇尔关于“真实成本体现为代价的主观负效用”的观点,早先已被迈因特(Myint 1948, p. 133)与斯密相联系,并与李嘉图形成对照。感谢一位审稿人提供此文献来源。}。

\subsection{简单计价物(Simple Numeraire)}

选择工资作为计价标准的理由可能源于工资数据的相对可得性。然而自亚当斯密以来,经济学家就意识到界定恒定特征与质量的劳动极其困难。实际上,斯密明确指出工资数据难以标准化。在论证谷物价格与基础工资高度相关后,他得出结论:实践中最佳选择是以谷物价格为计价标准(Smith 1776/1937, p. 38)。

马歇尔从未提出基于可得性的工资率论证。相反,他指出历史工资的错误估算极易导致误导性劳动价值。他特别批评某研究者将“人口中较优越阶层的工资视为整体代表”(Marshall 1920, p. 675)。若马歇尔唯一的考虑是数据可得性,最自然的选择应是选取基础商品(如谷物)价格,用以代表整体消费组合。这种做法确有充分先例。

\subsection{新古典方案}
新古典主义方案试图将边际分析运用于“真实世界”\footnote{关于边际效用学派与新古典主义者之间的区别,参见Myint(1948)的论述。}的实用经济学,而非形式主义的机械套用。在马歇尔对新古典方案的整体承诺下,我们可以运用何种理论论证来合理化其对劳动支配价值的使用?

首先,马歇尔明确认为对一般价值衡量标准的普遍需求具有合理性。这种工具对历史研究与国际比较不可或缺。作为新古典主义者,马歇尔不能像某些边际主义者那样耸肩回避对这种尺度工具的实际需求。

如前所述,马歇尔对斯密与马尔萨斯理论中隐含的主观主义立场抱以同情。尽管包含多重主题,马歇尔的论述始终围绕新古典作家共有的主观边际效用理论展开。对他而言,劳动负效用是明确定义的核心概念。

新古典主义者马歇尔也未必会反对斯密与马尔萨斯进行的跨期与跨人际效用比较。当然,他承认此处存在严重问题:并非所有个体都相同,有人可能更健康或更偏好工作等。但马歇尔并不排斥效用平均化,他认为将劳动负效用曲线视为基本给定并无内在困难\footnote{参见马歇尔(Marshall 1920, pp. 18-20)对个体差异与群体福利的讨论。马歇尔得出结论认为,对于足够大规模的群体,其内部差异将会相互抵消。该方法与其对代表性个体或企业的运用完全一致。}。

但我们仍要追问:为何马歇尔关注劳动负效用而非直接关注商品效用?他明确做出这种选择体现在对劳动价值的使用。但原因何在?为何以努力而非获得的享受作为更恰当的衡量标准?答案可能在于马歇尔深信物质消费面临显著的边际效用递减,因此货币的边际效用随收入递减。马歇尔明确指出:“一先令对富人产生的愉悦或满足感小于对穷人”(Marshall 1920,p. 19)。这意味着“获得的享受量”不能简单地以消费品数量衡量。因为若社会收入分配改变,商品价值必然随之变化。

随着时间推移,技术进步带来劳动生产率提升。因此我们可以预期大多数劳动者将享受越来越多的消费。马歇尔坚信这种普遍进步。但如果这些劳动者对后续消费增量的边际评价递减,那么固定商品组合的效用价值必然下降。因此,这样的商品篮子几乎不可能作为长期不变的价值尺度。

当然,马歇尔不会否认技术进步允许产出数量增长。为此,使用商品篮子的指数可能是最有用的工具。这种指数与马歇尔的“真实价值”概念一致。但此类指数无法捕捉人口体验的价值变化,因为其消费价值尺度在消费能力提升时也在改变。

在此背景下,新古典主义者马歇尔很可能被劳动价值吸引。遵循斯密的基本论证,马歇尔指出,劳动负效用作为主观价值标准,具有跨越时空的恒定性。。若想了解某商品价值如何变化,最佳方法就是将其价格与劳动价格比较。

这种解读为马歇尔本人零散且令人困惑的劳动价值论述提供了最清晰的合理化。特别值得注意的是,包含前文引述的关键文本(第二节第258页)的完整句子写道:“但如果发明极大地增强了人类对自然的掌控力,那么对某些目的而言,以劳动而非商品衡量货币的真实价值更为恰当”(Marshall 1920, p. 62)。马歇尔明确指出,劳动价值在生产率发生显著变化时最有用——实际上可能仅在此类情况下有用\footnote{应当指出,马歇尔将这一基本观点延伸至其货币政策主张中。根据彼得格罗尼维根的研究,马歇尔曾认真考虑在《货币、信用与商业》一书中加入如下论述:​“统一购买力标准应基于所需劳动量而非所获效用量:因此货币供给应如此调控,以使标准质量单位劳动的平均报酬作为一般购买力的计量单位”(格罗尼维根(Groenewegen),1995,721)。 

正如本文一位审稿人所指出的,这一主张很可能源于马歇尔(Marshall)试图制定应对工资黏性的货币政策。马歇尔认为,长期通缩有助于实际工资增长,从而提升劳动者福利。在马歇尔看来,长期通缩的分配效应比长期通胀更为有利(Laidler 1990, pp. 62-63)。

这种将“工资黏性/宏观经济学”视角解读马歇尔以工资作为基本购买力单位的做法,与正文自然推导出的“不变负效用”解读形成互补。无论我们还是马歇尔都无需在两者间做取舍。但必须强调的是,马歇尔对劳动价值计量法的承诺绝非非对称性的——即不仅限于生产率增长情景,同样适用于生产率衰退情形。参见正文下一段落。}。 

若劳动价值在生产率提升时期有效,则在生产率下降时期也应适用。对新古典主义者马歇尔而言,这种下降最可能发生在国家经历“人口对生存资料压力持续增加”之时。此类马尔萨斯式状况必然导致“人民贫困化”与“初级产品劳动价值的上升”(Marshall 1920, p. 632)。

\section{马歇尔论证的形式化}

若上述解读准确反映马歇尔思想,他最终将其价值尺度与相对标准脱钩,转而建立在假定恒定的劳动边际负效用这一绝对标准之上。

假设存在具有明确效用函数(跨时空恒定)的代表性工人\footnote{从马歇尔(Marshall)的著作中无法明确他主张采用何种工资率作为计价基准。他仅笼统提及“特定种类的劳动”。在实证分析中,他始终未具体说明所使用的工资标准。推测其思路与亚当斯密(Adam Smith)相似,应指主要从事体力劳动的低技能工人。​这种选择在低技能劳动者占劳动力主体时具有合理性。然而,当半技术工人和技术工人在劳动力中的比例持续扩大时(如当前美国经济扩张期所示),低技能工资率的相关性将面临挑战——此时“平均劳动者”的边际负效用已不等同于低技能工人的边际负效用。尤其当低技能工资增速显著落后于技术工人工资增长时,这一观察结论更具说服力。}。马歇尔试图论证劳动边际负效用的恒定性。但在何种情况下这种主张成立?正如马歇尔本人强调,特定类型劳动的边际负效用随工时长度变化。比较不同国家或时期时,工时通常并不相同。实际上,我们不禁要问:马歇尔在关于劳动强度恒定性的各种暗示中,为何回避了自己的结论——“劳动边际负效用通常随劳动量增加而增强”(Marshall 1920, p. 141)?若边际负效用随日/周工时变化,那么即使个体基本相似,不同历史时期的边际小时价值也会变化。

马歇尔对劳动供给的分析(1920年,数学附录注释X,p. 843)假设总效用实质上是消费产生的效用$U$与劳动负效用$V$之差。则最大化$U-V$的一阶条件为:
\begin{equation}
    \label{eq1}
    \frac{\dif v}{\dif l} = \left( \frac{\dif u}{\dif m} \right) * w
\end{equation}

其中$l$是劳动时间,$m$是名义收入,$w$是名义工资率。若马歇尔寻求与消费中商品边际效用成比例的价值衡量标准,则在其效用最大化描述中,只需找到与下式成比例的尺度:
\begin{equation}
    \label{eq2}
    \left( \frac{\dif u}{\dif m} \right) * p_i
\end{equation}

其中$p_i$是商品$i$的名义价格。若构建支配的劳动价值尺度,则取:
\begin{equation}
    \label{eq3}
    \frac{p_i}{w} = \left( \frac{\dif u}{\dif m} \right) * \frac{p_i}{\frac{\dif v}{\dif l}}
\end{equation}

若跨期(或跨空间)劳动者工时保持不变,则$\sfrac{\dif v}{\dif l}$为常数,劳动边际负效用不变化,此时劳动支配标准将与式\ref{eq2}成比例。

但如前所述,该结论仅在个体劳动供给曲线对工资变化无弹性时成立。若工时随工资变化,劳动支配标准的增速将快于或慢于应有水平。在深入分析这种偏差前,首先考虑传统价格指数框架下的相同问题。

我们的目标仍是估计式\ref{eq2}中的价值。此时的估计值为:
\begin{equation}
    \label{eq4}
    \frac{p_i}{P}
\end{equation}

其中$P$是价格指数。马歇尔称此类比值为真实价值。只要$\sfrac{\dif u}{\dif m}$与$\sfrac{1}{P}$成比例——即仅价格变化影响名义收入的边际效用——该式就成立。但马歇尔的基本观点是:无论价格如何变化,当劳动者获得更多购买力并消费更多商品时,收入边际效用下降将影响商品价值。令经价格调整的实际购买力$\sfrac{m}{P}$为$y$,则:
\begin{equation}
    \label{eq5}
    \frac{\dif u}{\dif m} = \left( \frac{\dif u}{\dif y} \right) * \left( \frac{\dif y}{\dif m} \right) = \frac{\frac{\dif u}{\dif y}}{P}
\end{equation}

其中$\left(\sfrac{\dif u}{\dif y}\right)$是经价格调整购买力的边际效用。因此:
\begin{equation}
    \label{eq6}
    \frac{p_i}{P} = \left( \frac{\dif u}{\dif m} \right) * \frac{p_i}{\frac{\dif u}{\dif y}}
\end{equation}

式\ref{eq6}右侧与式\ref{eq3}右侧相似,但$\sfrac{\dif v}{\dif l}$被替换为$\sfrac{\dif u}{\dif y}$。

秉承马歇尔的新古典方法,价尺度问题可归结为在式\ref{eq6}与式\ref{eq3}之间的选择——即在他的劳动价值与真实价值之间。由于$\sfrac{\dif v}{\dif l}$与$\sfrac{\dif u}{\dif y}$都可能跨时空变化,两种估计都不完美。但我们可以追问(正如马歇尔本人隐含的:$\sfrac{\dif v}{\dif l}$与$\sfrac{\dif u}{\dif y}$何者更易随时间空间变化?

将式\ref{eq5}代入式\ref{eq1}得:
\begin{equation}
    \label{eq7}
    \frac{w}{P} = \frac{\frac{\dif v}{\dif l}}{\frac{\dif u}{\dif y}}
\end{equation}

考虑马歇尔认为最适用劳动价值的情形:技术显著改进、生产率提升。此时预期$\sfrac{w}{P}$上升且$\sfrac{\dif u}{\dif y}$下降,但$\sfrac{\dif v}{\dif l}$可能依劳动供给弹性向任意方向移动。历史上,工资上升通常伴随工时下降。假设工时减少导致$\sfrac{\dif v}{\dif l}$下降。由于$\sfrac{w}{P}$上升,可知$\frac{\dif v}{\dif l}$的降幅必然小于$\sfrac{\dif u}{\dif y}$的降幅。此时我们可以较有把握地断言,马歇尔劳动价值的偏差小于其真实价值。

上述论证本身颇具说服力。我们也希望将其视为对马歇尔立场的合理化或形式化。但必须承认,这种主张可能略显牵强。马歇尔明确指出:对特定类型的固定劳动力,工资上升将带来更多工时。在若干次要限制条件下,他总结道:“大体而言,劳动者群体的努力程度将随报酬增减而升降”(Marshall 1920, p. 142)。他迫切希望建立劳动供给的正常向上倾斜曲线,这正是达成该目标所需的假设。但这意味着实际工资上升将导致$\sfrac{\dif v}{\dif l}$增加(尽管增速必然放缓)。此时我们无法确知$\sfrac{\dif u}{\dif y}$的降幅是否绝对大于$\sfrac{\dif v}{\dif l}$的增幅。

尽管如此,马歇尔注意到工时的长期下降趋势。在讨论“经济进步的普遍影响”时,他明确指出:

“除最高等级外,所有工种的劳动者较之以往更重视休闲,更不耐过度劳累导致的疲劳;总体而言,他们可能较过去更不愿为获取眼前奢侈品而忍受超长工时带来的持续增加的‘负效用’。”(Marshall 1920, pp. 680-81)

在重构马歇尔关于工时对工资敏感性的观点时,或许最佳结论是:他认为短期工资上升会增加工时,但存在暗示表明长期中他看到工时与工资率的反向关系。而后者正是马歇尔认为最适用劳动价值的情形——即足以产生重大技术进步的长时期。

\section{结论}

把支配劳动作为价值尺度在本质上比把劳动耗费作为价值尺度更契合新古典经济学。劳动支配方法在强调价值尺度的主观性时,并未对商品价值的源泉提供解释,也未曾言及劳动与资本间收入分配的正当性。它仅提供对结果的测量,而非价值判断。相比之下,劳动投入理论似乎不可避免地暗示对收入分配的批判。或许这正是马歇尔能轻易采用劳动支配价值作为尺度,同时援引新古典供需框架进行解释的原因。

剑桥学派对劳动支配价值的兴趣在后续发展中续写篇章。约翰梅纳德凯恩斯在《通论》中明确拒绝使用价格指数,转而建议采用基于总工资额与技能调整总工时的工资单位。当凯恩斯谈及物价上升时,他意指相对于此工资单位的上升。此处不宜详论凯恩斯的工资理论,但值得注意的是:在将工资单位作为明确的短期计价标准提出时,凯恩斯建议价格指数适用于长期,即“在特定(可能相当宽泛)限度内公开承认不精确与近似性的历史比较”(Keynes 1936, p.43)。

马歇尔最杰出的学生对其核心信息的这种讽刺性反转是否蕴含更深意涵?在《凯恩斯指南》中,阿尔文$\cdot$汉森认为凯恩斯使用价格指数同样可以处理短尺度问题。但他着重强调:长期中采用工资率平减将低估产出,因其未考虑生产率增长。汉森似乎完全忽视了价值理论的论证。这或许正是关键所在。不论好坏,在当代,价值尺度已成为虚幻而几被遗忘的实践。量化尺度已成为核心挑战\footnote{在完成最后这句论述后,我注意到近期有两项大规模的劳动价值测算研究。瑞士联合银行经济研究部(UBS Economic Research, 1997)在全球范围内测算了巨无霸汉堡、小麦和大米的劳动支配价值。达拉斯联邦储备银行(Cox与Alm,1997)的研究则通过对比二十世纪初与当代数据,测算了美国各类商品服务的劳动价值变迁。我相信马歇尔定会赞许此类研究。}。


% 书面翻译的参考文献
% 默认使用正文的参考文献样式;
% 如果使用 BibTeX,可以切换为其他兼容 natbib 的 BibTeX 样式。
% \bibliographystyle{unsrtnat}
% \bibliographystyle{IEEEtranN}

% 默认使用正文的参考文献 .bib 数据库;
% 如果使用 BibTeX,可以改为指定数据库,如 \bibliography{ref/refs}。
\notecite{almTimeWellSpent1997}
\notecite{groenewegenSOARINGEAGLEAlfred1995}
\notecite{hansenGuideKeynes1953}
\notecite{huttEconomistsPublicStudy1990}
\notecite{jevonsMoneyMechanismExchange1893}
\notecite{jevonsTheoryPoliticalEconomy}
\notecite{keynesGeneralTheoryEmployment1936}
\notecite{lockeConsiderationsConsequencesLowering1991}
\notecite{laidlerAlfredMarshallDevelopment1990}
\notecite{malthusPrinciplesPoliticalEconomy1989}
\notecite{marshallPrinciplesEconomics1920}
\notecite{marshallRemediesFluctuationsGeneral1925}
\notecite{milPrinciplesPoliticalEconomy1929}
\notecite{myintTheoriesWelfareEconomics1948}
\notecite{perskyMarshallsNeoClassicalLaborValues1999}
\notecite{ricardoPrinciplesPoliticalEconomy1951}
\notecite{roscherPrinciplesPoliticalEconomy1882}
\notecite{smithWealthNations1937}
\notecite{tookeHistoryPricesState1838}
\notecite{ubseconomicresearchPricesEarningsGlobe1997}
\notecite{walshMeasurementGeneralExchangeValue1901}
\notecite{whitakerHistoryCriticismLabor1968}

\printbibliography

% 书面翻译对应的原文索引
\begin{translation-index}
  \notecite{perskyMarshallsNeoClassicalLaborValues1999}
  \printbibliography 
\end{translation-index}

\end{translation}